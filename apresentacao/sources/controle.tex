%SEÇÃO------------------------------------------------------------------

\section{Controle Multimalha}

  \frame{
    \frametitle{\insertsection}
    \vfill
    \begin{itemize}
      \item Projeto do controlador da malha interna \cite{ref:DANNEHL}
        \begin{itemize}
          \item Variável intermediária: tensão do capacitor $v_C$
          \item Variável intermediária: corrente do capacitor $i_C$
        \end{itemize}
      \vfill
      \item Projeto do controlador da malha externa
    \end{itemize}
    \vfill
  }

  \frame{
    \frametitle{\insertsection}
    \vfill
    Variável controlada na malha interna: tensão do capacitor $v_C$
    \vfill
    \begin{equation}
      C_i(z) = \left( K_P + K_D \right) \frac{z- \frac{K_D}{K_P+K_D}}{z}
    \end{equation}
    \begin{equation}
      z = \frac{K_D}{K_P+K_D} > 1
    \end{equation}
    \vfill
  }

  \frame{
    \frametitle{\insertsection}
    \vfill
    \begin{small}
    \begin{figure}[htb]
      \centering
      %\raisebox{-0.5\height}{
        \def\svgwidth{0.5\textwidth}
        % This file was created by matlab2tikz v0.4.7 running on MATLAB 7.14.
% Copyright (c) 2008--2014, Nico Schlömer <nico.schloemer@gmail.com>
% All rights reserved.
% Minimal pgfplots version: 1.3
% 
% The latest updates can be retrieved from
%   http://www.mathworks.com/matlabcentral/fileexchange/22022-matlab2tikz
% where you can also make suggestions and rate matlab2tikz.
% 
%
% defining custom colors
\definecolor{mycolor1}{rgb}{0.66667,0.66667,0.66667}%
%
\begin{tikzpicture}

\begin{axis}[%
width=0.8\textwidth,
height=0.461611624834875\textwidth,
scale only axis,
xmin=-1.9,
xmax=1.9,
xtick={-1,  0,  1},
xlabel={Eixo Real},
ymin=-1.1,
ymax=1.1,
ytick={-1,  0,  1},
ylabel={Eixo Imaginário},
scaled y ticks = false,
y tick label style={/pgf/number format/.cd, fixed, fixed zerofill, precision=0},
scaled x ticks = false,
x tick label style={/pgf/number format/.cd, fixed, fixed zerofill, precision=0}
]
\addplot [color=mycolor1,dotted,line width=1.0pt,forget plot]
  table[row sep=crcr]{1	0\\
0.987688340595138	0.156434465040231\\
0.951056516295154	0.309016994374947\\
0.891006524188368	0.453990499739547\\
0.809016994374947	0.587785252292473\\
0.707106781186548	0.707106781186547\\
0.587785252292473	0.809016994374947\\
0.453990499739547	0.891006524188368\\
0.309016994374947	0.951056516295154\\
0.156434465040231	0.987688340595138\\
6.12323399573677e-17	1\\
-0.156434465040231	0.987688340595138\\
-0.309016994374947	0.951056516295154\\
-0.453990499739547	0.891006524188368\\
-0.587785252292473	0.809016994374947\\
-0.707106781186547	0.707106781186548\\
-0.809016994374947	0.587785252292473\\
-0.891006524188368	0.453990499739547\\
-0.951056516295154	0.309016994374948\\
-0.987688340595138	0.156434465040231\\
-1	1.22464679914735e-16\\
};
\addplot [color=mycolor1,dotted,line width=1.0pt,forget plot]
  table[row sep=crcr]{1	0\\
0.972218045703082	0.153984211042097\\
0.921496791140463	0.299412457456957\\
0.849791018366557	0.432990150609548\\
0.759508516401756	0.551815237548133\\
0.653437051516106	0.653437051516106\\
0.534664304371591	0.735902282058355\\
0.406492952796512	0.797787339572243\\
0.272353130096494	0.83821674479613\\
0.135714483753824	0.856867527364047\\
5.22899666167123e-17	0.853959960588124\\
-0.131496350792703	0.830235283991667\\
-0.255686250369537	0.786921363444393\\
-0.369756251949576	0.725687504552447\\
-0.47122811850853	0.648589862732789\\
-0.558009078759514	0.558009078759514\\
-0.628431083783263	0.456581908289041\\
-0.681278445058817	0.347128705949459\\
-0.715803538350409	0.232578668243255\\
-0.731730559897045	0.115894735197653\\
-0.729247614287671	8.9307075662324e-17\\
};
\addplot [color=mycolor1,dotted,line width=1.0pt,forget plot]
  table[row sep=crcr]{1	0\\
0.956521682769877	0.151498151383791\\
0.891982039736211	0.289822533396298\\
0.809292583610082	0.412355167436434\\
0.711634858347723	0.517032989000692\\
0.602364630186427	0.602364630186426\\
0.484917701774751	0.667431957639199\\
0.362719588349683	0.711877274647591\\
0.239101059900595	0.735877395777214\\
0.117221283062812	0.740106053489981\\
4.44354699422903e-17	0.725686295399261\\
-0.109940132237539	0.694134676438339\\
-0.210320240583588	0.647299141978847\\
-0.299240493845688	0.587292536894097\\
-0.375203754387119	0.516423664031337\\
-0.437127756959533	0.437127756959533\\
-0.484346224770267	0.351898130574443\\
-0.516599582217932	0.263220634338226\\
-0.534016141209622	0.173512362385299\\
-0.537084820159711	0.0850658786478001\\
-0.526620599330303	6.44924231334916e-17\\
};
\addplot [color=mycolor1,dotted,line width=1.0pt,forget plot]
  table[row sep=crcr]{1	0\\
0.940082788364644	0.148894486293864\\
0.86158608093073	0.279946287694513\\
0.768279786681378	0.391458103646313\\
0.663960650859636	0.482395649771645\\
0.552351927561387	0.552351927561387\\
0.437014383712816	0.601498696728173\\
0.321269860940431	0.630527604183869\\
0.208138182971344	0.640583459188477\\
0.100287778328637	0.633192112325829\\
3.73630739569174e-17	0.610185303761559\\
-0.0908532273476425	0.573624701779294\\
-0.170819110593797	0.525727164509449\\
-0.238861981883349	0.468793035010043\\
-0.294350736089166	0.40513903143051\\
-0.337037028702328	0.337037028702328\\
-0.367025664975262	0.266659754473976\\
-0.384738677688982	0.196034147687371\\
-0.390874612629551	0.127002860400749\\
-0.386364521398959	0.0611941284830315\\
-0.372326104926586	4.55967972637346e-17\\
};
\addplot [color=mycolor1,dotted,line width=1.0pt,forget plot]
  table[row sep=crcr]{1	0\\
0.922246029428501	0.146069421212559\\
0.829201462983264	0.269423887468404\\
0.725373165529273	0.369596088217499\\
0.614985835074999	0.446813363302878\\
0.501902475185001	0.501902475185001\\
0.38956496084428	0.536190168955128\\
0.280953750703967	0.551402782692681\\
0.178565395716864	0.549567778706978\\
0.0844061798404073	0.532919645815306\\
3.08495992433718e-17	0.50381218919366\\
-0.0735915548753502	0.464638791061534\\
-0.135739038566523	0.417761804348184\\
-0.186207014322272	0.365451842507638\\
-0.225110018726605	0.309837359892227\\
-0.252864584784672	0.252864584784672\\
-0.270139142845401	0.196267575756099\\
-0.277803443075876	0.141547924204336\\
-0.27687894570066	0.0899634229303457\\
-0.268491413422519	0.0425248622468694\\
-0.253826721980109	3.10848082611005e-17\\
};
\addplot [color=mycolor1,dotted,line width=1.0pt,forget plot]
  table[row sep=crcr]{1	0\\
0.902056570675584	0.142871725087523\\
0.793293726869509	0.257756756757911\\
0.678769666307395	0.345850419328459\\
0.562876391436159	0.408953636388562\\
0.449318435861499	0.449318435861499\\
0.341115747349381	0.469505547439701\\
0.24062663375655	0.472256359314945\\
0.149586755320764	0.460380694230177\\
0.069160331541479	0.436661148025441\\
2.47240337905553e-17	0.403774113610049\\
-0.057687904157153	0.364227092250654\\
-0.104075505986926	0.320311471399148\\
-0.139645446506012	0.274069620358657\\
-0.165124892040398	0.227274916024158\\
-0.18142315316863	0.18142315316863\\
-0.189574186252466	0.137733708524426\\
-0.190684892879101	0.0971588057560207\\
-0.185889806735969	0.0603992595374446\\
-0.176312467991919	0.0279251515651378\\
-0.16303353482158	1.99658496572927e-17\\
};
\addplot [color=mycolor1,dotted,line width=1.0pt,forget plot]
  table[row sep=crcr]{1	0\\
0.877921760602431	0.139049146701748\\
0.751411952540787	0.244148543365328\\
0.625732257688895	0.318826509862098\\
0.505011411193747	0.36691226736026\\
0.392341669548685	0.392341669548685\\
0.289890514921595	0.399000063654162\\
0.199020572856858	0.390599867100522\\
0.120411913496453	0.370589763847805\\
0.0541820486548744	0.34209199176291\\
1.88512313530119e-17	0.307863971328499\\
-0.042808222513439	0.270280479734772\\
-0.075164540154518	0.231332667812908\\
-0.0981551954909503	0.192640417840451\\
-0.112959067758674	0.155474818616317\\
-0.120787864504912	0.120787864504912\\
-0.122837744156894	0.0892468451742257\\
-0.120251626285951	0.0612712639357851\\
-0.114091236431876	0.0370704898842818\\
-0.105317799143806	0.0166807006736036\\
-0.0947802248421549	1.16072298975411e-17\\
};
\addplot [color=mycolor1,dotted,line width=1.0pt,forget plot]
  table[row sep=crcr]{1	0\\
0.84674396664984	0.134111069256013\\
0.698989566160914	0.227115477506975\\
0.561406498364893	0.286050898428465\\
0.437004973478602	0.317502698182059\\
0.327450698344114	0.327450698344114\\
0.233352139914598	0.321181666481713\\
0.154515499860225	0.303253743289057\\
0.0901653834921457	0.27750051639687\\
0.0391311034995635	0.247064063991275\\
1.31311479217367e-17	0.214447919692096\\
-0.0287598398096346	0.181582482159892\\
-0.0487044416812479	0.149896858349689\\
-0.0613430200940395	0.120392455675976\\
-0.06808779312177	0.0937148074575858\\
-0.0702211210616195	0.0702211210616195\\
-0.068876740220244	0.0500418809603846\\
-0.0650321343871793	0.0331355275052096\\
-0.0595094084547913	0.0193357789181307\\
-0.0529823745536039	0.00839158374073751\\
-0.0459879102602678	5.63189470997126e-18\\
};
\addplot [color=mycolor1,dotted,line width=1.0pt,forget plot]
  table[row sep=crcr]{1	0\\
0.801053465278425	0.126874404768154\\
0.625589539649299	0.203266363189334\\
0.475341369738971	0.242198525079547\\
0.350045057714404	0.254322621147605\\
0.24813777530853	0.24813777530853\\
0.167289292234614	0.230253957320141\\
0.104794333468008	0.205670459781722\\
0.0578515468028918	0.178048753195384\\
0.0237523524567972	0.149966451301199\\
7.54043881219821e-18	0.123144711070133\\
-0.0156239119173694	0.0986454975334405\\
-0.0250311775652273	0.0770380431086105\\
-0.0298254673925332	0.0585357756361869\\
-0.0313184856998208	0.0431061974937683\\
-0.0305568546459545	0.0305568546459545\\
-0.0283545570469435	0.0206007915573586\\
-0.0253272293454815	0.012904867916705\\
-0.021925762268643	0.00712411201600239\\
-0.0184675753156993	0.00292497658052826\\
-0.0151646198645466	1.85713031774033e-18\\
};
\addplot [color=mycolor1,dotted,line width=1.0pt,forget plot]
  table[row sep=crcr]{1	0\\
0.714110955679367	0.113104064044896\\
0.497161927717827	0.161537702532642\\
0.336758208677056	0.171586877647116\\
0.221075553032581	0.160620791180475\\
0.139705610200823	0.139705610200823\\
0.0839640345306934	0.115566579097864\\
0.0468885871776745	0.0920240337831981\\
0.0230753615892199	0.0710186604775083\\
0.00844586939409394	0.0533251206797082\\
2.39022368106624e-18	0.0390353150431684\\
-0.00441505277265522	0.0278755461307222\\
-0.00630567510971132	0.0194068724759343\\
-0.00669794782622876	0.0131454627690819\\
-0.0062698831862128	0.00862975386096949\\
-0.0054534525074872	0.0054534525074872\\
-0.00451117782354635	0.00327756254020109\\
-0.00359218715027031	0.00183031077240959\\
-0.00277223578568335	0.000900754009370227\\
-0.00208156288544739	0.000329687172611911\\
-0.00152375582051941	1.86606268828125e-19\\
};
\addplot [color=mycolor1,dotted,line width=1.0pt,forget plot]
  table[row sep=crcr]{1	-0\\
0.987688340595138	-0.156434465040231\\
0.951056516295154	-0.309016994374947\\
0.891006524188368	-0.453990499739547\\
0.809016994374947	-0.587785252292473\\
0.707106781186548	-0.707106781186547\\
0.587785252292473	-0.809016994374947\\
0.453990499739547	-0.891006524188368\\
0.309016994374947	-0.951056516295154\\
0.156434465040231	-0.987688340595138\\
6.12323399573677e-17	-1\\
-0.156434465040231	-0.987688340595138\\
-0.309016994374947	-0.951056516295154\\
-0.453990499739547	-0.891006524188368\\
-0.587785252292473	-0.809016994374947\\
-0.707106781186547	-0.707106781186548\\
-0.809016994374947	-0.587785252292473\\
-0.891006524188368	-0.453990499739547\\
-0.951056516295154	-0.309016994374948\\
-0.987688340595138	-0.156434465040231\\
-1	-1.22464679914735e-16\\
};
\addplot [color=mycolor1,dotted,line width=1.0pt,forget plot]
  table[row sep=crcr]{1	-0\\
0.972218045703082	-0.153984211042097\\
0.921496791140463	-0.299412457456957\\
0.849791018366557	-0.432990150609548\\
0.759508516401756	-0.551815237548133\\
0.653437051516106	-0.653437051516106\\
0.534664304371591	-0.735902282058355\\
0.406492952796512	-0.797787339572243\\
0.272353130096494	-0.83821674479613\\
0.135714483753824	-0.856867527364047\\
5.22899666167123e-17	-0.853959960588124\\
-0.131496350792703	-0.830235283991667\\
-0.255686250369537	-0.786921363444393\\
-0.369756251949576	-0.725687504552447\\
-0.47122811850853	-0.648589862732789\\
-0.558009078759514	-0.558009078759514\\
-0.628431083783263	-0.456581908289041\\
-0.681278445058817	-0.347128705949459\\
-0.715803538350409	-0.232578668243255\\
-0.731730559897045	-0.115894735197653\\
-0.729247614287671	-8.9307075662324e-17\\
};
\addplot [color=mycolor1,dotted,line width=1.0pt,forget plot]
  table[row sep=crcr]{1	-0\\
0.956521682769877	-0.151498151383791\\
0.891982039736211	-0.289822533396298\\
0.809292583610082	-0.412355167436434\\
0.711634858347723	-0.517032989000692\\
0.602364630186427	-0.602364630186426\\
0.484917701774751	-0.667431957639199\\
0.362719588349683	-0.711877274647591\\
0.239101059900595	-0.735877395777214\\
0.117221283062812	-0.740106053489981\\
4.44354699422903e-17	-0.725686295399261\\
-0.109940132237539	-0.694134676438339\\
-0.210320240583588	-0.647299141978847\\
-0.299240493845688	-0.587292536894097\\
-0.375203754387119	-0.516423664031337\\
-0.437127756959533	-0.437127756959533\\
-0.484346224770267	-0.351898130574443\\
-0.516599582217932	-0.263220634338226\\
-0.534016141209622	-0.173512362385299\\
-0.537084820159711	-0.0850658786478001\\
-0.526620599330303	-6.44924231334916e-17\\
};
\addplot [color=mycolor1,dotted,line width=1.0pt,forget plot]
  table[row sep=crcr]{1	-0\\
0.940082788364644	-0.148894486293864\\
0.86158608093073	-0.279946287694513\\
0.768279786681378	-0.391458103646313\\
0.663960650859636	-0.482395649771645\\
0.552351927561387	-0.552351927561387\\
0.437014383712816	-0.601498696728173\\
0.321269860940431	-0.630527604183869\\
0.208138182971344	-0.640583459188477\\
0.100287778328637	-0.633192112325829\\
3.73630739569174e-17	-0.610185303761559\\
-0.0908532273476425	-0.573624701779294\\
-0.170819110593797	-0.525727164509449\\
-0.238861981883349	-0.468793035010043\\
-0.294350736089166	-0.40513903143051\\
-0.337037028702328	-0.337037028702328\\
-0.367025664975262	-0.266659754473976\\
-0.384738677688982	-0.196034147687371\\
-0.390874612629551	-0.127002860400749\\
-0.386364521398959	-0.0611941284830315\\
-0.372326104926586	-4.55967972637346e-17\\
};
\addplot [color=mycolor1,dotted,line width=1.0pt,forget plot]
  table[row sep=crcr]{1	-0\\
0.922246029428501	-0.146069421212559\\
0.829201462983264	-0.269423887468404\\
0.725373165529273	-0.369596088217499\\
0.614985835074999	-0.446813363302878\\
0.501902475185001	-0.501902475185001\\
0.38956496084428	-0.536190168955128\\
0.280953750703967	-0.551402782692681\\
0.178565395716864	-0.549567778706978\\
0.0844061798404073	-0.532919645815306\\
3.08495992433718e-17	-0.50381218919366\\
-0.0735915548753502	-0.464638791061534\\
-0.135739038566523	-0.417761804348184\\
-0.186207014322272	-0.365451842507638\\
-0.225110018726605	-0.309837359892227\\
-0.252864584784672	-0.252864584784672\\
-0.270139142845401	-0.196267575756099\\
-0.277803443075876	-0.141547924204336\\
-0.27687894570066	-0.0899634229303457\\
-0.268491413422519	-0.0425248622468694\\
-0.253826721980109	-3.10848082611005e-17\\
};
\addplot [color=mycolor1,dotted,line width=1.0pt,forget plot]
  table[row sep=crcr]{1	-0\\
0.902056570675584	-0.142871725087523\\
0.793293726869509	-0.257756756757911\\
0.678769666307395	-0.345850419328459\\
0.562876391436159	-0.408953636388562\\
0.449318435861499	-0.449318435861499\\
0.341115747349381	-0.469505547439701\\
0.24062663375655	-0.472256359314945\\
0.149586755320764	-0.460380694230177\\
0.069160331541479	-0.436661148025441\\
2.47240337905553e-17	-0.403774113610049\\
-0.057687904157153	-0.364227092250654\\
-0.104075505986926	-0.320311471399148\\
-0.139645446506012	-0.274069620358657\\
-0.165124892040398	-0.227274916024158\\
-0.18142315316863	-0.18142315316863\\
-0.189574186252466	-0.137733708524426\\
-0.190684892879101	-0.0971588057560207\\
-0.185889806735969	-0.0603992595374446\\
-0.176312467991919	-0.0279251515651378\\
-0.16303353482158	-1.99658496572927e-17\\
};
\addplot [color=mycolor1,dotted,line width=1.0pt,forget plot]
  table[row sep=crcr]{1	-0\\
0.877921760602431	-0.139049146701748\\
0.751411952540787	-0.244148543365328\\
0.625732257688895	-0.318826509862098\\
0.505011411193747	-0.36691226736026\\
0.392341669548685	-0.392341669548685\\
0.289890514921595	-0.399000063654162\\
0.199020572856858	-0.390599867100522\\
0.120411913496453	-0.370589763847805\\
0.0541820486548744	-0.34209199176291\\
1.88512313530119e-17	-0.307863971328499\\
-0.042808222513439	-0.270280479734772\\
-0.075164540154518	-0.231332667812908\\
-0.0981551954909503	-0.192640417840451\\
-0.112959067758674	-0.155474818616317\\
-0.120787864504912	-0.120787864504912\\
-0.122837744156894	-0.0892468451742257\\
-0.120251626285951	-0.0612712639357851\\
-0.114091236431876	-0.0370704898842818\\
-0.105317799143806	-0.0166807006736036\\
-0.0947802248421549	-1.16072298975411e-17\\
};
\addplot [color=mycolor1,dotted,line width=1.0pt,forget plot]
  table[row sep=crcr]{1	-0\\
0.84674396664984	-0.134111069256013\\
0.698989566160914	-0.227115477506975\\
0.561406498364893	-0.286050898428465\\
0.437004973478602	-0.317502698182059\\
0.327450698344114	-0.327450698344114\\
0.233352139914598	-0.321181666481713\\
0.154515499860225	-0.303253743289057\\
0.0901653834921457	-0.27750051639687\\
0.0391311034995635	-0.247064063991275\\
1.31311479217367e-17	-0.214447919692096\\
-0.0287598398096346	-0.181582482159892\\
-0.0487044416812479	-0.149896858349689\\
-0.0613430200940395	-0.120392455675976\\
-0.06808779312177	-0.0937148074575858\\
-0.0702211210616195	-0.0702211210616195\\
-0.068876740220244	-0.0500418809603846\\
-0.0650321343871793	-0.0331355275052096\\
-0.0595094084547913	-0.0193357789181307\\
-0.0529823745536039	-0.00839158374073751\\
-0.0459879102602678	-5.63189470997126e-18\\
};
\addplot [color=mycolor1,dotted,line width=1.0pt,forget plot]
  table[row sep=crcr]{1	-0\\
0.801053465278425	-0.126874404768154\\
0.625589539649299	-0.203266363189334\\
0.475341369738971	-0.242198525079547\\
0.350045057714404	-0.254322621147605\\
0.24813777530853	-0.24813777530853\\
0.167289292234614	-0.230253957320141\\
0.104794333468008	-0.205670459781722\\
0.0578515468028918	-0.178048753195384\\
0.0237523524567972	-0.149966451301199\\
7.54043881219821e-18	-0.123144711070133\\
-0.0156239119173694	-0.0986454975334405\\
-0.0250311775652273	-0.0770380431086105\\
-0.0298254673925332	-0.0585357756361869\\
-0.0313184856998208	-0.0431061974937683\\
-0.0305568546459545	-0.0305568546459545\\
-0.0283545570469435	-0.0206007915573586\\
-0.0253272293454815	-0.012904867916705\\
-0.021925762268643	-0.00712411201600239\\
-0.0184675753156993	-0.00292497658052826\\
-0.0151646198645466	-1.85713031774033e-18\\
};
\addplot [color=mycolor1,dotted,line width=1.0pt,forget plot]
  table[row sep=crcr]{1	-0\\
0.714110955679367	-0.113104064044896\\
0.497161927717827	-0.161537702532642\\
0.336758208677056	-0.171586877647116\\
0.221075553032581	-0.160620791180475\\
0.139705610200823	-0.139705610200823\\
0.0839640345306934	-0.115566579097864\\
0.0468885871776745	-0.0920240337831981\\
0.0230753615892199	-0.0710186604775083\\
0.00844586939409394	-0.0533251206797082\\
2.39022368106624e-18	-0.0390353150431684\\
-0.00441505277265522	-0.0278755461307222\\
-0.00630567510971132	-0.0194068724759343\\
-0.00669794782622876	-0.0131454627690819\\
-0.0062698831862128	-0.00862975386096949\\
-0.0054534525074872	-0.0054534525074872\\
-0.00451117782354635	-0.00327756254020109\\
-0.00359218715027031	-0.00183031077240959\\
-0.00277223578568335	-0.000900754009370227\\
-0.00208156288544739	-0.000329687172611911\\
-0.00152375582051941	-1.86606268828125e-19\\
};
\addplot [color=mycolor1,dotted,line width=1.0pt,forget plot]
  table[row sep=crcr]{1	0\\
1	0\\
};
\addplot [color=mycolor1,dotted,line width=1.0pt,forget plot]
  table[row sep=crcr]{0.951056516295154	0.309016994374947\\
0.906577591518048	0.290693092532446\\
0.867277205719189	0.267124444629623\\
0.833330533437629	0.239553777474139\\
0.804679893747523	0.20903820245934\\
0.781122098933164	0.176433510213537\\
0.762382187756933	0.142402376999381\\
0.748171074803624	0.107437628337747\\
0.738227708485823	0.0718935522249012\\
0.732347931289196	0.0360204899377399\\
0.730402691048646	2.81010850277162e-17\\
};
\addplot [color=mycolor1,dotted,line width=1.0pt,forget plot]
  table[row sep=crcr]{0.809016994374947	0.587785252292473\\
0.737380455396588	0.527071687397996\\
0.6808142826414	0.463341883835339\\
0.637053765657314	0.399254954339046\\
0.603812761314093	0.336417677088311\\
0.579022949915482	0.275632227640288\\
0.560948163233974	0.217130071437151\\
0.54821711318997	0.160763451735608\\
0.539811466724715	0.106147624627789\\
0.535036016768209	0.0527590625798542\\
0.533488091091103	4.10502162512614e-17\\
};
\addplot [color=mycolor1,dotted,line width=1.0pt,forget plot]
  table[row sep=crcr]{0.587785252292473	0.809016994374947\\
0.515276498489902	0.692182785870847\\
0.46668476526979	0.583707991451876\\
0.435233321876477	0.485319980094308\\
0.415551842123536	0.396928474901319\\
0.403656860517901	0.317461475735791\\
0.396737049613855	0.245416450708029\\
0.392888142823216	0.179177710929467\\
0.39087245229983	0.11717008156476\\
0.389932112762627	0.0579102497954412\\
0.389661137375347	4.49747826270753e-17\\
};
\addplot [color=mycolor1,dotted,line width=1.0pt,forget plot]
  table[row sep=crcr]{0.309016994374947	0.951056516295154\\
0.265925372344309	0.777304721760365\\
0.248822386132444	0.630899544522142\\
0.24643298177389	0.508693744238057\\
0.251412997268255	0.406266573115132\\
0.25929445161488	0.31919477108011\\
0.267517373913267	0.243597429511063\\
0.274697115780397	0.1762665508339\\
0.280129101393366	0.114599409879343\\
0.283480020554887	0.0564559973822998\\
0.284609543336029	4.37996030135249e-17\\
};
\addplot [color=mycolor1,dotted,line width=1.0pt,forget plot]
  table[row sep=crcr]{6.12323399573677e-17	1\\
0.0151248701748503	0.781989711398728\\
0.0472692933177679	0.613631335769719\\
0.0835006201485716	0.482943980920436\\
0.117381749765263	0.379469463911326\\
0.146223972383669	0.295078319831894\\
0.169261627793672	0.223809451176491\\
0.186582776182006	0.161390341419992\\
0.198560105942714	0.104739935929793\\
0.205572433929556	0.0515565221197431\\
0.207879576350762	3.99891848849262e-17\\
};
\addplot [color=mycolor1,dotted,line width=1.0pt,forget plot]
  table[row sep=crcr]{-0.309016994374947	0.951056516295154\\
-0.213607139159912	0.713331044437031\\
-0.122920349149865	0.544815253953639\\
-0.0461074386071066	0.422454854218942\\
0.0151311193047769	0.329888717873058\\
0.0621570724668225	0.256291005261778\\
0.0971710522581798	0.194731597161215\\
0.122256440677152	0.140793596164787\\
0.13905244595299	0.0915971142347777\\
0.148693455532156	0.0451621321066958\\
0.151835801980649	3.50498499033503e-17\\
};
\addplot [color=mycolor1,dotted,line width=1.0pt,forget plot]
  table[row sep=crcr]{-0.587785252292473	0.809016994374948\\
-0.401011853057454	0.584595820351374\\
-0.252139489074835	0.439670821081765\\
-0.139623392550337	0.340999317931598\\
-0.0567836371213516	0.268617800427267\\
0.0033339412143336	0.21116115844769\\
0.0463512370945915	0.162297289886265\\
0.0763422025660035	0.118472638203443\\
0.0960671266193367	0.0776165020305757\\
0.1072685824301	0.0384304051397518\\
0.110901278364195	2.98672554719679e-17\\
};
\addplot [color=mycolor1,dotted,line width=1.0pt,forget plot]
  table[row sep=crcr]{-0.809016994374948	0.587785252292473\\
-0.533486326814499	0.413410095118228\\
-0.336121655437603	0.313963860155743\\
-0.198040110920963	0.250717832404618\\
-0.101844033235305	0.204281393673556\\
-0.0346516892466227	0.165530866251108\\
0.0122258376810533	0.130333089269644\\
0.0443886084751889	0.0968398262452624\\
0.065339288702761	0.0642052594194206\\
0.0771736424133687	0.032008294596762\\
0.0810025921579431	2.49315700239574e-17\\
};
\addplot [color=mycolor1,dotted,line width=1.0pt,forget plot]
  table[row sep=crcr]{-0.951056516295154	0.309016994374948\\
-0.60382220830535	0.219707538176049\\
-0.375378071887508	0.182507388795924\\
-0.225093275108467	0.161489568357552\\
-0.124654461172013	0.143091836517116\\
-0.0562723920172722	0.12318609851568\\
-0.00923898083522721	0.101104614081101\\
0.0228060316514889	0.0772217637055013\\
0.0436193292019494	0.0520955750986265\\
0.0553650029180358	0.0262210407420432\\
0.0591645112940776	2.0486346567262e-17\\
};
\addplot [color=mycolor1,dotted,line width=1.0pt,forget plot]
  table[row sep=crcr]{-1	1.22464679914735e-16\\
-0.61127914703566	0.0236549857259488\\
-0.374309030147768	0.0580118391989452\\
-0.226262535142082	0.0806522438077526\\
-0.130218598863194	0.0890855793127953\\
-0.0656897647351535	0.0862950481802363\\
-0.0214411717925588	0.0757647040434823\\
0.00876629006412298	0.0602253159022079\\
0.0284556614934045	0.0415943455493057\\
0.039601950618638	0.0211971994741971\\
0.0432139182637723	1.66258696249815e-17\\
};
\addplot [color=mycolor1,dotted,line width=1.0pt,forget plot]
  table[row sep=crcr]{1	-0\\
1	-0\\
};
\addplot [color=mycolor1,dotted,line width=1.0pt,forget plot]
  table[row sep=crcr]{0.951056516295154	-0.309016994374947\\
0.906577591518048	-0.290693092532446\\
0.867277205719189	-0.267124444629623\\
0.833330533437629	-0.239553777474139\\
0.804679893747523	-0.20903820245934\\
0.781122098933164	-0.176433510213537\\
0.762382187756933	-0.142402376999381\\
0.748171074803624	-0.107437628337747\\
0.738227708485823	-0.0718935522249012\\
0.732347931289196	-0.0360204899377399\\
0.730402691048646	-2.81010850277162e-17\\
};
\addplot [color=mycolor1,dotted,line width=1.0pt,forget plot]
  table[row sep=crcr]{0.809016994374947	-0.587785252292473\\
0.737380455396588	-0.527071687397996\\
0.6808142826414	-0.463341883835339\\
0.637053765657314	-0.399254954339046\\
0.603812761314093	-0.336417677088311\\
0.579022949915482	-0.275632227640288\\
0.560948163233974	-0.217130071437151\\
0.54821711318997	-0.160763451735608\\
0.539811466724715	-0.106147624627789\\
0.535036016768209	-0.0527590625798542\\
0.533488091091103	-4.10502162512614e-17\\
};
\addplot [color=mycolor1,dotted,line width=1.0pt,forget plot]
  table[row sep=crcr]{0.587785252292473	-0.809016994374947\\
0.515276498489902	-0.692182785870847\\
0.46668476526979	-0.583707991451876\\
0.435233321876477	-0.485319980094308\\
0.415551842123536	-0.396928474901319\\
0.403656860517901	-0.317461475735791\\
0.396737049613855	-0.245416450708029\\
0.392888142823216	-0.179177710929467\\
0.39087245229983	-0.11717008156476\\
0.389932112762627	-0.0579102497954412\\
0.389661137375347	-4.49747826270753e-17\\
};
\addplot [color=mycolor1,dotted,line width=1.0pt,forget plot]
  table[row sep=crcr]{0.309016994374947	-0.951056516295154\\
0.265925372344309	-0.777304721760365\\
0.248822386132444	-0.630899544522142\\
0.24643298177389	-0.508693744238057\\
0.251412997268255	-0.406266573115132\\
0.25929445161488	-0.31919477108011\\
0.267517373913267	-0.243597429511063\\
0.274697115780397	-0.1762665508339\\
0.280129101393366	-0.114599409879343\\
0.283480020554887	-0.0564559973822998\\
0.284609543336029	-4.37996030135249e-17\\
};
\addplot [color=mycolor1,dotted,line width=1.0pt,forget plot]
  table[row sep=crcr]{6.12323399573677e-17	-1\\
0.0151248701748503	-0.781989711398728\\
0.0472692933177679	-0.613631335769719\\
0.0835006201485716	-0.482943980920436\\
0.117381749765263	-0.379469463911326\\
0.146223972383669	-0.295078319831894\\
0.169261627793672	-0.223809451176491\\
0.186582776182006	-0.161390341419992\\
0.198560105942714	-0.104739935929793\\
0.205572433929556	-0.0515565221197431\\
0.207879576350762	-3.99891848849262e-17\\
};
\addplot [color=mycolor1,dotted,line width=1.0pt,forget plot]
  table[row sep=crcr]{-0.309016994374947	-0.951056516295154\\
-0.213607139159912	-0.713331044437031\\
-0.122920349149865	-0.544815253953639\\
-0.0461074386071066	-0.422454854218942\\
0.0151311193047769	-0.329888717873058\\
0.0621570724668225	-0.256291005261778\\
0.0971710522581798	-0.194731597161215\\
0.122256440677152	-0.140793596164787\\
0.13905244595299	-0.0915971142347777\\
0.148693455532156	-0.0451621321066958\\
0.151835801980649	-3.50498499033503e-17\\
};
\addplot [color=mycolor1,dotted,line width=1.0pt,forget plot]
  table[row sep=crcr]{-0.587785252292473	-0.809016994374948\\
-0.401011853057454	-0.584595820351374\\
-0.252139489074835	-0.439670821081765\\
-0.139623392550337	-0.340999317931598\\
-0.0567836371213516	-0.268617800427267\\
0.0033339412143336	-0.21116115844769\\
0.0463512370945915	-0.162297289886265\\
0.0763422025660035	-0.118472638203443\\
0.0960671266193367	-0.0776165020305757\\
0.1072685824301	-0.0384304051397518\\
0.110901278364195	-2.98672554719679e-17\\
};
\addplot [color=mycolor1,dotted,line width=1.0pt,forget plot]
  table[row sep=crcr]{-0.809016994374948	-0.587785252292473\\
-0.533486326814499	-0.413410095118228\\
-0.336121655437603	-0.313963860155743\\
-0.198040110920963	-0.250717832404618\\
-0.101844033235305	-0.204281393673556\\
-0.0346516892466227	-0.165530866251108\\
0.0122258376810533	-0.130333089269644\\
0.0443886084751889	-0.0968398262452624\\
0.065339288702761	-0.0642052594194206\\
0.0771736424133687	-0.032008294596762\\
0.0810025921579431	-2.49315700239574e-17\\
};
\addplot [color=mycolor1,dotted,line width=1.0pt,forget plot]
  table[row sep=crcr]{-0.951056516295154	-0.309016994374948\\
-0.60382220830535	-0.219707538176049\\
-0.375378071887508	-0.182507388795924\\
-0.225093275108467	-0.161489568357552\\
-0.124654461172013	-0.143091836517116\\
-0.0562723920172722	-0.12318609851568\\
-0.00923898083522721	-0.101104614081101\\
0.0228060316514889	-0.0772217637055013\\
0.0436193292019494	-0.0520955750986265\\
0.0553650029180358	-0.0262210407420432\\
0.0591645112940776	-2.0486346567262e-17\\
};
\addplot [color=mycolor1,dotted,line width=1.0pt,forget plot]
  table[row sep=crcr]{-1	-1.22464679914735e-16\\
-0.61127914703566	-0.0236549857259488\\
-0.374309030147768	-0.0580118391989452\\
-0.226262535142082	-0.0806522438077526\\
-0.130218598863194	-0.0890855793127953\\
-0.0656897647351535	-0.0862950481802363\\
-0.0214411717925588	-0.0757647040434823\\
0.00876629006412298	-0.0602253159022079\\
0.0284556614934045	-0.0415943455493057\\
0.039601950618638	-0.0211971994741971\\
0.0432139182637723	-1.66258696249815e-17\\
};
\addplot [color=mycolor1,line width=1.5pt,mark size=4.0pt,only marks,mark=x,mark options={solid},forget plot]
  table[row sep=crcr]{0	0\\
0.91444306659383	0.404714563561125\\
-0.124253880684544	0\\
0.89755799120607	0.418082106216459\\
-0.168612823306847	0\\
0.880031509804686	0.432694736532765\\
-0.20035214433653	0\\
0.862049033868094	0.448898382097043\\
-0.225714039094344	0\\
0.843968536244412	0.467035231637572\\
-0.247083156478973	0\\
0.826353726809843	0.487334319047914\\
-0.265667627304716	0\\
0.80991520639231	0.509761287368884\\
-0.28217711202824	0\\
0.795322964447454	0.533928890391685\\
-0.297069157418281	0\\
0.78298488576272	0.559181443716326\\
-0.310658437059747	0\\
0.772959683909908	0.584813903534905\\
-0.323171911393944	0\\
0.765041430382214	0.610258212917158\\
-0.334779380319558	0\\
0.758904835963185	0.635145159493358\\
-0.345611626818817	0\\
0.754213373942793	0.659276394587029\\
};
\addplot [color=black!50!mycolor1,line width=1.5pt,mark size=4.0pt,only marks,mark=x,mark options={solid},forget plot]
  table[row sep=crcr]{0	0\\
0.91444306659383	-0.404714563561125\\
0.158024031460067	0\\
0.89755799120607	-0.418082106216459\\
0.237435936885133	0\\
0.880031509804686	-0.432694736532765\\
0.305140209788	0\\
0.862049033868094	-0.448898382097043\\
0.36666309979318	0\\
0.843968536244412	-0.467035231637572\\
0.423261836046949	0\\
0.826353726809843	-0.487334319047914\\
0.474723347707758	0\\
0.80991520639231	-0.509761287368884\\
0.520417316320995	0\\
0.795322964447454	-0.533928890391685\\
0.559985519080503	0\\
0.78298488576272	-0.559181443716326\\
0.593625202427589	0\\
0.772959683909908	-0.584813903534905\\
0.621975183817178	0\\
0.765041430382214	-0.610258212917158\\
0.645855841580849	0\\
0.758904835963185	-0.635145159493358\\
0.666071012120892	0\\
};
\addplot [color=black,line width=1.5pt,mark size=4.0pt,only marks,mark=o,mark options={solid},forget plot]
  table[row sep=crcr]{-1	0\\
0.9	0\\
};
\end{axis}
\end{tikzpicture}%%}
    \end{figure}
    \end{small}
    \vfill
  }

  \frame{
    \frametitle{\insertsection}
    \vfill
    Variável controlada na malha interna: corrente do capacitor $i_C$
    \begin{equation}
      C_i(z) = K_P
    \end{equation}
    \vfill
    Equação característica da função de transferência $\frac{i_C}{U_c}$:
    \begin{equation}
      z^3 - 2 \cos \left( \omega_n T_s \right) z^2 + \left( 1 + K_P K_{id} \right) z - K_P K_{id} = 0
      \label{eq:eq_caracteristica}
    \end{equation}
    Onde:
    \begin{equation}
      K_{id} = \frac{\sin(\omega_n T_s)}{\omega_n L_1}
    \end{equation}
    \vfill
  }

  \frame{
    \frametitle{\insertsection}
    \vfill
    Utilizando a transformação bilinear:
    \begin{equation}
      z = \frac{w + 1}{w - 1}
    \end{equation}
    \vfill
    Critério de estabilidade de Routh-Hurwitz como em um sistema em tempo contínuo \cite{ref:OGATA}
    \begin{equation}
      \overline{K_P} = \frac{2 \cos \left( \omega_n T_s \right) - 1}
        {\sen \left( \omega_n T_s \right)} \omega_n L_1
    \end{equation}
    \vfill
  }

  \frame{
    \frametitle{\insertsection}
    \vfill
    \begin{small}
    \begin{figure}[htb]
      \centering
      \raisebox{-0.5\height}{
        \def\svgwidth{0.5\textwidth}
        % This file was created by matlab2tikz v0.4.7 running on MATLAB 7.14.
% Copyright (c) 2008--2014, Nico Schlömer <nico.schloemer@gmail.com>
% All rights reserved.
% Minimal pgfplots version: 1.3
% 
% The latest updates can be retrieved from
%   http://www.mathworks.com/matlabcentral/fileexchange/22022-matlab2tikz
% where you can also make suggestions and rate matlab2tikz.
% 
%
% defining custom colors
\definecolor{mycolor1}{rgb}{0.66667,0.66667,0.66667}%
%
\begin{tikzpicture}

\begin{axis}[%
width=0.8\textwidth,
height=0.461611624834875\textwidth,
scale only axis,
xmin=-1.9,
xmax=1.9,
xtick={-1,  0,  1},
xlabel={Eixo Real},
ymin=-1.1,
ymax=1.1,
ytick={-1,  0,  1},
ylabel={Eixo Imaginário},
scaled y ticks = false,
y tick label style={/pgf/number format/.cd, fixed, fixed zerofill, precision=0},
scaled x ticks = false,
x tick label style={/pgf/number format/.cd, fixed, fixed zerofill, precision=0}
]
\addplot [color=mycolor1,dotted,line width=1.0pt,forget plot]
  table[row sep=crcr]{1	0\\
0.987688340595138	0.156434465040231\\
0.951056516295154	0.309016994374947\\
0.891006524188368	0.453990499739547\\
0.809016994374947	0.587785252292473\\
0.707106781186548	0.707106781186547\\
0.587785252292473	0.809016994374947\\
0.453990499739547	0.891006524188368\\
0.309016994374947	0.951056516295154\\
0.156434465040231	0.987688340595138\\
6.12323399573677e-17	1\\
-0.156434465040231	0.987688340595138\\
-0.309016994374947	0.951056516295154\\
-0.453990499739547	0.891006524188368\\
-0.587785252292473	0.809016994374947\\
-0.707106781186547	0.707106781186548\\
-0.809016994374947	0.587785252292473\\
-0.891006524188368	0.453990499739547\\
-0.951056516295154	0.309016994374948\\
-0.987688340595138	0.156434465040231\\
-1	1.22464679914735e-16\\
};
\addplot [color=mycolor1,dotted,line width=1.0pt,forget plot]
  table[row sep=crcr]{1	0\\
0.972218045703082	0.153984211042097\\
0.921496791140463	0.299412457456957\\
0.849791018366557	0.432990150609548\\
0.759508516401756	0.551815237548133\\
0.653437051516106	0.653437051516106\\
0.534664304371591	0.735902282058355\\
0.406492952796512	0.797787339572243\\
0.272353130096494	0.83821674479613\\
0.135714483753824	0.856867527364047\\
5.22899666167123e-17	0.853959960588124\\
-0.131496350792703	0.830235283991667\\
-0.255686250369537	0.786921363444393\\
-0.369756251949576	0.725687504552447\\
-0.47122811850853	0.648589862732789\\
-0.558009078759514	0.558009078759514\\
-0.628431083783263	0.456581908289041\\
-0.681278445058817	0.347128705949459\\
-0.715803538350409	0.232578668243255\\
-0.731730559897045	0.115894735197653\\
-0.729247614287671	8.9307075662324e-17\\
};
\addplot [color=mycolor1,dotted,line width=1.0pt,forget plot]
  table[row sep=crcr]{1	0\\
0.956521682769877	0.151498151383791\\
0.891982039736211	0.289822533396298\\
0.809292583610082	0.412355167436434\\
0.711634858347723	0.517032989000692\\
0.602364630186427	0.602364630186426\\
0.484917701774751	0.667431957639199\\
0.362719588349683	0.711877274647591\\
0.239101059900595	0.735877395777214\\
0.117221283062812	0.740106053489981\\
4.44354699422903e-17	0.725686295399261\\
-0.109940132237539	0.694134676438339\\
-0.210320240583588	0.647299141978847\\
-0.299240493845688	0.587292536894097\\
-0.375203754387119	0.516423664031337\\
-0.437127756959533	0.437127756959533\\
-0.484346224770267	0.351898130574443\\
-0.516599582217932	0.263220634338226\\
-0.534016141209622	0.173512362385299\\
-0.537084820159711	0.0850658786478001\\
-0.526620599330303	6.44924231334916e-17\\
};
\addplot [color=mycolor1,dotted,line width=1.0pt,forget plot]
  table[row sep=crcr]{1	0\\
0.940082788364644	0.148894486293864\\
0.86158608093073	0.279946287694513\\
0.768279786681378	0.391458103646313\\
0.663960650859636	0.482395649771645\\
0.552351927561387	0.552351927561387\\
0.437014383712816	0.601498696728173\\
0.321269860940431	0.630527604183869\\
0.208138182971344	0.640583459188477\\
0.100287778328637	0.633192112325829\\
3.73630739569174e-17	0.610185303761559\\
-0.0908532273476425	0.573624701779294\\
-0.170819110593797	0.525727164509449\\
-0.238861981883349	0.468793035010043\\
-0.294350736089166	0.40513903143051\\
-0.337037028702328	0.337037028702328\\
-0.367025664975262	0.266659754473976\\
-0.384738677688982	0.196034147687371\\
-0.390874612629551	0.127002860400749\\
-0.386364521398959	0.0611941284830315\\
-0.372326104926586	4.55967972637346e-17\\
};
\addplot [color=mycolor1,dotted,line width=1.0pt,forget plot]
  table[row sep=crcr]{1	0\\
0.922246029428501	0.146069421212559\\
0.829201462983264	0.269423887468404\\
0.725373165529273	0.369596088217499\\
0.614985835074999	0.446813363302878\\
0.501902475185001	0.501902475185001\\
0.38956496084428	0.536190168955128\\
0.280953750703967	0.551402782692681\\
0.178565395716864	0.549567778706978\\
0.0844061798404073	0.532919645815306\\
3.08495992433718e-17	0.50381218919366\\
-0.0735915548753502	0.464638791061534\\
-0.135739038566523	0.417761804348184\\
-0.186207014322272	0.365451842507638\\
-0.225110018726605	0.309837359892227\\
-0.252864584784672	0.252864584784672\\
-0.270139142845401	0.196267575756099\\
-0.277803443075876	0.141547924204336\\
-0.27687894570066	0.0899634229303457\\
-0.268491413422519	0.0425248622468694\\
-0.253826721980109	3.10848082611005e-17\\
};
\addplot [color=mycolor1,dotted,line width=1.0pt,forget plot]
  table[row sep=crcr]{1	0\\
0.902056570675584	0.142871725087523\\
0.793293726869509	0.257756756757911\\
0.678769666307395	0.345850419328459\\
0.562876391436159	0.408953636388562\\
0.449318435861499	0.449318435861499\\
0.341115747349381	0.469505547439701\\
0.24062663375655	0.472256359314945\\
0.149586755320764	0.460380694230177\\
0.069160331541479	0.436661148025441\\
2.47240337905553e-17	0.403774113610049\\
-0.057687904157153	0.364227092250654\\
-0.104075505986926	0.320311471399148\\
-0.139645446506012	0.274069620358657\\
-0.165124892040398	0.227274916024158\\
-0.18142315316863	0.18142315316863\\
-0.189574186252466	0.137733708524426\\
-0.190684892879101	0.0971588057560207\\
-0.185889806735969	0.0603992595374446\\
-0.176312467991919	0.0279251515651378\\
-0.16303353482158	1.99658496572927e-17\\
};
\addplot [color=mycolor1,dotted,line width=1.0pt,forget plot]
  table[row sep=crcr]{1	0\\
0.877921760602431	0.139049146701748\\
0.751411952540787	0.244148543365328\\
0.625732257688895	0.318826509862098\\
0.505011411193747	0.36691226736026\\
0.392341669548685	0.392341669548685\\
0.289890514921595	0.399000063654162\\
0.199020572856858	0.390599867100522\\
0.120411913496453	0.370589763847805\\
0.0541820486548744	0.34209199176291\\
1.88512313530119e-17	0.307863971328499\\
-0.042808222513439	0.270280479734772\\
-0.075164540154518	0.231332667812908\\
-0.0981551954909503	0.192640417840451\\
-0.112959067758674	0.155474818616317\\
-0.120787864504912	0.120787864504912\\
-0.122837744156894	0.0892468451742257\\
-0.120251626285951	0.0612712639357851\\
-0.114091236431876	0.0370704898842818\\
-0.105317799143806	0.0166807006736036\\
-0.0947802248421549	1.16072298975411e-17\\
};
\addplot [color=mycolor1,dotted,line width=1.0pt,forget plot]
  table[row sep=crcr]{1	0\\
0.84674396664984	0.134111069256013\\
0.698989566160914	0.227115477506975\\
0.561406498364893	0.286050898428465\\
0.437004973478602	0.317502698182059\\
0.327450698344114	0.327450698344114\\
0.233352139914598	0.321181666481713\\
0.154515499860225	0.303253743289057\\
0.0901653834921457	0.27750051639687\\
0.0391311034995635	0.247064063991275\\
1.31311479217367e-17	0.214447919692096\\
-0.0287598398096346	0.181582482159892\\
-0.0487044416812479	0.149896858349689\\
-0.0613430200940395	0.120392455675976\\
-0.06808779312177	0.0937148074575858\\
-0.0702211210616195	0.0702211210616195\\
-0.068876740220244	0.0500418809603846\\
-0.0650321343871793	0.0331355275052096\\
-0.0595094084547913	0.0193357789181307\\
-0.0529823745536039	0.00839158374073751\\
-0.0459879102602678	5.63189470997126e-18\\
};
\addplot [color=mycolor1,dotted,line width=1.0pt,forget plot]
  table[row sep=crcr]{1	0\\
0.801053465278425	0.126874404768154\\
0.625589539649299	0.203266363189334\\
0.475341369738971	0.242198525079547\\
0.350045057714404	0.254322621147605\\
0.24813777530853	0.24813777530853\\
0.167289292234614	0.230253957320141\\
0.104794333468008	0.205670459781722\\
0.0578515468028918	0.178048753195384\\
0.0237523524567972	0.149966451301199\\
7.54043881219821e-18	0.123144711070133\\
-0.0156239119173694	0.0986454975334405\\
-0.0250311775652273	0.0770380431086105\\
-0.0298254673925332	0.0585357756361869\\
-0.0313184856998208	0.0431061974937683\\
-0.0305568546459545	0.0305568546459545\\
-0.0283545570469435	0.0206007915573586\\
-0.0253272293454815	0.012904867916705\\
-0.021925762268643	0.00712411201600239\\
-0.0184675753156993	0.00292497658052826\\
-0.0151646198645466	1.85713031774033e-18\\
};
\addplot [color=mycolor1,dotted,line width=1.0pt,forget plot]
  table[row sep=crcr]{1	0\\
0.714110955679367	0.113104064044896\\
0.497161927717827	0.161537702532642\\
0.336758208677056	0.171586877647116\\
0.221075553032581	0.160620791180475\\
0.139705610200823	0.139705610200823\\
0.0839640345306934	0.115566579097864\\
0.0468885871776745	0.0920240337831981\\
0.0230753615892199	0.0710186604775083\\
0.00844586939409394	0.0533251206797082\\
2.39022368106624e-18	0.0390353150431684\\
-0.00441505277265522	0.0278755461307222\\
-0.00630567510971132	0.0194068724759343\\
-0.00669794782622876	0.0131454627690819\\
-0.0062698831862128	0.00862975386096949\\
-0.0054534525074872	0.0054534525074872\\
-0.00451117782354635	0.00327756254020109\\
-0.00359218715027031	0.00183031077240959\\
-0.00277223578568335	0.000900754009370227\\
-0.00208156288544739	0.000329687172611911\\
-0.00152375582051941	1.86606268828125e-19\\
};
\addplot [color=mycolor1,dotted,line width=1.0pt,forget plot]
  table[row sep=crcr]{1	-0\\
0.987688340595138	-0.156434465040231\\
0.951056516295154	-0.309016994374947\\
0.891006524188368	-0.453990499739547\\
0.809016994374947	-0.587785252292473\\
0.707106781186548	-0.707106781186547\\
0.587785252292473	-0.809016994374947\\
0.453990499739547	-0.891006524188368\\
0.309016994374947	-0.951056516295154\\
0.156434465040231	-0.987688340595138\\
6.12323399573677e-17	-1\\
-0.156434465040231	-0.987688340595138\\
-0.309016994374947	-0.951056516295154\\
-0.453990499739547	-0.891006524188368\\
-0.587785252292473	-0.809016994374947\\
-0.707106781186547	-0.707106781186548\\
-0.809016994374947	-0.587785252292473\\
-0.891006524188368	-0.453990499739547\\
-0.951056516295154	-0.309016994374948\\
-0.987688340595138	-0.156434465040231\\
-1	-1.22464679914735e-16\\
};
\addplot [color=mycolor1,dotted,line width=1.0pt,forget plot]
  table[row sep=crcr]{1	-0\\
0.972218045703082	-0.153984211042097\\
0.921496791140463	-0.299412457456957\\
0.849791018366557	-0.432990150609548\\
0.759508516401756	-0.551815237548133\\
0.653437051516106	-0.653437051516106\\
0.534664304371591	-0.735902282058355\\
0.406492952796512	-0.797787339572243\\
0.272353130096494	-0.83821674479613\\
0.135714483753824	-0.856867527364047\\
5.22899666167123e-17	-0.853959960588124\\
-0.131496350792703	-0.830235283991667\\
-0.255686250369537	-0.786921363444393\\
-0.369756251949576	-0.725687504552447\\
-0.47122811850853	-0.648589862732789\\
-0.558009078759514	-0.558009078759514\\
-0.628431083783263	-0.456581908289041\\
-0.681278445058817	-0.347128705949459\\
-0.715803538350409	-0.232578668243255\\
-0.731730559897045	-0.115894735197653\\
-0.729247614287671	-8.9307075662324e-17\\
};
\addplot [color=mycolor1,dotted,line width=1.0pt,forget plot]
  table[row sep=crcr]{1	-0\\
0.956521682769877	-0.151498151383791\\
0.891982039736211	-0.289822533396298\\
0.809292583610082	-0.412355167436434\\
0.711634858347723	-0.517032989000692\\
0.602364630186427	-0.602364630186426\\
0.484917701774751	-0.667431957639199\\
0.362719588349683	-0.711877274647591\\
0.239101059900595	-0.735877395777214\\
0.117221283062812	-0.740106053489981\\
4.44354699422903e-17	-0.725686295399261\\
-0.109940132237539	-0.694134676438339\\
-0.210320240583588	-0.647299141978847\\
-0.299240493845688	-0.587292536894097\\
-0.375203754387119	-0.516423664031337\\
-0.437127756959533	-0.437127756959533\\
-0.484346224770267	-0.351898130574443\\
-0.516599582217932	-0.263220634338226\\
-0.534016141209622	-0.173512362385299\\
-0.537084820159711	-0.0850658786478001\\
-0.526620599330303	-6.44924231334916e-17\\
};
\addplot [color=mycolor1,dotted,line width=1.0pt,forget plot]
  table[row sep=crcr]{1	-0\\
0.940082788364644	-0.148894486293864\\
0.86158608093073	-0.279946287694513\\
0.768279786681378	-0.391458103646313\\
0.663960650859636	-0.482395649771645\\
0.552351927561387	-0.552351927561387\\
0.437014383712816	-0.601498696728173\\
0.321269860940431	-0.630527604183869\\
0.208138182971344	-0.640583459188477\\
0.100287778328637	-0.633192112325829\\
3.73630739569174e-17	-0.610185303761559\\
-0.0908532273476425	-0.573624701779294\\
-0.170819110593797	-0.525727164509449\\
-0.238861981883349	-0.468793035010043\\
-0.294350736089166	-0.40513903143051\\
-0.337037028702328	-0.337037028702328\\
-0.367025664975262	-0.266659754473976\\
-0.384738677688982	-0.196034147687371\\
-0.390874612629551	-0.127002860400749\\
-0.386364521398959	-0.0611941284830315\\
-0.372326104926586	-4.55967972637346e-17\\
};
\addplot [color=mycolor1,dotted,line width=1.0pt,forget plot]
  table[row sep=crcr]{1	-0\\
0.922246029428501	-0.146069421212559\\
0.829201462983264	-0.269423887468404\\
0.725373165529273	-0.369596088217499\\
0.614985835074999	-0.446813363302878\\
0.501902475185001	-0.501902475185001\\
0.38956496084428	-0.536190168955128\\
0.280953750703967	-0.551402782692681\\
0.178565395716864	-0.549567778706978\\
0.0844061798404073	-0.532919645815306\\
3.08495992433718e-17	-0.50381218919366\\
-0.0735915548753502	-0.464638791061534\\
-0.135739038566523	-0.417761804348184\\
-0.186207014322272	-0.365451842507638\\
-0.225110018726605	-0.309837359892227\\
-0.252864584784672	-0.252864584784672\\
-0.270139142845401	-0.196267575756099\\
-0.277803443075876	-0.141547924204336\\
-0.27687894570066	-0.0899634229303457\\
-0.268491413422519	-0.0425248622468694\\
-0.253826721980109	-3.10848082611005e-17\\
};
\addplot [color=mycolor1,dotted,line width=1.0pt,forget plot]
  table[row sep=crcr]{1	-0\\
0.902056570675584	-0.142871725087523\\
0.793293726869509	-0.257756756757911\\
0.678769666307395	-0.345850419328459\\
0.562876391436159	-0.408953636388562\\
0.449318435861499	-0.449318435861499\\
0.341115747349381	-0.469505547439701\\
0.24062663375655	-0.472256359314945\\
0.149586755320764	-0.460380694230177\\
0.069160331541479	-0.436661148025441\\
2.47240337905553e-17	-0.403774113610049\\
-0.057687904157153	-0.364227092250654\\
-0.104075505986926	-0.320311471399148\\
-0.139645446506012	-0.274069620358657\\
-0.165124892040398	-0.227274916024158\\
-0.18142315316863	-0.18142315316863\\
-0.189574186252466	-0.137733708524426\\
-0.190684892879101	-0.0971588057560207\\
-0.185889806735969	-0.0603992595374446\\
-0.176312467991919	-0.0279251515651378\\
-0.16303353482158	-1.99658496572927e-17\\
};
\addplot [color=mycolor1,dotted,line width=1.0pt,forget plot]
  table[row sep=crcr]{1	-0\\
0.877921760602431	-0.139049146701748\\
0.751411952540787	-0.244148543365328\\
0.625732257688895	-0.318826509862098\\
0.505011411193747	-0.36691226736026\\
0.392341669548685	-0.392341669548685\\
0.289890514921595	-0.399000063654162\\
0.199020572856858	-0.390599867100522\\
0.120411913496453	-0.370589763847805\\
0.0541820486548744	-0.34209199176291\\
1.88512313530119e-17	-0.307863971328499\\
-0.042808222513439	-0.270280479734772\\
-0.075164540154518	-0.231332667812908\\
-0.0981551954909503	-0.192640417840451\\
-0.112959067758674	-0.155474818616317\\
-0.120787864504912	-0.120787864504912\\
-0.122837744156894	-0.0892468451742257\\
-0.120251626285951	-0.0612712639357851\\
-0.114091236431876	-0.0370704898842818\\
-0.105317799143806	-0.0166807006736036\\
-0.0947802248421549	-1.16072298975411e-17\\
};
\addplot [color=mycolor1,dotted,line width=1.0pt,forget plot]
  table[row sep=crcr]{1	-0\\
0.84674396664984	-0.134111069256013\\
0.698989566160914	-0.227115477506975\\
0.561406498364893	-0.286050898428465\\
0.437004973478602	-0.317502698182059\\
0.327450698344114	-0.327450698344114\\
0.233352139914598	-0.321181666481713\\
0.154515499860225	-0.303253743289057\\
0.0901653834921457	-0.27750051639687\\
0.0391311034995635	-0.247064063991275\\
1.31311479217367e-17	-0.214447919692096\\
-0.0287598398096346	-0.181582482159892\\
-0.0487044416812479	-0.149896858349689\\
-0.0613430200940395	-0.120392455675976\\
-0.06808779312177	-0.0937148074575858\\
-0.0702211210616195	-0.0702211210616195\\
-0.068876740220244	-0.0500418809603846\\
-0.0650321343871793	-0.0331355275052096\\
-0.0595094084547913	-0.0193357789181307\\
-0.0529823745536039	-0.00839158374073751\\
-0.0459879102602678	-5.63189470997126e-18\\
};
\addplot [color=mycolor1,dotted,line width=1.0pt,forget plot]
  table[row sep=crcr]{1	-0\\
0.801053465278425	-0.126874404768154\\
0.625589539649299	-0.203266363189334\\
0.475341369738971	-0.242198525079547\\
0.350045057714404	-0.254322621147605\\
0.24813777530853	-0.24813777530853\\
0.167289292234614	-0.230253957320141\\
0.104794333468008	-0.205670459781722\\
0.0578515468028918	-0.178048753195384\\
0.0237523524567972	-0.149966451301199\\
7.54043881219821e-18	-0.123144711070133\\
-0.0156239119173694	-0.0986454975334405\\
-0.0250311775652273	-0.0770380431086105\\
-0.0298254673925332	-0.0585357756361869\\
-0.0313184856998208	-0.0431061974937683\\
-0.0305568546459545	-0.0305568546459545\\
-0.0283545570469435	-0.0206007915573586\\
-0.0253272293454815	-0.012904867916705\\
-0.021925762268643	-0.00712411201600239\\
-0.0184675753156993	-0.00292497658052826\\
-0.0151646198645466	-1.85713031774033e-18\\
};
\addplot [color=mycolor1,dotted,line width=1.0pt,forget plot]
  table[row sep=crcr]{1	-0\\
0.714110955679367	-0.113104064044896\\
0.497161927717827	-0.161537702532642\\
0.336758208677056	-0.171586877647116\\
0.221075553032581	-0.160620791180475\\
0.139705610200823	-0.139705610200823\\
0.0839640345306934	-0.115566579097864\\
0.0468885871776745	-0.0920240337831981\\
0.0230753615892199	-0.0710186604775083\\
0.00844586939409394	-0.0533251206797082\\
2.39022368106624e-18	-0.0390353150431684\\
-0.00441505277265522	-0.0278755461307222\\
-0.00630567510971132	-0.0194068724759343\\
-0.00669794782622876	-0.0131454627690819\\
-0.0062698831862128	-0.00862975386096949\\
-0.0054534525074872	-0.0054534525074872\\
-0.00451117782354635	-0.00327756254020109\\
-0.00359218715027031	-0.00183031077240959\\
-0.00277223578568335	-0.000900754009370227\\
-0.00208156288544739	-0.000329687172611911\\
-0.00152375582051941	-1.86606268828125e-19\\
};
\addplot [color=mycolor1,dotted,line width=1.0pt,forget plot]
  table[row sep=crcr]{1	0\\
1	0\\
};
\addplot [color=mycolor1,dotted,line width=1.0pt,forget plot]
  table[row sep=crcr]{0.951056516295154	0.309016994374947\\
0.906577591518048	0.290693092532446\\
0.867277205719189	0.267124444629623\\
0.833330533437629	0.239553777474139\\
0.804679893747523	0.20903820245934\\
0.781122098933164	0.176433510213537\\
0.762382187756933	0.142402376999381\\
0.748171074803624	0.107437628337747\\
0.738227708485823	0.0718935522249012\\
0.732347931289196	0.0360204899377399\\
0.730402691048646	2.81010850277162e-17\\
};
\addplot [color=mycolor1,dotted,line width=1.0pt,forget plot]
  table[row sep=crcr]{0.809016994374947	0.587785252292473\\
0.737380455396588	0.527071687397996\\
0.6808142826414	0.463341883835339\\
0.637053765657314	0.399254954339046\\
0.603812761314093	0.336417677088311\\
0.579022949915482	0.275632227640288\\
0.560948163233974	0.217130071437151\\
0.54821711318997	0.160763451735608\\
0.539811466724715	0.106147624627789\\
0.535036016768209	0.0527590625798542\\
0.533488091091103	4.10502162512614e-17\\
};
\addplot [color=mycolor1,dotted,line width=1.0pt,forget plot]
  table[row sep=crcr]{0.587785252292473	0.809016994374947\\
0.515276498489902	0.692182785870847\\
0.46668476526979	0.583707991451876\\
0.435233321876477	0.485319980094308\\
0.415551842123536	0.396928474901319\\
0.403656860517901	0.317461475735791\\
0.396737049613855	0.245416450708029\\
0.392888142823216	0.179177710929467\\
0.39087245229983	0.11717008156476\\
0.389932112762627	0.0579102497954412\\
0.389661137375347	4.49747826270753e-17\\
};
\addplot [color=mycolor1,dotted,line width=1.0pt,forget plot]
  table[row sep=crcr]{0.309016994374947	0.951056516295154\\
0.265925372344309	0.777304721760365\\
0.248822386132444	0.630899544522142\\
0.24643298177389	0.508693744238057\\
0.251412997268255	0.406266573115132\\
0.25929445161488	0.31919477108011\\
0.267517373913267	0.243597429511063\\
0.274697115780397	0.1762665508339\\
0.280129101393366	0.114599409879343\\
0.283480020554887	0.0564559973822998\\
0.284609543336029	4.37996030135249e-17\\
};
\addplot [color=mycolor1,dotted,line width=1.0pt,forget plot]
  table[row sep=crcr]{6.12323399573677e-17	1\\
0.0151248701748503	0.781989711398728\\
0.0472692933177679	0.613631335769719\\
0.0835006201485716	0.482943980920436\\
0.117381749765263	0.379469463911326\\
0.146223972383669	0.295078319831894\\
0.169261627793672	0.223809451176491\\
0.186582776182006	0.161390341419992\\
0.198560105942714	0.104739935929793\\
0.205572433929556	0.0515565221197431\\
0.207879576350762	3.99891848849262e-17\\
};
\addplot [color=mycolor1,dotted,line width=1.0pt,forget plot]
  table[row sep=crcr]{-0.309016994374947	0.951056516295154\\
-0.213607139159912	0.713331044437031\\
-0.122920349149865	0.544815253953639\\
-0.0461074386071066	0.422454854218942\\
0.0151311193047769	0.329888717873058\\
0.0621570724668225	0.256291005261778\\
0.0971710522581798	0.194731597161215\\
0.122256440677152	0.140793596164787\\
0.13905244595299	0.0915971142347777\\
0.148693455532156	0.0451621321066958\\
0.151835801980649	3.50498499033503e-17\\
};
\addplot [color=mycolor1,dotted,line width=1.0pt,forget plot]
  table[row sep=crcr]{-0.587785252292473	0.809016994374948\\
-0.401011853057454	0.584595820351374\\
-0.252139489074835	0.439670821081765\\
-0.139623392550337	0.340999317931598\\
-0.0567836371213516	0.268617800427267\\
0.0033339412143336	0.21116115844769\\
0.0463512370945915	0.162297289886265\\
0.0763422025660035	0.118472638203443\\
0.0960671266193367	0.0776165020305757\\
0.1072685824301	0.0384304051397518\\
0.110901278364195	2.98672554719679e-17\\
};
\addplot [color=mycolor1,dotted,line width=1.0pt,forget plot]
  table[row sep=crcr]{-0.809016994374948	0.587785252292473\\
-0.533486326814499	0.413410095118228\\
-0.336121655437603	0.313963860155743\\
-0.198040110920963	0.250717832404618\\
-0.101844033235305	0.204281393673556\\
-0.0346516892466227	0.165530866251108\\
0.0122258376810533	0.130333089269644\\
0.0443886084751889	0.0968398262452624\\
0.065339288702761	0.0642052594194206\\
0.0771736424133687	0.032008294596762\\
0.0810025921579431	2.49315700239574e-17\\
};
\addplot [color=mycolor1,dotted,line width=1.0pt,forget plot]
  table[row sep=crcr]{-0.951056516295154	0.309016994374948\\
-0.60382220830535	0.219707538176049\\
-0.375378071887508	0.182507388795924\\
-0.225093275108467	0.161489568357552\\
-0.124654461172013	0.143091836517116\\
-0.0562723920172722	0.12318609851568\\
-0.00923898083522721	0.101104614081101\\
0.0228060316514889	0.0772217637055013\\
0.0436193292019494	0.0520955750986265\\
0.0553650029180358	0.0262210407420432\\
0.0591645112940776	2.0486346567262e-17\\
};
\addplot [color=mycolor1,dotted,line width=1.0pt,forget plot]
  table[row sep=crcr]{-1	1.22464679914735e-16\\
-0.61127914703566	0.0236549857259488\\
-0.374309030147768	0.0580118391989452\\
-0.226262535142082	0.0806522438077526\\
-0.130218598863194	0.0890855793127953\\
-0.0656897647351535	0.0862950481802363\\
-0.0214411717925588	0.0757647040434823\\
0.00876629006412298	0.0602253159022079\\
0.0284556614934045	0.0415943455493057\\
0.039601950618638	0.0211971994741971\\
0.0432139182637723	1.66258696249815e-17\\
};
\addplot [color=mycolor1,dotted,line width=1.0pt,forget plot]
  table[row sep=crcr]{1	-0\\
1	-0\\
};
\addplot [color=mycolor1,dotted,line width=1.0pt,forget plot]
  table[row sep=crcr]{0.951056516295154	-0.309016994374947\\
0.906577591518048	-0.290693092532446\\
0.867277205719189	-0.267124444629623\\
0.833330533437629	-0.239553777474139\\
0.804679893747523	-0.20903820245934\\
0.781122098933164	-0.176433510213537\\
0.762382187756933	-0.142402376999381\\
0.748171074803624	-0.107437628337747\\
0.738227708485823	-0.0718935522249012\\
0.732347931289196	-0.0360204899377399\\
0.730402691048646	-2.81010850277162e-17\\
};
\addplot [color=mycolor1,dotted,line width=1.0pt,forget plot]
  table[row sep=crcr]{0.809016994374947	-0.587785252292473\\
0.737380455396588	-0.527071687397996\\
0.6808142826414	-0.463341883835339\\
0.637053765657314	-0.399254954339046\\
0.603812761314093	-0.336417677088311\\
0.579022949915482	-0.275632227640288\\
0.560948163233974	-0.217130071437151\\
0.54821711318997	-0.160763451735608\\
0.539811466724715	-0.106147624627789\\
0.535036016768209	-0.0527590625798542\\
0.533488091091103	-4.10502162512614e-17\\
};
\addplot [color=mycolor1,dotted,line width=1.0pt,forget plot]
  table[row sep=crcr]{0.587785252292473	-0.809016994374947\\
0.515276498489902	-0.692182785870847\\
0.46668476526979	-0.583707991451876\\
0.435233321876477	-0.485319980094308\\
0.415551842123536	-0.396928474901319\\
0.403656860517901	-0.317461475735791\\
0.396737049613855	-0.245416450708029\\
0.392888142823216	-0.179177710929467\\
0.39087245229983	-0.11717008156476\\
0.389932112762627	-0.0579102497954412\\
0.389661137375347	-4.49747826270753e-17\\
};
\addplot [color=mycolor1,dotted,line width=1.0pt,forget plot]
  table[row sep=crcr]{0.309016994374947	-0.951056516295154\\
0.265925372344309	-0.777304721760365\\
0.248822386132444	-0.630899544522142\\
0.24643298177389	-0.508693744238057\\
0.251412997268255	-0.406266573115132\\
0.25929445161488	-0.31919477108011\\
0.267517373913267	-0.243597429511063\\
0.274697115780397	-0.1762665508339\\
0.280129101393366	-0.114599409879343\\
0.283480020554887	-0.0564559973822998\\
0.284609543336029	-4.37996030135249e-17\\
};
\addplot [color=mycolor1,dotted,line width=1.0pt,forget plot]
  table[row sep=crcr]{6.12323399573677e-17	-1\\
0.0151248701748503	-0.781989711398728\\
0.0472692933177679	-0.613631335769719\\
0.0835006201485716	-0.482943980920436\\
0.117381749765263	-0.379469463911326\\
0.146223972383669	-0.295078319831894\\
0.169261627793672	-0.223809451176491\\
0.186582776182006	-0.161390341419992\\
0.198560105942714	-0.104739935929793\\
0.205572433929556	-0.0515565221197431\\
0.207879576350762	-3.99891848849262e-17\\
};
\addplot [color=mycolor1,dotted,line width=1.0pt,forget plot]
  table[row sep=crcr]{-0.309016994374947	-0.951056516295154\\
-0.213607139159912	-0.713331044437031\\
-0.122920349149865	-0.544815253953639\\
-0.0461074386071066	-0.422454854218942\\
0.0151311193047769	-0.329888717873058\\
0.0621570724668225	-0.256291005261778\\
0.0971710522581798	-0.194731597161215\\
0.122256440677152	-0.140793596164787\\
0.13905244595299	-0.0915971142347777\\
0.148693455532156	-0.0451621321066958\\
0.151835801980649	-3.50498499033503e-17\\
};
\addplot [color=mycolor1,dotted,line width=1.0pt,forget plot]
  table[row sep=crcr]{-0.587785252292473	-0.809016994374948\\
-0.401011853057454	-0.584595820351374\\
-0.252139489074835	-0.439670821081765\\
-0.139623392550337	-0.340999317931598\\
-0.0567836371213516	-0.268617800427267\\
0.0033339412143336	-0.21116115844769\\
0.0463512370945915	-0.162297289886265\\
0.0763422025660035	-0.118472638203443\\
0.0960671266193367	-0.0776165020305757\\
0.1072685824301	-0.0384304051397518\\
0.110901278364195	-2.98672554719679e-17\\
};
\addplot [color=mycolor1,dotted,line width=1.0pt,forget plot]
  table[row sep=crcr]{-0.809016994374948	-0.587785252292473\\
-0.533486326814499	-0.413410095118228\\
-0.336121655437603	-0.313963860155743\\
-0.198040110920963	-0.250717832404618\\
-0.101844033235305	-0.204281393673556\\
-0.0346516892466227	-0.165530866251108\\
0.0122258376810533	-0.130333089269644\\
0.0443886084751889	-0.0968398262452624\\
0.065339288702761	-0.0642052594194206\\
0.0771736424133687	-0.032008294596762\\
0.0810025921579431	-2.49315700239574e-17\\
};
\addplot [color=mycolor1,dotted,line width=1.0pt,forget plot]
  table[row sep=crcr]{-0.951056516295154	-0.309016994374948\\
-0.60382220830535	-0.219707538176049\\
-0.375378071887508	-0.182507388795924\\
-0.225093275108467	-0.161489568357552\\
-0.124654461172013	-0.143091836517116\\
-0.0562723920172722	-0.12318609851568\\
-0.00923898083522721	-0.101104614081101\\
0.0228060316514889	-0.0772217637055013\\
0.0436193292019494	-0.0520955750986265\\
0.0553650029180358	-0.0262210407420432\\
0.0591645112940776	-2.0486346567262e-17\\
};
\addplot [color=mycolor1,dotted,line width=1.0pt,forget plot]
  table[row sep=crcr]{-1	-1.22464679914735e-16\\
-0.61127914703566	-0.0236549857259488\\
-0.374309030147768	-0.0580118391989452\\
-0.226262535142082	-0.0806522438077526\\
-0.130218598863194	-0.0890855793127953\\
-0.0656897647351535	-0.0862950481802363\\
-0.0214411717925588	-0.0757647040434823\\
0.00876629006412298	-0.0602253159022079\\
0.0284556614934045	-0.0415943455493057\\
0.039601950618638	-0.0211971994741971\\
0.0432139182637723	-1.66258696249815e-17\\
};
\addplot [color=lightgray,line width=1.5pt,mark size=4.0pt,only marks,mark=x,mark options={solid},forget plot]
  table[row sep=crcr]{0	0\\
0.0332891817456478	0\\
0.0685951920820552	0\\
0.106261074191997	0\\
0.146718073740713	0\\
0.190507000794354	0\\
0.238288964560797	0\\
0.290802156451388	0\\
0.34863201227429	0\\
0.411482455545339	0\\
0.476716764894891	0\\
0.538701889630014	0\\
0.592009697191867	0\\
0.635111673402557	0\\
0.669445798454785	0\\
0.697088115438512	0\\
0.719764016628818	0\\
0.738726481861229	0\\
0.754859260015905	0\\
0.768789529287501	0\\
0.780970447060097	0\\
0.791736388624927	0\\
0.801339260556346	0\\
0.809972543832126	0\\
0.817787474350972	0\\
0.824904144359498	0\\
0.831419270792714	0\\
0.83741173557266	0\\
0.842946608042182	0\\
0.848078114087544	0\\
0.852851861415261	0\\
0.857306530845705	0\\
0.861475178388775	0\\
0.865386249582643	0\\
};
\addplot [color=gray,line width=1.5pt,mark size=4.0pt,only marks,mark=x,mark options={solid},forget plot]
  table[row sep=crcr]{0.91444306659383	0.404714563561125\\
0.897798475721005	0.408118988555352\\
0.880145470552802	0.411522521388228\\
0.861312529497832	0.414999075403529\\
0.841084029723473	0.418666762346229\\
0.819189566196653	0.422719274912659\\
0.795298584313431	0.427482631171093\\
0.769041988368137	0.433516364723153\\
0.740127060456686	0.441775148924797\\
0.708701838821161	0.453761439128821\\
0.676084684146385	0.471252951045491\\
0.645092121778823	0.494954826602702\\
0.618438217997896	0.523275555707163\\
0.596887229892552	0.553578075862308\\
0.57972016736644	0.583971213409664\\
0.565899008874575	0.613557528499428\\
0.554561058279421	0.642020626188909\\
0.545079825663216	0.669305600011421\\
0.537013436585878	0.695461292765181\\
0.530048301950081	0.720571832020281\\
0.523957843063781	0.744728338733943\\
0.518574872281366	0.768017822317744\\
0.513773436315659	0.790519346317733\\
0.509456794677767	0.812303231185067\\
0.505549329418344	0.833431472307299\\
0.50199099441408	0.853958585180083\\
0.498733431197472	0.873932540554951\\
0.495737198807501	0.893395651952642\\
0.49296976257274	0.912385366558339\\
0.490404009550058	0.930934949262658\\
0.488017135886199	0.949074065677673\\
0.485789801170979	0.966829275718118\\
0.483705477399443	0.984224450514024\\
0.48174994180251	1.00128112467398\\
};
\addplot [color=darkgray,line width=1.5pt,mark size=4.0pt,only marks,mark=x,mark options={solid},forget plot]
  table[row sep=crcr]{0.91444306659383	-0.404714563561125\\
0.897798475721005	-0.408118988555352\\
0.880145470552802	-0.411522521388228\\
0.861312529497832	-0.414999075403529\\
0.841084029723473	-0.418666762346229\\
0.819189566196653	-0.422719274912659\\
0.795298584313431	-0.427482631171093\\
0.769041988368137	-0.433516364723153\\
0.740127060456686	-0.441775148924797\\
0.708701838821161	-0.453761439128821\\
0.676084684146385	-0.471252951045491\\
0.645092121778823	-0.494954826602702\\
0.618438217997896	-0.523275555707163\\
0.596887229892552	-0.553578075862308\\
0.57972016736644	-0.583971213409664\\
0.565899008874575	-0.613557528499428\\
0.554561058279421	-0.642020626188909\\
0.545079825663216	-0.669305600011421\\
0.537013436585878	-0.695461292765181\\
0.530048301950081	-0.720571832020281\\
0.523957843063781	-0.744728338733943\\
0.518574872281366	-0.768017822317744\\
0.513773436315659	-0.790519346317733\\
0.509456794677767	-0.812303231185067\\
0.505549329418344	-0.833431472307299\\
0.50199099441408	-0.853958585180083\\
0.498733431197472	-0.873932540554951\\
0.495737198807501	-0.893395651952642\\
0.49296976257274	-0.912385366558339\\
0.490404009550058	-0.930934949262658\\
0.488017135886199	-0.949074065677673\\
0.485789801170979	-0.966829275718118\\
0.483705477399443	-0.984224450514024\\
};
\addplot [color=black,line width=1.5pt,mark size=4.0pt,only marks,mark=o,mark options={solid},forget plot]
  table[row sep=crcr]{1	0\\
};
\end{axis}
\end{tikzpicture}%}
    \end{figure}
    \end{small}
    \vfill
  }

  \frame{
    \frametitle{\insertsection}
    \vfill
    Projeto do controlador da malha externa:
    \vfill
    \begin{itemize}
      \item Tratamento do zero de fase não-mínima devido à discretização
      \vfill
      \item Explorar a característica robusta do controlador para reduzir a ordem do sistema
    \end{itemize}
    \vfill
  }

  \frame{
    \frametitle{\insertsection}
    \vfill
    Reescrita da expressão da planta em termos de uma \emph{dinâmica não-modelada}
    \begin{equation}
      G(z) = G_o(z) + \Delta(z)
      \label{eq:planta_go_delta_aditiva}
    \end{equation}
    \vfill
    Considerando $G_o(z)$ como sendo apenas um indutor $L_1 + L_2$:
    \begin{equation}
      G_o(z) = k \frac{T_s}{L_1+L_2} \frac{1}{z(z-1)}
      \label{eq:go_L1_L2}
    \end{equation}
    \vfill
    Logo, a dinâmica não-modelada é dada por
    \begin{equation}
      \Delta(z) = G(z) - G_o(z)
      \label{eq:dnm_deltaz}
    \end{equation}
    \vfill
  }

  \frame{
    \frametitle{\insertsection}
    \vfill
    Margem de ganho para $G(z)$, $G_o(z)$ e $\Delta(z)$ e diagrama de pólos e zeros para $\Delta(z)$ para a corrente do capacitor $i_C$ como variável controlada na malha interna
    \vfill
  }

  \frame{
    \frametitle{\insertsection}
    \vfill
    \begin{small}
    \begin{figure}[htb]
      \centering
      \raisebox{-0.5\height}{
        \def\svgwidth{0.5\textwidth}
        % This file was created by matlab2tikz v0.4.7 running on MATLAB 7.14.
% Copyright (c) 2008--2014, Nico Schlömer <nico.schloemer@gmail.com>
% All rights reserved.
% Minimal pgfplots version: 1.3
% 
% The latest updates can be retrieved from
%   http://www.mathworks.com/matlabcentral/fileexchange/22022-matlab2tikz
% where you can also make suggestions and rate matlab2tikz.
% 
%
% defining custom colors
\definecolor{mycolor1}{rgb}{0.66667,0.66667,0.66667}%
%
\begin{tikzpicture}

\begin{axis}[%
width=0.8\textwidth,
height=0.461611624834875\textwidth,
scale only axis,
xmode=log,
xmin=10,
xmax=31622.7766016838,
xminorticks=true,
xlabel={Frequência (rad/s)},
xmajorgrids,
xminorgrids,
ymin=-60,
ymax=60,
ylabel={Magnitude (dB)},
ymajorgrids,
legend style={draw=black,fill=white,legend cell align=left}
]
\addplot [color=mycolor1,solid,line width=1.5pt]
  table[row sep=crcr]{10	-10.0200805450291\\
10.0461810465159	-10.020080477148\\
10.0925753619375	-10.0200804086386\\
10.139183931163	-10.0200803394949\\
10.1860077436388	-10.0200802697111\\
10.2330477933808	-10.0200801992812\\
10.2803050789954	-10.0200801281994\\
10.3277806037004	-10.0200800564595\\
10.375475375347	-10.0200799840555\\
10.4233904064403	-10.0200799109812\\
10.4715267141616	-10.0200798372304\\
10.5198853203895	-10.0200797627968\\
10.5684672517218	-10.0200796876742\\
10.6172735394972	-10.0200796118561\\
10.6663052198171	-10.0200795353361\\
10.715563333568	-10.0200794581078\\
10.7650489264432	-10.0200793801645\\
10.814763048965	-10.0200793014996\\
10.8647067565072	-10.0200792221065\\
10.9148811093176	-10.0200791419783\\
10.9652871725401	-10.0200790611084\\
11.0159260162376	-10.0200789794899\\
11.0667987154147	-10.0200788971157\\
11.1179063500406	-10.020078813979\\
11.1692500050717	-10.0200787300726\\
11.2208307704748	-10.0200786453894\\
11.2726497412507	-10.0200785599223\\
11.3247080174565	-10.020078473664\\
11.3770067042298	-10.0200783866071\\
11.4295469118117	-10.0200782987443\\
11.4823297555707	-10.0200782100681\\
11.535356356026	-10.020078120571\\
11.5886278388715	-10.0200780302453\\
11.6421453349997	-10.0200779390834\\
11.6959099805258	-10.0200778470776\\
11.7499229168114	-10.0200777542201\\
11.8041852904894	-10.0200776605029\\
11.8586982534876	-10.0200775659181\\
11.9134629630538	-10.0200774704577\\
11.96848058178	-10.0200773741135\\
12.0237522776272	-10.0200772768775\\
12.0792792239501	-10.0200771787413\\
12.135062599522	-10.0200770796965\\
12.1911035885602	-10.0200769797349\\
12.2474033807505	-10.0200768788478\\
12.3039631712731	-10.0200767770268\\
12.3607841608273	-10.0200766742632\\
12.4178675556577	-10.0200765705482\\
12.4752145675793	-10.020076465873\\
12.5328264140034	-10.0200763602288\\
12.5907043179635	-10.0200762536067\\
12.6488495081411	-10.0200761459974\\
12.7072632188919	-10.020076037392\\
12.765946690272	-10.0200759277811\\
12.8249011680643	-10.0200758171555\\
12.8841279038047	-10.0200757055057\\
12.9436281548089	-10.0200755928224\\
13.0034031841991	-10.0200754790959\\
13.0634542609305	-10.0200753643165\\
13.1237826598188	-10.0200752484746\\
13.1843896615665	-10.0200751315602\\
13.2452765527909	-10.0200750135635\\
13.306444626051	-10.0200748944744\\
13.3678951798747	-10.0200747742829\\
13.4296295187868	-10.0200746529786\\
13.4916489533366	-10.0200745305514\\
13.5539548001256	-10.0200744069908\\
13.6165483818355	-10.0200742822863\\
13.6794310272563	-10.0200741564274\\
13.7426040713143	-10.0200740294033\\
13.806068855101	-10.0200739012033\\
13.8698267259009	-10.0200737718164\\
13.9338790372205	-10.0200736412318\\
13.998227148817	-10.0200735094382\\
14.062872426727	-10.0200733764245\\
14.1278162432955	-10.0200732421795\\
14.1930599772055	-10.0200731066916\\
14.2586050135065	-10.0200729699495\\
14.3244527436445	-10.0200728319414\\
14.3906045654914	-10.0200726926558\\
14.4570618833745	-10.0200725520807\\
14.5238261081064	-10.0200724102042\\
14.5908986570152	-10.0200722670142\\
14.658280953974	-10.0200721224987\\
14.7259744294318	-10.0200719766453\\
14.7939805204436	-10.0200718294416\\
14.8623006707006	-10.0200716808752\\
14.9309363305612	-10.0200715309334\\
14.999888957082	-10.0200713796035\\
15.069160014048	-10.0200712268727\\
15.1387509720044	-10.0200710727279\\
15.2086633082875	-10.0200709171561\\
15.2788985070559	-10.0200707601441\\
15.3494580593225	-10.0200706016786\\
15.4203434629857	-10.020070441746\\
15.4915562228612	-10.0200702803328\\
15.5630978507143	-10.0200701174253\\
15.6349698652918	-10.0200699530097\\
15.7071737923542	-10.020069787072\\
15.779711164708	-10.0200696195981\\
15.8525835222385	-10.0200694505738\\
15.9257924119422	-10.0200692799847\\
15.9993393879601	-10.0200691078164\\
16.07322601161	-10.0200689340542\\
16.1474538514202	-10.0200687586834\\
16.2220244831628	-10.0200685816891\\
16.2969394898867	-10.0200684030562\\
16.3722004619516	-10.0200682227696\\
16.4478089970617	-10.020068040814\\
16.5237667002994	-10.0200678571739\\
16.6000751841599	-10.0200676718337\\
16.6767360685846	-10.0200674847777\\
16.7537509809962	-10.02006729599\\
16.8311215563331	-10.0200671054546\\
16.9088494370839	-10.0200669131553\\
16.9869362733223	-10.0200667190757\\
17.0653837227424	-10.0200665231994\\
17.1441934506935	-10.0200663255098\\
17.2233671302159	-10.02006612599\\
17.302906442076	-10.0200659246232\\
17.3828130748022	-10.0200657213922\\
17.4630887247206	-10.0200655162797\\
17.5437350959914	-10.0200653092684\\
17.6247539006444	-10.0200651003406\\
17.7061468586161	-10.0200648894786\\
17.7879156977856	-10.0200646766646\\
17.8700621540116	-10.0200644618804\\
17.9525879711692	-10.0200642451078\\
18.035494901187	-10.0200640263284\\
18.1187847040838	-10.0200638055237\\
18.2024591480069	-10.0200635826747\\
18.2865200092687	-10.0200633577628\\
18.3709690723849	-10.0200631307687\\
18.4558081301122	-10.0200629016731\\
18.5410389834868	-10.0200626704567\\
18.6266634418617	-10.0200624370997\\
18.7126833229461	-10.0200622015824\\
18.7991004528435	-10.0200619638847\\
18.8859166660905	-10.0200617239866\\
18.9731338056957	-10.0200614818675\\
19.060753723179	-10.020061237507\\
19.1487782786108	-10.0200609908842\\
19.2372093406515	-10.0200607419783\\
19.3260487865912	-10.0200604907682\\
19.4152985023894	-10.0200602372324\\
19.5049603827153	-10.0200599813494\\
19.5950363309877	-10.0200597230976\\
19.6855282594159	-10.020059462455\\
19.7764380890397	-10.0200591993994\\
19.8677677497706	-10.0200589339085\\
19.9595191804325	-10.0200586659598\\
20.0516943288031	-10.0200583955306\\
20.1442951516552	-10.0200581225978\\
20.2373236147981	-10.0200578471382\\
20.3307816931193	-10.0200575691286\\
20.4246713706267	-10.0200572885452\\
20.5189946404906	-10.0200570053642\\
20.6137535050858	-10.0200567195617\\
20.7089499760343	-10.0200564311133\\
20.8045860742482	-10.0200561399945\\
20.900663829972	-10.0200558461806\\
20.9971852828265	-10.0200555496467\\
21.0941524818514	-10.0200552503676\\
21.1915674855492	-10.0200549483179\\
21.2894323619287	-10.0200546434718\\
21.387749188549	-10.0200543358036\\
21.4865200525637	-10.0200540252871\\
21.5857470507649	-10.0200537118959\\
21.685432289628	-10.0200533956035\\
21.7855778853565	-10.0200530763829\\
21.8861859639264	-10.020052754207\\
21.987258661132	-10.0200524290486\\
22.0887981226306	-10.0200521008799\\
22.1908065039887	-10.0200517696731\\
22.2932859707273	-10.0200514354001\\
22.396238698368	-10.0200510980325\\
22.499666872479	-10.0200507575417\\
22.603572688722	-10.0200504138987\\
22.7079583528983	-10.0200500670743\\
22.8128260809959	-10.0200497170391\\
22.9181780992365	-10.0200493637634\\
23.0240166441225	-10.0200490072172\\
23.130343962485	-10.0200486473702\\
23.237162311531	-10.0200482841918\\
23.3444739588916	-10.0200479176512\\
23.4522811826701	-10.0200475477173\\
23.5605862714901	-10.0200471743586\\
23.6693915245447	-10.0200467975434\\
23.7786992516445	-10.0200464172398\\
23.8885117732672	-10.0200460334155\\
23.9988314206069	-10.0200456460378\\
24.1096605356231	-10.0200452550739\\
24.2210014710909	-10.0200448604905\\
24.3328565906507	-10.0200444622542\\
24.4452282688584	-10.0200440603311\\
24.558118891236	-10.0200436546871\\
24.6715308543219	-10.0200432452877\\
24.7854665657221	-10.0200428320983\\
24.899928444161	-10.0200424150836\\
25.0149189195332	-10.0200419942083\\
25.1304404329547	-10.0200415694366\\
25.2464954368146	-10.0200411407326\\
25.3630863948276	-10.0200407080596\\
25.4802157820863	-10.0200402713811\\
25.597886085113	-10.0200398306599\\
25.7160998019135	-10.0200393858587\\
25.8348594420295	-10.0200389369395\\
25.9541675265918	-10.0200384838644\\
26.0740265883745	-10.0200380265948\\
26.194439171848	-10.0200375650919\\
26.3154078332332	-10.0200370993165\\
26.4369351405564	-10.020036629229\\
26.5590236737027	-10.0200361547896\\
26.6816760244719	-10.020035675958\\
26.8048947966327	-10.0200351926934\\
26.9286826059784	-10.0200347049548\\
27.0530420803822	-10.0200342127009\\
27.1779758598533	-10.0200337158897\\
27.3034865965925	-10.0200332144793\\
27.4295769550488	-10.0200327084268\\
27.556249611976	-10.0200321976894\\
27.6835072564894	-10.0200316822237\\
27.8113525901229	-10.0200311619859\\
27.9397883268864	-10.0200306369319\\
28.0688171933232	-10.020030107017\\
28.1984419285683	-10.0200295721962\\
28.3286652844062	-10.0200290324242\\
28.4594900253294	-10.0200284876551\\
28.5909189285972	-10.0200279378426\\
28.7229547842946	-10.02002738294\\
28.8556003953913	-10.0200268229002\\
28.9888585778017	-10.0200262576757\\
29.1227321604441	-10.0200256872184\\
29.2572239853012	-10.0200251114799\\
29.3923369074803	-10.0200245304113\\
29.5280737952738	-10.0200239439633\\
29.6644375302202	-10.0200233520861\\
29.8014310071653	-10.0200227547293\\
29.9390571343235	-10.0200221518423\\
30.0773188333397	-10.0200215433738\\
30.2162190393513	-10.0200209292722\\
30.3557607010504	-10.0200203094854\\
30.4959467807464	-10.0200196839607\\
30.6367802544292	-10.0200190526449\\
30.7782641118319	-10.0200184154845\\
30.9204013564946	-10.0200177724254\\
31.063195005828	-10.0200171234129\\
31.2066480911777	-10.0200164683919\\
31.350763657888	-10.0200158073068\\
31.4955447653674	-10.0200151401014\\
31.6409944871527	-10.0200144667192\\
31.7871159109747	-10.0200137871028\\
31.9339121388238	-10.0200131011947\\
32.0813862870155	-10.0200124089364\\
32.229541486257	-10.0200117102693\\
32.3783808817133	-10.020011005134\\
32.5279076330741	-10.0200102934707\\
32.6781249146208	-10.0200095752188\\
32.8290359152942	-10.0200088503174\\
32.9806438387618	-10.0200081187049\\
33.132951903486	-10.0200073803192\\
33.2859633427924	-10.0200066350976\\
33.4396814049384	-10.0200058829768\\
33.5941093531822	-10.0200051238929\\
33.7492504658521	-10.0200043577815\\
33.9051080364161	-10.0200035845774\\
34.0616853735517	-10.0200028042151\\
34.2189858012163	-10.0200020166282\\
34.3770126587175	-10.0200012217498\\
34.5357693007845	-10.0200004195125\\
34.6952590976386	-10.0199996098482\\
34.8554854350656	-10.019998792688\\
35.0164517144866	-10.0199979679625\\
35.1781613530314	-10.0199971356018\\
35.3406177836102	-10.0199962955351\\
35.5038244549868	-10.0199954476911\\
35.6677848318515	-10.0199945919977\\
35.8325023948954	-10.0199937283824\\
35.9979806408833	-10.0199928567717\\
36.1642230827288	-10.0199919770916\\
36.3312332495683	-10.0199910892675\\
36.4990146868361	-10.0199901932238\\
36.6675709563398	-10.0199892888846\\
36.8369056363357	-10.0199883761731\\
37.007022321605	-10.0199874550116\\
37.1779246235299	-10.019986525322\\
37.3496161701703	-10.0199855870253\\
37.5221006063408	-10.0199846400418\\
37.6953815936883	-10.0199836842911\\
37.8694628107696	-10.019982719692\\
38.0443479531292	-10.0199817461627\\
38.2200407333782	-10.0199807636203\\
38.3965448812729	-10.0199797719815\\
38.5738641437941	-10.0199787711621\\
38.7520022852263	-10.019977761077\\
38.9309630872381	-10.0199767416405\\
39.1107503489621	-10.019975712766\\
39.2913678870758	-10.019974674366\\
39.4728195358824	-10.0199736263525\\
39.6551091473924	-10.0199725686364\\
39.8382405914052	-10.0199715011278\\
40.0222177555915	-10.0199704237361\\
40.2070445455755	-10.0199693363698\\
40.3927248850181	-10.0199682389366\\
40.5792627157	-10.0199671313432\\
40.7666619976054	-10.0199660134956\\
40.9549267090063	-10.0199648852988\\
41.1440608465467	-10.019963746657\\
41.3340684253274	-10.0199625974735\\
41.5249534789915	-10.0199614376508\\
41.7167200598098	-10.0199602670902\\
41.9093722387671	-10.0199590856923\\
42.1029141056481	-10.0199578933569\\
42.2973497691249	-10.0199566899827\\
42.4926833568435	-10.0199554754674\\
42.6889190155123	-10.0199542497079\\
42.8860609109892	-10.0199530126002\\
43.0841132283705	-10.019951764039\\
43.2830801720801	-10.0199505039184\\
43.4829659659579	-10.0199492321314\\
43.6837748533501	-10.0199479485699\\
43.8855110971994	-10.0199466531249\\
44.0881789801347	-10.0199453456864\\
44.2917828045629	-10.0199440261434\\
44.4963268927599	-10.0199426943837\\
44.701815586962	-10.0199413502943\\
44.9082532494586	-10.0199399937609\\
45.1156442626847	-10.0199386246685\\
45.3239930293136	-10.0199372429006\\
45.5333039723509	-10.0199358483399\\
45.7435815352278	-10.0199344408681\\
45.954830181896	-10.0199330203654\\
46.167054396922	-10.0199315867114\\
46.3802586855826	-10.0199301397841\\
46.5944475739604	-10.0199286794608\\
46.8096256090399	-10.0199272056174\\
47.0257973588042	-10.0199257181286\\
47.2429674123316	-10.0199242168682\\
47.4611403798933	-10.0199227017087\\
47.6803208930514	-10.0199211725213\\
47.9005136047569	-10.0199196291761\\
48.1217231894485	-10.0199180715421\\
48.3439543431522	-10.019916499487\\
48.5672117835806	-10.0199149128773\\
48.791500250233	-10.0199133115781\\
49.0168245044966	-10.0199116954534\\
49.243189329747	-10.0199100643661\\
49.4705995314498	-10.0199084181775\\
49.6990599372628	-10.0199067567479\\
49.9285753971387	-10.019905079936\\
50.1591507834275	-10.0199033875995\\
50.3907909909802	-10.0199016795946\\
50.6235009372529	-10.0198999557763\\
50.8572855624109	-10.0198982159981\\
51.0921498294339	-10.0198964601123\\
51.3280987242209	-10.0198946879696\\
51.5651372556965	-10.0198928994196\\
51.8032704559168	-10.0198910943104\\
52.0425033801768	-10.0198892724886\\
52.2828411071172	-10.0198874337995\\
52.5242887388322	-10.0198855780868\\
52.7668514009784	-10.019883705193\\
53.010534242883	-10.019881814959\\
53.2553424376533	-10.0198799072242\\
53.5012811822866	-10.0198779818265\\
53.7483556977805	-10.0198760386025\\
53.9965712292437	-10.0198740773869\\
54.2459330460073	-10.0198720980134\\
54.4964464417369	-10.0198701003137\\
54.7481167345445	-10.019868084118\\
55.0009492671021	-10.0198660492553\\
55.2549494067543	-10.0198639955525\\
55.510122545633	-10.0198619228353\\
55.7664741007712	-10.0198598309276\\
56.0240095142187	-10.0198577196516\\
56.282734253157	-10.0198555888281\\
56.5426538100157	-10.019853438276\\
56.8037737025889	-10.0198512678127\\
57.0660994741527	-10.0198490772537\\
57.3296366935823	-10.0198468664131\\
57.5943909554708	-10.0198446351029\\
57.8603678802477	-10.0198423831337\\
58.1275731142982	-10.0198401103141\\
58.3960123300829	-10.0198378164511\\
58.6656912262587	-10.0198355013498\\
58.9366155277994	-10.0198331648135\\
59.2087909861172	-10.0198308066438\\
59.4822233791852	-10.0198284266403\\
59.7569185116595	-10.0198260246009\\
60.0328822150028	-10.0198236003214\\
60.3101203476082	-10.0198211535961\\
60.5886387949234	-10.0198186842169\\
60.8684434695757	-10.0198161919741\\
61.1495403114975	-10.019813676656\\
61.4319352880526	-10.019811138049\\
61.7156343941624	-10.0198085759373\\
62.0006436524339	-10.0198059901033\\
62.2869691132868	-10.0198033803273\\
62.5746168550822	-10.0198007463877\\
62.8635929842521	-10.0197980880606\\
63.1539036354283	-10.0197954051202\\
63.445554971573	-10.0197926973386\\
63.7385531841099	-10.0197899644858\\
64.032904493055	-10.0197872063295\\
64.3286151471491	-10.0197844226355\\
64.6256914239905	-10.0197816131673\\
64.9241396301678	-10.0197787776862\\
65.2239661013943	-10.0197759159512\\
65.5251772026422	-10.0197730277194\\
65.827779328278	-10.0197701127452\\
66.1317789021976	-10.0197671707811\\
66.4371823779638	-10.0197642015771\\
66.7439962389419	-10.0197612048809\\
67.0522269984386	-10.0197581804379\\
67.3618811998395	-10.0197551279913\\
67.6729654167483	-10.0197520472816\\
67.9854862531262	-10.0197489380472\\
68.2994503434323	-10.0197458000238\\
68.6148643527643	-10.0197426329448\\
68.931734977	-10.0197394365412\\
69.2500689429393	-10.0197362105414\\
69.5698730084476	-10.0197329546713\\
69.8911539625984	-10.0197296686543\\
70.213918625818	-10.0197263522112\\
70.5381738500302	-10.0197230050602\\
70.8639265188016	-10.019719626917\\
71.191183547488	-10.0197162174944\\
71.5199518833808	-10.0197127765029\\
71.8502385058548	-10.01970930365\\
72.1820504265165	-10.0197057986407\\
72.5153946893524	-10.0197022611772\\
72.8502783708792	-10.0196986909589\\
73.1867085802933	-10.0196950876824\\
73.5246924596224	-10.0196914510417\\
73.8642371838769	-10.0196877807277\\
74.2053499612018	-10.0196840764285\\
74.5480380330304	-10.0196803378294\\
74.8923086742377	-10.0196765646127\\
75.2381691932944	-10.0196727564578\\
75.585626932423	-10.0196689130412\\
75.9346892677529	-10.0196650340362\\
76.2853636094773	-10.0196611191133\\
76.6376574020103	-10.0196571679397\\
76.9915781241455	-10.0196531801798\\
77.3471332892138	-10.0196491554947\\
77.7043304452438	-10.0196450935424\\
78.0631771751216	-10.0196409939777\\
78.4236810967518	-10.0196368564523\\
78.7858498632195	-10.0196326806146\\
79.1496911629522	-10.0196284661097\\
79.5152127198837	-10.0196242125796\\
79.8824222936175	-10.0196199196627\\
80.2513276795919	-10.0196155869943\\
80.6219367092452	-10.0196112142062\\
80.9942572501823	-10.0196068009267\\
81.3682972063414	-10.0196023467809\\
81.7440645181619	-10.0195978513903\\
82.121567162753	-10.0195933143727\\
82.5008131540631	-10.0195887353427\\
82.8818105430498	-10.0195841139112\\
83.2645674178507	-10.0195794496853\\
83.6490919039556	-10.0195747422687\\
84.0353921643786	-10.0195699912614\\
84.423476399831	-10.0195651962596\\
84.8133528488963	-10.0195603568557\\
85.2050297882047	-10.0195554726386\\
85.5985155326084	-10.0195505431931\\
85.9938184353586	-10.0195455681002\\
86.3909468882829	-10.0195405469373\\
86.7899093219629	-10.0195354792774\\
87.1907142059136	-10.01953036469\\
87.5933700487633	-10.0195252027404\\
87.9978853984339	-10.0195199929897\\
88.4042688423224	-10.0195147349954\\
88.8125290074835	-10.0195094283104\\
89.2226745608124	-10.0195040724837\\
89.6347142092288	-10.0194986670602\\
90.0486566998624	-10.0194932115804\\
90.4645108202374	-10.0194877055806\\
90.88228539846	-10.0194821485928\\
91.3019893034057	-10.0194765401448\\
91.7236314449071	-10.0194708797599\\
92.1472207739435	-10.0194651669568\\
92.5727662828307	-10.01945940125\\
93.0002770054119	-10.0194535821496\\
93.4297620172496	-10.0194477091607\\
93.8612304358183	-10.0194417817843\\
94.2946914206979	-10.0194357995165\\
94.730154173768	-10.0194297618488\\
95.1676279394036	-10.019423668268\\
95.6071220046713	-10.019417518256\\
96.0486456995261	-10.0194113112902\\
96.4922083970099	-10.0194050468429\\
96.9378195134503	-10.0193987243816\\
97.3854885086602	-10.0193923433689\\
97.8352248861393	-10.0193859032622\\
98.2870381932753	-10.0193794035142\\
98.7409380215466	-10.0193728435723\\
99.1969340067261	-10.0193662228789\\
99.655035829086	-10.019359540871\\
100.115253213603	-10.0193527969807\\
100.577595930163	-10.0193459906347\\
101.042073793774	-10.0193391212543\\
101.508696664767	-10.0193321882555\\
101.977474449012	-10.0193251910489\\
102.448417098122	-10.0193181290397\\
102.921534609671	-10.0193110016274\\
103.3968370274	-10.019303808206\\
103.874334441436	-10.0192965481641\\
104.354036988501	-10.0192892208843\\
104.83595485213	-10.0192818257437\\
105.320098262886	-10.0192743621136\\
105.80647749858	-10.0192668293593\\
106.295102884484	-10.0192592268403\\
106.785984793556	-10.0192515539104\\
107.279133646656	-10.019243809917\\
107.774559912768	-10.0192359942018\\
108.272274109224	-10.0192281061002\\
108.772286801926	-10.0192201449414\\
109.27460860557	-10.0192121100485\\
109.779250183872	-10.0192040007384\\
110.286222249794	-10.0191958163214\\
110.795535565772	-10.0191875561016\\
111.307200943943	-10.0191792193766\\
111.821229246378	-10.0191708054375\\
112.337631385307	-10.0191623135688\\
112.856418323356	-10.0191537430483\\
113.377601073777	-10.0191450931473\\
113.901190700682	-10.0191363631302\\
114.427198319278	-10.0191275522544\\
114.955635096105	-10.0191186597709\\
115.486512249268	-10.0191096849232\\
116.019841048683	-10.0191006269481\\
116.555632816306	-10.0190914850754\\
117.093898926384	-10.0190822585274\\
117.634650805689	-10.0190729465195\\
118.177899933763	-10.0190635482598\\
118.723657843162	-10.0190540629488\\
119.271936119701	-10.0190444897799\\
119.822746402699	-10.0190348279388\\
120.376100385228	-10.0190250766037\\
120.932009814357	-10.0190152349453\\
121.490486491407	-10.0190053021264\\
122.051542272196	-10.0189952773021\\
122.615189067297	-10.0189851596198\\
123.181438842284	-10.0189749482188\\
123.750303617991	-10.0189646422305\\
124.321795470765	-10.0189542407782\\
124.895926532723	-10.0189437429772\\
125.472708992008	-10.0189331479345\\
126.052155093052	-10.0189224547487\\
126.63427713683	-10.0189116625102\\
127.219087481126	-10.0189007703008\\
127.806598540793	-10.0188897771941\\
128.396822788018	-10.0188786822546\\
128.989772752585	-10.0188674845386\\
129.585461022141	-10.0188561830933\\
130.183900242466	-10.0188447769573\\
130.785103117737	-10.0188332651602\\
131.389082410804	-10.0188216467224\\
131.995850943453	-10.0188099206556\\
132.605421596686	-10.0187980859619\\
133.217807310988	-10.0187861416344\\
133.833021086605	-10.0187740866568\\
134.451075983821	-10.0187619200034\\
135.071985123233	-10.0187496406388\\
135.69576168603	-10.0187372475183\\
136.322418914273	-10.0187247395872\\
136.951970111177	-10.0187121157813\\
137.584428641392	-10.0186993750261\\
138.219807931287	-10.0186865162376\\
138.858121469236	-10.0186735383213\\
139.499382805904	-10.018660440173\\
140.143605554534	-10.0186472206778\\
140.790803391236	-10.0186338787108\\
141.440990055278	-10.0186204131364\\
142.094179349378	-10.0186068228086\\
142.750385139995	-10.0185931065707\\
143.409621357626	-10.0185792632552\\
144.0719019971	-10.018565291684\\
144.737241117876	-10.0185511906678\\
145.405652844341	-10.0185369590064\\
146.077151366109	-10.0185225954883\\
146.751750938323	-10.0185080988909\\
147.42946588196	-10.0184934679803\\
148.110310584131	-10.0184787015109\\
148.794299498388	-10.0184637982257\\
149.481447145032	-10.018448756856\\
150.171768111418	-10.0184335761213\\
150.865277052271	-10.0184182547292\\
151.561988689989	-10.0184027913753\\
152.261917814963	-10.0183871847432\\
152.965079285884	-10.018371433504\\
153.671488030065	-10.0183555363167\\
154.381159043753	-10.0183394918278\\
155.094107392451	-10.018323298671\\
155.810348211234	-10.0183069554677\\
156.529896705074	-10.0182904608261\\
157.25276814916	-10.0182738133418\\
157.978977889225	-10.018257011597\\
158.708541341869	-10.0182400541611\\
159.441473994887	-10.0182229395899\\
160.177791407599	-10.018205666426\\
160.917509211179	-10.0181882331983\\
161.66064310899	-10.0181706384222\\
162.40720887691	-10.018152880599\\
163.157222363676	-10.0181349582163\\
163.910699491214	-10.0181168697477\\
164.66765625498	-10.0180986136525\\
165.428108724297	-10.0180801883756\\
166.192073042701	-10.0180615923476\\
166.959565428276	-10.0180428239845\\
167.730602174008	-10.0180238816874\\
168.505199648122	-10.0180047638427\\
169.283374294433	-10.0179854688218\\
170.065142632699	-10.0179659949809\\
170.850521258965	-10.0179463406609\\
171.639526845917	-10.0179265041874\\
172.432176143241	-10.0179064838701\\
173.228485977972	-10.0178862780035\\
174.028473254854	-10.0178658848658\\
174.832154956701	-10.0178453027193\\
175.639548144754	-10.0178245298102\\
176.450669959044	-10.0178035643685\\
177.265537618758	-10.0177824046075\\
178.084168422602	-10.017761048724\\
178.906579749169	-10.017739494898\\
179.732789057308	-10.0177177412926\\
180.562813886497	-10.0176957860538\\
181.39667185721	-10.0176736273103\\
182.234380671297	-10.0176512631736\\
183.075958112354	-10.0176286917373\\
183.921422046107	-10.0176059110774\\
184.770790420785	-10.0175829192522\\
185.624081267505	-10.0175597143016\\
186.481312700654	-10.0175362942476\\
187.342502918271	-10.0175126570934\\
188.207670202438	-10.017488800824\\
189.076832919665	-10.0174647234054\\
189.950009521279	-10.0174404227847\\
190.827218543818	-10.01741589689\\
191.708478609426	-10.0173911436301\\
192.593808426241	-10.0173661608942\\
193.483226788801	-10.017340946552\\
194.376752578439	-10.0173154984532\\
195.274404763682	-10.0172898144277\\
196.176202400658	-10.017263892285\\
197.082164633495	-10.0172377298143\\
197.992310694735	-10.0172113247842\\
198.906659905733	-10.0171846749425\\
199.825231677076	-10.017157778016\\
200.748045508988	-10.0171306317105\\
201.675120991751	-10.0171032337102\\
202.606477806113	-10.0170755816779\\
203.542135723711	-10.0170476732544\\
204.482114607491	-10.0170195060589\\
205.426434412127	-10.0169910776881\\
206.375115184445	-10.0169623857164\\
207.32817706385	-10.0169334276957\\
208.285640282755	-10.0169042011549\\
209.247525167004	-10.0168747036\\
210.213852136311	-10.0168449325138\\
211.18464170469	-10.0168148853556\\
212.159914480891	-10.016784559561\\
213.139691168835	-10.0167539525416\\
214.123992568061	-10.016723061685\\
215.112839574156	-10.0166918843546\\
216.10625317921	-10.0166604178888\\
217.104254472254	-10.0166286596016\\
218.106864639713	-10.0165966067818\\
219.114104965849	-10.0165642566928\\
220.12599683322	-10.0165316065726\\
221.142561723131	-10.0164986536336\\
222.163821216089	-10.016465395062\\
223.189796992262	-10.0164318280176\\
224.220510831939	-10.0163979496342\\
225.255984615994	-10.0163637570184\\
226.296240326348	-10.01632924725\\
227.341300046436	-10.0162944173816\\
228.391185961678	-10.0162592644383\\
229.44592035995	-10.0162237854173\\
230.505525632052	-10.0161879772879\\
231.570024272191	-10.016151836991\\
232.639438878451	-10.0161153614393\\
233.713792153278	-10.0160785475162\\
234.793106903962	-10.0160413920763\\
235.877406043117	-10.0160038919448\\
236.966712589169	-10.0159660439173\\
238.061049666849	-10.0159278447594\\
239.160440507677	-10.0158892912065\\
240.264908450462	-10.0158503799636\\
241.374476941791	-10.0158111077048\\
242.48916953653	-10.0157714710734\\
243.609009898327	-10.015731466681\\
244.734021800108	-10.0156910911078\\
245.864229124585	-10.015650340902\\
246.999655864764	-10.0156092125796\\
248.140326124455	-10.0155677026238\\
249.286264118777	-10.0155258074854\\
250.437494174681	-10.0154835235816\\
251.594040731461	-10.0154408472963\\
252.755928341275	-10.0153977749796\\
253.923181669665	-10.0153543029476\\
255.09582549608	-10.0153104274818\\
256.273884714404	-10.0152661448289\\
257.457384333485	-10.0152214512007\\
258.646349477661	-10.0151763427734\\
259.840805387301	-10.0151308156876\\
261.040777419332	-10.0150848660478\\
262.246291047787	-10.015038489922\\
263.457371864337	-10.0149916833413\\
264.674045578839	-10.0149444422999\\
265.896338019882	-10.0148967627544\\
267.124275135332	-10.0148486406237\\
268.357882992886	-10.0148000717884\\
269.597187780627	-10.0147510520906\\
270.842215807572	-10.0147015773334\\
272.092993504239	-10.0146516432806\\
273.349547423206	-10.0146012456567\\
274.611904239671	-10.0145503801457\\
275.880090752022	-10.0144990423915\\
277.154133882405	-10.0144472279972\\
278.434060677294	-10.0143949325244\\
279.719898308069	-10.0143421514936\\
281.011674071587	-10.0142888803831\\
282.309415390768	-10.0142351146289\\
283.613149815172	-10.0141808496241\\
284.922905021585	-10.0141260807189\\
286.238708814609	-10.0140708032197\\
287.560589127251	-10.014015012389\\
288.888574021513	-10.013958703445\\
290.222691688993	-10.0139018715609\\
291.562970451479	-10.0138445118649\\
292.909438761552	-10.0137866194392\\
294.262125203191	-10.0137281893201\\
295.621058492378	-10.0136692164973\\
296.98626747771	-10.0136096959136\\
298.357781141006	-10.0135496224643\\
299.735628597932	-10.0134889909968\\
301.119839098607	-10.01342779631\\
302.510442028234	-10.0133660331544\\
303.907466907719	-10.0133036962309\\
305.310943394298	-10.0132407801908\\
306.720901282168	-10.0131772796352\\
308.137370503119	-10.0131131891144\\
309.560381127168	-10.0130485031277\\
310.989963363199	-10.0129832161226\\
312.426147559604	-10.0129173224946\\
313.868964204927	-10.0128508165865\\
315.318443928511	-10.0127836926879\\
316.774617501149	-10.0127159450349\\
318.237515835737	-10.0126475678092\\
319.707169987928	-10.0125785551381\\
321.183611156796	-10.0125089010935\\
322.666870685493	-10.0124385996917\\
324.156980061919	-10.0123676448926\\
325.653970919389	-10.0122960305996\\
327.1578750373	-10.0122237506585\\
328.668724341813	-10.0121507988572\\
330.186550906528	-10.0120771689254\\
331.711386953162	-10.0120028545337\\
333.243264852235	-10.011927849293\\
334.78221712376	-10.0118521467543\\
336.328276437929	-10.0117757404077\\
337.881475615808	-10.0116986236823\\
339.441847630035	-10.011620789945\\
341.009425605519	-10.0115422325006\\
342.584242820144	-10.0114629445905\\
344.166332705473	-10.0113829193926\\
345.75572884746	-10.0113021500206\\
347.352464987164	-10.0112206295233\\
348.956575021463	-10.0111383508839\\
350.568093003772	-10.0110553070195\\
352.187053144771	-10.0109714907804\\
353.813489813129	-10.0108868949496\\
355.447437536229	-10.0108015122421\\
357.088931000911	-10.010715335304\\
358.738005054198	-10.0106283567121\\
360.39469470404	-10.0105405689732\\
362.059035120061	-10.0104519645234\\
363.731061634299	-10.0103625357275\\
365.410809741959	-10.010272274878\\
367.09831510217	-10.0101811741949\\
368.793613538734	-10.0100892258245\\
370.496741040893	-10.0099964218392\\
372.207733764093	-10.0099027542363\\
373.926628030746	-10.0098082149375\\
375.653460331008	-10.0097127957883\\
377.388267323548	-10.0096164885571\\
379.13108583633	-10.0095192849344\\
380.881952867393	-10.0094211765322\\
382.640905585636	-10.0093221548833\\
384.407981331609	-10.0092222114402\\
386.183217618305	-10.0091213375746\\
387.966652131954	-10.0090195245768\\
389.758322732826	-10.0089167636543\\
391.558267456034	-10.0088130459317\\
393.366524512341	-10.0087083624494\\
395.183132288971	-10.0086027041631\\
397.008129350424	-10.0084960619427\\
398.841554439296	-10.0083884265716\\
400.683446477099	-10.008279788746\\
402.533844565089	-10.0081701390739\\
404.392787985098	-10.0080594680743\\
406.260316200361	-10.0079477661763\\
408.136468856361	-10.0078350237181\\
410.021285781671	-10.0077212309465\\
411.914806988789	-10.0076063780156\\
413.817072675003	-10.0074904549862\\
415.72812322323	-10.0073734518246\\
417.647999202884	-10.0072553584021\\
419.576741370729	-10.0071361644936\\
421.514390671752	-10.0070158597771\\
423.460988240025	-10.0068944338323\\
425.416575399583	-10.0067718761402\\
427.381193665299	-10.0066481760815\\
429.354884743766	-10.0065233229362\\
431.337690534184	-10.0063973058823\\
433.329653129246	-10.0062701139949\\
435.330814816033	-10.0061417362451\\
437.341218076915	-10.0060121614993\\
439.360905590448	-10.0058813785175\\
441.389920232282	-10.0057493759532\\
443.428305076071	-10.0056161423516\\
445.476103394389	-10.0054816661488\\
447.533358659646	-10.0053459356707\\
449.600114545014	-10.0052089391322\\
451.67641492535	-10.0050706646356\\
453.762303878129	-10.00493110017\\
455.857825684385	-10.0047902336098\\
457.963024829641	-10.0046480527137\\
460.077946004862	-10.0045045451239\\
462.202634107401	-10.0043596983644\\
464.33713424195	-10.0042134998401\\
466.481491721497	-10.0040659368357\\
468.635752068297	-10.0039169965146\\
470.799961014824	-10.0037666659173\\
472.974164504755	-10.0036149319606\\
475.158408693936	-10.0034617814362\\
477.352739951367	-10.0033072010095\\
479.557204860185	-10.0031511772185\\
481.771850218653	-10.0029936964721\\
483.996723041153	-10.0028347450495\\
486.231870559183	-10.0026743090982\\
488.477340222363	-10.0025123746334\\
490.73317969944	-10.0023489275361\\
492.999436879299	-10.002183953552\\
495.276159871982	-10.0020174382905\\
497.563397009708	-10.0018493672226\\
499.861196847899	-10.0016797256804\\
502.169608166212	-10.001508498855\\
504.48867996957	-10.0013356717955\\
506.818461489212	-10.0011612294075\\
509.159002183727	-10.0009851564517\\
511.51035174011	-10.0008074375422\\
513.872560074817	-10.0006280571456\\
516.245677334823	-10.0004469995791\\
518.629753898685	-10.0002642490089\\
521.024840377617	-10.0000797894494\\
523.430987616559	-9.99989360476065\\
525.848246695256	-9.99970567864782\\
528.27666892935	-9.99951599465902\\
530.716305871459	-9.99932453618395\\
533.167209312278	-9.99913128645232\\
535.629431281677	-9.99893622853221\\
538.103024049808	-9.9987393453285\\
540.588040128206	-9.99854061958119\\
543.084532270915	-9.99834003386382\\
545.592553475602	-9.99813757058174\\
548.11215698468	-9.99793321197044\\
550.643396286443	-9.99772694009389\\
553.186325116201	-9.99751873684276\\
555.740997457415	-9.99730858393274\\
558.307467542852	-9.99709646290274\\
560.885789855729	-9.99688235511313\\
563.476019130871	-9.99666624174396\\
566.078210355879	-9.99644810379312\\
568.692418772287	-9.99622792207455\\
571.318699876742	-9.99600567721632\\
573.957109422183	-9.99578134965883\\
576.607703419019	-9.99555491965285\\
579.27053813632	-9.99532636725768\\
581.945670103016	-9.99509567233917\\
584.633156109091	-9.99486281456777\\
587.333053206791	-9.99462777341654\\
590.045418711838	-9.9943905281592\\
592.77031020464	-9.9941510578681\\
595.507785531519	-9.99390934141212\\
598.25790280594	-9.99366535745468\\
601.020720409738	-9.99341908445162\\
603.796296994364	-9.99317050064912\\
606.584691482126	-9.99291958408152\\
609.385963067442	-9.99266631256923\\
612.200171218097	-9.9924106637165\\
615.027375676503	-9.99215261490924\\
617.867636460969	-9.99189214331281\\
620.721013866976	-9.99162922586975\\
623.587568468455	-9.99136383929752\\
626.467361119072	-9.99109596008617\\
629.360452953524	-9.99082556449608\\
632.266905388836	-9.99055262855554\\
635.186780125658	-9.99027712805842\\
638.120139149584	-9.98999903856176\\
641.067044732464	-9.98971833538334\\
644.027559433723	-9.98943499359921\\
647.001746101695	-9.98914898804122\\
649.989667874954	-9.98886029329454\\
652.991388183652	-9.98856888369507\\
656.00697075087	-9.98827473332693\\
659.03647959397	-9.98797781601979\\
662.07997902595	-9.98767810534634\\
665.137533656814	-9.98737557461959\\
668.209208394941	-9.98707019689016\\
671.295068448464	-9.98676194494364\\
674.395179326655	-9.9864507912978\\
677.509606841313	-9.98613670819981\\
680.638417108163	-9.98581966762348\\
683.78167654826	-9.98549964126641\\
686.9394518894	-9.98517660054707\\
690.11181016753	-9.98485051660201\\
693.298818728181	-9.98452136028282\\
696.500545227891	-9.98418910215323\\
699.717057635642	-9.9838537124861\\
702.948424234306	-9.98351516126039\\
706.19471362209	-9.98317341815806\\
709.455994713995	-9.98282845256105\\
712.732336743282	-9.98248023354804\\
716.023809262934	-9.98212872989138\\
719.330482147139	-9.9817739100538\\
722.652425592773	-9.98141574218524\\
725.989710120886	-9.9810541941195\\
729.3424065782	-9.98068923337097\\
732.71058613862	-9.98032082713127\\
736.094320304735	-9.97994894226584\\
739.493680909343	-9.97957354531052\\
742.908740116972	-9.97919460246805\\
746.339570425412	-9.97881207960461\\
749.786244667258	-9.97842594224623\\
753.248836011454	-9.97803615557519\\
756.727417964843	-9.97764268442642\\
760.222064373731	-9.97724549328381\\
763.732849425456	-9.97684454627647\\
767.259847649959	-9.976439807175\\
770.803133921369	-9.97603123938765\\
774.362783459591	-9.9756188059565\\
777.938871831903	-9.97520246955359\\
781.531474954562	-9.97478219247689\\
785.140669094413	-9.97435793664641\\
788.76653087051	-9.97392966360012\\
792.40913725574	-9.97349733448989\\
796.068565578463	-9.97306091007736\\
799.744893524145	-9.97262035072974\\
803.438199137013	-9.97217561641563\\
807.148560821713	-9.97172666670073\\
810.876057344967	-9.97127346074348\\
814.620767837254	-9.97081595729077\\
818.382771794484	-9.97035411467342\\
822.162149079689	-9.96988789080177\\
825.958979924714	-9.96941724316114\\
829.773344931927	-9.9689421288072\\
833.605325075921	-9.96846250436137\\
837.455001705243	-9.96797832600616\\
841.322456544116	-9.96748954948036\\
845.207771694168	-9.96699613007423\\
849.111029636188	-9.96649802262472\\
853.032313231867	-9.96599518151046\\
856.971705725559	-9.96548756064683\\
860.92929074605	-9.9649751134809\\
864.905152308334	-9.96445779298637\\
868.899374815392	-9.96393555165835\\
872.91204305999	-9.96340834150818\\
876.943242226474	-9.96287611405815\\
880.993057892579	-9.9623388203361\\
885.061576031251	-9.96179641087012\\
889.148883012464	-9.96124883568291\\
893.255065605058	-9.96069604428639\\
897.380210978585	-9.96013798567601\\
901.52440670515	-9.95957460832507\\
905.687740761275	-9.95900586017903\\
909.870301529772	-9.95843168864961\\
914.07217780161	-9.95785204060897\\
918.293458777803	-9.95726686238374\\
922.534234071309	-9.95667609974895\\
926.794593708924	-9.95607969792197\\
931.074628133199	-9.9554776015563\\
935.374428204358	-9.95486975473533\\
939.694085202226	-9.95425610096602\\
944.033690828168	-9.95363658317246\\
948.39333720704	-9.95301114368939\\
952.773116889131	-9.95237972425566\\
957.173122852146	-9.95174226600754\\
961.593448503166	-9.95109870947203\\
966.034187680636	-9.95044899456002\\
970.495434656357	-9.94979306055938\\
974.977284137491	-9.94913084612801\\
979.479831268559	-9.94846228928676\\
984.003171633478	-9.94778732741226\\
988.547401257578	-9.94710589722968\\
993.112616609642	-9.9464179348054\\
997.698914603958	-9.94572337553955\\
1002.30639260238	-9.94502215415857\\
1006.93514841637	-9.94431420470749\\
1011.58528030912	-9.9435994605423\\
1016.2568869976	-9.94287785432208\\
1020.95006765465	-9.94214931800118\\
1025.66492191112	-9.94141378282113\\
1030.40154985798	-9.9406711793026\\
1035.16005204838	-9.93992143723714\\
1039.94052949988	-9.93916448567891\\
1044.74308369654	-9.93840025293628\\
1049.56781659108	-9.93762866656324\\
1054.41483060703	-9.93684965335084\\
1059.28422864096	-9.93606313931844\\
1064.17611406461	-9.93526904970482\\
1069.09059072708	-9.93446730895928\\
1074.02776295708	-9.93365784073253\\
1078.98773556513	-9.93284056786748\\
1083.97061384575	-9.93201541239\\
1088.97650357974	-9.93118229549942\\
1094.00551103639	-9.93034113755904\\
1099.05774297578	-9.92949185808643\\
1104.13330665098	-9.92863437574366\\
1109.2323098104	-9.92776860832733\\
1114.35486070002	-9.92689447275864\\
1119.50106806574	-9.92601188507308\\
1124.67104115564	-9.92512076041022\\
1129.86488972231	-9.92422101300323\\
1135.0827240252	-9.92331255616831\\
1140.32465483296	-9.92239530229402\\
1145.59079342577	-9.92146916283037\\
1150.8812515977	-9.9205340482779\\
1156.19614165913	-9.91958986817652\\
1161.53557643908	-9.91863653109419\\
1166.89966928762	-9.9176739446156\\
1172.28853407829	-9.91670201533051\\
1177.70228521052	-9.91572064882209\\
1183.14103761204	-9.91472974965502\\
1188.60490674132	-9.91372922136346\\
1194.09400859005	-9.91271896643889\\
1199.60845968555	-9.91169888631775\\
1205.14837709331	-9.91066888136895\\
1210.71387841942	-9.90962885088117\\
1216.30508181309	-9.90857869305007\\
1221.92210596917	-9.90751830496526\\
1227.56507013062	-9.90644758259713\\
1233.23409409112	-9.90536642078351\\
1238.92929819754	-9.90427471321614\\
1244.65080335254	-9.90317235242698\\
1250.3987310171	-9.90205922977433\\
1256.17320321316	-9.90093523542874\\
1261.97434252612	-9.89980025835878\\
1267.80227210752	-9.89865418631664\\
1273.65711567764	-9.89749690582341\\
1279.53899752808	-9.89632830215433\\
1285.44804252445	-9.89514825932374\\
1291.38437610901	-9.89395666006987\\
1297.34812430331	-9.89275338583941\\
1303.33941371088	-9.89153831677189\\
1309.35837151994	-9.89031133168383\\
1315.40512550606	-9.88907230805271\\
1321.47980403488	-9.88782112200071\\
1327.58253606487	-9.88655764827821\\
1333.71345115004	-9.88528176024712\\
1339.87267944269	-9.88399332986393\\
1346.06035169616	-9.88269222766261\\
1352.27659926764	-9.88137832273715\\
1358.52155412096	-9.88005148272403\\
1364.79534882933	-9.87871157378435\\
1371.09811657822	-9.87735846058572\\
1377.42999116818	-9.87599200628397\\
1383.79110701763	-9.87461207250451\\
1390.18159916578	-9.87321851932359\\
1396.60160327544	-9.87181120524916\\
1403.05125563594	-9.87038998720154\\
1409.53069316601	-9.86895472049386\\
1416.04005341668	-9.86750525881214\\
1422.5794745742	-9.86604145419525\\
1429.14909546299	-9.86456315701438\\
1435.74905554856	-9.86307021595249\\
1442.3794949405	-9.86156247798329\\
1449.04055439544	-9.86003978834999\\
1455.73237532004	-9.85850199054383\\
1462.45509977397	-9.85694892628218\\
1469.20887047298	-9.85538043548651\\
1475.99383079187	-9.8537963562599\\
1482.81012476756	-9.8521965248644\\
1489.65789710218	-9.85058077569793\\
1496.53729316606	-9.84894894127095\\
1503.44845900091	-9.8473008521829\\
1510.39154132284	-9.84563633709808\\
1517.36668752555	-9.84395522272143\\
1524.37404568338	-9.8422573337739\\
1531.41376455451	-9.84054249296746\\
1538.4859935841	-9.83881052097979\\
1545.59088290748	-9.83706123642864\\
1552.72858335329	-9.83529445584581\\
1559.89924644673	-9.83350999365081\\
1567.10302441275	-9.83170766212417\\
1574.34007017931	-9.8298872713803\\
1581.61053738059	-9.82804862934012\\
1588.91458036027	-9.82619154170316\\
1596.25235417481	-9.82431581191945\\
1603.62401459674	-9.82242124116084\\
1611.02971811795	-9.82050762829213\\
1618.46962195303	-9.81857476984162\\
1625.94388404263	-9.81662245997139\\
1633.45266305675	-9.81465049044712\\
1640.99611839816	-9.81265865060752\\
1648.57441020578	-9.81064672733329\\
1656.18769935804	-9.80861450501575\\
1663.83614747635	-9.80656176552498\\
1671.51991692849	-9.80448828817752\\
1679.23917083208	-9.80239384970367\\
1686.99407305804	-9.80027822421433\\
1694.78478823403	-9.79814118316737\\
1702.61148174801	-9.7959824953336\\
1710.47431975172	-9.79380192676216\\
1718.37346916419	-9.79159924074561\\
1726.30909767531	-9.7893741977844\\
1734.28137374936	-9.78712655555093\\
1742.29046662864	-9.78485606885311\\
1750.33654633699	-9.78256248959741\\
1758.41978368348	-9.78024556675146\\
1766.54035026596	-9.77790504630608\\
1774.69841847474	-9.77554067123681\\
1782.89416149627	-9.77315218146502\\
1791.12775331676	-9.77073931381831\\
1799.39936872594	-9.76830180199056\\
1807.70918332073	-9.76583937650133\\
1816.05737350894	-9.76335176465477\\
1824.44411651309	-9.76083869049794\\
1832.86959037413	-9.75829987477859\\
1841.33397395519	-9.75573503490242\\
1849.83744694544	-9.75314388488965\\
1858.38018986386	-9.75052613533116\\
1866.9623840631	-9.74788149334391\\
1875.58421173328	-9.74520966252589\\
1884.24585590593	-9.74251034291035\\
1892.94750045783	-9.73978323091952\\
1901.68933011491	-9.7370280193177\\
1910.47153045619	-9.73424439716363\\
1919.29428791772	-9.73143204976244\\
1928.15778979652	-9.72859065861672\\
1937.06222425457	-9.72571990137714\\
1946.00778032282	-9.72281945179235\\
1954.99464790516	-9.71988897965821\\
1964.02301778248	-9.71692815076638\\
1973.09308161673	-9.71393662685226\\
1982.20503195496	-9.71091406554219\\
1991.35906223344	-9.70786012030007\\
2000.55536678172	-9.7047744403732\\
2009.79414082682	-9.70165667073744\\
2019.07558049731	-9.69850645204172\\
2028.39988282751	-9.6953234205518\\
2037.76724576168	-9.69210720809328\\
2047.17786815819	-9.688857441994\\
2056.63194979376	-9.68557374502553\\
2066.12969136771	-9.68225573534417\\
2075.6712945062	-9.67890302643091\\
2085.25696176653	-9.67551522703096\\
2094.88689664142	-9.67209194109225\\
2104.56130356336	-9.66863276770333\\
2114.2803879089	-9.66513730103045\\
2124.04435600306	-9.6616051302539\\
2133.8534151237	-9.65803583950352\\
2143.70777350589	-9.65442900779347\\
2153.60764034637	-9.65078420895619\\
2163.55322580795	-9.64710101157558\\
2173.54474102401	-9.64337897891936\\
2183.58239810298	-9.63961766887065\\
2193.66641013278	-9.63581663385874\\
2203.79699118545	-9.63197542078904\\
2213.97435632161	-9.62809357097218\\
2224.19872159503	-9.62417062005234\\
2234.47030405729	-9.6202060979348\\
2244.78932176229	-9.61619952871248\\
2255.15599377096	-9.6121504305919\\
2265.57054015585	-9.6080583158181\\
2276.03318200585	-9.60392269059883\\
2286.54414143084	-9.59974305502789\\
2297.10364156645	-9.59551890300763\\
2307.71190657875	-9.5912497221706\\
2318.36916166905	-9.58693499380036\\
2329.07563307865	-9.58257419275149\\
2339.83154809368	-9.57816678736872\\
2350.63713504986	-9.57371223940523\\
2361.49262333744	-9.56921000394014\\
2372.39824340596	-9.5646595292952\\
2383.35422676926	-9.56006025695051\\
2394.36080601029	-9.55541162145962\\
2405.4182147861	-9.55071305036364\\
2416.52668783283	-9.5459639641046\\
2427.6864609706	-9.54116377593807\\
2438.8977711086	-9.53631189184484\\
2450.16085625011	-9.53140771044191\\
2461.4759554975	-9.52645062289268\\
2472.84330905736	-9.52144001281634\\
2484.26315824557	-9.5163752561965\\
2495.73574549243	-9.5112557212891\\
2507.26131434783	-9.50608076852956\\
2518.84010948637	-9.50084975043918\\
2530.4723767126	-9.4955620115309\\
2542.15836296621	-9.49021688821431\\
2553.8983163273	-9.4848137087\\
2565.69248602162	-9.47935179290328\\
2577.54112242586	-9.47383045234725\\
2589.444477073	-9.46824899006527\\
2601.4028026576	-9.46260670050283\\
2613.41635304121	-9.45690286941889\\
2625.48538325773	-9.45113677378668\\
2637.61014951883	-9.44530768169397\\
2649.7909092194	-9.43941485224296\\
2662.02792094301	-9.43345753544964\\
2674.32144446738	-9.42743497214279\\
2686.67174076992	-9.4213463938627\\
2699.07907203326	-9.41519102275943\\
2711.54370165082	-9.40896807149093\\
2724.0658942324	-9.40267674312088\\
2736.64591560979	-9.39631623101636\\
2749.28403284242	-9.38988571874542\\
2761.98051422303	-9.38338437997454\\
2774.73562928336	-9.37681137836617\\
2787.54964879989	-9.37016586747623\\
2800.42284479955	-9.36344699065177\\
2813.35549056553	-9.35665388092889\\
2826.34786064309	-9.3497856609308\\
2839.40023084533	-9.34284144276634\\
2852.51287825912	-9.33582032792886\\
2865.68608125093	-9.32872140719564\\
2878.92011947275	-9.32154376052787\\
2892.21527386804	-9.31428645697141\\
2905.57182667769	-9.30694855455831\\
2918.99006144599	-9.2995291002092\\
2932.4702630267	-9.29202712963675\\
2946.01271758903	-9.28444166725025\\
2959.61771262377	-9.27677172606147\\
2973.28553694936	-9.26901630759188\\
2987.01648071805	-9.26117440178141\\
3000.81083542203	-9.25324498689889\\
3014.66889389963	-9.24522702945437\\
3028.59095034155	-9.23711948411325\\
3042.57730029708	-9.22892129361279\\
3056.6282406804	-9.22063138868077\\
3070.74406977686	-9.21224868795676\\
3084.92508724934	-9.20377209791601\\
3099.17159414457	-9.19520051279634\\
3113.48389289956	-9.18653281452808\\
3127.86228734801	-9.17776787266739\\
3142.30708272674	-9.16890454433309\\
3156.8185856822	-9.15994167414743\\
3171.39710427697	-9.15087809418081\\
3186.04294799627	-9.14171262390094\\
3200.75642775457	-9.13244407012658\\
3215.53785590219	-9.12307122698622\\
3230.38754623189	-9.11359287588203\\
3245.30581398558	-9.10400778545927\\
3260.29297586097	-9.09431471158178\\
3275.34935001834	-9.08451239731353\\
3290.47525608724	-9.07459957290695\\
3305.67101517332	-9.0645749557982\\
3320.93694986511	-9.05443725060994\\
3336.27338424092	-9.04418514916191\\
3351.68064387565	-9.03381733048981\\
3367.15905584778	-9.02333246087308\\
3382.70894874623	-9.01272919387177\\
3398.3306526774	-9.00200617037338\\
3414.02449927217	-8.99116201864987\\
3429.7908216929	-8.98019535442574\\
3445.62995464054	-8.96910478095739\\
3461.54223436172	-8.95788888912474\\
3477.5279986559	-8.94654625753554\\
3493.58758688252	-8.93507545264309\\
3509.72133996824	-8.92347502887809\\
3525.92960041413	-8.91174352879541\\
3542.21271230298	-8.89987948323641\\
3558.57102130658	-8.88788141150777\\
3575.00487469309	-8.87574782157765\\
3591.51462133436	-8.86347721028993\\
3608.1006117134	-8.85106806359762\\
3624.76319793175	-8.83851885681635\\
3641.50273371703	-8.82582805489888\\
3658.31957443038	-8.81299411273179\\
3675.21407707406	-8.80001547545543\\
3692.18660029898	-8.78689057880818\\
3709.23750441235	-8.77361784949637\\
3726.36715138533	-8.76019570559103\\
3743.57590486067	-8.74662255695287\\
3760.86413016048	-8.73289680568668\\
3778.23219429398	-8.71901684662681\\
3795.68046596523	-8.70498106785513\\
3813.20931558105	-8.69078785125297\\
3830.81911525882	-8.6764355730888\\
3848.51023883439	-8.66192260464327\\
3866.28306187004	-8.64724731287345\\
3884.13796166242	-8.63240806111807\\
3902.07531725059	-8.61740320984574\\
3920.09550942403	-8.60223111744823\\
3938.19892073078	-8.58689014108077\\
3956.38593548549	-8.57137863755183\\
3974.65693977764	-8.55569496426436\\
3993.0123214797	-8.53983748021122\\
4011.45247025537	-8.52380454702705\\
4029.97777756789	-8.50759453009941\\
4048.58863668828	-8.49120579974164\\
4067.28544270374	-8.47463673243066\\
4086.06859252603	-8.45788571211227\\
4104.93848489988	-8.44095113157734\\
4123.89552041149	-8.42383139391197\\
4142.94010149697	-8.40652491402494\\
4162.07263245094	-8.38903012025604\\
4181.29351943512	-8.37134545606885\\
4200.60317048688	-8.35346938183173\\
4220.00199552799	-8.335400376691\\
4239.49040637325	-8.31713694054035\\
4259.06881673929	-8.29867759609079\\
4278.73764225331	-8.2800208910455\\
4298.49730046193	-8.2611654003842\\
4318.34821084004	-8.24210972876184\\
4338.2907947997	-8.22285251302649\\
4358.32547569911	-8.20339242486162\\
4378.45267885158	-8.18372817355817\\
4398.67283153454	-8.16385850892169\\
4418.98636299868	-8.14378222432068\\
4439.39370447695	-8.12349815988167\\
4459.89528919383	-8.1030052058374\\
4480.49155237446	-8.08230230603442\\
4501.18293125388	-8.06138846160671\\
4521.96986508636	-8.04026273482192\\
4542.85279515466	-8.01892425310756\\
4563.83216477945	-7.99737221326409\\
4584.90841932869	-7.97560588587241\\
4606.0820062271	-7.95362461990358\\
4627.35337496566	-7.93142784753841\\
4648.72297711113	-7.90901508920527\\
4670.19126631568	-7.88638595884416\\
4691.75869832646	-7.86354016940576\\
4713.42573099534	-7.84047753859424\\
4735.19282428857	-7.81719799486232\\
4757.06044029659	-7.79370158366827\\
4779.02904324381	-7.76998847400354\\
4801.09909949849	-7.74605896520078\\
4823.27107758263	-7.72191349403162\\
4845.54544818189	-7.6975526421039\\
4867.92268415562	-7.67297714356805\\
4890.4032605469	-7.64818789314237\\
4912.98765459258	-7.62318595446715\\
4935.67634573345	-7.59797256879723\\
4958.46981562442	-7.57254916404285\\
4981.36854814472	-7.54691736416848\\
5004.37302940819	-7.52107899895888\\
5027.48374777359	-7.49503611416193\\
5050.70119385497	-7.46879098201703\\
5074.02586053209	-7.44234611217781\\
5097.4582429609	-7.41570426303717\\
5120.998838584	-7.38886845346264\\
5144.64814714125	-7.36184197494877\\
5168.40667068035	-7.33462840419311\\
5192.27491356753	-7.30723161610114\\
5216.2533824982	-7.27965579722474\\
5240.34258650778	-7.25190545963753\\
5264.54303698246	-7.22398545524908\\
5288.85524767004	-7.19590099055901\\
5313.27973469088	-7.16765764184958\\
5337.81701654885	-7.13926137081433\\
5362.46761414231	-7.11071854061796\\
5387.23205077517	-7.0820359323802\\
5412.11085216805	-7.05322076207443\\
5437.10454646936	-7.0242806978286\\
5462.21366426659	-6.99522387761313\\
5487.43873859752	-6.9660589272974\\
5512.78030496154	-6.9367949790523\\
5538.23890133107	-6.90744169007286\\
5563.81506816292	-6.87800926159043\\
5589.50934840979	-6.84850845813923\\
5615.32228753178	-6.81895062703725\\
5641.254433508	-6.78934771803573\\
5667.30633684818	-6.75971230308626\\
5693.47855060436	-6.7300575961674\\
5719.77163038263	-6.70039747310721\\
5746.18613435493	-6.67074649133012\\
5772.72262327089	-6.64111990944968\\
5799.38166046975	-6.61153370662064\\
5826.16381189231	-6.58200460155564\\
5853.06964609293	-6.55255007110329\\
5880.09973425162	-6.52318836827509\\
5907.25465018618	-6.49393853959993\\
5934.53497036433	-6.46482044167477\\
5961.94127391598	-6.43585475677056\\
5989.47414264556	-6.4070630073428\\
6017.13416104428	-6.37846756928552\\
6044.92191630263	-6.35009168375791\\
6072.83799832281	-6.32195946740278\\
6100.88299973121	-6.29409592076634\\
6129.05751589107	-6.26652693471926\\
6157.36214491506	-6.23927929467067\\
6185.79748767801	-6.21238068235798\\
6214.36414782964	-6.18585967498866\\
6243.06273180741	-6.15974574150336\\
6271.89384884933	-6.13406923572546\\
6300.85811100697	-6.1088613861579\\
6329.95613315842	-6.084154282187\\
6359.18853302131	-6.05998085645306\\
6388.55593116599	-6.03637486314987\\
6418.05895102864	-6.01337085202059\\
6447.69821892457	-5.99100413782503\\
6477.47436406142	-5.96931076506436\\
6507.38801855264	-5.94832746776326\\
6537.43981743082	-5.9280916241274\\
6567.63039866118	-5.90864120591525\\
6597.96040315515	-5.89001472238882\\
6628.43047478397	-5.87225115873697\\
6659.0412603923	-5.85538990889854\\
6689.79340981204	-5.8394707027502\\
6720.68757587607	-5.82453352766553\\
6751.7244144321	-5.81061854449771\\
6782.90458435663	-5.79776599808758\\
6814.22874756894	-5.78601612245206\\
6845.69756904508	-5.77540904086409\\
6877.31171683206	-5.76598466109386\\
6909.07186206199	-5.75778256614255\\
6940.97867896634	-5.75084190086147\\
6973.03284489025	-5.74520125491281\\
7005.23504030693	-5.74089854259072\\
7037.58594883204	-5.7379708800827\\
7070.0862572383	-5.73645446081045\\
7102.73665547	-5.73638442954502\\
7135.53783665763	-5.73779475604253\\
7168.49049713269	-5.74071810899197\\
7201.59533644237	-5.74518573110618\\
7234.85305736446	-5.75122731621764\\
7268.26436592224	-5.75887088926388\\
7301.82997139948	-5.76814269005932\\
7335.55058635552	-5.77906706175368\\
7369.42692664033	-5.79166634486766\\
7403.45971140979	-5.80596077777737\\
7437.64966314091	-5.82196840448685\\
7471.99750764715	-5.83970499048526\\
7506.50397409388	-5.85918394743067\\
7541.16979501382	-5.88041626733689\\
7575.99570632259	-5.90341046686453\\
7610.98244733438	-5.92817254223282\\
7646.13076077758	-5.95470593517576\\
7681.44139281058	-5.98301151026702\\
7716.91509303763	-6.01308754383374\\
7752.55261452471	-6.04492972457076\\
7788.35471381553	-6.07853116585743\\
7824.32215094763	-6.11388242966926\\
7860.45568946846	-6.15097156186807\\
7896.7560964516	-6.18978413855005\\
7933.22414251308	-6.23030332303125\\
7969.86060182773	-6.27250993295682\\
8006.66625214554	-6.31638251693557\\
8043.6418748083	-6.36189744002502\\
8080.78825476606	-6.40902897732659\\
8118.10618059391	-6.45774941489499\\
8155.5964445086	-6.50802915712279\\
8193.25984238547	-6.5598368397284\\
8231.09717377527	-6.61313944745527\\
8269.10924192116	-6.66790243558115\\
8307.29685377578	-6.7240898543381\\
8345.66082001834	-6.78166447535589\\
8384.20195507185	-6.84058791926398\\
8422.92107712042	-6.90082078361769\\
8461.81900812665	-6.96232277035285\\
8500.89657384898	-7.02505281201867\\
8540.15460385935	-7.08896919608976\\
8579.59393156072	-7.15402968671333\\
8619.21539420481	-7.22019164330709\\
8659.01983290983	-7.28741213548457\\
8699.0080926784	-7.35564805384695\\
8739.18102241541	-7.42485621624368\\
8779.5394749461	-7.49499346916674\\
8820.08430703417	-7.56601678400408\\
8860.81637939989	-7.63788334793701\\
8901.73655673847	-7.7105506493231\\
8942.84570773836	-7.78397655745872\\
8984.14470509972	-7.85811939666589\\
9025.63442555289	-7.93293801469365\\
9067.31574987707	-8.00839184546663\\
9109.18956291901	-8.08444096625084\\
9151.25675361173	-8.16104614934097\\
9193.51821499347	-8.23816890840305\\
9235.97484422661	-8.31577153963156\\
9278.62754261669	-8.39381715790196\\
9321.47721563161	-8.47226972811801\\
9364.5247729208	-8.55109409196653\\
9407.77112833455	-8.6302559903045\\
9451.21719994339	-8.70972208141091\\
9494.86391005764	-8.78945995534069\\
9538.71218524688	-8.86943814462158\\
9582.76295635973	-8.9496261315346\\
9627.01715854358	-9.02999435221728\\
9671.47573126437	-9.11051419782581\\
9716.13961832666	-9.19115801298736\\
9761.00976789355	-9.27189909176719\\
9806.08713250686	-9.3527116713689\\
9851.37266910737	-9.43357092377737\\
9896.86733905512	-9.51445294554549\\
9942.57210814977	-9.5953347459165\\
9988.48794665118	-9.67619423346437\\
10034.6158293	-9.75701020142422\\
10080.9567353382	-9.83776231187534\\
10127.5116485301	-9.91843107892928\\
10174.2815571832	-9.99899785106545\\
10221.267454169	-10.0794447927468\\
10268.4703369442	-10.1597548654388\\
10315.891207572	-10.2399118081458\\
10363.531072743	-10.3199001175701\\
10411.3909437969	-10.3997050279885\\
10459.4718367439	-10.4793124909368\\
10507.7747722864	-10.5587091547808\\
10556.3007758401	-10.6378823442466\\
10605.0508775566	-10.7168200399772\\
10654.0261123446	-10.7955108581738\\
10703.2275198921	-10.8739440303745\\
10752.6561446888	-10.9521093834183\\
10802.3130360475	-11.0299973196351\\
10852.1992481272	-11.1075987972995\\
10902.315839955	-11.1849053113797\\
10952.6638754485	-11.2619088746095\\
11003.244423439	-11.3386019989072\\
11054.0585576935	-11.4149776771606\\
11105.1073569377	-11.4910293653969\\
11156.3919048792	-11.5667509653488\\
11207.9132902301	-11.6421368074291\\
11259.6726067303	-11.7171816341227\\
11311.6709531708	-11.7918805838005\\
11363.9094334169	-11.8662291749606\\
11416.3891564316	-11.9402232908989\\
11469.1112362992	-12.0138591648091\\
11522.0767922492	-12.0871333653117\\
11575.2869486794	-12.1600427824098\\
11628.7428351806	-12.2325846138685\\
11682.4455865599	-12.3047563520132\\
11736.3963428651	-12.3765557709415\\
11790.596249409	-12.4479809141437\\
11845.0464567934	-12.5190300825242\\
11899.7481209338	-12.5897018228169\\
11954.7024030838	-12.6599949163877\\
12009.9104698599	-12.7299083684149\\
12065.3734932659	-12.7994413974399\\
12121.0926507183	-12.8685934252794\\
12177.069125071	-12.9373640672908\\
12233.3041046402	-13.0057531229808\\
12289.7987832301	-13.0737605669494\\
12346.554360158	-13.1413865401595\\
12403.5720402798	-13.208631341523\\
12460.8530340153	-13.2754954197952\\
12518.3985573745	-13.3419793657675\\
12576.2098319827	-13.4080839047498\\
12634.2880851072	-13.4738098893348\\
12692.6345496825	-13.5391582924332\\
12751.2504643373	-13.604130200574\\
12810.1370734203	-13.6687268074592\\
12869.2956270265	-13.7329494077664\\
12928.7273810244	-13.7967993911897\\
12988.4335970818	-13.8602782367122\\
13048.4155426933	-13.9233875071016\\
13108.6744912069	-13.9861288436214\\
13169.2117218509	-14.0485039609515\\
13230.0285197614	-14.110514642309\\
13291.1261760091	-14.1721627347638\\
13352.5059876274	-14.2334501447421\\
13414.1692576393	-14.2943788337102\\
13476.1172950852	-14.3549508140336\\
13538.351415051	-14.4151681450038\\
13600.8729386957	-14.4750329290276\\
13663.6831932795	-14.5345473079739\\
13726.7835121922	-14.5937134596698\\
13790.1752349813	-14.6525335945433\\
13853.8597073802	-14.7110099524058\\
13917.8382813373	-14.7691447993692\\
13982.1123150444	-14.826940424894\\
14046.6831729655	-14.8843991389617\\
14111.552225866	-14.9415232693684\\
14176.7208508414	-14.9983151591342\\
14242.190431347	-15.054777164025\\
14307.9623572268	-15.1109116501814\\
14374.0380247435	-15.1667209918516\\
14440.4188366077	-15.2222075692237\\
14507.1062020079	-15.2773737663545\\
14574.1015366405	-15.3322219691903\\
14641.4062627396	-15.3867545636764\\
14709.0218091073	-15.4409739339529\\
14776.9496111443	-15.4948824606314\\
14845.1911108798	-15.548482519152\\
14913.7477570027	-15.601776478216\\
14982.6210048919	-15.6547666982911\\
15051.8123166476	-15.7074555301878\\
15121.323161122	-15.7598453137024\\
15191.1550139505	-15.8119383763263\\
15261.3093575835	-15.8637370320159\\
15331.7876813171	-15.9152435800241\\
15402.5914813253	-15.9664603037883\\
15473.7222606918	-16.0173894698741\\
15545.1815294413	-16.0680333269725\\
15616.9708045722	-16.1183941049471\\
15689.0916100885	-16.1684740139314\\
15761.5454770323	-16.2182752434721\\
15834.333943516	-16.2677999617184\\
15907.4585547554	-16.3170503146535\\
15980.9208631021	-16.3660284253687\\
16054.7224280766	-16.4147363933762\\
16128.8648164017	-16.4631762939612\\
16203.3496020352	-16.5113501775692\\
16278.1783662036	-16.5592600692295\\
16353.352697436	-16.6069079680116\\
16428.8741915971	-16.6542958465137\\
16504.7444519217	-16.7014256503822\\
16580.9650890484	-16.7482992978607\\
16657.537721054	-16.7949186793665\\
16734.4639734876	-16.8412856570949\\
16811.7454794054	-16.8874020646481\\
16889.3838794052	-16.9332697066899\\
16967.3808216612	-16.9788903586228\\
17045.7379619589	-17.0242657662881\\
17124.4569637308	-17.0693976456871\\
17203.539498091	-17.1142876827227\\
17282.9872438709	-17.1589375329601\\
17362.8018876552	-17.2033488214068\\
17442.9851238172	-17.2475231423091\\
17523.5386545551	-17.2914620589661\\
17604.464189928	-17.335167103559\\
17685.7634478922	-17.3786397769961\\
17767.4381543379	-17.4218815487716\\
17849.4900431252	-17.4648938568378\\
17931.9208561219	-17.5076781074907\\
18014.7323432394	-17.5502356752676\\
18097.9262624709	-17.5925679028553\\
18181.5043799277	-17.6346761010105\\
18265.4684698776	-17.6765615484892\\
18349.8203147818	-17.7182254919862\\
18434.5617053333	-17.7596691460835\\
18519.6944404947	-17.8008936932071\\
18605.2203275364	-17.8419002835917\\
18691.1411820748	-17.8826900352526\\
18777.4588281113	-17.9232640339648\\
18864.1750980704	-17.9636233332477\\
18951.2918328392	-18.0037689543568\\
19038.810881806	-18.0437018862798\\
19126.7341029	-18.0834230857379\\
19215.0633626303	-18.1229334771922\\
19303.8005361259	-18.1622339528537\\
19392.9475071751	-18.2013253726969\\
19482.506168266	-18.2402085644772\\
19572.4784206263	-18.278884323751\\
19662.8661742637	-18.3173534138988\\
19753.6713480067	-18.3556165661494\\
19844.8958695449	-18.3936744796079\\
19936.5416754703	-18.4315278212833\\
20028.6107113184	-18.4691772261191\\
20121.1049316092	-18.5066232970233\\
20214.026299889	-18.5438666049007\\
20307.3767887718	-18.580907688684\\
20401.1583799817	-18.6177470553662\\
20495.373064394	-18.6543851800322\\
20590.0228420787	-18.69082250589\\
20685.1097223421	-18.7270594443018\\
20780.6357237695	-18.7630963748132\\
20876.6028742684	-18.798933645182\\
20973.0132111115	-18.8345715714053\\
21069.8687809795	-18.8700104377451\\
21167.1716400053	-18.9052504967516\\
21264.923853817	-18.9402919692849\\
21363.127497582	-18.9751350445346\\
21461.7846560511	-19.0097798800358\\
21560.8974236026	-19.0442266016837\\
21660.467904287	-19.0784753037448\\
21760.4982118713	-19.1125260488644\\
21860.9904698845	-19.1463788680721\\
21961.9468116618	-19.1800337607824\\
22063.3693803907	-19.2134906947928\\
22165.260329156	-19.2467496062775\\
22267.6218209858	-19.2798103997771\\
22370.4560288971	-19.3126729481844\\
22473.7651359423	-19.3453370927256\\
22577.5513352553	-19.377802642937\\
22681.8168300982	-19.4100693766371\\
22786.5638339077	-19.4421370398939\\
22891.7945703429	-19.4740053469869\\
22997.5112733313	-19.5056739803642\\
23103.7161871177	-19.5371425905941\\
23210.4115663104	-19.5684107963115\\
23317.5996759301	-19.5994781841579\\
23425.2827914574	-19.6303443087162\\
23533.4631988815	-19.6610086924397\\
23642.1431947482	-19.6914708255741\\
23751.3250862094	-19.7217301660739\\
23861.0111910714	-19.7517861395118\\
23971.2038378446	-19.7816381389822\\
24081.9053657923	-19.811285524997\\
24193.1181249812	-19.8407276253751\\
24304.8444763306	-19.869963735124\\
24417.0867916629	-19.8989931163155\\
24529.8474537537	-19.9278149979518\\
24643.1288563827	-19.9564285758261\\
24756.933404384	-19.9848330123739\\
24871.2635136979	-20.0130274365166\\
24986.1216114214	-20.0410109434976\\
25101.5101358603	-20.0687825947088\\
25217.4315365806	-20.0963414175096\\
25333.8882744609	-20.1236864050365\\
25450.882821744	-20.1508165160039\\
25568.4176620901	-20.1777306744959\\
25686.4952906292	-20.2044277697483\\
25805.1182140139	-20.2309066559212\\
25924.2889504728	-20.2571661518623\\
26044.0100298642	-20.283205040859\\
26164.2839937291	-20.3090220703814\\
26285.113395346	-20.3346159518142\\
26406.5007997846	-20.3599853601776\\
26528.4487839603	-20.3851289338382\\
26650.959936689	-20.4100452742072\\
26774.0368587419	-20.4347329454286\\
26897.6821629011	-20.4591904740542\\
27021.8984740145	-20.4834163487079\\
27146.6884290521	-20.5074090197361\\
27272.0546771616	-20.5311668988471\\
27397.9998797246	-20.554688358736\\
27524.5267104134	-20.5779717326972\\
27651.6378552475	-20.6010153142234\\
27779.3360126509	-20.6238173565896\\
27907.623893509	-20.646376072424\\
28036.5042212264	-20.6686896332635\\
28165.9797317848	-20.6907561690943\\
28296.0531738006	-20.7125737678769\\
28426.7273085842	-20.7341404750557\\
28558.0049101974	-20.7554542930516\\
28689.8887655133	-20.7765131807387\\
28822.3816742749	-20.7973150529028\\
28955.4864491548	-20.8178577796835\\
29089.2059158146	-20.8381391859969\\
29223.5429129655	-20.8581570509406\\
29358.5002924277	-20.8779091071794\\
29494.0809191919	-20.8973930403105\\
29630.2876714792	-20.9166064882095\\
29767.1234408028	-20.9355470403548\\
29904.5911320291	-20.954212237131\\
30042.6936634399	-20.9725995691098\\
30181.4339667932	-20.9907064763082\\
30320.8149873869	-21.0085303474237\\
30460.8396841202	-21.0260685190444\\
30601.5110295567	-21.0433182748354\\
30742.8320099879	-21.0602768446983\\
30884.8056254962	-21.0769414039049\\
31027.4348900188	-21.0933090722038\\
31170.7228314113	-21.1093769128976\\
31314.6724915125	-21.1251419318925\\
31459.2869262085	-21.1406010767167\\
31604.5692054981	-21.1557512355083\\
31750.5224135574	-21.1705892359706\\
};
\addlegendentry{$\Delta\text{(z)}$};

\addplot [color=black!50!mycolor1,dashed,line width=1.5pt]
  table[row sep=crcr]{10	46.0288004247686\\
10.0461810465159	45.9887804170902\\
10.0925753619375	45.9487604094335\\
10.139183931163	45.9087404017984\\
10.1860077436388	45.8687203941853\\
10.2330477933808	45.8287003865943\\
10.2803050789954	45.7886803790257\\
10.3277806037004	45.7486603714797\\
10.375475375347	45.7086403639564\\
10.4233904064403	45.668620356456\\
10.4715267141616	45.6286003489789\\
10.5198853203895	45.5885803415252\\
10.5684672517218	45.5485603340951\\
10.6172735394972	45.5085403266888\\
10.6663052198171	45.4685203193065\\
10.715563333568	45.4285003119486\\
10.7650489264432	45.3884803046152\\
10.814763048965	45.3484602973065\\
10.8647067565072	45.3084402900227\\
10.9148811093176	45.2684202827642\\
10.9652871725401	45.2284002755311\\
11.0159260162376	45.1883802683236\\
11.0667987154147	45.148360261142\\
11.1179063500406	45.1083402539866\\
11.1692500050717	45.0683202468576\\
11.2208307704748	45.0283002397551\\
11.2726497412507	44.9882802326796\\
11.3247080174565	44.9482602256311\\
11.3770067042298	44.9082402186101\\
11.4295469118117	44.8682202116166\\
11.4823297555707	44.8282002046511\\
11.535356356026	44.7881801977137\\
11.5886278388715	44.7481601908046\\
11.6421453349997	44.7081401839243\\
11.6959099805258	44.6681201770728\\
11.7499229168114	44.6281001702506\\
11.8041852904894	44.5880801634578\\
11.8586982534876	44.5480601566948\\
11.9134629630538	44.5080401499617\\
11.96848058178	44.468020143259\\
12.0237522776272	44.4280001365868\\
12.0792792239501	44.3879801299455\\
12.135062599522	44.3479601233354\\
12.1911035885602	44.3079401167566\\
12.2474033807505	44.2679201102096\\
12.3039631712731	44.2279001036946\\
12.3607841608273	44.1878800972119\\
12.4178675556577	44.1478600907618\\
12.4752145675793	44.1078400843446\\
12.5328264140034	44.0678200779606\\
12.5907043179635	44.0278000716102\\
12.6488495081411	43.9877800652935\\
12.7072632188919	43.9477600590111\\
12.765946690272	43.9077400527631\\
12.8249011680643	43.8677200465498\\
12.8841279038047	43.8277000403717\\
12.9436281548089	43.787680034229\\
13.0034031841991	43.747660028122\\
13.0634542609305	43.7076400220512\\
13.1237826598188	43.6676200160167\\
13.1843896615665	43.6276000100191\\
13.2452765527909	43.5875800040585\\
13.306444626051	43.5475599981353\\
13.3678951798747	43.50753999225\\
13.4296295187868	43.4675199864028\\
13.4916489533366	43.427499980594\\
13.5539548001256	43.3874799748242\\
13.6165483818355	43.3474599690935\\
13.6794310272563	43.3074399634024\\
13.7426040713143	43.2674199577512\\
13.806068855101	43.2273999521404\\
13.8698267259009	43.1873799465702\\
13.9338790372205	43.1473599410411\\
13.998227148817	43.1073399355534\\
14.062872426727	43.0673199301075\\
14.1278162432955	43.0272999247038\\
14.1930599772055	42.9872799193428\\
14.2586050135065	42.9472599140247\\
14.3244527436445	42.90723990875\\
14.3906045654914	42.8672199035191\\
14.4570618833745	42.8271998983324\\
14.5238261081064	42.7871798931903\\
14.5908986570152	42.7471598880932\\
14.658280953974	42.7071398830416\\
14.7259744294318	42.6671198780358\\
14.7939805204436	42.6270998730763\\
14.8623006707006	42.5870798681635\\
14.9309363305612	42.5470598632978\\
14.999888957082	42.5070398584798\\
15.069160014048	42.4670198537097\\
15.1387509720044	42.4269998489881\\
15.2086633082875	42.3869798443154\\
15.2788985070559	42.3469598396921\\
15.3494580593225	42.3069398351186\\
15.4203434629857	42.2669198305954\\
15.4915562228612	42.2268998261229\\
15.5630978507143	42.1868798217017\\
15.6349698652918	42.1468598173321\\
15.7071737923542	42.1068398130147\\
15.779711164708	42.06681980875\\
15.8525835222385	42.0267998045384\\
15.9257924119422	41.9867798003804\\
15.9993393879601	41.9467597962766\\
16.07322601161	41.9067397922274\\
16.1474538514202	41.8667197882333\\
16.2220244831628	41.8266997842949\\
16.2969394898867	41.7866797804127\\
16.3722004619516	41.7466597765871\\
16.4478089970617	41.7066397728187\\
16.5237667002994	41.6666197691081\\
16.6000751841599	41.6265997654557\\
16.6767360685846	41.5865797618622\\
16.7537509809962	41.546559758328\\
16.8311215563331	41.5065397548537\\
16.9088494370839	41.4665197514398\\
16.9869362733223	41.426499748087\\
17.0653837227424	41.3864797447958\\
17.1441934506935	41.3464597415667\\
17.2233671302159	41.3064397384003\\
17.302906442076	41.2664197352973\\
17.3828130748022	41.2263997322581\\
17.4630887247206	41.1863797292835\\
17.5437350959914	41.1463597263739\\
17.6247539006444	41.10633972353\\
17.7061468586161	41.0663197207524\\
17.7879156977856	41.0262997180417\\
17.8700621540116	40.9862797153985\\
17.9525879711692	40.9462597128235\\
18.035494901187	40.9062397103173\\
18.1187847040838	40.8662197078805\\
18.2024591480069	40.8261997055137\\
18.2865200092687	40.7861797032177\\
18.3709690723849	40.746159700993\\
18.4558081301122	40.7061396988404\\
18.5410389834868	40.6661196967604\\
18.6266634418617	40.6260996947538\\
18.7126833229461	40.5860796928213\\
18.7991004528435	40.5460596909635\\
18.8859166660905	40.5060396891811\\
18.9731338056957	40.4660196874748\\
19.060753723179	40.4259996858454\\
19.1487782786108	40.3859796842935\\
19.2372093406515	40.3459596828199\\
19.3260487865912	40.3059396814252\\
19.4152985023894	40.2659196801102\\
19.5049603827153	40.2258996788757\\
19.5950363309877	40.1858796777224\\
19.6855282594159	40.1458596766511\\
19.7764380890397	40.1058396756624\\
19.8677677497706	40.0658196747572\\
19.9595191804325	40.0257996739362\\
20.0516943288031	39.9857796732003\\
20.1442951516552	39.9457596725502\\
20.2373236147981	39.9057396719867\\
20.3307816931193	39.8657196715105\\
20.4246713706267	39.8256996711226\\
20.5189946404906	39.7856796708238\\
20.6137535050858	39.7456596706148\\
20.7089499760343	39.7056396704964\\
20.8045860742482	39.6656196704696\\
20.900663829972	39.6255996705352\\
20.9971852828265	39.585579670694\\
21.0941524818514	39.5455596709469\\
21.1915674855492	39.5055396712947\\
21.2894323619287	39.4655196717384\\
21.387749188549	39.4254996722789\\
21.4865200525637	39.3854796729169\\
21.5857470507649	39.3454596736535\\
21.685432289628	39.3054396744895\\
21.7855778853565	39.2654196754259\\
21.8861859639264	39.2253996764636\\
21.987258661132	39.1853796776035\\
22.0887981226306	39.1453596788465\\
22.1908065039887	39.1053396801937\\
22.2932859707273	39.065319681646\\
22.396238698368	39.0252996832044\\
22.499666872479	38.9852796848698\\
22.603572688722	38.9452596866432\\
22.7079583528983	38.9052396885257\\
22.8128260809959	38.8652196905183\\
22.9181780992365	38.8251996926219\\
23.0240166441225	38.7851796948376\\
23.130343962485	38.7451596971664\\
23.237162311531	38.7051396996094\\
23.3444739588916	38.6651197021677\\
23.4522811826701	38.6250997048422\\
23.5605862714901	38.5850797076342\\
23.6693915245447	38.5450597105446\\
23.7786992516445	38.5050397135746\\
23.8885117732672	38.4650197167252\\
23.9988314206069	38.4249997199976\\
24.1096605356231	38.384979723393\\
24.2210014710909	38.3449597269124\\
24.3328565906507	38.304939730557\\
24.4452282688584	38.264919734328\\
24.558118891236	38.2248997382265\\
24.6715308543219	38.1848797422537\\
24.7854665657221	38.1448597464108\\
24.899928444161	38.104839750699\\
25.0149189195332	38.0648197551196\\
25.1304404329547	38.0247997596736\\
25.2464954368146	37.9847797643625\\
25.3630863948276	37.9447597691874\\
25.4802157820863	37.9047397741496\\
25.597886085113	37.8647197792503\\
25.7160998019135	37.8246997844909\\
25.8348594420295	37.7846797898726\\
25.9541675265918	37.7446597953968\\
26.0740265883745	37.7046398010647\\
26.194439171848	37.6646198068778\\
26.3154078332332	37.6245998128372\\
26.4369351405564	37.5845798189445\\
26.5590236737027	37.5445598252009\\
26.6816760244719	37.5045398316078\\
26.8048947966327	37.4645198381667\\
26.9286826059784	37.4244998448789\\
27.0530420803822	37.3844798517459\\
27.1779758598533	37.3444598587691\\
27.3034865965925	37.3044398659499\\
27.4295769550488	37.2644198732898\\
27.556249611976	37.2243998807903\\
27.6835072564894	37.1843798884529\\
27.8113525901229	37.144359896279\\
27.9397883268864	37.1043399042701\\
28.0688171933232	37.0643199124279\\
28.1984419285683	37.0242999207538\\
28.3286652844062	36.9842799292494\\
28.4594900253294	36.9442599379163\\
28.5909189285972	36.9042399467561\\
28.7229547842946	36.8642199557703\\
28.8556003953913	36.8241999649605\\
28.9888585778017	36.7841799743285\\
29.1227321604441	36.7441599838758\\
29.2572239853012	36.7041399936042\\
29.3923369074803	36.6641200035152\\
29.5280737952738	36.6241000136105\\
29.6644375302202	36.584080023892\\
29.8014310071653	36.5440600343612\\
29.9390571343235	36.50404004502\\
30.0773188333397	36.4640200558702\\
30.2162190393513	36.4240000669133\\
30.3557607010504	36.3839800781513\\
30.4959467807464	36.343960089586\\
30.6367802544292	36.3039401012192\\
30.7782641118319	36.2639201130526\\
30.9204013564946	36.2239001250883\\
31.063195005828	36.183880137328\\
31.2066480911777	36.1438601497736\\
31.350763657888	36.103840162427\\
31.4955447653674	36.0638201752902\\
31.6409944871527	36.0238001883652\\
31.7871159109747	35.9837802016537\\
31.9339121388238	35.943760215158\\
32.0813862870155	35.9037402288798\\
32.229541486257	35.8637202428213\\
32.3783808817133	35.8237002569845\\
32.5279076330741	35.7836802713715\\
32.6781249146208	35.7436602859842\\
32.8290359152942	35.7036403008249\\
32.9806438387618	35.6636203158955\\
33.132951903486	35.6236003311983\\
33.2859633427924	35.5835803467354\\
33.4396814049384	35.5435603625089\\
33.5941093531822	35.5035403785211\\
33.7492504658521	35.4635203947741\\
33.9051080364161	35.4235004112703\\
34.0616853735517	35.3834804280117\\
34.2189858012163	35.3434604450008\\
34.3770126587175	35.3034404622397\\
34.5357693007845	35.2634204797309\\
34.6952590976386	35.2234004974766\\
34.8554854350656	35.1833805154793\\
35.0164517144866	35.1433605337412\\
35.1781613530314	35.1033405522648\\
35.3406177836102	35.0633205710525\\
35.5038244549868	35.0233005901067\\
35.6677848318515	34.98328060943\\
35.8325023948954	34.9432606290247\\
35.9979806408833	34.9032406488935\\
36.1642230827288	34.8632206690389\\
36.3312332495683	34.8232006894633\\
36.4990146868361	34.7831807101695\\
36.6675709563398	34.74316073116\\
36.8369056363357	34.7031407524374\\
37.007022321605	34.6631207740044\\
37.1779246235299	34.6231007958637\\
37.3496161701703	34.583080818018\\
37.5221006063408	34.54306084047\\
37.6953815936883	34.5030408632224\\
37.8694628107696	34.4630208862782\\
38.0443479531292	34.4230009096399\\
38.2200407333782	34.3829809333106\\
38.3965448812729	34.342960957293\\
38.5738641437941	34.3029409815901\\
38.7520022852263	34.2629210062047\\
38.9309630872381	34.2229010311399\\
39.1107503489621	34.1828810563984\\
39.2913678870758	34.1428610819835\\
39.4728195358824	34.102841107898\\
39.6551091473924	34.062821134145\\
39.8382405914052	34.0228011607276\\
40.0222177555915	33.982781187649\\
40.2070445455755	33.9427612149122\\
40.3927248850181	33.9027412425204\\
40.5792627157	33.8627212704768\\
40.7666619976054	33.8227012987846\\
40.9549267090063	33.7826813274471\\
41.1440608465467	33.7426613564676\\
41.3340684253274	33.7026413858494\\
41.5249534789915	33.6626214155958\\
41.7167200598098	33.6226014457102\\
41.9093722387671	33.582581476196\\
42.1029141056481	33.5425615070566\\
42.2973497691249	33.5025415382956\\
42.4926833568435	33.4625215699164\\
42.6889190155123	33.4225016019225\\
42.8860609109892	33.3824816343176\\
43.0841132283705	33.3424616671051\\
43.2830801720801	33.3024417002888\\
43.4829659659579	33.2624217338724\\
43.6837748533501	33.2224017678595\\
43.8855110971994	33.1823818022538\\
44.0881789801347	33.1423618370591\\
44.2917828045629	33.1023418722793\\
44.4963268927599	33.0623219079182\\
44.701815586962	33.0223019439796\\
44.9082532494586	32.9822819804675\\
45.1156442626847	32.9422620173858\\
45.3239930293136	32.9022420547384\\
45.5333039723509	32.8622220925295\\
45.7435815352278	32.8222021307631\\
45.954830181896	32.7821821694432\\
46.167054396922	32.7421622085741\\
46.3802586855826	32.7021422481598\\
46.5944475739604	32.6621222882046\\
46.8096256090399	32.6221023287128\\
47.0257973588042	32.5820823696885\\
47.2429674123316	32.5420624111363\\
47.4611403798933	32.5020424530603\\
47.6803208930514	32.4620224954651\\
47.9005136047569	32.4220025383551\\
48.1217231894485	32.3819825817347\\
48.3439543431522	32.3419626256086\\
48.5672117835806	32.3019426699812\\
48.791500250233	32.2619227148573\\
49.0168245044966	32.2219027602414\\
49.243189329747	32.1818828061383\\
49.4705995314498	32.1418628525527\\
49.6990599372628	32.1018428994894\\
49.9285753971387	32.0618229469532\\
50.1591507834275	32.0218029949491\\
50.3907909909802	31.9817830434819\\
50.6235009372529	31.9417630925566\\
50.8572855624109	31.9017431421783\\
51.0921498294339	31.8617231923519\\
51.3280987242209	31.8217032430827\\
51.5651372556965	31.7816832943757\\
51.8032704559168	31.7416633462362\\
52.0425033801768	31.7016433986694\\
52.2828411071172	31.6616234516807\\
52.5242887388322	31.6216035052753\\
52.7668514009784	31.5815835594587\\
53.010534242883	31.5415636142363\\
53.2553424376533	31.5015436696136\\
53.5012811822866	31.4615237255963\\
53.7483556977805	31.4215037821898\\
53.9965712292437	31.3814838393998\\
54.2459330460073	31.3414638972321\\
54.4964464417369	31.3014439556924\\
54.7481167345445	31.2614240147865\\
55.0009492671021	31.2214040745203\\
55.2549494067543	31.1813841348997\\
55.510122545633	31.1413641959307\\
55.7664741007712	31.1013442576193\\
56.0240095142187	31.0613243199717\\
56.282734253157	31.0213043829938\\
56.5426538100157	30.9812844466921\\
56.8037737025889	30.9412645110726\\
57.0660994741527	30.9012445761418\\
57.3296366935823	30.8612246419059\\
57.5943909554708	30.8212047083715\\
57.8603678802477	30.7811847755451\\
58.1275731142982	30.7411648434331\\
58.3960123300829	30.7011449120422\\
58.6656912262587	30.6611249813791\\
58.9366155277994	30.6211050514505\\
59.2087909861172	30.5810851222632\\
59.4822233791852	30.5410651938241\\
59.7569185116595	30.5010452661401\\
60.0328822150028	30.4610253392182\\
60.3101203476082	30.4210054130654\\
60.5886387949234	30.3809854876889\\
60.8684434695757	30.3409655630958\\
61.1495403114975	30.3009456392934\\
61.4319352880526	30.2609257162891\\
61.7156343941624	30.2209057940902\\
62.0006436524339	30.1808858727041\\
62.2869691132868	30.1408659521384\\
62.5746168550822	30.1008460324008\\
62.8635929842521	30.0608261134987\\
63.1539036354283	30.0208061954401\\
63.445554971573	29.9807862782327\\
63.7385531841099	29.9407663618843\\
64.032904493055	29.900746446403\\
64.3286151471491	29.8607265317967\\
64.6256914239905	29.8207066180736\\
64.9241396301678	29.7806867052418\\
65.2239661013943	29.7406667933096\\
65.5251772026422	29.7006468822854\\
65.827779328278	29.6606269721774\\
66.1317789021976	29.6206070629942\\
66.4371823779638	29.5805871547444\\
66.7439962389419	29.5405672474366\\
67.0522269984386	29.5005473410796\\
67.3618811998395	29.460527435682\\
67.6729654167483	29.4205075312529\\
67.9854862531262	29.3804876278011\\
68.2994503434323	29.3404677253358\\
68.6148643527643	29.300447823866\\
68.931734977	29.2604279234009\\
69.2500689429393	29.22040802395\\
69.5698730084476	29.1803881255225\\
69.8911539625984	29.1403682281279\\
70.213918625818	29.1003483317759\\
70.5381738500302	29.0603284364759\\
70.8639265188016	29.0203085422379\\
71.191183547488	28.9802886490716\\
71.5199518833808	28.9402687569869\\
71.8502385058548	28.9002488659939\\
72.1820504265165	28.8602289761026\\
72.5153946893524	28.8202090873233\\
72.8502783708792	28.7801891996663\\
73.1867085802933	28.7401693131419\\
73.5246924596224	28.7001494277606\\
73.8642371838769	28.660129543533\\
74.2053499612018	28.6201096604698\\
74.5480380330304	28.5800897785818\\
74.8923086742377	28.5400698978798\\
75.2381691932944	28.5000500183749\\
75.585626932423	28.460030140078\\
75.9346892677529	28.4200102630005\\
76.2853636094773	28.3799903871535\\
76.6376574020103	28.3399705125486\\
76.9915781241455	28.299950639197\\
77.3471332892138	28.2599307671106\\
77.7043304452438	28.219910896301\\
78.0631771751216	28.1798910267799\\
78.4236810967518	28.1398711585594\\
78.7858498632195	28.0998512916515\\
79.1496911629522	28.0598314260683\\
79.5152127198837	28.0198115618221\\
79.8824222936175	27.9797916989252\\
80.2513276795919	27.9397718373902\\
80.6219367092452	27.8997519772297\\
80.9942572501823	27.8597321184564\\
81.3682972063414	27.8197122610831\\
81.7440645181619	27.7796924051228\\
82.121567162753	27.7396725505886\\
82.5008131540631	27.6996526974937\\
82.8818105430498	27.6596328458513\\
83.2645674178507	27.619612995675\\
83.6490919039556	27.5795931469784\\
84.0353921643786	27.539573299775\\
84.423476399831	27.4995534540788\\
84.8133528488963	27.4595336099037\\
85.2050297882047	27.4195137672637\\
85.5985155326084	27.3794939261732\\
85.9938184353586	27.3394740866464\\
86.3909468882829	27.2994542486978\\
86.7899093219629	27.259434412342\\
87.1907142059136	27.2194145775938\\
87.5933700487633	27.179394744468\\
87.9978853984339	27.1393749129797\\
88.4042688423224	27.099355083144\\
88.8125290074835	27.0593352549763\\
89.2226745608124	27.0193154284919\\
89.6347142092288	26.9792956037065\\
90.0486566998624	26.9392757806357\\
90.4645108202374	26.8992559592955\\
90.88228539846	26.8592361397019\\
91.3019893034057	26.819216321871\\
91.7236314449071	26.7791965058192\\
92.1472207739435	26.7391766915629\\
92.5727662828307	26.6991568791187\\
93.0002770054119	26.6591370685035\\
93.4297620172496	26.6191172597342\\
93.8612304358183	26.5790974528278\\
94.2946914206979	26.5390776478015\\
94.730154173768	26.4990578446729\\
95.1676279394036	26.4590380434595\\
95.6071220046713	26.4190182441789\\
96.0486456995261	26.3789984468492\\
96.4922083970099	26.3389786514883\\
96.9378195134503	26.2989588581144\\
97.3854885086602	26.2589390667461\\
97.8352248861393	26.2189192774018\\
98.2870381932753	26.1788994901002\\
98.7409380215466	26.1388797048604\\
99.1969340067261	26.0988599217013\\
99.655035829086	26.0588401406423\\
100.115253213603	26.0188203617027\\
100.577595930163	25.9788005849022\\
101.042073793774	25.9387808102607\\
101.508696664767	25.898761037798\\
101.977474449012	25.8587412675344\\
102.448417098122	25.8187214994902\\
102.921534609671	25.7787017336859\\
103.3968370274	25.7386819701424\\
103.874334441436	25.6986622088805\\
104.354036988501	25.6586424499214\\
104.83595485213	25.6186226932863\\
105.320098262886	25.5786029389968\\
105.80647749858	25.5385831870746\\
106.295102884484	25.4985634375416\\
106.785984793556	25.4585436904199\\
107.279133646656	25.4185239457319\\
107.774559912768	25.3785042035001\\
108.272274109224	25.3384844637472\\
108.772286801926	25.2984647264962\\
109.27460860557	25.2584449917702\\
109.779250183872	25.2184252595926\\
110.286222249794	25.1784055299871\\
110.795535565772	25.1383858029773\\
111.307200943943	25.0983660785874\\
111.821229246378	25.0583463568415\\
112.337631385307	25.0183266377643\\
112.856418323356	24.9783069213803\\
113.377601073777	24.9382872077146\\
113.901190700682	24.8982674967922\\
114.427198319278	24.8582477886386\\
114.955635096105	24.8182280832793\\
115.486512249268	24.7782083807404\\
116.019841048683	24.7381886810478\\
116.555632816306	24.698168984228\\
117.093898926384	24.6581492903075\\
117.634650805689	24.6181295993132\\
118.177899933763	24.5781099112721\\
118.723657843162	24.5380902262117\\
119.271936119701	24.4980705441594\\
119.822746402699	24.4580508651431\\
120.376100385228	24.4180311891911\\
120.932009814357	24.3780115163315\\
121.490486491407	24.337991846593\\
122.051542272196	24.2979721800046\\
122.615189067297	24.2579525165954\\
123.181438842284	24.2179328563948\\
123.750303617991	24.1779131994326\\
124.321795470765	24.1378935457386\\
124.895926532723	24.0978738953433\\
125.472708992008	24.057854248277\\
126.052155093052	24.0178346045707\\
126.63427713683	23.9778149642554\\
127.219087481126	23.9377953273625\\
127.806598540793	23.8977756939237\\
128.396822788018	23.857756063971\\
128.989772752585	23.8177364375367\\
129.585461022141	23.7777168146533\\
130.183900242466	23.7376971953536\\
130.785103117737	23.697677579671\\
131.389082410804	23.6576579676388\\
131.995850943453	23.6176383592909\\
132.605421596686	23.5776187546613\\
133.217807310988	23.5375991537845\\
133.833021086605	23.4975795566953\\
134.451075983821	23.4575599634286\\
135.071985123233	23.4175403740199\\
135.69576168603	23.377520788505\\
136.322418914273	23.3375012069197\\
136.951970111177	23.2974816293006\\
137.584428641392	23.2574620556843\\
138.219807931287	23.2174424861079\\
138.858121469236	23.1774229206088\\
139.499382805904	23.1374033592247\\
140.143605554534	23.0973838019938\\
140.790803391236	23.0573642489544\\
141.440990055278	23.0173447001455\\
142.094179349378	22.9773251556061\\
142.750385139995	22.9373056153757\\
143.409621357626	22.8972860794943\\
144.0719019971	22.8572665480022\\
144.737241117876	22.8172470209399\\
145.405652844341	22.7772274983485\\
146.077151366109	22.7372079802693\\
146.751750938323	22.6971884667442\\
147.42946588196	22.6571689578153\\
148.110310584131	22.6171494535251\\
148.794299498388	22.5771299539166\\
149.481447145032	22.5371104590331\\
150.171768111418	22.4970909689184\\
150.865277052271	22.4570714836166\\
151.561988689989	22.4170520031723\\
152.261917814963	22.3770325276304\\
152.965079285884	22.3370130570363\\
153.671488030065	22.2969935914359\\
154.381159043753	22.2569741308753\\
155.094107392451	22.2169546754013\\
155.810348211234	22.1769352250608\\
156.529896705074	22.1369157799015\\
157.25276814916	22.0968963399713\\
157.978977889225	22.0568769053186\\
158.708541341869	22.0168574759922\\
159.441473994887	21.9768380520415\\
160.177791407599	21.9368186335163\\
160.917509211179	21.8967992204666\\
161.66064310899	21.8567798129434\\
162.40720887691	21.8167604109977\\
163.157222363676	21.7767410146811\\
163.910699491214	21.7367216240458\\
164.66765625498	21.6967022391443\\
165.428108724297	21.6566828600298\\
166.192073042701	21.6166634867558\\
166.959565428276	21.5766441193763\\
167.730602174008	21.536624757946\\
168.505199648122	21.49660540252\\
169.283374294433	21.4565860531537\\
170.065142632699	21.4165667099033\\
170.850521258965	21.3765473728255\\
171.639526845917	21.3365280419774\\
172.432176143241	21.2965087174165\\
173.228485977972	21.2564893992013\\
174.028473254854	21.2164700873903\\
174.832154956701	21.1764507820429\\
175.639548144754	21.1364314832189\\
176.450669959044	21.0964121909787\\
177.265537618758	21.0563929053833\\
178.084168422602	21.0163736264942\\
178.906579749169	20.9763543543734\\
179.732789057308	20.9363350890836\\
180.562813886497	20.8963158306881\\
181.39667185721	20.8562965792507\\
182.234380671297	20.8162773348357\\
183.075958112354	20.7762580975083\\
183.921422046107	20.7362388673339\\
184.770790420785	20.6962196443789\\
185.624081267505	20.6562004287101\\
186.481312700654	20.6161812203949\\
187.342502918271	20.5761620195015\\
188.207670202438	20.5361428260984\\
189.076832919665	20.4961236402551\\
189.950009521279	20.4561044620415\\
190.827218543818	20.4160852915283\\
191.708478609426	20.3760661287867\\
192.593808426241	20.3360469738888\\
193.483226788801	20.296027826907\\
194.376752578439	20.2560086879148\\
195.274404763682	20.215989556986\\
196.176202400658	20.1759704341954\\
197.082164633495	20.1359513196182\\
197.992310694735	20.0959322133305\\
198.906659905733	20.0559131154091\\
199.825231677076	20.0158940259313\\
200.748045508988	19.9758749449754\\
201.675120991751	19.9358558726202\\
202.606477806113	19.8958368089454\\
203.542135723711	19.8558177540313\\
204.482114607491	19.8157987079591\\
205.426434412127	19.7757796708105\\
206.375115184445	19.7357606426682\\
207.32817706385	19.6957416236157\\
208.285640282755	19.6557226137369\\
209.247525167004	19.6157036131169\\
210.213852136311	19.5756846218413\\
211.18464170469	19.5356656399967\\
212.159914480891	19.4956466676705\\
213.139691168835	19.4556277049506\\
214.123992568061	19.415608751926\\
215.112839574156	19.3755898086865\\
216.10625317921	19.3355708753227\\
217.104254472254	19.295551951926\\
218.106864639713	19.2555330385887\\
219.114104965849	19.2155141354039\\
220.12599683322	19.1754952424655\\
221.142561723131	19.1354763598685\\
222.163821216089	19.0954574877086\\
223.189796992262	19.0554386260823\\
224.220510831939	19.0154197750873\\
225.255984615994	18.975400934822\\
226.296240326348	18.9353821053856\\
227.341300046436	18.8953632868784\\
228.391185961678	18.8553444794016\\
229.44592035995	18.8153256830574\\
230.505525632052	18.7753068979487\\
231.570024272191	18.7352881241796\\
232.639438878451	18.6952693618551\\
233.713792153278	18.6552506110811\\
234.793106903962	18.6152318719645\\
235.877406043117	18.5752131446134\\
236.966712589169	18.5351944291365\\
238.061049666849	18.4951757256438\\
239.160440507677	18.4551570342463\\
240.264908450462	18.4151383550559\\
241.374476941791	18.3751196881857\\
242.48916953653	18.3351010337497\\
243.609009898327	18.295082391863\\
244.734021800108	18.2550637626418\\
245.864229124585	18.2150451462033\\
246.999655864764	18.175026542666\\
248.140326124455	18.1350079521491\\
249.286264118777	18.0949893747733\\
250.437494174681	18.0549708106602\\
251.594040731461	18.0149522599326\\
252.755928341275	17.9749337227144\\
253.923181669665	17.9349151991307\\
255.09582549608	17.8948966893077\\
256.273884714404	17.8548781933728\\
257.457384333485	17.8148597114546\\
258.646349477661	17.7748412436828\\
259.840805387301	17.7348227901883\\
261.040777419332	17.6948043511035\\
262.246291047787	17.6547859265616\\
263.457371864337	17.6147675166973\\
264.674045578839	17.5747491216465\\
265.896338019882	17.5347307415463\\
267.124275135332	17.4947123765351\\
268.357882992886	17.4546940267527\\
269.597187780627	17.4146756923399\\
270.842215807572	17.3746573734391\\
272.092993504239	17.334639070194\\
273.349547423206	17.2946207827493\\
274.611904239671	17.2546025112515\\
275.880090752022	17.2145842558481\\
277.154133882405	17.1745660166881\\
278.434060677294	17.1345477939219\\
279.719898308069	17.0945295877013\\
281.011674071587	17.0545113981795\\
282.309415390768	17.014493225511\\
283.613149815172	16.9744750698519\\
284.922905021585	16.9344569313596\\
286.238708814609	16.894438810193\\
287.560589127251	16.8544207065127\\
288.888574021513	16.8144026204803\\
290.222691688993	16.7743845522594\\
291.562970451479	16.7343665020148\\
292.909438761552	16.6943484699129\\
294.262125203191	16.6543304561217\\
295.621058492378	16.6143124608108\\
296.98626747771	16.5742944841511\\
298.357781141006	16.5342765263153\\
299.735628597932	16.4942585874778\\
301.119839098607	16.4542406678145\\
302.510442028234	16.4142227675027\\
303.907466907719	16.3742048867217\\
305.310943394298	16.3341870256523\\
306.720901282168	16.2941691844769\\
308.137370503119	16.2541513633798\\
309.560381127168	16.2141335625468\\
310.989963363199	16.1741157821655\\
312.426147559604	16.1340980224253\\
313.868964204927	16.0940802835172\\
315.318443928511	16.0540625656341\\
316.774617501149	16.0140448689707\\
318.237515835737	15.9740271937234\\
319.707169987928	15.9340095400904\\
321.183611156796	15.8939919082719\\
322.666870685493	15.8539742984698\\
324.156980061919	15.8139567108879\\
325.653970919389	15.7739391457319\\
327.1578750373	15.7339216032096\\
328.668724341813	15.6939040835303\\
330.186550906528	15.6538865869056\\
331.711386953162	15.6138691135489\\
333.243264852235	15.5738516636756\\
334.78221712376	15.5338342375031\\
336.328276437929	15.4938168352509\\
337.881475615808	15.4537994571403\\
339.441847630035	15.4137821033949\\
341.009425605519	15.3737647742403\\
342.584242820144	15.3337474699041\\
344.166332705473	15.293730190616\\
345.75572884746	15.253712936608\\
347.352464987164	15.2136957081141\\
348.956575021463	15.1736785053705\\
350.568093003772	15.1336613286156\\
352.187053144771	15.09364417809\\
353.813489813129	15.0536270540365\\
355.447437536229	15.0136099567002\\
357.088931000911	14.9735928863284\\
358.738005054198	14.9335758431708\\
360.39469470404	14.8935588274793\\
362.059035120061	14.8535418395082\\
363.731061634299	14.8135248795141\\
365.410809741959	14.773507947756\\
367.09831510217	14.7334910444953\\
368.793613538734	14.6934741699959\\
370.496741040893	14.653457324524\\
372.207733764093	14.6134405083484\\
373.926628030746	14.5734237217402\\
375.653460331008	14.5334069649732\\
377.388267323548	14.4933902383237\\
379.13108583633	14.4533735420705\\
380.881952867393	14.4133568764949\\
382.640905585636	14.3733402418811\\
384.407981331609	14.3333236385156\\
386.183217618305	14.2933070666877\\
387.966652131954	14.2532905266894\\
389.758322732826	14.2132740188154\\
391.558267456034	14.1732575433631\\
393.366524512341	14.1332411006326\\
395.183132288971	14.0932246909269\\
397.008129350424	14.0532083145517\\
398.841554439296	14.0131919718155\\
400.683446477099	13.9731756630299\\
402.533844565089	13.933159388509\\
404.392787985098	13.8931431485702\\
406.260316200361	13.8531269435336\\
408.136468856361	13.8131107737223\\
410.021285781671	13.7730946394624\\
411.914806988789	13.7330785410831\\
413.817072675003	13.6930624789165\\
415.72812322323	13.6530464532979\\
417.647999202884	13.6130304645656\\
419.576741370729	13.5730145130612\\
421.514390671752	13.5329985991292\\
423.460988240025	13.4929827231176\\
425.416575399583	13.4529668853774\\
427.381193665299	13.4129510862629\\
429.354884743766	13.3729353261316\\
431.337690534184	13.3329196053446\\
433.329653129246	13.292903924266\\
435.330814816033	13.2528882832635\\
437.341218076915	13.2128726827081\\
439.360905590448	13.1728571229742\\
441.389920232282	13.1328416044398\\
443.428305076071	13.0928261274863\\
445.476103394389	13.0528106924987\\
447.533358659646	13.0127952998654\\
449.600114545014	12.9727799499786\\
451.67641492535	12.932764643234\\
453.762303878129	12.892749380031\\
455.857825684385	12.8527341607728\\
457.963024829641	12.8127189858662\\
460.077946004862	12.7727038557218\\
462.202634107401	12.732688770754\\
464.33713424195	12.692673731381\\
466.481491721497	12.6526587380249\\
468.635752068297	12.6126437911119\\
470.799961014824	12.5726288910718\\
472.974164504755	12.5326140383386\\
475.158408693936	12.4925992333503\\
477.352739951367	12.4525844765489\\
479.557204860185	12.4125697683805\\
481.771850218653	12.3725551092953\\
483.996723041153	12.3325404997478\\
486.231870559183	12.2925259401966\\
488.477340222363	12.2525114311046\\
490.73317969944	12.2124969729389\\
492.999436879299	12.172482566171\\
495.276159871982	12.1324682112767\\
497.563397009708	12.0924539087363\\
499.861196847899	12.0524396590345\\
502.169608166212	12.0124254626604\\
504.48867996957	11.9724113201078\\
506.818461489212	11.932397231875\\
509.159002183727	11.8923831984649\\
511.51035174011	11.8523692203849\\
513.872560074817	11.8123552981474\\
516.245677334823	11.7723414322694\\
518.629753898685	11.7323276232726\\
521.024840377617	11.6923138716837\\
523.430987616559	11.6523001780341\\
525.848246695256	11.6122865428603\\
528.27666892935	11.5722729667036\\
530.716305871459	11.5322594501104\\
533.167209312278	11.4922459936321\\
535.629431281677	11.4522325978254\\
538.103024049808	11.4122192632519\\
540.588040128206	11.3722059904786\\
543.084532270915	11.3321927800775\\
545.592553475602	11.2921796326262\\
548.11215698468	11.2521665487074\\
550.643396286443	11.2121535289093\\
553.186325116201	11.1721405738256\\
555.740997457415	11.1321276840555\\
558.307467542852	11.0921148602035\\
560.885789855729	11.05210210288\\
563.476019130871	11.0120894127009\\
566.078210355879	10.9720767902879\\
568.692418772287	10.9320642362683\\
571.318699876742	10.8920517512754\\
573.957109422183	10.8520393359482\\
576.607703419019	10.8120269909317\\
579.27053813632	10.7720147168768\\
581.945670103016	10.7320025144406\\
584.633156109091	10.6919903842861\\
587.333053206791	10.6519783270824\\
590.045418711838	10.6119663435051\\
592.77031020464	10.5719544342357\\
595.507785531519	10.5319425999623\\
598.25790280594	10.4919308413791\\
601.020720409738	10.4519191591869\\
603.796296994364	10.411907554093\\
606.584691482126	10.3718960268111\\
609.385963067442	10.3318845780617\\
612.200171218097	10.2918732085719\\
615.027375676503	10.2518619190755\\
617.867636460969	10.211850710313\\
620.721013866976	10.171839583032\\
623.587568468455	10.1318285379867\\
626.467361119072	10.0918175759387\\
629.360452953524	10.0518066976563\\
632.266905388836	10.011795903915\\
635.186780125658	9.97178519549763\\
638.120139149584	9.93177457319405\\
641.067044732464	9.89176403780156\\
644.027559433723	9.85175359012485\\
647.001746101695	9.811743230976\\
649.989667874954	9.77173296117464\\
652.991388183652	9.73172278154801\\
656.00697075087	9.69171269293098\\
659.03647959397	9.65170269616614\\
662.07997902595	9.61169279210393\\
665.137533656814	9.57168298160262\\
668.209208394941	9.53167326552842\\
671.295068448464	9.49166364475563\\
674.395179326655	9.45165412016656\\
677.509606841313	9.41164469265174\\
680.638417108163	9.37163536310994\\
683.78167654826	9.33162613244825\\
686.9394518894	9.29161700158213\\
690.11181016753	9.25160797143557\\
693.298818728181	9.21159904294108\\
696.500545227891	9.17159021703979\\
699.717057635642	9.13158149468161\\
702.948424234306	9.09157287682516\\
706.19471362209	9.051564364438\\
709.455994713995	9.01155595849664\\
712.732336743282	8.97154765998659\\
716.023809262934	8.93153946990254\\
719.330482147139	8.89153138924837\\
722.652425592773	8.85152341903722\\
725.989710120886	8.81151556029168\\
729.3424065782	8.77150781404377\\
732.71058613862	8.73150018133504\\
736.094320304735	8.69149266321674\\
739.493680909343	8.65148526074982\\
742.908740116972	8.61147797500505\\
746.339570425412	8.57147080706316\\
749.786244667258	8.53146375801482\\
753.248836011454	8.49145682896085\\
756.727417964843	8.45145002101226\\
760.222064373731	8.41144333529032\\
763.732849425456	8.37143677292669\\
767.259847649959	8.33143033506355\\
770.803133921369	8.2914240228536\\
774.362783459591	8.25141783746023\\
777.938871831903	8.21141178005764\\
781.531474954562	8.17140585183087\\
785.140669094413	8.13140005397591\\
788.76653087051	8.09139438769989\\
792.40913725574	8.05138885422107\\
796.068565578463	8.01138345476899\\
799.744893524145	7.97137819058462\\
803.438199137013	7.93137306292036\\
807.148560821713	7.89136807304024\\
810.876057344967	7.85136322222002\\
814.620767837254	7.81135851174722\\
818.382771794484	7.77135394292133\\
822.162149079689	7.73134951705385\\
825.958979924714	7.69134523546843\\
829.773344931927	7.65134109950097\\
833.605325075921	7.61133711049976\\
837.455001705243	7.57133326982555\\
841.322456544116	7.53132957885172\\
845.207771694168	7.49132603896434\\
849.111029636188	7.45132265156235\\
853.032313231867	7.4113194180576\\
856.971705725559	7.37131633987506\\
860.92929074605	7.33131341845286\\
864.905152308334	7.29131065524247\\
868.899374815392	7.2513080517088\\
872.91204305999	7.21130560933031\\
876.943242226474	7.17130332959917\\
880.993057892579	7.13130121402138\\
885.061576031251	7.09129926411683\\
889.148883012464	7.05129748141955\\
893.255065605058	7.01129586747776\\
897.380210978585	6.97129442385397\\
901.52440670515	6.93129315212522\\
905.687740761275	6.89129205388317\\
909.870301529772	6.85129113073412\\
914.07217780161	6.81129038429935\\
918.293458777803	6.77128981621512\\
922.534234071309	6.73128942813281\\
926.794593708924	6.69128922171915\\
931.074628133199	6.65128919865627\\
935.374428204358	6.61128936064187\\
939.694085202226	6.57128970938942\\
944.033690828168	6.53129024662821\\
948.39333720704	6.49129097410355\\
952.773116889131	6.45129189357697\\
957.173122852146	6.41129300682624\\
961.593448503166	6.37129431564563\\
966.034187680636	6.33129582184604\\
970.495434656357	6.29129752725512\\
974.977284137491	6.25129943371745\\
979.479831268559	6.2113015430947\\
984.003171633478	6.17130385726579\\
988.547401257578	6.131306378127\\
993.112616609642	6.09130910759224\\
997.698914603958	6.0513120475931\\
1002.30639260238	6.01131520007904\\
1006.93514841637	5.97131856701766\\
1011.58528030912	5.93132215039469\\
1016.2568869976	5.8913259522143\\
1020.95006765465	5.85132997449922\\
1025.66492191112	5.81133421929091\\
1030.40154985798	5.77133868864971\\
1035.16005204838	5.73134338465511\\
1039.94052949988	5.69134830940579\\
1044.74308369654	5.65135346501991\\
1049.56781659108	5.61135885363524\\
1054.41483060703	5.57136447740934\\
1059.28422864096	5.53137033851976\\
1064.17611406461	5.49137643916421\\
1069.09059072708	5.4513827815608\\
1074.02776295708	5.41138936794813\\
1078.98773556513	5.37139620058554\\
1083.97061384575	5.33140328175335\\
1088.97650357974	5.29141061375294\\
1094.00551103639	5.25141819890703\\
1099.05774297578	5.21142603955987\\
1104.13330665098	5.17143413807741\\
1109.2323098104	5.13144249684751\\
1114.35486070002	5.09145111828015\\
1119.50106806574	5.05146000480765\\
1124.67104115564	5.01146915888483\\
1129.86488972231	4.97147858298928\\
1135.0827240252	4.93148827962153\\
1140.32465483296	4.89149825130525\\
1145.59079342577	4.85150850058752\\
1150.8812515977	4.811519030039\\
1156.19614165913	4.77152984225415\\
1161.53557643908	4.73154093985149\\
1166.89966928762	4.69155232547378\\
1172.28853407829	4.65156400178825\\
1177.70228521052	4.61157597148686\\
1183.14103761204	4.57158823728649\\
1188.60490674132	4.5316008019292\\
1194.09400859005	4.49161366818244\\
1199.60845968555	4.45162683883931\\
1205.14837709331	4.41164031671877\\
1210.71387841942	4.3716541046659\\
1216.30508181309	4.33166820555215\\
1221.92210596917	4.29168262227555\\
1227.56507013062	4.25169735776097\\
1233.23409409112	4.21171241496041\\
1238.92929819754	4.1717277968532\\
1244.65080335254	4.13174350644625\\
1250.3987310171	4.09175954677437\\
1256.17320321316	4.05177592090044\\
1261.97434252612	4.01179263191573\\
1267.80227210752	3.97180968294018\\
1273.65711567764	3.9318270771226\\
1279.53899752808	3.89184481764096\\
1285.44804252445	3.85186290770272\\
1291.38437610901	3.81188135054502\\
1297.34812430331	3.771900149435\\
1303.33941371088	3.73191930767007\\
1309.35837151994	3.69193882857818\\
1315.40512550606	3.65195871551812\\
1321.47980403488	3.61197897187983\\
1327.58253606487	3.5719996010846\\
1333.71345115004	3.53202060658545\\
1339.87267944269	3.49204199186739\\
1346.06035169616	3.45206376044774\\
1352.27659926764	3.41208591587636\\
1358.52155412096	3.37210846173604\\
1364.79534882933	3.33213140164275\\
1371.09811657822	3.29215473924599\\
1377.42999116818	3.25217847822902\\
1383.79110701763	3.21220262230928\\
1390.18159916578	3.17222717523864\\
1396.60160327544	3.13225214080371\\
1403.05125563594	3.09227752282622\\
1409.53069316601	3.0523033251633\\
1416.04005341668	3.01232955170781\\
1422.5794745742	2.9723562063887\\
1429.14909546299	2.9323832931713\\
1435.74905554856	2.8924108160577\\
1442.3794949405	2.85243877908709\\
1449.04055439544	2.81246718633607\\
1455.73237532004	2.77249604191901\\
1462.45509977397	2.73252534998842\\
1469.20887047298	2.69255511473528\\
1475.99383079187	2.65258534038941\\
1482.81012476756	2.61261603121984\\
1489.65789710218	2.57264719153515\\
1496.53729316606	2.53267882568384\\
1503.44845900091	2.4927109380547\\
1510.39154132284	2.45274353307724\\
1517.36668752555	2.41277661522197\\
1524.37404568338	2.37281018900083\\
1531.41376455451	2.33284425896762\\
1538.4859935841	2.2928788297183\\
1545.59088290748	2.25291390589143\\
1552.72858335329	2.21294949216857\\
1559.89924644673	2.17298559327467\\
1567.10302441275	2.13302221397845\\
1574.34007017931	2.09305935909287\\
1581.61053738059	2.05309703347544\\
1588.91458036027	2.01313524202871\\
1596.25235417481	1.9731739897007\\
1603.62401459674	1.93321328148527\\
1611.02971811795	1.89325312242253\\
1618.46962195303	1.85329351759938\\
1625.94388404263	1.8133344721498\\
1633.45266305675	1.77337599125538\\
1640.99611839816	1.73341808014576\\
1648.57441020578	1.69346074409902\\
1656.18769935804	1.65350398844217\\
1663.83614747635	1.61354781855162\\
1671.51991692849	1.57359223985359\\
1679.23917083208	1.53363725782462\\
1686.99407305804	1.49368287799198\\
1694.78478823403	1.45372910593423\\
1702.61148174801	1.4137759472816\\
1710.47431975172	1.3738234077165\\
1718.37346916419	1.33387149297408\\
1726.30909767531	1.2939202088426\\
1734.28137374936	1.25396956116398\\
1742.29046662864	1.21401955583433\\
1750.33654633699	1.17407019880439\\
1758.41978368348	1.13412149608007\\
1766.54035026596	1.09417345372298\\
1774.69841847474	1.05422607785093\\
1782.89416149627	1.01427937463839\\
1791.12775331676	0.97433335031715\\
1799.39936872594	0.93438801117673\\
1807.70918332073	0.894443363564966\\
1816.05737350894	0.854499413888576\\
1824.44411651309	0.814556168613648\\
1832.86959037413	0.774613634266224\\
1841.33397395519	0.734671817432884\\
1849.83744694544	0.694730724761245\\
1858.38018986386	0.654790362960579\\
1866.9623840631	0.614850738802355\\
1875.58421173328	0.57491185912086\\
1884.24585590593	0.534973730813716\\
1892.94750045783	0.495036360842517\\
1901.68933011491	0.455099756233431\\
1910.47153045619	0.415163924077755\\
1919.29428791772	0.375228871532537\\
1928.15778979652	0.335294605821238\\
1937.06222425457	0.295361134234259\\
1946.00778032282	0.255428464129619\\
1954.99464790516	0.215496602933597\\
1964.02301778248	0.175565558141325\\
1973.09308161673	0.135635337317429\\
1982.20503195496	0.0957059480967274\\
1991.35906223344	0.0557773981848182\\
2000.55536678172	0.0158496953587651\\
2009.79414082682	-0.024077152532218\\
2019.07558049731	-0.0640031375661358\\
2028.39988282751	-0.103928251747521\\
2037.76724576168	-0.143852487006716\\
2047.17786815819	-0.183775835199256\\
2056.63194979376	-0.223698288105111\\
2066.12969136771	-0.263619837427994\\
2075.6712945062	-0.303540474794691\\
2085.25696176653	-0.343460191754303\\
2094.88689664142	-0.383378979777548\\
2104.56130356336	-0.423296830256\\
2114.2803879089	-0.463213734501404\\
2124.04435600306	-0.503129683744904\\
2133.8534151237	-0.543044669136284\\
2143.70777350589	-0.582958681743229\\
2153.60764034637	-0.622871712550576\\
2163.55322580795	-0.662783752459487\\
2173.54474102401	-0.702694792286739\\
2183.58239810298	-0.74260482276392\\
2193.66641013278	-0.782513834536581\\
2203.79699118545	-0.822421818163527\\
2213.97435632161	-0.86232876411595\\
2224.19872159503	-0.90223466277661\\
2234.47030405729	-0.942139504439061\\
2244.78932176229	-0.982043279306782\\
2255.15599377096	-1.02194597749233\\
2265.57054015585	-1.06184758901655\\
2276.03318200585	-1.10174810380766\\
2286.54414143084	-1.1416475117004\\
2297.10364156645	-1.18154580243523\\
2307.71190657875	-1.22144296565737\\
2318.36916166905	-1.26133899091592\\
2329.07563307865	-1.30123386766302\\
2339.83154809368	-1.34112758525293\\
2350.63713504986	-1.38102013294109\\
2361.49262333744	-1.4209114998832\\
2372.39824340596	-1.46080167513434\\
2383.35422676926	-1.50069064764801\\
2394.36080601029	-1.54057840627512\\
2405.4182147861	-1.58046493976317\\
2416.52668783283	-1.62035023675517\\
2427.6864609706	-1.66023428578871\\
2438.8977711086	-1.70011707529497\\
2450.16085625011	-1.73999859359778\\
2461.4759554975	-1.77987882891253\\
2472.84330905736	-1.81975776934522\\
2484.26315824557	-1.85963540289146\\
2495.73574549243	-1.89951171743536\\
2507.26131434783	-1.93938670074859\\
2518.84010948637	-1.97926034048927\\
2530.4723767126	-2.01913262420093\\
2542.15836296621	-2.05900353931143\\
2553.8983163273	-2.09887307313194\\
2565.69248602162	-2.13874121285573\\
2577.54112242586	-2.17860794555724\\
2589.444477073	-2.21847325819083\\
2601.4028026576	-2.25833713758972\\
2613.41635304121	-2.29819957046487\\
2625.48538325773	-2.33806054340382\\
2637.61014951883	-2.37792004286958\\
2649.7909092194	-2.41777805519937\\
2662.02792094301	-2.45763456660357\\
2674.32144446738	-2.49748956316449\\
2686.67174076992	-2.53734303083513\\
2699.07907203326	-2.57719495543804\\
2711.54370165082	-2.6170453226641\\
2724.0658942324	-2.65689411807123\\
2736.64591560979	-2.69674132708321\\
2749.28403284242	-2.73658693498844\\
2761.98051422303	-2.77643092693861\\
2774.73562928336	-2.8162732879475\\
2787.54964879989	-2.85611400288964\\
2800.42284479955	-2.89595305649905\\
2813.35549056553	-2.93579043336788\\
2826.34786064309	-2.97562611794518\\
2839.40023084533	-3.01546009453544\\
2852.51287825912	-3.05529234729734\\
2865.68608125093	-3.09512286024237\\
2878.92011947275	-3.13495161723341\\
2892.21527386804	-3.1747786019834\\
2905.57182667769	-3.21460379805392\\
2918.99006144599	-3.25442718885376\\
2932.4702630267	-3.29424875763753\\
2946.01271758903	-3.3340684875042\\
2959.61771262377	-3.37388636139566\\
2973.28553694936	-3.41370236209521\\
2987.01648071805	-3.45351647222614\\
3000.81083542203	-3.49332867425023\\
3014.66889389963	-3.53313895046618\\
3028.59095034155	-3.57294728300815\\
3042.57730029708	-3.61275365384424\\
3056.6282406804	-3.65255804477485\\
3070.74406977686	-3.69236043743119\\
3084.92508724934	-3.73216081327371\\
3099.17159414457	-3.77195915359041\\
3113.48389289956	-3.81175543949537\\
3127.86228734801	-3.85154965192699\\
3142.30708272674	-3.89134177164642\\
3156.8185856822	-3.93113177923593\\
3171.39710427697	-3.97091965509717\\
3186.04294799627	-4.01070537944949\\
3200.75642775457	-4.05048893232832\\
3215.53785590219	-4.09027029358337\\
3230.38754623189	-4.13004944287688\\
3245.30581398558	-4.16982635968194\\
3260.29297586097	-4.20960102328069\\
3275.34935001834	-4.24937341276252\\
3290.47525608724	-4.28914350702227\\
3305.67101517332	-4.32891128475844\\
3320.93694986511	-4.36867672447133\\
3336.27338424092	-4.40843980446116\\
3351.68064387565	-4.44820050282628\\
3367.15905584778	-4.48795879746121\\
3382.70894874623	-4.52771466605473\\
3398.3306526774	-4.56746808608802\\
3414.02449927217	-4.60721903483266\\
3429.7908216929	-4.64696748934868\\
3445.62995464054	-4.6867134264826\\
3461.54223436172	-4.72645682286542\\
3477.5279986559	-4.76619765491059\\
3493.58758688252	-4.80593589881198\\
3509.72133996824	-4.84567153054187\\
3525.92960041413	-4.88540452584878\\
3542.21271230298	-4.92513486025547\\
3558.57102130658	-4.96486250905678\\
3575.00487469309	-5.00458744731748\\
3591.51462133436	-5.04430964987016\\
3608.1006117134	-5.08402909131303\\
3624.76319793175	-5.12374574600771\\
3641.50273371703	-5.16345958807704\\
3658.31957443038	-5.20317059140283\\
3675.21407707406	-5.24287872962362\\
3692.18660029898	-5.28258397613235\\
3709.23750441235	-5.32228630407411\\
3726.36715138533	-5.3619856863438\\
3743.57590486067	-5.40168209558377\\
3760.86413016048	-5.44137550418147\\
3778.23219429398	-5.48106588426706\\
3795.68046596523	-5.52075320771098\\
3813.20931558105	-5.56043744612152\\
3830.81911525882	-5.60011857084239\\
3848.51023883439	-5.63979655295016\\
3866.28306187004	-5.67947136325186\\
3884.13796166242	-5.71914297228237\\
3902.07531725059	-5.7588113503019\\
3920.09550942403	-5.79847646729337\\
3938.19892073078	-5.8381382929599\\
3956.38593548549	-5.87779679672206\\
3974.65693977764	-5.91745194771533\\
3993.0123214797	-5.95710371478733\\
4011.45247025537	-5.99675206649518\\
4029.97777756789	-6.03639697110274\\
4048.58863668828	-6.07603839657788\\
4067.28544270374	-6.11567631058965\\
4086.06859252603	-6.15531068050552\\
4104.93848489988	-6.19494147338852\\
4123.89552041149	-6.23456865599443\\
4142.94010149697	-6.27419219476877\\
4162.07263245094	-6.31381205584401\\
4181.29351943512	-6.35342820503658\\
4200.60317048688	-6.39304060784384\\
4220.00199552799	-6.43264922944119\\
4239.49040637325	-6.47225403467897\\
4259.06881673929	-6.51185498807938\\
4278.73764225331	-6.55145205383343\\
4298.49730046193	-6.59104519579786\\
4318.34821084004	-6.63063437749189\\
4338.2907947997	-6.67021956209414\\
4358.32547569911	-6.70980071243937\\
4378.45267885158	-6.74937779101523\\
4398.67283153454	-6.78895075995906\\
4418.98636299868	-6.82851958105452\\
4439.39370447695	-6.86808421572826\\
4459.89528919383	-6.9076446250466\\
4480.49155237446	-6.94720076971212\\
4501.18293125388	-6.98675261006016\\
4521.96986508636	-7.02630010605548\\
4542.85279515466	-7.06584321728868\\
4563.83216477945	-7.10538190297265\\
4584.90841932869	-7.14491612193911\\
4606.0820062271	-7.18444583263489\\
4627.35337496566	-7.22397099311842\\
4648.72297711113	-7.26349156105593\\
4670.19126631568	-7.30300749371787\\
4691.75869832646	-7.3425187479751\\
4713.42573099534	-7.38202528029512\\
4735.19282428857	-7.4215270467383\\
4757.06044029659	-7.46102400295401\\
4779.02904324381	-7.5005161041767\\
4801.09909949849	-7.54000330522203\\
4823.27107758263	-7.57948556048292\\
4845.54544818189	-7.61896282392546\\
4867.92268415562	-7.658435049085\\
4890.4032605469	-7.697902189062\\
4912.98765459258	-7.73736419651788\\
4935.67634573345	-7.77682102367098\\
4958.46981562442	-7.81627262229227\\
4981.36854814472	-7.85571894370113\\
5004.37302940819	-7.8951599387611\\
5027.48374777359	-7.93459555787558\\
5050.70119385497	-7.97402575098337\\
5074.02586053209	-8.01345046755442\\
5097.4582429609	-8.05286965658524\\
5120.998838584	-8.09228326659454\\
5144.64814714125	-8.13169124561857\\
5168.40667068035	-8.17109354120666\\
5192.27491356753	-8.21049010041656\\
5216.2533824982	-8.24988086980971\\
5240.34258650778	-8.28926579544666\\
5264.54303698246	-8.3286448228822\\
5288.85524767004	-8.36801789716062\\
5313.27973469088	-8.40738496281084\\
5337.81701654885	-8.44674596384154\\
5362.46761414231	-8.48610084373616\\
5387.23205077517	-8.52544954544798\\
5412.11085216805	-8.56479201139503\\
5437.10454646936	-8.60412818345502\\
5462.21366426659	-8.6434580029602\\
5487.43873859752	-8.68278141069217\\
5512.78030496154	-8.72209834687665\\
5538.23890133107	-8.76140875117817\\
5563.81506816292	-8.80071256269474\\
5589.50934840979	-8.84000971995246\\
5615.32228753178	-8.87930016090005\\
5641.254433508	-8.91858382290341\\
5667.30633684818	-8.95786064273996\\
5693.47855060436	-8.99713055659313\\
5719.77163038263	-9.03639350004667\\
5746.18613435493	-9.0756494080789\\
5772.72262327089	-9.11489821505695\\
5799.38166046975	-9.15413985473094\\
5826.16381189231	-9.19337426022811\\
5853.06964609293	-9.23260136404679\\
5880.09973425162	-9.27182109805049\\
5907.25465018618	-9.3110333934618\\
5934.53497036433	-9.35023818085623\\
5961.94127391598	-9.38943539015607\\
5989.47414264556	-9.42862495062414\\
6017.13416104428	-9.46780679085746\\
6044.92191630263	-9.50698083878089\\
6072.83799832281	-9.54614702164072\\
6100.88299973121	-9.58530526599812\\
6129.05751589107	-9.62445549772263\\
6157.36214491506	-9.66359764198555\\
6185.79748767801	-9.70273162325315\\
6214.36414782964	-9.74185736528003\\
6243.06273180741	-9.78097479110225\\
6271.89384884933	-9.82008382303039\\
6300.85811100697	-9.85918438264268\\
6329.95613315842	-9.89827639077791\\
6359.18853302131	-9.93735976752833\\
6388.55593116599	-9.97643443223254\\
6418.05895102864	-10.0155003034681\\
6447.69821892457	-10.0545572990446\\
6477.47436406142	-10.0936053359956\\
6507.38801855264	-10.1326443305718\\
6537.43981743082	-10.1716741982334\\
6567.63039866118	-10.210694853642\\
6597.96040315515	-10.2497062106536\\
6628.43047478397	-10.2887081823105\\
6659.0412603923	-10.3277006808333\\
6689.79340981204	-10.3666836176134\\
6720.68757587607	-10.4056569032044\\
6751.7244144321	-10.4446204473149\\
6782.90458435663	-10.4835741587994\\
6814.22874756894	-10.5225179456507\\
6845.69756904508	-10.5614517149913\\
6877.31171683206	-10.6003753730652\\
6909.07186206199	-10.6392888252292\\
6940.97867896634	-10.6781919759442\\
6973.03284489025	-10.7170847287671\\
7005.23504030693	-10.7559669863414\\
7037.58594883204	-10.7948386503889\\
7070.0862572383	-10.8336996217003\\
7102.73665547	-10.8725498001266\\
7135.53783665763	-10.9113890845696\\
7168.49049713269	-10.9502173729731\\
7201.59533644237	-10.9890345623132\\
7234.85305736446	-11.0278405485895\\
7268.26436592224	-11.0666352268147\\
7301.82997139948	-11.1054184910061\\
7335.55058635552	-11.144190234175\\
7369.42692664033	-11.1829503483175\\
7403.45971140979	-11.2216987244044\\
7437.64966314091	-11.2604352523713\\
7471.99750764715	-11.2991598211084\\
7506.50397409388	-11.3378723184505\\
7541.16979501382	-11.3765726311666\\
7575.99570632259	-11.4152606449496\\
7610.98244733438	-11.4539362444055\\
7646.13076077758	-11.4925993130436\\
7681.44139281058	-11.5312497332646\\
7716.91509303763	-11.5698873863508\\
7752.55261452471	-11.6085121524548\\
7788.35471381553	-11.6471239105882\\
7824.32215094763	-11.6857225386111\\
7860.45568946846	-11.72430791322\\
7896.7560964516	-11.7628799099372\\
7933.22414251308	-11.8014384030986\\
7969.86060182773	-11.8399832658428\\
8006.66625214554	-11.8785143700986\\
8043.6418748083	-11.9170315865738\\
8080.78825476606	-11.9555347847428\\
8118.10618059391	-11.9940238328346\\
8155.5964445086	-12.0324985978207\\
8193.25984238547	-12.0709589454024\\
8231.09717377527	-12.1094047399988\\
8269.10924192116	-12.1478358447338\\
8307.29685377578	-12.1862521214234\\
8345.66082001834	-12.2246534305629\\
8384.20195507185	-12.2630396313139\\
8422.92107712042	-12.3014105814911\\
8461.81900812665	-12.339766137549\\
8500.89657384898	-12.3781061545684\\
8540.15460385935	-12.4164304862431\\
8579.59393156072	-12.4547389848659\\
8619.21539420481	-12.4930315013147\\
8659.01983290983	-12.5313078850388\\
8699.0080926784	-12.5695679840446\\
8739.18102241541	-12.6078116448812\\
8779.5394749461	-12.646038712626\\
8820.08430703417	-12.6842490308704\\
8860.81637939989	-12.7224424417045\\
8901.73655673847	-12.7606187857025\\
8942.84570773836	-12.7987779019078\\
8984.14470509972	-12.8369196278173\\
9025.63442555289	-12.8750437993666\\
9067.31574987707	-12.9131502509137\\
9109.18956291901	-12.9512388152242\\
9151.25675361173	-12.9893093234545\\
9193.51821499347	-13.0273616051364\\
9235.97484422661	-13.0653954881606\\
9278.62754261669	-13.1034107987603\\
9321.47721563161	-13.1414073614946\\
9364.5247729208	-13.179384999232\\
9407.77112833455	-13.2173435331331\\
9451.21719994339	-13.2552827826337\\
9494.86391005764	-13.2932025654275\\
9538.71218524688	-13.3311026974485\\
9582.76295635973	-13.3689829928535\\
9627.01715854358	-13.4068432640041\\
9671.47573126437	-13.4446833214486\\
9716.13961832666	-13.4825029739039\\
9761.00976789355	-13.5203020282371\\
9806.08713250686	-13.5580802894465\\
9851.37266910737	-13.5958375606431\\
9896.86733905512	-13.6335736430314\\
9942.57210814977	-13.6712883358901\\
9988.48794665118	-13.7089814365529\\
10034.6158293	-13.7466527403882\\
10080.9567353382	-13.78430204078\\
10127.5116485301	-13.821929129107\\
10174.2815571832	-13.8595337947229\\
10221.267454169	-13.8971158249355\\
10268.4703369442	-13.9346750049859\\
10315.891207572	-13.9722111180275\\
10363.531072743	-14.009723945105\\
10411.3909437969	-14.0472132651324\\
10459.4718367439	-14.0846788548719\\
10507.7747722864	-14.1221204889115\\
10556.3007758401	-14.1595379396428\\
10605.0508775566	-14.1969309772388\\
10654.0261123446	-14.234299369631\\
10703.2275198921	-14.2716428824864\\
10752.6561446888	-14.3089612791847\\
10802.3130360475	-14.3462543207941\\
10852.1992481272	-14.3835217660482\\
10902.315839955	-14.4207633713217\\
10952.6638754485	-14.4579788906062\\
11003.244423439	-14.4951680754857\\
11054.0585576935	-14.5323306751116\\
11105.1073569377	-14.5694664361778\\
11156.3919048792	-14.6065751028953\\
11207.9132902301	-14.6436564169663\\
11259.6726067303	-14.6807101175585\\
11311.6709531708	-14.7177359412788\\
11363.9094334169	-14.7547336221465\\
11416.3891564316	-14.7917028915668\\
11469.1112362992	-14.8286434783033\\
11522.0767922492	-14.8655551084508\\
11575.2869486794	-14.9024375054073\\
11628.7428351806	-14.939290389846\\
11682.4455865599	-14.9761134796868\\
11736.3963428651	-15.0129064900675\\
11790.596249409	-15.049669133315\\
11845.0464567934	-15.0864011189152\\
11899.7481209338	-15.1231021534842\\
11954.7024030838	-15.1597719407373\\
12009.9104698599	-15.196410181459\\
12065.3734932659	-15.2330165734719\\
12121.0926507183	-15.269590811606\\
12177.069125071	-15.3061325876664\\
12233.3041046402	-15.3426415904021\\
12289.7987832301	-15.3791175054734\\
12346.554360158	-15.4155600154192\\
12403.5720402798	-15.4519687996238\\
12460.8530340153	-15.4883435342841\\
12518.3985573745	-15.5246838923748\\
12576.2098319827	-15.5609895436151\\
12634.2880851072	-15.5972601544332\\
12692.6345496825	-15.633495387932\\
12751.2504643373	-15.6696949038531\\
12810.1370734203	-15.7058583585411\\
12869.2956270265	-15.7419854049074\\
12928.7273810244	-15.7780756923929\\
12988.4335970818	-15.8141288669315\\
13048.4155426933	-15.8501445709121\\
13108.6744912069	-15.8861224431403\\
13169.2117218509	-15.9220621188002\\
13230.0285197614	-15.957963229415\\
13291.1261760091	-15.9938254028078\\
13352.5059876274	-16.0296482630616\\
13414.1692576393	-16.0654314304784\\
13476.1172950852	-16.1011745215388\\
13538.351415051	-16.1368771488599\\
13600.8729386957	-16.1725389211541\\
13663.6831932795	-16.2081594431857\\
13726.7835121922	-16.2437383157286\\
13790.1752349813	-16.2792751355223\\
13853.8597073802	-16.3147694952281\\
13917.8382813373	-16.3502209833842\\
13982.1123150444	-16.3856291843605\\
14046.6831729655	-16.4209936783131\\
14111.552225866	-16.4563140411377\\
14176.7208508414	-16.4915898444229\\
14242.190431347	-16.5268206554024\\
14307.9623572268	-16.5620060369071\\
14374.0380247435	-16.5971455473163\\
14440.4188366077	-16.6322387405086\\
14507.1062020079	-16.6672851658114\\
14574.1015366405	-16.7022843679511\\
14641.4062627396	-16.737235887001\\
14709.0218091073	-16.77213925833\\
14776.9496111443	-16.8069940125499\\
14845.1911108798	-16.8417996754621\\
14913.7477570027	-16.8765557680038\\
14982.6210048919	-16.9112618061931\\
15051.8123166476	-16.9459173010741\\
15121.323161122	-16.9805217586605\\
15191.1550139505	-17.0150746798792\\
15261.3093575835	-17.0495755605123\\
15331.7876813171	-17.0840238911393\\
15402.5914813253	-17.118419157078\\
15473.7222606918	-17.1527608383244\\
15545.1815294413	-17.1870484094925\\
15616.9708045722	-17.2212813397525\\
15689.0916100885	-17.2554590927691\\
15761.5454770323	-17.2895811266381\\
15834.333943516	-17.3236468938227\\
15907.4585547554	-17.3576558410887\\
15980.9208631021	-17.391607409439\\
16054.7224280766	-17.4255010340471\\
16128.8648164017	-17.4593361441897\\
16203.3496020352	-17.4931121631785\\
16278.1783662036	-17.5268285082909\\
16353.352697436	-17.5604845907001\\
16428.8741915971	-17.5940798154036\\
16504.7444519217	-17.6276135811518\\
16580.9650890484	-17.661085280374\\
16657.537721054	-17.6944942991055\\
16734.4639734876	-17.7278400169115\\
16811.7454794054	-17.7611218068116\\
16889.3838794052	-17.7943390352023\\
16967.3808216612	-17.8274910617791\\
17045.7379619589	-17.860577239457\\
17124.4569637308	-17.8935969142899\\
17203.539498091	-17.9265494253893\\
17282.9872438709	-17.9594341048416\\
17362.8018876552	-17.9922502776237\\
17442.9851238172	-18.0249972615188\\
17523.5386545551	-18.0576743670293\\
17604.464189928	-18.0902808972899\\
17685.7634478922	-18.1228161479785\\
17767.4381543379	-18.1552794072262\\
17849.4900431252	-18.1876699555261\\
17931.9208561219	-18.2199870656405\\
18014.7323432394	-18.2522300025072\\
18097.9262624709	-18.2843980231439\\
18181.5043799277	-18.3164903765513\\
18265.4684698776	-18.3485063036155\\
18349.8203147818	-18.3804450370079\\
18434.5617053333	-18.4123058010844\\
18519.6944404947	-18.4440878117826\\
18605.2203275364	-18.4757902765182\\
18691.1411820748	-18.5074123940788\\
18777.4588281113	-18.5389533545172\\
18864.1750980704	-18.5704123390424\\
18951.2918328392	-18.601788519909\\
19038.810881806	-18.6330810603055\\
19126.7341029	-18.6642891142401\\
19215.0633626303	-18.6954118264256\\
19303.8005361259	-18.7264483321618\\
19392.9475071751	-18.7573977572164\\
19482.506168266	-18.7882592177046\\
19572.4784206263	-18.8190318199657\\
19662.8661742637	-18.8497146604389\\
19753.6713480067	-18.8803068255367\\
19844.8958695449	-18.910807391516\\
19936.5416754703	-18.9412154243478\\
20028.6107113184	-18.9715299795849\\
20121.1049316092	-19.0017501022267\\
20214.026299889	-19.0318748265824\\
20307.3767887718	-19.0619031761326\\
20401.1583799817	-19.0918341633875\\
20495.373064394	-19.1216667897437\\
20590.0228420787	-19.1514000453386\\
20685.1097223421	-19.1810329089021\\
20780.6357237695	-19.210564347606\\
20876.6028742684	-19.2399933169117\\
20973.0132111115	-19.269318760414\\
21069.8687809795	-19.2985396096839\\
21167.1716400053	-19.3276547841075\\
21264.923853817	-19.3566631907231\\
21363.127497582	-19.3855637240555\\
21461.7846560511	-19.4143552659469\\
21560.8974236026	-19.4430366853861\\
21660.467904287	-19.4716068383336\\
21760.4982118713	-19.5000645675448\\
21860.9904698845	-19.5284087023898\\
21961.9468116618	-19.5566380586699\\
22063.3693803907	-19.5847514384312\\
22165.260329156	-19.6127476297757\\
22267.6218209858	-19.6406254066675\\
22370.4560288971	-19.6683835287377\\
22473.7651359423	-19.6960207410841\\
22577.5513352553	-19.7235357740688\\
22681.8168300982	-19.7509273431115\\
22786.5638339077	-19.7781941484793\\
22891.7945703429	-19.805334875073\\
22997.5112733313	-19.8323481922099\\
23103.7161871177	-19.8592327534016\\
23210.4115663104	-19.8859871961294\\
23317.5996759301	-19.9126101416145\\
23425.2827914574	-19.9391001945847\\
23533.4631988815	-19.9654559430366\\
23642.1431947482	-19.9916759579937\\
23751.3250862094	-20.0177587932597\\
23861.0111910714	-20.0437029851682\\
23971.2038378446	-20.0695070523263\\
24081.9053657923	-20.0951694953554\\
24193.1181249812	-20.1206887966253\\
24304.8444763306	-20.1460634199849\\
24417.0867916629	-20.1712918104867\\
24529.8474537537	-20.1963723941068\\
24643.1288563827	-20.2213035774599\\
24756.933404384	-20.2460837475076\\
24871.2635136979	-20.2707112712628\\
24986.1216114214	-20.2951844954872\\
25101.5101358603	-20.3195017463839\\
25217.4315365806	-20.3436613292832\\
25333.8882744609	-20.3676615283227\\
25450.882821744	-20.3915006061219\\
25568.4176620901	-20.4151768034488\\
25686.4952906292	-20.4386883388819\\
25805.1182140139	-20.4620334084636\\
25924.2889504728	-20.4852101853483\\
26044.0100298642	-20.5082168194424\\
26164.2839937291	-20.531051437038\\
26285.113395346	-20.5537121404382\\
26406.5007997846	-20.5761970075758\\
26528.4487839603	-20.5985040916236\\
26650.959936689	-20.6206314205968\\
26774.0368587419	-20.642576996948\\
26897.6821629011	-20.6643387971522\\
27021.8984740145	-20.6859147712854\\
27146.6884290521	-20.7073028425923\\
27272.0546771616	-20.7285009070463\\
27397.9998797246	-20.7495068328997\\
27524.5267104134	-20.7703184602246\\
27651.6378552475	-20.7909336004439\\
27779.3360126509	-20.811350035852\\
27907.623893509	-20.831565519126\\
28036.5042212264	-20.8515777728254\\
28165.9797317848	-20.8713844888817\\
28296.0531738006	-20.8909833280762\\
28426.7273085842	-20.9103719195069\\
28558.0049101974	-20.9295478600434\\
28689.8887655133	-20.9485087137694\\
28822.3816742749	-20.9672520114135\\
28955.4864491548	-20.9857752497664\\
29089.2059158146	-21.0040758910858\\
29223.5429129655	-21.0221513624871\\
29358.5002924277	-21.0399990553203\\
29494.0809191919	-21.0576163245334\\
29630.2876714792	-21.07500048802\\
29767.1234408028	-21.0921488259528\\
29904.5911320291	-21.1090585801005\\
30042.6936634399	-21.12572695313\\
30181.4339667932	-21.1421511078909\\
30320.8149873869	-21.158328166684\\
30460.8396841202	-21.1742552105119\\
30601.5110295567	-21.1899292783114\\
30742.8320099879	-21.2053473661681\\
30884.8056254962	-21.2205064265118\\
31027.4348900188	-21.2354033672913\\
31170.7228314113	-21.2500350511305\\
31314.6724915125	-21.2643982944626\\
31459.2869262085	-21.2784898666433\\
31604.5692054981	-21.2923064890415\\
31750.5224135574	-21.3058448341077\\
};
\addlegendentry{$\text{G}_\text{o}\text{(z)}$};

\addplot [color=black,solid,line width=1.5pt]
  table[row sep=crcr]{10	46.0206064866912\\
10.0461810465159	45.9805865375398\\
10.0925753619375	45.9405665889519\\
10.139183931163	45.9005466409324\\
10.1860077436388	45.8605266934869\\
10.2330477933808	45.8205067466205\\
10.2803050789954	45.7804868003385\\
10.3277806037004	45.7404668546465\\
10.375475375347	45.7004469095499\\
10.4233904064403	45.6604269650542\\
10.4715267141616	45.620407021165\\
10.5198853203895	45.5803870778878\\
10.5684672517218	45.5403671352283\\
10.6172735394972	45.5003471931923\\
10.6663052198171	45.4603272517855\\
10.715563333568	45.4203073110138\\
10.7650489264432	45.380287370883\\
10.814763048965	45.3402674313991\\
10.8647067565072	45.300247492568\\
10.9148811093176	45.2602275543958\\
10.9652871725401	45.2202076168887\\
11.0159260162376	45.1801876800526\\
11.0667987154147	45.140167743894\\
11.1179063500406	45.100147808419\\
11.1692500050717	45.0601278736339\\
11.2208307704748	45.0201079395452\\
11.2726497412507	44.9800880061593\\
11.3247080174565	44.9400680734827\\
11.3770067042298	44.900048141522\\
11.4295469118117	44.8600282102838\\
11.4823297555707	44.8200082797747\\
11.535356356026	44.7799883500016\\
11.5886278388715	44.7399684209713\\
11.6421453349997	44.6999484926906\\
11.6959099805258	44.6599285651664\\
11.7499229168114	44.6199086384059\\
11.8041852904894	44.5798887124159\\
11.8586982534876	44.5398687872038\\
11.9134629630538	44.4998488627766\\
11.96848058178	44.4598289391416\\
12.0237522776272	44.4198090163063\\
12.0792792239501	44.3797890942779\\
12.135062599522	44.3397691730639\\
12.1911035885602	44.299749252672\\
12.2474033807505	44.2597293331097\\
12.3039631712731	44.2197094143846\\
12.3607841608273	44.1796894965045\\
12.4178675556577	44.1396695794774\\
12.4752145675793	44.0996496633109\\
12.5328264140034	44.0596297480132\\
12.5907043179635	44.0196098335922\\
12.6488495081411	43.9795899200562\\
12.7072632188919	43.9395700074131\\
12.765946690272	43.8995500956715\\
12.8249011680643	43.8595301848395\\
12.8841279038047	43.8195102749256\\
12.9436281548089	43.7794903659383\\
13.0034031841991	43.7394704578861\\
13.0634542609305	43.6994505507779\\
13.1237826598188	43.6594306446221\\
13.1843896615665	43.6194107394278\\
13.2452765527909	43.5793908352038\\
13.306444626051	43.539370931959\\
13.3678951798747	43.4993510297026\\
13.4296295187868	43.4593311284437\\
13.4916489533366	43.4193112281915\\
13.5539548001256	43.3792913289554\\
13.6165483818355	43.3392714307447\\
13.6794310272563	43.2992515335689\\
13.7426040713143	43.2592316374377\\
13.806068855101	43.2192117423607\\
13.8698267259009	43.1791918483476\\
13.9338790372205	43.1391719554083\\
13.998227148817	43.0991520635528\\
14.062872426727	43.059132172791\\
14.1278162432955	43.0191122831332\\
14.1930599772055	42.9790923945895\\
14.2586050135065	42.9390725071702\\
14.3244527436445	42.8990526208857\\
14.3906045654914	42.8590327357467\\
14.4570618833745	42.8190128517635\\
14.5238261081064	42.7789929689471\\
14.5908986570152	42.738973087308\\
14.658280953974	42.6989532068574\\
14.7259744294318	42.6589333276061\\
14.7939805204436	42.6189134495653\\
14.8623006707006	42.5788935727461\\
14.9309363305612	42.5388736971599\\
14.999888957082	42.4988538228181\\
15.069160014048	42.4588339497323\\
15.1387509720044	42.4188140779139\\
15.2086633082875	42.3787942073749\\
15.2788985070559	42.3387743381269\\
15.3494580593225	42.298754470182\\
15.4203434629857	42.2587346035523\\
15.4915562228612	42.2187147382499\\
15.5630978507143	42.1786948742871\\
15.6349698652918	42.1386750116762\\
15.7071737923542	42.0986551504299\\
15.779711164708	42.0586352905607\\
15.8525835222385	42.0186154320815\\
15.9257924119422	41.9785955750049\\
15.9993393879601	41.9385757193442\\
16.07322601161	41.8985558651123\\
16.1474538514202	41.8585360123225\\
16.2220244831628	41.8185161609881\\
16.2969394898867	41.7784963111226\\
16.3722004619516	41.7384764627396\\
16.4478089970617	41.6984566158529\\
16.5237667002994	41.6584367704762\\
16.6000751841599	41.6184169266236\\
16.6767360685846	41.5783970843092\\
16.7537509809962	41.5383772435472\\
16.8311215563331	41.498357404352\\
16.9088494370839	41.458337566738\\
16.9869362733223	41.41831773072\\
17.0653837227424	41.3782978963127\\
17.1441934506935	41.338278063531\\
17.2233671302159	41.2982582323899\\
17.302906442076	41.2582384029048\\
17.3828130748022	41.2182185750908\\
17.4630887247206	41.1781987489634\\
17.5437350959914	41.1381789245383\\
17.6247539006444	41.0981591018312\\
17.7061468586161	41.0581392808581\\
17.7879156977856	41.018119461635\\
17.8700621540116	40.978099644178\\
17.9525879711692	40.9380798285036\\
18.035494901187	40.8980600146282\\
18.1187847040838	40.8580402025685\\
18.2024591480069	40.8180203923413\\
18.2865200092687	40.7780005839635\\
18.3709690723849	40.7379807774524\\
18.4558081301122	40.6979609728251\\
18.5410389834868	40.6579411700991\\
18.6266634418617	40.617921369292\\
18.7126833229461	40.5779015704216\\
18.7991004528435	40.5378817735058\\
18.8859166660905	40.4978619785627\\
18.9731338056957	40.4578421856106\\
19.060753723179	40.4178223946678\\
19.1487782786108	40.3778026057531\\
19.2372093406515	40.3377828188851\\
19.3260487865912	40.2977630340828\\
19.4152985023894	40.2577432513654\\
19.5049603827153	40.2177234707521\\
19.5950363309877	40.1777036922624\\
19.6855282594159	40.137683915916\\
19.7764380890397	40.0976641417328\\
19.8677677497706	40.0576443697327\\
19.9595191804325	40.0176245999359\\
20.0516943288031	39.9776048323629\\
20.1442951516552	39.9375850670343\\
20.2373236147981	39.8975653039707\\
20.3307816931193	39.8575455431933\\
20.4246713706267	39.8175257847231\\
20.5189946404906	39.7775060285815\\
20.6137535050858	39.7374862747901\\
20.7089499760343	39.6974665233707\\
20.8045860742482	39.6574467743451\\
20.900663829972	39.6174270277356\\
20.9971852828265	39.5774072835645\\
21.0941524818514	39.5373875418543\\
21.1915674855492	39.497367802628\\
21.2894323619287	39.4573480659084\\
21.387749188549	39.4173283317188\\
21.4865200525637	39.3773086000827\\
21.5857470507649	39.3372888710235\\
21.685432289628	39.2972691445652\\
21.7855778853565	39.2572494207319\\
21.8861859639264	39.2172296995479\\
21.987258661132	39.1772099810377\\
22.0887981226306	39.1371902652261\\
22.1908065039887	39.0971705521379\\
22.2932859707273	39.0571508417985\\
22.396238698368	39.0171311342334\\
22.499666872479	38.9771114294681\\
22.603572688722	38.9370917275286\\
22.7079583528983	38.8970720284411\\
22.8128260809959	38.8570523322319\\
22.9181780992365	38.8170326389277\\
23.0240166441225	38.7770129485555\\
23.130343962485	38.7369932611423\\
23.237162311531	38.6969735767155\\
23.3444739588916	38.6569538953029\\
23.4522811826701	38.6169342169322\\
23.5605862714901	38.5769145416317\\
23.6693915245447	38.5368948694297\\
23.7786992516445	38.4968752003551\\
23.8885117732672	38.4568555344366\\
23.9988314206069	38.4168358717036\\
24.1096605356231	38.3768162121855\\
24.2210014710909	38.3367965559121\\
24.3328565906507	38.2967769029134\\
24.4452282688584	38.2567572532198\\
24.558118891236	38.2167376068618\\
24.6715308543219	38.1767179638703\\
24.7854665657221	38.1366983242766\\
24.899928444161	38.0966786881119\\
25.0149189195332	38.0566590554082\\
25.1304404329547	38.0166394261974\\
25.2464954368146	37.9766198005118\\
25.3630863948276	37.9366001783842\\
25.4802157820863	37.8965805598474\\
25.597886085113	37.8565609449347\\
25.7160998019135	37.8165413336796\\
25.8348594420295	37.7765217261161\\
25.9541675265918	37.7365021222782\\
26.0740265883745	37.6964825222005\\
26.194439171848	37.6564629259177\\
26.3154078332332	37.6164433334651\\
26.4369351405564	37.576423744878\\
26.5590236737027	37.5364041601923\\
26.6816760244719	37.496384579444\\
26.8048947966327	37.4563650026697\\
26.9286826059784	37.4163454299061\\
27.0530420803822	37.3763258611904\\
27.1779758598533	37.3363062965599\\
27.3034865965925	37.2962867360527\\
27.4295769550488	37.2562671797067\\
27.556249611976	37.2162476275605\\
27.6835072564894	37.1762280796531\\
27.8113525901229	37.1362085360237\\
27.9397883268864	37.0961889967118\\
28.0688171933232	37.0561694617574\\
28.1984419285683	37.016149931201\\
28.3286652844062	36.9761304050831\\
28.4594900253294	36.9361108834449\\
28.5909189285972	36.8960913663279\\
28.7229547842946	36.8560718537739\\
28.8556003953913	36.8160523458251\\
28.9888585778017	36.7760328425243\\
29.1227321604441	36.7360133439144\\
29.2572239853012	36.6959938500388\\
29.3923369074803	36.6559743609414\\
29.5280737952738	36.6159548766664\\
29.6644375302202	36.5759353972585\\
29.8014310071653	36.5359159227627\\
29.9390571343235	36.4958964532245\\
30.0773188333397	36.4558769886898\\
30.2162190393513	36.4158575292049\\
30.3557607010504	36.3758380748166\\
30.4959467807464	36.3358186255721\\
30.6367802544292	36.2957991815189\\
30.7782641118319	36.2557797427052\\
30.9204013564946	36.2157603091794\\
31.063195005828	36.1757408809905\\
31.2066480911777	36.1357214581879\\
31.350763657888	36.0957020408214\\
31.4955447653674	36.0556826289415\\
31.6409944871527	36.0156632225989\\
31.7871159109747	35.9756438218448\\
31.9339121388238	35.935624426731\\
32.0813862870155	35.8956050373097\\
32.229541486257	35.8555856536337\\
32.3783808817133	35.815566275756\\
32.5279076330741	35.7755469037305\\
32.6781249146208	35.7355275376112\\
32.8290359152942	35.6955081774528\\
32.9806438387618	35.6554888233106\\
33.132951903486	35.6154694752402\\
33.2859633427924	35.5754501332978\\
33.4396814049384	35.5354307975403\\
33.5941093531822	35.4954114680247\\
33.7492504658521	35.455392144809\\
33.9051080364161	35.4153728279514\\
34.0616853735517	35.3753535175109\\
34.2189858012163	35.3353342135467\\
34.3770126587175	35.295314916119\\
34.5357693007845	35.2552956252881\\
34.6952590976386	35.2152763411152\\
34.8554854350656	35.1752570636619\\
35.0164517144866	35.1352377929904\\
35.1781613530314	35.0952185291635\\
35.3406177836102	35.0551992722446\\
35.5038244549868	35.0151800222976\\
35.6677848318515	34.9751607793871\\
35.8325023948954	34.9351415435782\\
35.9979806408833	34.8951223149367\\
36.1642230827288	34.8551030935289\\
36.3312332495683	34.8150838794218\\
36.4990146868361	34.775064672683\\
36.6675709563398	34.7350454733806\\
36.8369056363357	34.6950262815836\\
37.007022321605	34.6550070973615\\
37.1779246235299	34.6149879207843\\
37.3496161701703	34.5749687519228\\
37.5221006063408	34.5349495908485\\
37.6953815936883	34.4949304376335\\
37.8694628107696	34.4549112923505\\
38.0443479531292	34.414892155073\\
38.2200407333782	34.374873025875\\
38.3965448812729	34.3348539048315\\
38.5738641437941	34.2948347920178\\
38.7520022852263	34.2548156875102\\
38.9309630872381	34.2147965913855\\
39.1107503489621	34.1747775037215\\
39.2913678870758	34.1347584245963\\
39.4728195358824	34.0947393540891\\
39.6551091473924	34.0547202922797\\
39.8382405914052	34.0147012392486\\
40.0222177555915	33.974682195077\\
40.2070445455755	33.9346631598469\\
40.3927248850181	33.8946441336412\\
40.5792627157	33.8546251165435\\
40.7666619976054	33.8146061086379\\
40.9549267090063	33.7745871100097\\
41.1440608465467	33.7345681207448\\
41.3340684253274	33.6945491409297\\
41.5249534789915	33.654530170652\\
41.7167200598098	33.6145112100001\\
41.9093722387671	33.574492259063\\
42.1029141056481	33.5344733179306\\
42.2973497691249	33.4944543866937\\
42.4926833568435	33.4544354654441\\
42.6889190155123	33.414416554274\\
42.8860609109892	33.3743976532769\\
43.0841132283705	33.3343787625469\\
43.2830801720801	33.2943598821791\\
43.4829659659579	33.2543410122694\\
43.6837748533501	33.2143221529146\\
43.8855110971994	33.1743033042125\\
44.0881789801347	33.1342844662617\\
44.2917828045629	33.0942656391617\\
44.4963268927599	33.054246823013\\
44.701815586962	33.014228017917\\
44.9082532494586	32.9742092239761\\
45.1156442626847	32.9341904412933\\
45.3239930293136	32.8941716699732\\
45.5333039723509	32.8541529101207\\
45.7435815352278	32.8141341618421\\
45.954830181896	32.7741154252446\\
46.167054396922	32.7340967004363\\
46.3802586855826	32.6940779875264\\
46.5944475739604	32.6540592866249\\
46.8096256090399	32.6140405978432\\
47.0257973588042	32.5740219212933\\
47.2429674123316	32.5340032570886\\
47.4611403798933	32.4939846053433\\
47.6803208930514	32.4539659661729\\
47.9005136047569	32.4139473396936\\
48.1217231894485	32.3739287260232\\
48.3439543431522	32.33391012528\\
48.5672117835806	32.2938915375838\\
48.791500250233	32.2538729630554\\
49.0168245044966	32.2138544018168\\
49.243189329747	32.1738358539909\\
49.4705995314498	32.133817319702\\
49.6990599372628	32.0937987990753\\
49.9285753971387	32.0537802922374\\
50.1591507834275	32.013761799316\\
50.3907909909802	31.9737433204398\\
50.6235009372529	31.933724855739\\
50.8572855624109	31.8937064053447\\
51.0921498294339	31.8536879693895\\
51.3280987242209	31.8136695480069\\
51.5651372556965	31.7736511413321\\
51.8032704559168	31.733632749501\\
52.0425033801768	31.6936143726512\\
52.2828411071172	31.6535960109213\\
52.5242887388322	31.6135776644514\\
52.7668514009784	31.5735593333827\\
53.010534242883	31.5335410178578\\
53.2553424376533	31.4935227180207\\
53.5012811822866	31.4535044340166\\
53.7483556977805	31.4134861659921\\
53.9965712292437	31.3734679140951\\
54.2459330460073	31.333449678475\\
54.4964464417369	31.2934314592824\\
54.7481167345445	31.2534132566695\\
55.0009492671021	31.2133950707898\\
55.2549494067543	31.1733769017982\\
55.510122545633	31.133358749851\\
55.7664741007712	31.0933406151062\\
56.0240095142187	31.0533224977228\\
56.282734253157	31.0133043978618\\
56.5426538100157	30.9732863156854\\
56.8037737025889	30.9332682513572\\
57.0660994741527	30.8932502050425\\
57.3296366935823	30.8532321769082\\
57.5943909554708	30.8132141671226\\
57.8603678802477	30.7731961758554\\
58.1275731142982	30.7331782032783\\
58.3960123300829	30.6931602495643\\
58.6656912262587	30.6531423148879\\
58.9366155277994	30.6131243994256\\
59.2087909861172	30.5731065033551\\
59.4822233791852	30.5330886268561\\
59.7569185116595	30.4930707701097\\
60.0328822150028	30.4530529332988\\
60.3101203476082	30.4130351166081\\
60.5886387949234	30.3730173202238\\
60.8684434695757	30.3329995443339\\
61.1495403114975	30.2929817891282\\
61.4319352880526	30.2529640547983\\
61.7156343941624	30.2129463415374\\
62.0006436524339	30.1729286495405\\
62.2869691132868	30.1329109790046\\
62.5746168550822	30.0928933301285\\
62.8635929842521	30.0528757031125\\
63.1539036354283	30.0128580981592\\
63.445554971573	29.9728405154729\\
63.7385531841099	29.9328229552597\\
64.032904493055	29.8928054177276\\
64.3286151471491	29.8527879030868\\
64.6256914239905	29.8127704115492\\
64.9241396301678	29.7727529433286\\
65.2239661013943	29.732735498641\\
65.5251772026422	29.6927180777043\\
65.827779328278	29.6527006807385\\
66.1317789021976	29.6126833079654\\
66.4371823779638	29.572665959609\\
66.7439962389419	29.5326486358955\\
67.0522269984386	29.4926313370531\\
67.3618811998395	29.452614063312\\
67.6729654167483	29.4125968149047\\
67.9854862531262	29.3725795920657\\
68.2994503434323	29.3325623950317\\
68.6148643527643	29.2925452240419\\
68.931734977	29.2525280793372\\
69.2500689429393	29.2125109611611\\
69.5698730084476	29.1724938697592\\
69.8911539625984	29.1324768053794\\
70.213918625818	29.092459768272\\
70.5381738500302	29.0524427586895\\
70.8639265188016	29.0124257768868\\
71.191183547488	28.972408823121\\
71.5199518833808	28.9323918976519\\
71.8502385058548	28.8923750007414\\
72.1820504265165	28.852358132654\\
72.5153946893524	28.8123412936565\\
72.8502783708792	28.7723244840184\\
73.1867085802933	28.7323077040115\\
73.5246924596224	28.6922909539102\\
73.8642371838769	28.6522742339913\\
74.2053499612018	28.6122575445345\\
74.5480380330304	28.5722408858217\\
74.8923086742377	28.5322242581377\\
75.2381691932944	28.4922076617698\\
75.585626932423	28.4521910970079\\
75.9346892677529	28.4121745641448\\
76.2853636094773	28.3721580634758\\
76.6376574020103	28.332141595299\\
76.9915781241455	28.2921251599154\\
77.3471332892138	28.2521087576286\\
77.7043304452438	28.212092388745\\
78.0631771751216	28.172076053574\\
78.4236810967518	28.1320597524278\\
78.7858498632195	28.0920434856214\\
79.1496911629522	28.0520272534729\\
79.5152127198837	28.0120110563031\\
79.8824222936175	27.9719948944361\\
80.2513276795919	27.9319787681987\\
80.6219367092452	27.8919626779208\\
80.9942572501823	27.8519466239354\\
81.3682972063414	27.8119306065787\\
81.7440645181619	27.7719146261899\\
82.121567162753	27.7318986831112\\
82.5008131540631	27.6918827776881\\
82.8818105430498	27.6518669102695\\
83.2645674178507	27.6118510812072\\
83.6490919039556	27.5718352908564\\
84.0353921643786	27.5318195395756\\
84.423476399831	27.4918038277267\\
84.8133528488963	27.4517881556747\\
85.2050297882047	27.4117725237883\\
85.5985155326084	27.3717569324393\\
85.9938184353586	27.3317413820033\\
86.3909468882829	27.2917258728591\\
86.7899093219629	27.251710405389\\
87.1907142059136	27.211694979979\\
87.5933700487633	27.1716795970187\\
87.9978853984339	27.1316642569011\\
88.4042688423224	27.091648960023\\
88.8125290074835	27.0516337067849\\
89.2226745608124	27.0116184975909\\
89.6347142092288	26.9716033328488\\
90.0486566998624	26.9315882129704\\
90.4645108202374	26.8915731383711\\
90.88228539846	26.8515581094703\\
91.3019893034057	26.8115431266912\\
91.7236314449071	26.7715281904609\\
92.1472207739435	26.7315133012105\\
92.5727662828307	26.6914984593752\\
93.0002770054119	26.651483665394\\
93.4297620172496	26.6114689197101\\
93.8612304358183	26.5714542227709\\
94.2946914206979	26.5314395750277\\
94.730154173768	26.4914249769362\\
95.1676279394036	26.4514104289562\\
95.6071220046713	26.4113959315519\\
96.0486456995261	26.3713814851916\\
96.4922083970099	26.3313670903481\\
96.9378195134503	26.2913527474985\\
97.3854885086602	26.2513384571244\\
97.8352248861393	26.2113242197117\\
98.2870381932753	26.1713100357511\\
98.7409380215466	26.1312959057375\\
99.1969340067261	26.0912818301705\\
99.655035829086	26.0512678095545\\
100.115253213603	26.0112538443985\\
100.577595930163	25.971239935216\\
101.042073793774	25.9312260825254\\
101.508696664767	25.8912122868501\\
101.977474449012	25.8511985487181\\
102.448417098122	25.8111848686624\\
102.921534609671	25.7711712472207\\
103.3968370274	25.7311576849362\\
103.874334441436	25.6911441823566\\
104.354036988501	25.651130740035\\
104.83595485213	25.6111173585295\\
105.320098262886	25.5711040384034\\
105.80647749858	25.5310907802253\\
106.295102884484	25.4910775845688\\
106.785984793556	25.4510644520131\\
107.279133646656	25.4110513831426\\
107.774559912768	25.3710383785474\\
108.272274109224	25.3310254388226\\
108.772286801926	25.2910125645692\\
109.27460860557	25.2509997563937\\
109.779250183872	25.210987014908\\
110.286222249794	25.1709743407298\\
110.795535565772	25.1309617344828\\
111.307200943943	25.0909491967961\\
111.821229246378	25.0509367283047\\
112.337631385307	25.0109243296496\\
112.856418323356	24.9709120014777\\
113.377601073777	24.9308997444419\\
113.901190700682	24.8908875592012\\
114.427198319278	24.8508754464206\\
114.955635096105	24.8108634067712\\
115.486512249268	24.7708514409306\\
116.019841048683	24.7308395495824\\
116.555632816306	24.6908277334167\\
117.093898926384	24.6508159931299\\
117.634650805689	24.610804329425\\
118.177899933763	24.5707927430113\\
118.723657843162	24.5307812346049\\
119.271936119701	24.4907698049283\\
119.822746402699	24.4507584547109\\
120.376100385228	24.4107471846889\\
120.932009814357	24.370735995605\\
121.490486491407	24.3307248882092\\
122.051542272196	24.2907138632581\\
122.615189067297	24.2507029215156\\
123.181438842284	24.2106920637524\\
123.750303617991	24.1706812907466\\
124.321795470765	24.1306706032833\\
124.895926532723	24.0906600021551\\
125.472708992008	24.0506494881618\\
126.052155093052	24.0106390621106\\
126.63427713683	23.9706287248163\\
127.219087481126	23.9306184771011\\
127.806598540793	23.890608319795\\
128.396822788018	23.8505982537355\\
128.989772752585	23.8105882797681\\
129.585461022141	23.7705783987459\\
130.183900242466	23.7305686115301\\
130.785103117737	23.6905589189897\\
131.389082410804	23.6505493220019\\
131.995850943453	23.6105398214521\\
132.605421596686	23.5705304182336\\
133.217807310988	23.5305211132484\\
133.833021086605	23.4905119074065\\
134.451075983821	23.4505028016266\\
135.071985123233	23.4104937968357\\
135.69576168603	23.3704848939695\\
136.322418914273	23.3304760939725\\
136.951970111177	23.2904673977978\\
137.584428641392	23.2504588064074\\
138.219807931287	23.2104503207721\\
138.858121469236	23.1704419418718\\
139.499382805904	23.1304336706957\\
140.143605554534	23.0904255082417\\
140.790803391236	23.0504174555174\\
141.440990055278	23.0104095135396\\
142.094179349378	22.9704016833343\\
142.750385139995	22.9303939659373\\
143.409621357626	22.890386362394\\
144.0719019971	22.8503788737593\\
144.737241117876	22.8103715010981\\
145.405652844341	22.770364245485\\
146.077151366109	22.7303571080047\\
146.751750938323	22.6903500897519\\
147.42946588196	22.6503431918314\\
148.110310584131	22.6103364153584\\
148.794299498388	22.5703297614583\\
149.481447145032	22.530323231267\\
150.171768111418	22.490316825931\\
150.865277052271	22.4503105466072\\
151.561988689989	22.4103043944636\\
152.261917814963	22.3702983706787\\
152.965079285884	22.3302924764421\\
153.671488030065	22.2902867129545\\
154.381159043753	22.2502810814275\\
155.094107392451	22.2102755830842\\
155.810348211234	22.1702702191589\\
156.529896705074	22.1302649908973\\
157.25276814916	22.0902598995569\\
157.978977889225	22.0502549464066\\
158.708541341869	22.0102501327272\\
159.441473994887	21.9702454598115\\
160.177791407599	21.930240928964\\
160.917509211179	21.8902365415016\\
161.66064310899	21.8502322987532\\
162.40720887691	21.8102282020602\\
163.157222363676	21.7702242527763\\
163.910699491214	21.7302204522679\\
164.66765625498	21.690216801914\\
165.428108724297	21.6502133031065\\
166.192073042701	21.61020995725\\
166.959565428276	21.5702067657623\\
167.730602174008	21.5302037300744\\
168.505199648122	21.4902008516305\\
169.283374294433	21.4501981318882\\
170.065142632699	21.4101955723186\\
170.850521258965	21.3701931744066\\
171.639526845917	21.3301909396507\\
172.432176143241	21.2901888695634\\
173.228485977972	21.2501869656713\\
174.028473254854	21.2101852295151\\
174.832154956701	21.1701836626498\\
175.639548144754	21.1301822666448\\
176.450669959044	21.0901810430842\\
177.265537618758	21.0501799935667\\
178.084168422602	21.0101791197058\\
178.906579749169	20.9701784231301\\
179.732789057308	20.9301779054834\\
180.562813886497	20.8901775684244\\
181.39667185721	20.8501774136276\\
182.234380671297	20.8101774427829\\
183.075958112354	20.7701776575957\\
183.921422046107	20.7301780597876\\
184.770790420785	20.6901786510959\\
185.624081267505	20.6501794332743\\
186.481312700654	20.6101804080923\\
187.342502918271	20.5701815773365\\
188.207670202438	20.5301829428096\\
189.076832919665	20.4901845063311\\
189.950009521279	20.4501862697377\\
190.827218543818	20.4101882348828\\
191.708478609426	20.3701904036372\\
192.593808426241	20.3301927778891\\
193.483226788801	20.2901953595441\\
194.376752578439	20.2501981505257\\
195.274404763682	20.210201152775\\
196.176202400658	20.1702043682515\\
197.082164633495	20.1302077989325\\
197.992310694735	20.090211446814\\
198.906659905733	20.0502153139104\\
199.825231677076	20.0102194022548\\
200.748045508988	19.9702237138991\\
201.675120991751	19.9302282509146\\
202.606477806113	19.8902330153915\\
203.542135723711	19.8502380094396\\
204.482114607491	19.8102432351883\\
205.426434412127	19.7702486947867\\
206.375115184445	19.7302543904039\\
207.32817706385	19.6902603242292\\
208.285640282755	19.6502664984723\\
209.247525167004	19.6102729153633\\
210.213852136311	19.5702795771532\\
211.18464170469	19.5302864861137\\
212.159914480891	19.4902936445377\\
213.139691168835	19.4503010547397\\
214.123992568061	19.4103087190552\\
215.112839574156	19.3703166398419\\
216.10625317921	19.330324819479\\
217.104254472254	19.2903332603682\\
218.106864639713	19.2503419649333\\
219.114104965849	19.2103509356207\\
220.12599683322	19.1703601748995\\
221.142561723131	19.1303696852618\\
222.163821216089	19.0903794692229\\
223.189796992262	19.0503895293215\\
224.220510831939	19.0103998681199\\
225.255984615994	18.9704104882041\\
226.296240326348	18.9304213921843\\
227.341300046436	18.890432582695\\
228.391185961678	18.8504440623951\\
229.44592035995	18.8104558339682\\
230.505525632052	18.7704679001231\\
231.570024272191	18.7304802635936\\
232.639438878451	18.6904929271389\\
233.713792153278	18.650505893544\\
234.793106903962	18.6105191656197\\
235.877406043117	18.570532746203\\
236.966712589169	18.5305466381575\\
238.061049666849	18.490560844373\\
239.160440507677	18.4505753677667\\
240.264908450462	18.4105902112825\\
241.374476941791	18.3706053778919\\
242.48916953653	18.3306208705942\\
243.609009898327	18.2906366924164\\
244.734021800108	18.2506528464137\\
245.864229124585	18.2106693356698\\
246.999655864764	18.1706861632972\\
248.140326124455	18.1307033324371\\
249.286264118777	18.0907208462602\\
250.437494174681	18.0507387079666\\
251.594040731461	18.0107569207863\\
252.755928341275	17.9707754879793\\
253.923181669665	17.9307944128359\\
255.09582549608	17.8908136986771\\
256.273884714404	17.8508333488549\\
257.457384333485	17.8108533667524\\
258.646349477661	17.7708737557843\\
259.840805387301	17.730894519397\\
261.040777419332	17.6909156610692\\
262.246291047787	17.6509371843118\\
263.457371864337	17.6109590926684\\
264.674045578839	17.5709813897158\\
265.896338019882	17.5310040790641\\
267.124275135332	17.4910271643568\\
268.357882992886	17.4510506492716\\
269.597187780627	17.4110745375204\\
270.842215807572	17.3710988328497\\
272.092993504239	17.3311235390409\\
273.349547423206	17.2911486599105\\
274.611904239671	17.2511741993109\\
275.880090752022	17.2112001611302\\
277.154133882405	17.1712265492927\\
278.434060677294	17.1312533677593\\
279.719898308069	17.0912806205281\\
281.011674071587	17.0513083116341\\
282.309415390768	17.0113364451502\\
283.613149815172	16.9713650251871\\
284.922905021585	16.931394055894\\
286.238708814609	16.8914235414586\\
287.560589127251	16.8514534861078\\
288.888574021513	16.8114838941079\\
290.222691688993	16.771514769765\\
291.562970451479	16.7315461174255\\
292.909438761552	16.691577941476\\
294.262125203191	16.6516102463444\\
295.621058492378	16.6116430364997\\
296.98626747771	16.5716763164527\\
298.357781141006	16.5317100907562\\
299.735628597932	16.4917443640057\\
301.119839098607	16.4517791408393\\
302.510442028234	16.4118144259386\\
303.907466907719	16.3718502240289\\
305.310943394298	16.3318865398796\\
306.720901282168	16.2919233783044\\
308.137370503119	16.2519607441624\\
309.560381127168	16.2119986423576\\
310.989963363199	16.1720370778401\\
312.426147559604	16.1320760556062\\
313.868964204927	16.0921155806988\\
315.318443928511	16.0521556582079\\
316.774617501149	16.012196293271\\
318.237515835737	15.9722374910738\\
319.707169987928	15.9322792568502\\
321.183611156796	15.8923215958833\\
322.666870685493	15.8523645135052\\
324.156980061919	15.8124080150982\\
325.653970919389	15.7724521060946\\
327.1578750373	15.7324967919778\\
328.668724341813	15.6925420782821\\
330.186550906528	15.6525879705937\\
331.711386953162	15.6126344745513\\
333.243264852235	15.5726815958458\\
334.78221712376	15.5327293402217\\
336.328276437929	15.4927777134771\\
337.881475615808	15.4528267214644\\
339.441847630035	15.4128763700907\\
341.009425605519	15.3729266653182\\
342.584242820144	15.3329776131653\\
344.166332705473	15.2930292197062\\
345.75572884746	15.2530814910723\\
347.352464987164	15.2131344334523\\
348.956575021463	15.1731880530928\\
350.568093003772	15.1332423562988\\
352.187053144771	15.0932973494344\\
353.813489813129	15.0533530389233\\
355.447437536229	15.0134094312494\\
357.088931000911	14.973466532957\\
358.738005054198	14.9335243506522\\
360.39469470404	14.8935828910026\\
362.059035120061	14.8536421607383\\
363.731061634299	14.8137021666526\\
365.410809741959	14.7737629156023\\
367.09831510217	14.7338244145085\\
368.793613538734	14.6938866703572\\
370.496741040893	14.6539496901998\\
372.207733764093	14.6140134811538\\
373.926628030746	14.5740780504036\\
375.653460331008	14.5341434052007\\
377.388267323548	14.4942095528648\\
379.13108583633	14.454276500784\\
380.881952867393	14.414344256416\\
382.640905585636	14.3744128272881\\
384.407981331609	14.3344822209986\\
386.183217618305	14.2945524452168\\
387.966652131954	14.2546235076841\\
389.758322732826	14.2146954162144\\
391.558267456034	14.1747681786951\\
393.366524512341	14.1348418030876\\
395.183132288971	14.0949162974279\\
397.008129350424	14.0549916698277\\
398.841554439296	14.0150679284747\\
400.683446477099	13.9751450816334\\
402.533844565089	13.9352231376462\\
404.392787985098	13.8953021049336\\
406.260316200361	13.8553819919954\\
408.136468856361	13.8154628074112\\
410.021285781671	13.7755445598411\\
411.914806988789	13.7356272580268\\
413.817072675003	13.6957109107921\\
415.72812322323	13.6557955270437\\
417.647999202884	13.615881115772\\
419.576741370729	13.5759676860522\\
421.514390671752	13.5360552470445\\
423.460988240025	13.4961438079955\\
425.416575399583	13.4562333782388\\
427.381193665299	13.4163239671958\\
429.354884743766	13.3764155843763\\
431.337690534184	13.33650823938\\
433.329653129246	13.2966019418967\\
435.330814816033	13.2566967017076\\
437.341218076915	13.2167925286858\\
439.360905590448	13.1768894327976\\
441.389920232282	13.136987424103\\
443.428305076071	13.0970865127569\\
445.476103394389	13.0571867090098\\
447.533358659646	13.0172880232088\\
449.600114545014	12.9773904657984\\
451.67641492535	12.9374940473219\\
453.762303878129	12.8975987784217\\
455.857825684385	12.8577046698406\\
457.963024829641	12.8178117324228\\
460.077946004862	12.7779199771147\\
462.202634107401	12.7380294149661\\
464.33713424195	12.6981400571309\\
466.481491721497	12.6582519148683\\
468.635752068297	12.6183649995438\\
470.799961014824	12.5784793226303\\
472.974164504755	12.5385948957089\\
475.158408693936	12.4987117304701\\
477.352739951367	12.4588298387148\\
479.557204860185	12.4189492323554\\
481.771850218653	12.379069923417\\
483.996723041153	12.339191924038\\
486.231870559183	12.2993152464719\\
488.477340222363	12.2594399030878\\
490.73317969944	12.2195659063719\\
492.999436879299	12.1796932689285\\
495.276159871982	12.139822003481\\
497.563397009708	12.0999521228732\\
499.861196847899	12.0600836400705\\
502.169608166212	12.0202165681611\\
504.48867996957	11.9803509203569\\
506.818461489212	11.940486709995\\
509.159002183727	11.9006239505389\\
511.51035174011	11.8607626555794\\
513.872560074817	11.8209028388362\\
516.245677334823	11.7810445141592\\
518.629753898685	11.7411876955291\\
521.024840377617	11.7013323970596\\
523.430987616559	11.6614786329979\\
525.848246695256	11.6216264177265\\
528.27666892935	11.5817757657643\\
530.716305871459	11.541926691768\\
533.167209312278	11.5020792105332\\
535.629431281677	11.4622333369962\\
538.103024049808	11.4223890862349\\
540.588040128206	11.3825464734704\\
543.084532270915	11.3427055140685\\
545.592553475602	11.302866223541\\
548.11215698468	11.2630286175469\\
550.643396286443	11.2231927118943\\
553.186325116201	11.1833585225412\\
555.740997457415	11.1435260655978\\
558.307467542852	11.1036953573273\\
560.885789855729	11.0638664141476\\
563.476019130871	11.0240392526328\\
566.078210355879	10.984213889515\\
568.692418772287	10.9443903416853\\
571.318699876742	10.9045686261958\\
573.957109422183	10.864748760261\\
576.607703419019	10.8249307612595\\
579.27053813632	10.7851146467353\\
581.945670103016	10.7453004343998\\
584.633156109091	10.7054881421333\\
587.333053206791	10.6656777879865\\
590.045418711838	10.6258693901823\\
592.77031020464	10.5860629671176\\
595.507785531519	10.5462585373649\\
598.25790280594	10.5064561196738\\
601.020720409738	10.4666557329732\\
603.796296994364	10.4268573963727\\
606.584691482126	10.3870611291643\\
609.385963067442	10.3472669508248\\
612.200171218097	10.3074748810169\\
615.027375676503	10.2676849395913\\
617.867636460969	10.2278971465887\\
620.721013866976	10.1881115222415\\
623.587568468455	10.1483280869759\\
626.467361119072	10.1085468614133\\
629.360452953524	10.0687678663728\\
632.266905388836	10.028991122873\\
635.186780125658	9.98921665213372\\
638.120139149584	9.9494444755782\\
641.067044732464	9.9096746148351\\
644.027559433723	9.86990709174052\\
647.001746101695	9.83014192834\\
649.989667874954	9.79037914689064\\
652.991388183652	9.7506187698632\\
656.00697075087	9.71086081994418\\
659.03647959397	9.67110532003796\\
662.07997902595	9.63135229326906\\
665.137533656814	9.59160176298417\\
668.209208394941	9.55185375275443\\
671.295068448464	9.51210828637775\\
674.395179326655	9.47236538788088\\
677.509606841313	9.43262508152183\\
680.638417108163	9.39288739179214\\
683.78167654826	9.35315234341919\\
686.9394518894	9.3134199613685\\
690.11181016753	9.27369027084622\\
693.298818728181	9.23396329730146\\
696.500545227891	9.19423906642869\\
699.717057635642	9.15451760417026\\
702.948424234306	9.11479893671881\\
706.19471362209	9.07508309051983\\
709.455994713995	9.03537009227418\\
712.732336743282	8.99565996894056\\
716.023809262934	8.95595274773825\\
719.330482147139	8.9162484561496\\
722.652425592773	8.87654712192269\\
725.989710120886	8.83684877307403\\
729.3424065782	8.79715343789128\\
732.71058613862	8.75746114493583\\
736.094320304735	8.71777192304577\\
739.493680909343	8.67808580133848\\
742.908740116972	8.63840280921356\\
746.339570425412	8.59872297635564\\
749.786244667258	8.55904633273722\\
753.248836011454	8.51937290862163\\
756.727417964843	8.47970273456593\\
760.222064373731	8.44003584142387\\
763.732849425456	8.40037226034893\\
767.259847649959	8.36071202279733\\
770.803133921369	8.32105516053105\\
774.362783459591	8.281401705621\\
777.938871831903	8.24175169045012\\
781.531474954562	8.20210514771652\\
785.140669094413	8.16246211043672\\
788.76653087051	8.12282261194885\\
792.40913725574	8.08318668591594\\
796.068565578463	8.04355436632921\\
799.744893524145	8.00392568751144\\
803.438199137013	7.96430068412028\\
807.148560821713	7.92467939115173\\
810.876057344967	7.88506184394356\\
814.620767837254	7.84544807817882\\
818.382771794484	7.80583812988931\\
822.162149079689	7.76623203545923\\
825.958979924714	7.7266298316287\\
829.773344931927	7.68703155549744\\
833.605325075921	7.64743724452847\\
837.455001705243	7.60784693655183\\
841.322456544116	7.56826066976827\\
845.207771694168	7.52867848275318\\
849.111029636188	7.48910041446033\\
853.032313231867	7.44952650422582\\
856.971705725559	7.40995679177201\\
860.92929074605	7.37039131721145\\
864.905152308334	7.33083012105094\\
868.899374815392	7.2912732441956\\
872.91204305999	7.25172072795297\\
876.943242226474	7.21217261403713\\
880.993057892579	7.17262894457297\\
885.061576031251	7.13308976210038\\
889.148883012464	7.09355510957857\\
893.255065605058	7.05402503039042\\
897.380210978585	7.01449956834685\\
901.52440670515	6.97497876769126\\
905.687740761275	6.9354626731041\\
909.870301529772	6.89595132970723\\
914.07217780161	6.85644478306873\\
918.293458777803	6.81694307920738\\
922.534234071309	6.77744626459743\\
926.794593708924	6.73795438617333\\
931.074628133199	6.69846749133452\\
935.374428204358	6.65898562795031\\
939.694085202226	6.61950884436477\\
944.033690828168	6.5800371894017\\
948.39333720704	6.54057071236965\\
952.773116889131	6.501109463067\\
957.173122852146	6.46165349178709\\
961.593448503166	6.42220284932342\\
966.034187680636	6.38275758697491\\
970.495434656357	6.34331775655121\\
974.977284137491	6.30388341037801\\
979.479831268559	6.26445460130259\\
984.003171633478	6.22503138269922\\
988.547401257578	6.18561380847473\\
993.112616609642	6.14620193307422\\
997.698914603958	6.10679581148659\\
1002.30639260238	6.06739549925043\\
1006.93514841637	6.02800105245975\\
1011.58528030912	5.98861252776993\\
1016.2568869976	5.94922998240359\\
1020.95006765465	5.90985347415668\\
1025.66492191112	5.87048306140453\\
1030.40154985798	5.83111880310799\\
1035.16005204838	5.79176075881973\\
1039.94052949988	5.75240898869046\\
1044.74308369654	5.71306355347534\\
1049.56781659108	5.67372451454041\\
1054.41483060703	5.63439193386914\\
1059.28422864096	5.59506587406901\\
1064.17611406461	5.55574639837812\\
1069.09059072708	5.51643357067209\\
1074.02776295708	5.47712745547071\\
1078.98773556513	5.43782811794497\\
1083.97061384575	5.39853562392402\\
1088.97650357974	5.35925003990218\\
1094.00551103639	5.31997143304617\\
1099.05774297578	5.28069987120232\\
1104.13330665098	5.24143542290385\\
1109.2323098104	5.20217815737829\\
1114.35486070002	5.16292814455502\\
1119.50106806574	5.1236854550728\\
1124.67104115564	5.08445016028741\\
1129.86488972231	5.04522233227948\\
1135.0827240252	5.00600204386228\\
1140.32465483296	4.96678936858966\\
1145.59079342577	4.92758438076416\\
1150.8812515977	4.88838715544499\\
1156.19614165913	4.84919776845639\\
1161.53557643908	4.8100162963959\\
1166.89966928762	4.77084281664275\\
1172.28853407829	4.7316774073664\\
1177.70228521052	4.69252014753519\\
1183.14103761204	4.653371116925\\
1188.60490674132	4.6142303961281\\
1194.09400859005	4.57509806656206\\
1199.60845968555	4.53597421047883\\
1205.14837709331	4.49685891097382\\
1210.71387841942	4.45775225199518\\
1216.30508181309	4.41865431835316\\
1221.92210596917	4.37956519572955\\
1227.56507013062	4.34048497068729\\
1233.23409409112	4.30141373068015\\
1238.92929819754	4.26235156406255\\
1244.65080335254	4.22329856009948\\
1250.3987310171	4.18425480897656\\
1256.17320321316	4.14522040181016\\
1261.97434252612	4.10619543065775\\
1267.80227210752	4.06717998852829\\
1273.65711567764	4.02817416939272\\
1279.53899752808	3.98917806819467\\
1285.44804252445	3.95019178086124\\
1291.38437610901	3.9112154043139\\
1297.34812430331	3.87224903647957\\
1303.33941371088	3.83329277630174\\
1309.35837151994	3.79434672375186\\
1315.40512550606	3.75541097984071\\
1321.47980403488	3.7164856466301\\
1327.58253606487	3.67757082724445\\
1333.71345115004	3.6386666258828\\
1339.87267944269	3.59977314783071\\
1346.06035169616	3.56089049947249\\
1352.27659926764	3.52201878830348\\
1358.52155412096	3.48315812294245\\
1364.79534882933	3.44430861314432\\
1371.09811657822	3.40547036981278\\
1377.42999116818	3.36664350501324\\
1383.79110701763	3.32782813198597\\
1390.18159916578	3.28902436515919\\
1396.60160327544	3.25023232016252\\
1403.05125563594	3.21145211384055\\
1409.53069316601	3.17268386426647\\
1416.04005341668	3.13392769075595\\
1422.5794745742	3.09518371388126\\
1429.14909546299	3.05645205548537\\
1435.74905554856	3.01773283869638\\
1442.3794949405	2.97902618794208\\
1449.04055439544	2.94033222896464\\
1455.73237532004	2.90165108883555\\
1462.45509977397	2.86298289597068\\
1469.20887047298	2.82432778014558\\
1475.99383079187	2.78568587251087\\
1482.81012476756	2.74705730560799\\
1489.65789710218	2.70844221338488\\
1496.53729316606	2.66984073121213\\
1503.44845900091	2.63125299589911\\
1510.39154132284	2.59267914571044\\
1517.36668752555	2.55411932038255\\
1524.37404568338	2.51557366114049\\
1531.41376455451	2.47704231071497\\
1538.4859935841	2.43852541335954\\
1545.59088290748	2.40002311486805\\
1552.72858335329	2.36153556259226\\
1559.89924644673	2.32306290545968\\
1567.10302441275	2.28460529399164\\
1574.34007017931	2.24616288032157\\
1581.61053738059	2.2077358182135\\
1588.91458036027	2.16932426308076\\
1596.25235417481	2.13092837200495\\
1603.62401459674	2.09254830375509\\
1611.02971811795	2.05418421880704\\
1618.46962195303	2.01583627936314\\
1625.94388404263	1.97750464937203\\
1633.45266305675	1.93918949454882\\
1640.99611839816	1.90089098239542\\
1648.57441020578	1.8626092822211\\
1656.18769935804	1.82434456516335\\
1663.83614747635	1.78609700420897\\
1671.51991692849	1.74786677421542\\
1679.23917083208	1.70965405193233\\
1686.99407305804	1.67145901602343\\
1694.78478823403	1.63328184708864\\
1702.61148174801	1.59512272768638\\
1710.47431975172	1.55698184235626\\
1718.37346916419	1.51885937764196\\
1726.30909767531	1.48075552211435\\
1734.28137374936	1.44267046639498\\
1742.29046662864	1.40460440317979\\
1750.33654633699	1.36655752726306\\
1758.41978368348	1.32853003556171\\
1766.54035026596	1.29052212713984\\
1774.69841847474	1.25253400323359\\
1782.89416149627	1.21456586727619\\
1791.12775331676	1.1766179249235\\
1799.39936872594	1.1386903840796\\
1807.70918332073	1.10078345492285\\
1816.05737350894	1.06289734993222\\
1824.44411651309	1.0250322839138\\
1832.86959037413	0.987188474027817\\
1841.33397395519	0.949366139815774\\
1849.83744694544	0.911565503227985\\
1858.38018986386	0.873786788651433\\
1866.9623840631	0.836030222937872\\
1875.58421173328	0.798296035432339\\
1884.24585590593	0.760584458001877\\
1892.94750045783	0.722895725064662\\
1901.68933011491	0.685230073619429\\
1910.47153045619	0.647587743275195\\
1919.29428791772	0.609968976281325\\
1928.15778979652	0.572374017557965\\
1937.06222425457	0.534803114726736\\
1946.00778032282	0.497256518141786\\
1954.99464790516	0.459734480921236\\
1964.02301778248	0.422237258978856\\
1973.09308161673	0.384765111056124\\
1982.20503195496	0.347318298754715\\
1991.35906223344	0.309897086569142\\
2000.55536678172	0.272501741919876\\
2009.79414082682	0.235132535186828\\
2019.07558049731	0.197789739743043\\
2028.39988282751	0.160473631988832\\
2037.76724576168	0.123184491386315\\
2047.17786815819	0.0859226004940843\\
2056.63194979376	0.0486882450024715\\
2066.12969136771	0.0114817137690109\\
2075.6712945062	-0.0256967011457409\\
2085.25696176653	-0.0628467044419791\\
2094.88689664142	-0.0999679975441312\\
2104.56130356336	-0.137060278563956\\
2114.2803879089	-0.174123242263338\\
2124.04435600306	-0.211156580016705\\
2133.8534151237	-0.248159979773105\\
2143.70777350589	-0.285133126017966\\
2153.60764034637	-0.322075699734483\\
2163.55322580795	-0.358987378364669\\
2173.54474102401	-0.395867835770135\\
2183.58239810298	-0.432716742192456\\
2193.66641013278	-0.469533764213188\\
2203.79699118545	-0.506318564713697\\
2213.97435632161	-0.543070802834522\\
2224.19872159503	-0.579790133934459\\
2234.47030405729	-0.616476209549372\\
2244.78932176229	-0.653128677350657\\
2255.15599377096	-0.689747181103357\\
2265.57054015585	-0.726331360624109\\
2276.03318200585	-0.762880851738659\\
2286.54414143084	-0.799395286239111\\
2297.10364156645	-0.835874291841031\\
2307.71190657875	-0.872317492140089\\
2318.36916166905	-0.908724506568534\\
2329.07563307865	-0.945094950351455\\
2339.83154809368	-0.981428434462693\\
2350.63713504986	-1.01772456558059\\
2361.49262333744	-1.05398294604348\\
2372.39824340596	-1.09020317380498\\
2383.35422676926	-1.12638484238913\\
2394.36080601029	-1.16252754084517\\
2405.4182147861	-1.1986308537024\\
2416.52668783283	-1.23469436092466\\
2427.6864609706	-1.27071763786476\\
2438.8977711086	-1.30670025521883\\
2450.16085625011	-1.34264177898042\\
2461.4759554975	-1.37854177039463\\
2472.84330905736	-1.41439978591217\\
2484.26315824557	-1.45021537714326\\
2495.73574549243	-1.4859880908116\\
2507.26131434783	-1.52171746870831\\
2518.84010948637	-1.5574030476459\\
2530.4723767126	-1.59304435941221\\
2542.15836296621	-1.62864093072453\\
2553.8983163273	-1.66419228318377\\
2565.69248602162	-1.69969793322871\\
2577.54112242586	-1.73515739209053\\
2589.444477073	-1.77057016574744\\
2601.4028026576	-1.80593575487956\\
2613.41635304121	-1.8412536548241\\
2625.48538325773	-1.87652335553082\\
2637.61014951883	-1.91174434151787\\
2649.7909092194	-1.94691609182795\\
2662.02792094301	-1.98203807998501\\
2674.32144446738	-2.01710977395138\\
2686.67174076992	-2.05213063608538\\
2699.07907203326	-2.08710012309965\\
2711.54370165082	-2.12201768602004\\
2724.0658942324	-2.15688277014514\\
2736.64591560979	-2.19169481500673\\
2749.28403284242	-2.22645325433094\\
2761.98051422303	-2.26115751600031\\
2774.73562928336	-2.29580702201685\\
2787.54964879989	-2.33040118846614\\
2800.42284479955	-2.36493942548249\\
2813.35549056553	-2.39942113721532\\
2826.34786064309	-2.4338457217969\\
2839.40023084533	-2.46821257131127\\
2852.51287825912	-2.50252107176482\\
2865.68608125093	-2.53677060305833\\
2878.92011947275	-2.57096053896061\\
2892.21527386804	-2.60509024708405\\
2905.57182667769	-2.63915908886197\\
2918.99006144599	-2.6731664195279\\
2932.4702630267	-2.70711158809708\\
2946.01271758903	-2.74099393735022\\
2959.61771262377	-2.77481280381952\\
2973.28553694936	-2.80856751777728\\
2987.01648071805	-2.84225740322718\\
3000.81083542203	-2.87588177789846\\
3014.66889389963	-2.90943995324281\\
3028.59095034155	-2.94293123443481\\
3042.57730029708	-2.97635492037531\\
3056.6282406804	-3.00971030369858\\
3070.74406977686	-3.04299667078305\\
3084.92508724934	-3.07621330176598\\
3099.17159414457	-3.10935947056222\\
3113.48389289956	-3.14243444488737\\
3127.86228734801	-3.17543748628544\\
3142.30708272674	-3.20836785016132\\
3156.8185856822	-3.2412247858184\\
3171.39710427697	-3.27400753650134\\
3186.04294799627	-3.30671533944466\\
3200.75642775457	-3.33934742592699\\
3215.53785590219	-3.37190302133172\\
3230.38754623189	-3.40438134521396\\
3245.30581398558	-3.43678161137452\\
3260.29297586097	-3.46910302794087\\
3275.34935001834	-3.50134479745586\\
3290.47525608724	-3.53350611697412\\
3305.67101517332	-3.56558617816698\\
3320.93694986511	-3.59758416743597\\
3336.27338424092	-3.6294992660355\\
3351.68064387565	-3.66133065020518\\
3367.15905584778	-3.69307749131219\\
3382.70894874623	-3.72473895600417\\
3398.3306526774	-3.75631420637322\\
3414.02449927217	-3.78780240013152\\
3429.7908216929	-3.81920269079909\\
3445.62995464054	-3.85051422790431\\
3461.54223436172	-3.88173615719782\\
3477.5279986559	-3.91286762088043\\
3493.58758688252	-3.94390775784569\\
3509.72133996824	-3.97485570393785\\
3525.92960041413	-4.00571059222588\\
3542.21271230298	-4.03647155329443\\
3558.57102130658	-4.06713771555244\\
3575.00487469309	-4.09770820556013\\
3591.51462133436	-4.12818214837556\\
3608.1006117134	-4.15855866792129\\
3624.76319793175	-4.18883688737234\\
3641.50273371703	-4.21901592956637\\
3658.31957443038	-4.24909491743711\\
3675.21407707406	-4.27907297447209\\
3692.18660029898	-4.30894922519591\\
3709.23750441235	-4.33872279568005\\
3726.36715138533	-4.36839281408071\\
3743.57590486067	-4.39795841120565\\
3760.86413016048	-4.42741872111172\\
3778.23219429398	-4.45677288173416\\
3795.68046596523	-4.48602003554935\\
3813.20931558105	-4.51515933027255\\
3830.81911525882	-4.54418991959203\\
3848.51023883439	-4.57311096394143\\
3866.28306187004	-4.60192163131217\\
3884.13796166242	-4.63062109810744\\
3902.07531725059	-4.6592085500399\\
3920.09550942403	-4.68768318307506\\
3938.19892073078	-4.71604420442237\\
3956.38593548549	-4.74429083357598\\
3974.65693977764	-4.77242230340791\\
3993.0123214797	-4.80043786131539\\
4011.45247025537	-4.82833677042525\\
4029.97777756789	-4.85611831085778\\
4048.58863668828	-4.88378178105258\\
4067.28544270374	-4.91132649915943\\
4086.06859252603	-4.93875180449686\\
4104.93848489988	-4.96605705908157\\
4123.89552041149	-4.9932416492317\\
4142.94010149697	-5.02030498724739\\
4162.07263245094	-5.04724651317189\\
4181.29351943512	-5.07406569663687\\
4200.60317048688	-5.10076203879547\\
4220.00199552799	-5.12733507434714\\
4239.49040637325	-5.15378437365816\\
4259.06881673929	-5.18010954498177\\
4278.73764225331	-5.2063102367828\\
4298.49730046193	-5.23238614017059\\
4318.34821084004	-5.25833699144538\\
4338.2907947997	-5.28416257476282\\
4358.32547569911	-5.30986272492154\\
4378.45267885158	-5.33543733027919\\
4398.67283153454	-5.36088633580213\\
4418.98636299868	-5.38620974625463\\
4439.39370447695	-5.41140762953308\\
4459.89528919383	-5.43648012015151\\
4480.49155237446	-5.4614274228844\\
4501.18293125388	-5.48624981657335\\
4521.96986508636	-5.51094765810421\\
4542.85279515466	-5.53552138656144\\
4563.83216477945	-5.55997152756694\\
4584.90841932869	-5.58429869781046\\
4606.0820062271	-5.60850360977915\\
4627.35337496566	-5.6325870766941\\
4648.72297711113	-5.65655001766146\\
4670.19126631568	-5.68039346304666\\
4691.75869832646	-5.7041185600798\\
4713.42573099534	-5.72772657870083\\
4735.19282428857	-5.7512189176532\\
4757.06044029659	-5.77459711083495\\
4779.02904324381	-5.79786283391609\\
4801.09909949849	-5.82101791123173\\
4823.27107758263	-5.84406432296012\\
4845.54544818189	-5.86700421259484\\
4867.92268415562	-5.88983989472126\\
4890.4032605469	-5.91257386310597\\
4912.98765459258	-5.93520879910949\\
4935.67634573345	-5.95774758043145\\
4958.46981562442	-5.98019329019781\\
4981.36854814472	-6.00254922639952\\
5004.37302940819	-6.02481891169191\\
5027.48374777359	-6.04700610356372\\
5050.70119385497	-6.06911480488459\\
5074.02586053209	-6.0911492748394\\
5097.4582429609	-6.11311404025729\\
5120.998838584	-6.13501390734291\\
5144.64814714125	-6.15685397381662\\
5168.40667068035	-6.17863964146973\\
5192.27491356753	-6.20037662913994\\
5216.2533824982	-6.22207098611127\\
5240.34258650778	-6.2437291059416\\
5264.54303698246	-6.26535774071962\\
5288.85524767004	-6.28696401575154\\
5313.27973469088	-6.30855544467653\\
5337.81701654885	-6.33013994500761\\
5362.46761414231	-6.35172585409308\\
5387.23205077517	-6.37332194549118\\
5412.11085216805	-6.39493744574817\\
5437.10454646936	-6.41658205156713\\
5462.21366426659	-6.43826594735224\\
5487.43873859752	-6.45999982310929\\
5512.78030496154	-6.48179489268016\\
5538.23890133107	-6.50366291228474\\
5563.81506816292	-6.52561619933951\\
5589.50934840979	-6.54766765151738\\
5615.32228753178	-6.5698307660086\\
5641.254433508	-6.59211965893655\\
5667.30633684818	-6.61454908487718\\
5693.47855060436	-6.63713445642403\\
5719.77163038263	-6.6598918637347\\
5746.18613435493	-6.68283809398704\\
5772.72262327089	-6.70599065066633\\
5799.38166046975	-6.72936777259696\\
5826.16381189231	-6.75298845262321\\
5853.06964609293	-6.77687245583606\\
5880.09973425162	-6.80104033723326\\
5907.25465018618	-6.82551345869093\\
5934.53497036433	-6.85031400511551\\
5961.94127391598	-6.87546499963468\\
5989.47414264556	-6.9009903176766\\
6017.13416104428	-6.92691469977603\\
6044.92191630263	-6.95326376293667\\
6072.83799832281	-6.98006401036844\\
6100.88299973121	-7.00734283940928\\
6129.05751589107	-7.03512854743156\\
6157.36214491506	-7.06345033552415\\
6185.79748767801	-7.09233830973356\\
6214.36414782964	-7.12182347963983\\
6243.06273180741	-7.15193775403689\\
6271.89384884933	-7.18271393348208\\
6300.85811100697	-7.21418569947609\\
6329.95613315842	-7.24638760003288\\
6359.18853302131	-7.27935503139938\\
6388.55593116599	-7.31312421568741\\
6418.05895102864	-7.3477321741851\\
6447.69821892457	-7.38321669612328\\
6477.47436406142	-7.41961630268262\\
6507.38801855264	-7.45697020604229\\
6537.43981743082	-7.49531826328757\\
6567.63039866118	-7.53470092501611\\
6597.96040315515	-7.57515917850759\\
6628.43047478397	-7.61673448535051\\
6659.0412603923	-7.65946871345378\\
6689.79340981204	-7.70340406340864\\
6720.68757587607	-7.74858298920711\\
6751.7244144321	-7.79504811337031\\
6782.90458435663	-7.84284213658869\\
6814.22874756894	-7.89200774202937\\
6845.69756904508	-7.94258749452242\\
6877.31171683206	-7.9946237348964\\
6909.07186206199	-8.04815846979471\\
6940.97867896634	-8.10323325736616\\
6973.03284489025	-8.15988908928692\\
7005.23504030693	-8.21816626963272\\
7037.58594883204	-8.27810429118203\\
7070.0862572383	-8.3397417097904\\
7102.73665547	-8.40311601753082\\
7135.53783665763	-8.46826351534758\\
7168.49049713269	-8.53521918601587\\
7201.59533644237	-8.60401656823849\\
7234.85305736446	-8.67468763274281\\
7268.26436592224	-8.74726266126286\\
7301.82997139948	-8.82177012930485\\
7335.55058635552	-8.89823659359655\\
7369.42692664033	-8.97668658511244\\
7403.45971140979	-9.05714250854703\\
7437.64966314091	-9.13962454907628\\
7471.99750764715	-9.224150587205\\
7506.50397409388	-9.31073612244283\\
7541.16979501382	-9.39939420648638\\
7575.99570632259	-9.4901353865096\\
7610.98244733438	-9.58296765907994\\
7646.13076077758	-9.67789643512481\\
7681.44139281058	-9.7749245162739\\
7716.91509303763	-9.87405208279832\\
7752.55261452471	-9.9752766932594\\
7788.35471381553	-10.07859329587\\
7824.32215094763	-10.1839942514619\\
7860.45568946846	-10.2914693678438\\
7896.7560964516	-10.4010059452304\\
7933.22414251308	-10.5125888323234\\
7969.86060182773	-10.626200492531\\
8006.66625214554	-10.7418210797294\\
8043.6418748083	-10.8594285228922\\
8080.78825476606	-10.9789986188476\\
8118.10618059391	-11.1005051323699\\
8155.5964445086	-11.2239199027658\\
8193.25984238547	-11.3492129560858\\
8231.09717377527	-11.4763526220685\\
8269.10924192116	-11.6053056549187\\
8307.29685377578	-11.7360373570198\\
8345.66082001834	-11.8685117046948\\
8384.20195507185	-12.0026914751509\\
8422.92107712042	-12.1385383737762\\
8461.81900812665	-12.2760131609908\\
8500.89657384898	-12.4150757779055\\
8540.15460385935	-12.5556854700878\\
8579.59393156072	-12.6978009087926\\
8619.21539420481	-12.8413803090748\\
8659.01983290983	-12.986381544259\\
8699.0080926784	-13.1327622563083\\
8739.18102241541	-13.2804799616929\\
8779.5394749461	-13.4294921524259\\
8820.08430703417	-13.5797563919908\\
8860.81637939989	-13.7312304059461\\
8901.73655673847	-13.8838721670502\\
8942.84570773836	-14.0376399747994\\
8984.14470509972	-14.1924925293246\\
9025.63442555289	-14.3483889996379\\
9067.31574987707	-14.505289086261\\
9109.18956291901	-14.6631530783055\\
9151.25675361173	-14.8219419051105\\
9193.51821499347	-14.9816171825709\\
9235.97484422661	-15.1421412543146\\
9278.62754261669	-15.303477227911\\
9321.47721563161	-15.4655890063094\\
9364.5247729208	-15.6284413147193\\
9407.77112833455	-15.7919997231588\\
9451.21719994339	-15.956230664901\\
9494.86391005764	-16.121101451058\\
9538.71218524688	-16.286580281541\\
9582.76295635973	-16.4526362526379\\
9627.01715854358	-16.619239361447\\
9671.47573126437	-16.7863605074016\\
9716.13961832666	-16.9539714911178\\
9761.00976789355	-17.1220450107885\\
9806.08713250686	-17.2905546563411\\
9851.37266910737	-17.4594749015688\\
9896.86733905512	-17.6287810944357\\
9942.57210814977	-17.7984494457457\\
9988.48794665118	-17.9684570163586\\
10034.6158293	-18.1387817031233\\
10080.9567353382	-18.3094022236908\\
10127.5116485301	-18.4802981003573\\
10174.2815571832	-18.6514496430811\\
10221.267454169	-18.8228379318026\\
10268.4703369442	-18.9944447981914\\
10315.891207572	-19.1662528069335\\
10363.531072743	-19.3382452366609\\
10411.3909437969	-19.5104060606217\\
10459.4718367439	-19.6827199271748\\
10507.7747722864	-19.8551721401912\\
10556.3007758401	-20.0277486394314\\
10605.0508775566	-20.2004359809654\\
10654.0261123446	-20.3732213176925\\
10703.2275198921	-20.546092380013\\
10752.6561446888	-20.7190374566981\\
10802.3130360475	-20.892045375998\\
10852.1992481272	-21.0651054870249\\
10902.315839955	-21.2382076414405\\
10952.6638754485	-21.4113421754763\\
11003.244423439	-21.5844998923076\\
11054.0585576935	-21.7576720448017\\
11105.1073569377	-21.9308503186553\\
11156.3919048792	-22.1040268159332\\
11207.9132902301	-22.2771940390199\\
11259.6726067303	-22.4503448749897\\
11311.6709531708	-22.623472580402\\
11363.9094334169	-22.7965707665229\\
11416.3891564316	-22.9696333849765\\
11469.1112362992	-23.1426547138246\\
11522.0767922492	-23.315629344072\\
11575.2869486794	-23.4885521665966\\
11628.7428351806	-23.6614183594978\\
11682.4455865599	-23.8342233758593\\
11736.3963428651	-24.0069629319195\\
11790.596249409	-24.1796329956437\\
11845.0464567934	-24.3522297756891\\
11899.7481209338	-24.524749710756\\
11954.7024030838	-24.6971894593169\\
12009.9104698599	-24.8695458897122\\
12065.3734932659	-25.0418160706075\\
12121.0926507183	-25.2139972617996\\
12177.069125071	-25.3860869053635\\
12233.3041046402	-25.5580826171303\\
12289.7987832301	-25.7299821784877\\
12346.554360158	-25.9017835284901\\
12403.5720402798	-26.0734847562728\\
12460.8530340153	-26.2450840937578\\
12518.3985573745	-26.4165799086426\\
12576.2098319827	-26.5879706976628\\
12634.2880851072	-26.7592550801195\\
12692.6345496825	-26.9304317916606\\
12751.2504643373	-27.1014996783091\\
12810.1370734203	-27.2724576907288\\
12869.2956270265	-27.4433048787183\\
12928.7273810244	-27.6140403859251\\
12988.4335970818	-27.7846634447722\\
13048.4155426933	-27.9551733715883\\
13108.6744912069	-28.1255695619336\\
13169.2117218509	-28.2958514861148\\
13230.0285197614	-28.4660186848807\\
13291.1261760091	-28.6360707652915\\
13352.5059876274	-28.806007396755\\
13414.1692576393	-28.975828307223\\
13476.1172950852	-29.1455332795404\\
13538.351415051	-29.3151221479419\\
13600.8729386957	-29.4845947946897\\
13663.6831932795	-29.6539511468454\\
13726.7835121922	-29.8231911731716\\
13790.1752349813	-29.9923148811569\\
13853.8597073802	-30.1613223141584\\
13917.8382813373	-30.330213548658\\
13982.1123150444	-30.498988691626\\
14046.6831729655	-30.6676478779878\\
14111.552225866	-30.8361912681892\\
14176.7208508414	-31.0046190458547\\
14242.190431347	-31.172931415536\\
14307.9623572268	-31.3411286005453\\
14374.0380247435	-31.5092108408698\\
14440.4188366077	-31.6771783911635\\
14507.1062020079	-31.8450315188114\\
14574.1015366405	-32.0127705020654\\
14641.4062627396	-32.1803956282444\\
14709.0218091073	-32.3479071919982\\
14776.9496111443	-32.5153054936306\\
14845.1911108798	-32.6825908374793\\
14913.7477570027	-32.8497635303479\\
14982.6210048919	-33.0168238799897\\
15051.8123166476	-33.1837721936374\\
15121.323161122	-33.3506087765785\\
15191.1550139505	-33.5173339307725\\
15261.3093575835	-33.683947953507\\
15331.7876813171	-33.8504511360906\\
15402.5914813253	-34.0168437625817\\
15473.7222606918	-34.1831261085469\\
15545.1815294413	-34.3492984398511\\
15616.9708045722	-34.5153610114749\\
15689.0916100885	-34.6813140663565\\
15761.5454770323	-34.8471578342579\\
15834.333943516	-35.0128925306517\\
15907.4585547554	-35.1785183556276\\
15980.9208631021	-35.3440354928162\\
16054.7224280766	-35.509444108328\\
16128.8648164017	-35.6747443497065\\
16203.3496020352	-35.8399363448926\\
16278.1783662036	-36.0050202011994\\
16353.352697436	-36.1699960042954\\
16428.8741915971	-36.3348638171942\\
16504.7444519217	-36.4996236792488\\
16580.9650890484	-36.6642756051503\\
16657.537721054	-36.8288195839276\\
16734.4639734876	-36.9932555779466\\
16811.7454794054	-37.1575835219095\\
16889.3838794052	-37.3218033218501\\
16967.3808216612	-37.4859148541249\\
17045.7379619589	-37.6499179643981\\
17124.4569637308	-37.813812466619\\
17203.539498091	-37.9775981419906\\
17282.9872438709	-38.1412747379273\\
17362.8018876552	-38.3048419670012\\
17442.9851238172	-38.468299505874\\
17523.5386545551	-38.6316469942148\\
17604.464189928	-38.7948840336008\\
17685.7634478922	-38.9580101864002\\
17767.4381543379	-39.1210249746355\\
17849.4900431252	-39.2839278788263\\
17931.9208561219	-39.4467183368091\\
18014.7323432394	-39.6093957425329\\
18097.9262624709	-39.77195944483\\
18181.5043799277	-39.9344087461589\\
18265.4684698776	-40.0967429013188\\
18349.8203147818	-40.258961116134\\
18434.5617053333	-40.4210625461062\\
18519.6944404947	-40.5830462950331\\
18605.2203275364	-40.7449114135922\\
18691.1411820748	-40.9066568978873\\
18777.4588281113	-41.0682816879568\\
18864.1750980704	-41.2297846662413\\
18951.2918328392	-41.3911646560097\\
19038.810881806	-41.5524204197404\\
19126.7341029	-41.7135506574582\\
19215.0633626303	-41.8745540050216\\
19303.8005361259	-42.035429032363\\
19392.9475071751	-42.196174241675\\
19482.506168266	-42.3567880655453\\
19572.4784206263	-42.5172688650347\\
19662.8661742637	-42.6776149276981\\
19753.6713480067	-42.8378244655457\\
19844.8958695449	-42.9978956129426\\
19936.5416754703	-43.1578264244434\\
20028.6107113184	-43.3176148725622\\
20121.1049316092	-43.4772588454719\\
20214.026299889	-43.6367561446338\\
20307.3767887718	-43.7961044823533\\
20401.1583799817	-43.9553014792598\\
20495.373064394	-44.1143446617079\\
20590.0228420787	-44.2732314590978\\
20685.1097223421	-44.4319592011122\\
20780.6357237695	-44.5905251148658\\
20876.6028742684	-44.7489263219669\\
20973.0132111115	-44.9071598354863\\
21069.8687809795	-45.0652225568311\\
21167.1716400053	-45.223111272522\\
21264.923853817	-45.3808226508678\\
21363.127497582	-45.5383532385378\\
21461.7846560511	-45.6956994570262\\
21560.8974236026	-45.8528575990066\\
21660.467904287	-46.0098238245736\\
21760.4982118713	-46.166594157367\\
21860.9904698845	-46.3231644805764\\
21961.9468116618	-46.479530532822\\
22063.3693803907	-46.6356879039093\\
22165.260329156	-46.7916320304525\\
22267.6218209858	-46.947358191365\\
22370.4560288971	-47.1028615032112\\
22473.7651359423	-47.2581369154188\\
22577.5513352553	-47.4131792053447\\
22681.8168300982	-47.5679829731935\\
22786.5638339077	-47.7225426367836\\
22891.7945703429	-47.876852426158\\
22997.5112733313	-48.030906378034\\
23103.7161871177	-48.1846983300912\\
23210.4115663104	-48.3382219150916\\
23317.5996759301	-48.4914705548295\\
23425.2827914574	-48.6444374539059\\
23533.4631988815	-48.7971155933265\\
23642.1431947482	-48.9494977239159\\
23751.3250862094	-49.1015763595487\\
23861.0111910714	-49.2533437701903\\
23971.2038378446	-49.4047919747463\\
24081.9053657923	-49.5559127337165\\
24193.1181249812	-49.7066975416509\\
24304.8444763306	-49.8571376194027\\
24417.0867916629	-50.0072239061792\\
24529.8474537537	-50.156947051384\\
24643.1288563827	-50.306297406251\\
24756.933404384	-50.4552650152662\\
24871.2635136979	-50.6038396073767\\
24986.1216114214	-50.752010586985\\
25101.5101358603	-50.8997670247254\\
25217.4315365806	-51.0470976480259\\
25333.8882744609	-51.1939908314505\\
25450.882821744	-51.3404345868247\\
25568.4176620901	-51.4864165531441\\
25686.4952906292	-51.6319239862664\\
25805.1182140139	-51.7769437483892\\
25924.2889504728	-51.9214622973149\\
26044.0100298642	-52.0654656755072\\
26164.2839937291	-52.2089394989407\\
26285.113395346	-52.3518689457497\\
26406.5007997846	-52.4942387446807\\
26528.4487839603	-52.636033163355\\
26650.959936689	-52.7772359963484\\
26774.0368587419	-52.9178305530967\\
26897.6821629011	-53.057799645636\\
27021.8984740145	-53.197125576188\\
27146.6884290521	-53.3357901246034\\
27272.0546771616	-53.4737745356747\\
27397.9998797246	-53.6110595063351\\
27524.5267104134	-53.7476251727581\\
27651.6378552475	-53.8834510973767\\
27779.3360126509	-54.0185162558402\\
27907.623893509	-54.1527990239319\\
28036.5042212264	-54.2862771644687\\
28165.9797317848	-54.418927814209\\
28296.0531738006	-54.550727470796\\
28426.7273085842	-54.6816519797649\\
28558.0049101974	-54.8116765216457\\
28689.8887655133	-54.9407755991959\\
28822.3816742749	-55.0689230247982\\
28955.4864491548	-55.1960919080631\\
29089.2059158146	-55.3222546436758\\
29223.5429129655	-55.4473828995328\\
29358.5002924277	-55.5714476052124\\
29494.0809191919	-55.6944189408305\\
29630.2876714792	-55.8162663263303\\
29767.1234408028	-55.9369584112632\\
29904.5911320291	-56.0564630651155\\
30042.6936634399	-56.1747473682418\\
30181.4339667932	-56.2917776034673\\
30320.8149873869	-56.4075192484233\\
30460.8396841202	-56.5219369686842\\
30601.5110295567	-56.6349946117752\\
30742.8320099879	-56.7466552021232\\
30884.8056254962	-56.8568809370254\\
31027.4348900188	-56.9656331837112\\
31170.7228314113	-57.0728724775758\\
31314.6724915125	-57.1785585216655\\
31459.2869262085	-57.2826501874941\\
31604.5692054981	-57.3851055172733\\
31750.5224135574	-57.4858817276381\\
};
\addlegendentry{G(z)};

\end{axis}
\end{tikzpicture}%}
    \end{figure}
    \end{small}
    \vfill
  }

  \frame{
    \frametitle{\insertsection}
    \vfill
    \begin{small}
    \begin{figure}[htb]
      \centering
      \raisebox{-0.5\height}{
        \def\svgwidth{0.5\textwidth}
        % This file was created by matlab2tikz v0.4.7 running on MATLAB 8.3.
% Copyright (c) 2008--2014, Nico Schlömer <nico.schloemer@gmail.com>
% All rights reserved.
% Minimal pgfplots version: 1.3
% 
% The latest updates can be retrieved from
%   http://www.mathworks.com/matlabcentral/fileexchange/22022-matlab2tikz
% where you can also make suggestions and rate matlab2tikz.
% 
%
% defining custom colors
\definecolor{mycolor1}{rgb}{0.66667,0.66667,0.66667}%
%
\begin{tikzpicture}

\begin{axis}[%
width=0.8\textwidth,
height=0.461611624834875\textwidth,
scale only axis,
xmin=-1.9,
xmax=1.9,
xtick={-1,  0,  1},
xlabel={Eixo Real},
ymin=-1.1,
ymax=1.1,
ytick={-1,  0,  1},
ylabel={Eixo Imaginário}
]
\addplot [color=mycolor1,dotted,line width=1.0pt,forget plot]
  table[row sep=crcr]{1	0\\
0.987688340595138	0.156434465040231\\
0.951056516295154	0.309016994374947\\
0.891006524188368	0.453990499739547\\
0.809016994374947	0.587785252292473\\
0.707106781186548	0.707106781186547\\
0.587785252292473	0.809016994374947\\
0.453990499739547	0.891006524188368\\
0.309016994374947	0.951056516295154\\
0.156434465040231	0.987688340595138\\
6.12323399573677e-17	1\\
-0.156434465040231	0.987688340595138\\
-0.309016994374947	0.951056516295154\\
-0.453990499739547	0.891006524188368\\
-0.587785252292473	0.809016994374947\\
-0.707106781186547	0.707106781186548\\
-0.809016994374947	0.587785252292473\\
-0.891006524188368	0.453990499739547\\
-0.951056516295154	0.309016994374948\\
-0.987688340595138	0.156434465040231\\
-1	1.22464679914735e-16\\
};
\addplot [color=mycolor1,dotted,line width=1.0pt,forget plot]
  table[row sep=crcr]{1	0\\
0.972218045703082	0.153984211042097\\
0.921496791140463	0.299412457456957\\
0.849791018366557	0.432990150609548\\
0.759508516401756	0.551815237548133\\
0.653437051516106	0.653437051516106\\
0.534664304371591	0.735902282058355\\
0.406492952796512	0.797787339572243\\
0.272353130096494	0.83821674479613\\
0.135714483753824	0.856867527364047\\
5.22899666167123e-17	0.853959960588124\\
-0.131496350792703	0.830235283991667\\
-0.255686250369537	0.786921363444393\\
-0.369756251949576	0.725687504552447\\
-0.47122811850853	0.648589862732789\\
-0.558009078759514	0.558009078759514\\
-0.628431083783263	0.456581908289041\\
-0.681278445058817	0.347128705949459\\
-0.715803538350409	0.232578668243255\\
-0.731730559897045	0.115894735197653\\
-0.729247614287671	8.9307075662324e-17\\
};
\addplot [color=mycolor1,dotted,line width=1.0pt,forget plot]
  table[row sep=crcr]{1	0\\
0.956521682769877	0.151498151383791\\
0.891982039736211	0.289822533396298\\
0.809292583610082	0.412355167436434\\
0.711634858347723	0.517032989000692\\
0.602364630186427	0.602364630186426\\
0.484917701774751	0.667431957639199\\
0.362719588349683	0.711877274647591\\
0.239101059900595	0.735877395777214\\
0.117221283062812	0.740106053489981\\
4.44354699422903e-17	0.725686295399261\\
-0.109940132237539	0.694134676438339\\
-0.210320240583588	0.647299141978847\\
-0.299240493845688	0.587292536894097\\
-0.375203754387119	0.516423664031337\\
-0.437127756959533	0.437127756959533\\
-0.484346224770267	0.351898130574443\\
-0.516599582217932	0.263220634338226\\
-0.534016141209622	0.173512362385299\\
-0.537084820159711	0.0850658786478001\\
-0.526620599330303	6.44924231334916e-17\\
};
\addplot [color=mycolor1,dotted,line width=1.0pt,forget plot]
  table[row sep=crcr]{1	0\\
0.940082788364644	0.148894486293864\\
0.86158608093073	0.279946287694513\\
0.768279786681378	0.391458103646313\\
0.663960650859636	0.482395649771645\\
0.552351927561387	0.552351927561387\\
0.437014383712816	0.601498696728173\\
0.321269860940431	0.630527604183869\\
0.208138182971344	0.640583459188477\\
0.100287778328637	0.633192112325829\\
3.73630739569174e-17	0.610185303761559\\
-0.0908532273476425	0.573624701779294\\
-0.170819110593797	0.525727164509449\\
-0.238861981883349	0.468793035010043\\
-0.294350736089166	0.40513903143051\\
-0.337037028702328	0.337037028702328\\
-0.367025664975262	0.266659754473976\\
-0.384738677688982	0.196034147687371\\
-0.390874612629551	0.127002860400749\\
-0.386364521398959	0.0611941284830315\\
-0.372326104926586	4.55967972637346e-17\\
};
\addplot [color=mycolor1,dotted,line width=1.0pt,forget plot]
  table[row sep=crcr]{1	0\\
0.922246029428501	0.146069421212559\\
0.829201462983264	0.269423887468404\\
0.725373165529273	0.369596088217499\\
0.614985835074999	0.446813363302878\\
0.501902475185001	0.501902475185001\\
0.38956496084428	0.536190168955128\\
0.280953750703967	0.551402782692681\\
0.178565395716864	0.549567778706978\\
0.0844061798404073	0.532919645815306\\
3.08495992433718e-17	0.50381218919366\\
-0.0735915548753502	0.464638791061534\\
-0.135739038566523	0.417761804348184\\
-0.186207014322272	0.365451842507638\\
-0.225110018726605	0.309837359892227\\
-0.252864584784672	0.252864584784672\\
-0.270139142845401	0.196267575756099\\
-0.277803443075876	0.141547924204336\\
-0.27687894570066	0.0899634229303457\\
-0.268491413422519	0.0425248622468694\\
-0.253826721980109	3.10848082611005e-17\\
};
\addplot [color=mycolor1,dotted,line width=1.0pt,forget plot]
  table[row sep=crcr]{1	0\\
0.902056570675584	0.142871725087523\\
0.793293726869509	0.257756756757911\\
0.678769666307395	0.345850419328459\\
0.562876391436159	0.408953636388562\\
0.449318435861499	0.449318435861499\\
0.341115747349381	0.469505547439701\\
0.24062663375655	0.472256359314945\\
0.149586755320764	0.460380694230177\\
0.069160331541479	0.436661148025441\\
2.47240337905553e-17	0.403774113610049\\
-0.057687904157153	0.364227092250654\\
-0.104075505986926	0.320311471399148\\
-0.139645446506012	0.274069620358657\\
-0.165124892040398	0.227274916024158\\
-0.18142315316863	0.18142315316863\\
-0.189574186252466	0.137733708524426\\
-0.190684892879101	0.0971588057560207\\
-0.185889806735969	0.0603992595374446\\
-0.176312467991919	0.0279251515651378\\
-0.16303353482158	1.99658496572927e-17\\
};
\addplot [color=mycolor1,dotted,line width=1.0pt,forget plot]
  table[row sep=crcr]{1	0\\
0.877921760602431	0.139049146701748\\
0.751411952540787	0.244148543365328\\
0.625732257688895	0.318826509862098\\
0.505011411193747	0.36691226736026\\
0.392341669548685	0.392341669548685\\
0.289890514921595	0.399000063654162\\
0.199020572856858	0.390599867100522\\
0.120411913496453	0.370589763847805\\
0.0541820486548744	0.34209199176291\\
1.88512313530119e-17	0.307863971328499\\
-0.042808222513439	0.270280479734772\\
-0.075164540154518	0.231332667812908\\
-0.0981551954909503	0.192640417840451\\
-0.112959067758674	0.155474818616317\\
-0.120787864504912	0.120787864504912\\
-0.122837744156894	0.0892468451742257\\
-0.120251626285951	0.0612712639357851\\
-0.114091236431876	0.0370704898842818\\
-0.105317799143806	0.0166807006736036\\
-0.0947802248421549	1.16072298975411e-17\\
};
\addplot [color=mycolor1,dotted,line width=1.0pt,forget plot]
  table[row sep=crcr]{1	0\\
0.84674396664984	0.134111069256013\\
0.698989566160914	0.227115477506975\\
0.561406498364893	0.286050898428465\\
0.437004973478602	0.317502698182059\\
0.327450698344114	0.327450698344114\\
0.233352139914598	0.321181666481713\\
0.154515499860225	0.303253743289057\\
0.0901653834921457	0.27750051639687\\
0.0391311034995635	0.247064063991275\\
1.31311479217367e-17	0.214447919692096\\
-0.0287598398096346	0.181582482159892\\
-0.0487044416812479	0.149896858349689\\
-0.0613430200940395	0.120392455675976\\
-0.06808779312177	0.0937148074575858\\
-0.0702211210616195	0.0702211210616195\\
-0.068876740220244	0.0500418809603846\\
-0.0650321343871793	0.0331355275052096\\
-0.0595094084547913	0.0193357789181307\\
-0.0529823745536039	0.00839158374073751\\
-0.0459879102602678	5.63189470997126e-18\\
};
\addplot [color=mycolor1,dotted,line width=1.0pt,forget plot]
  table[row sep=crcr]{1	0\\
0.801053465278425	0.126874404768154\\
0.625589539649299	0.203266363189334\\
0.475341369738971	0.242198525079547\\
0.350045057714404	0.254322621147605\\
0.24813777530853	0.24813777530853\\
0.167289292234614	0.230253957320141\\
0.104794333468008	0.205670459781722\\
0.0578515468028918	0.178048753195384\\
0.0237523524567972	0.149966451301199\\
7.54043881219821e-18	0.123144711070133\\
-0.0156239119173694	0.0986454975334405\\
-0.0250311775652273	0.0770380431086105\\
-0.0298254673925332	0.0585357756361869\\
-0.0313184856998208	0.0431061974937683\\
-0.0305568546459545	0.0305568546459545\\
-0.0283545570469435	0.0206007915573586\\
-0.0253272293454815	0.012904867916705\\
-0.021925762268643	0.00712411201600239\\
-0.0184675753156993	0.00292497658052826\\
-0.0151646198645466	1.85713031774033e-18\\
};
\addplot [color=mycolor1,dotted,line width=1.0pt,forget plot]
  table[row sep=crcr]{1	0\\
0.714110955679367	0.113104064044896\\
0.497161927717827	0.161537702532642\\
0.336758208677056	0.171586877647116\\
0.221075553032581	0.160620791180475\\
0.139705610200823	0.139705610200823\\
0.0839640345306934	0.115566579097864\\
0.0468885871776745	0.0920240337831981\\
0.0230753615892199	0.0710186604775083\\
0.00844586939409394	0.0533251206797082\\
2.39022368106624e-18	0.0390353150431684\\
-0.00441505277265522	0.0278755461307222\\
-0.00630567510971132	0.0194068724759343\\
-0.00669794782622876	0.0131454627690819\\
-0.0062698831862128	0.00862975386096949\\
-0.0054534525074872	0.0054534525074872\\
-0.00451117782354635	0.00327756254020109\\
-0.00359218715027031	0.00183031077240959\\
-0.00277223578568335	0.000900754009370227\\
-0.00208156288544739	0.000329687172611911\\
-0.00152375582051941	1.86606268828125e-19\\
};
\addplot [color=mycolor1,dotted,line width=1.0pt,forget plot]
  table[row sep=crcr]{1	-0\\
0.987688340595138	-0.156434465040231\\
0.951056516295154	-0.309016994374947\\
0.891006524188368	-0.453990499739547\\
0.809016994374947	-0.587785252292473\\
0.707106781186548	-0.707106781186547\\
0.587785252292473	-0.809016994374947\\
0.453990499739547	-0.891006524188368\\
0.309016994374947	-0.951056516295154\\
0.156434465040231	-0.987688340595138\\
6.12323399573677e-17	-1\\
-0.156434465040231	-0.987688340595138\\
-0.309016994374947	-0.951056516295154\\
-0.453990499739547	-0.891006524188368\\
-0.587785252292473	-0.809016994374947\\
-0.707106781186547	-0.707106781186548\\
-0.809016994374947	-0.587785252292473\\
-0.891006524188368	-0.453990499739547\\
-0.951056516295154	-0.309016994374948\\
-0.987688340595138	-0.156434465040231\\
-1	-1.22464679914735e-16\\
};
\addplot [color=mycolor1,dotted,line width=1.0pt,forget plot]
  table[row sep=crcr]{1	-0\\
0.972218045703082	-0.153984211042097\\
0.921496791140463	-0.299412457456957\\
0.849791018366557	-0.432990150609548\\
0.759508516401756	-0.551815237548133\\
0.653437051516106	-0.653437051516106\\
0.534664304371591	-0.735902282058355\\
0.406492952796512	-0.797787339572243\\
0.272353130096494	-0.83821674479613\\
0.135714483753824	-0.856867527364047\\
5.22899666167123e-17	-0.853959960588124\\
-0.131496350792703	-0.830235283991667\\
-0.255686250369537	-0.786921363444393\\
-0.369756251949576	-0.725687504552447\\
-0.47122811850853	-0.648589862732789\\
-0.558009078759514	-0.558009078759514\\
-0.628431083783263	-0.456581908289041\\
-0.681278445058817	-0.347128705949459\\
-0.715803538350409	-0.232578668243255\\
-0.731730559897045	-0.115894735197653\\
-0.729247614287671	-8.9307075662324e-17\\
};
\addplot [color=mycolor1,dotted,line width=1.0pt,forget plot]
  table[row sep=crcr]{1	-0\\
0.956521682769877	-0.151498151383791\\
0.891982039736211	-0.289822533396298\\
0.809292583610082	-0.412355167436434\\
0.711634858347723	-0.517032989000692\\
0.602364630186427	-0.602364630186426\\
0.484917701774751	-0.667431957639199\\
0.362719588349683	-0.711877274647591\\
0.239101059900595	-0.735877395777214\\
0.117221283062812	-0.740106053489981\\
4.44354699422903e-17	-0.725686295399261\\
-0.109940132237539	-0.694134676438339\\
-0.210320240583588	-0.647299141978847\\
-0.299240493845688	-0.587292536894097\\
-0.375203754387119	-0.516423664031337\\
-0.437127756959533	-0.437127756959533\\
-0.484346224770267	-0.351898130574443\\
-0.516599582217932	-0.263220634338226\\
-0.534016141209622	-0.173512362385299\\
-0.537084820159711	-0.0850658786478001\\
-0.526620599330303	-6.44924231334916e-17\\
};
\addplot [color=mycolor1,dotted,line width=1.0pt,forget plot]
  table[row sep=crcr]{1	-0\\
0.940082788364644	-0.148894486293864\\
0.86158608093073	-0.279946287694513\\
0.768279786681378	-0.391458103646313\\
0.663960650859636	-0.482395649771645\\
0.552351927561387	-0.552351927561387\\
0.437014383712816	-0.601498696728173\\
0.321269860940431	-0.630527604183869\\
0.208138182971344	-0.640583459188477\\
0.100287778328637	-0.633192112325829\\
3.73630739569174e-17	-0.610185303761559\\
-0.0908532273476425	-0.573624701779294\\
-0.170819110593797	-0.525727164509449\\
-0.238861981883349	-0.468793035010043\\
-0.294350736089166	-0.40513903143051\\
-0.337037028702328	-0.337037028702328\\
-0.367025664975262	-0.266659754473976\\
-0.384738677688982	-0.196034147687371\\
-0.390874612629551	-0.127002860400749\\
-0.386364521398959	-0.0611941284830315\\
-0.372326104926586	-4.55967972637346e-17\\
};
\addplot [color=mycolor1,dotted,line width=1.0pt,forget plot]
  table[row sep=crcr]{1	-0\\
0.922246029428501	-0.146069421212559\\
0.829201462983264	-0.269423887468404\\
0.725373165529273	-0.369596088217499\\
0.614985835074999	-0.446813363302878\\
0.501902475185001	-0.501902475185001\\
0.38956496084428	-0.536190168955128\\
0.280953750703967	-0.551402782692681\\
0.178565395716864	-0.549567778706978\\
0.0844061798404073	-0.532919645815306\\
3.08495992433718e-17	-0.50381218919366\\
-0.0735915548753502	-0.464638791061534\\
-0.135739038566523	-0.417761804348184\\
-0.186207014322272	-0.365451842507638\\
-0.225110018726605	-0.309837359892227\\
-0.252864584784672	-0.252864584784672\\
-0.270139142845401	-0.196267575756099\\
-0.277803443075876	-0.141547924204336\\
-0.27687894570066	-0.0899634229303457\\
-0.268491413422519	-0.0425248622468694\\
-0.253826721980109	-3.10848082611005e-17\\
};
\addplot [color=mycolor1,dotted,line width=1.0pt,forget plot]
  table[row sep=crcr]{1	-0\\
0.902056570675584	-0.142871725087523\\
0.793293726869509	-0.257756756757911\\
0.678769666307395	-0.345850419328459\\
0.562876391436159	-0.408953636388562\\
0.449318435861499	-0.449318435861499\\
0.341115747349381	-0.469505547439701\\
0.24062663375655	-0.472256359314945\\
0.149586755320764	-0.460380694230177\\
0.069160331541479	-0.436661148025441\\
2.47240337905553e-17	-0.403774113610049\\
-0.057687904157153	-0.364227092250654\\
-0.104075505986926	-0.320311471399148\\
-0.139645446506012	-0.274069620358657\\
-0.165124892040398	-0.227274916024158\\
-0.18142315316863	-0.18142315316863\\
-0.189574186252466	-0.137733708524426\\
-0.190684892879101	-0.0971588057560207\\
-0.185889806735969	-0.0603992595374446\\
-0.176312467991919	-0.0279251515651378\\
-0.16303353482158	-1.99658496572927e-17\\
};
\addplot [color=mycolor1,dotted,line width=1.0pt,forget plot]
  table[row sep=crcr]{1	-0\\
0.877921760602431	-0.139049146701748\\
0.751411952540787	-0.244148543365328\\
0.625732257688895	-0.318826509862098\\
0.505011411193747	-0.36691226736026\\
0.392341669548685	-0.392341669548685\\
0.289890514921595	-0.399000063654162\\
0.199020572856858	-0.390599867100522\\
0.120411913496453	-0.370589763847805\\
0.0541820486548744	-0.34209199176291\\
1.88512313530119e-17	-0.307863971328499\\
-0.042808222513439	-0.270280479734772\\
-0.075164540154518	-0.231332667812908\\
-0.0981551954909503	-0.192640417840451\\
-0.112959067758674	-0.155474818616317\\
-0.120787864504912	-0.120787864504912\\
-0.122837744156894	-0.0892468451742257\\
-0.120251626285951	-0.0612712639357851\\
-0.114091236431876	-0.0370704898842818\\
-0.105317799143806	-0.0166807006736036\\
-0.0947802248421549	-1.16072298975411e-17\\
};
\addplot [color=mycolor1,dotted,line width=1.0pt,forget plot]
  table[row sep=crcr]{1	-0\\
0.84674396664984	-0.134111069256013\\
0.698989566160914	-0.227115477506975\\
0.561406498364893	-0.286050898428465\\
0.437004973478602	-0.317502698182059\\
0.327450698344114	-0.327450698344114\\
0.233352139914598	-0.321181666481713\\
0.154515499860225	-0.303253743289057\\
0.0901653834921457	-0.27750051639687\\
0.0391311034995635	-0.247064063991275\\
1.31311479217367e-17	-0.214447919692096\\
-0.0287598398096346	-0.181582482159892\\
-0.0487044416812479	-0.149896858349689\\
-0.0613430200940395	-0.120392455675976\\
-0.06808779312177	-0.0937148074575858\\
-0.0702211210616195	-0.0702211210616195\\
-0.068876740220244	-0.0500418809603846\\
-0.0650321343871793	-0.0331355275052096\\
-0.0595094084547913	-0.0193357789181307\\
-0.0529823745536039	-0.00839158374073751\\
-0.0459879102602678	-5.63189470997126e-18\\
};
\addplot [color=mycolor1,dotted,line width=1.0pt,forget plot]
  table[row sep=crcr]{1	-0\\
0.801053465278425	-0.126874404768154\\
0.625589539649299	-0.203266363189334\\
0.475341369738971	-0.242198525079547\\
0.350045057714404	-0.254322621147605\\
0.24813777530853	-0.24813777530853\\
0.167289292234614	-0.230253957320141\\
0.104794333468008	-0.205670459781722\\
0.0578515468028918	-0.178048753195384\\
0.0237523524567972	-0.149966451301199\\
7.54043881219821e-18	-0.123144711070133\\
-0.0156239119173694	-0.0986454975334405\\
-0.0250311775652273	-0.0770380431086105\\
-0.0298254673925332	-0.0585357756361869\\
-0.0313184856998208	-0.0431061974937683\\
-0.0305568546459545	-0.0305568546459545\\
-0.0283545570469435	-0.0206007915573586\\
-0.0253272293454815	-0.012904867916705\\
-0.021925762268643	-0.00712411201600239\\
-0.0184675753156993	-0.00292497658052826\\
-0.0151646198645466	-1.85713031774033e-18\\
};
\addplot [color=mycolor1,dotted,line width=1.0pt,forget plot]
  table[row sep=crcr]{1	-0\\
0.714110955679367	-0.113104064044896\\
0.497161927717827	-0.161537702532642\\
0.336758208677056	-0.171586877647116\\
0.221075553032581	-0.160620791180475\\
0.139705610200823	-0.139705610200823\\
0.0839640345306934	-0.115566579097864\\
0.0468885871776745	-0.0920240337831981\\
0.0230753615892199	-0.0710186604775083\\
0.00844586939409394	-0.0533251206797082\\
2.39022368106624e-18	-0.0390353150431684\\
-0.00441505277265522	-0.0278755461307222\\
-0.00630567510971132	-0.0194068724759343\\
-0.00669794782622876	-0.0131454627690819\\
-0.0062698831862128	-0.00862975386096949\\
-0.0054534525074872	-0.0054534525074872\\
-0.00451117782354635	-0.00327756254020109\\
-0.00359218715027031	-0.00183031077240959\\
-0.00277223578568335	-0.000900754009370227\\
-0.00208156288544739	-0.000329687172611911\\
-0.00152375582051941	-1.86606268828125e-19\\
};
\addplot [color=mycolor1,dotted,line width=1.0pt,forget plot]
  table[row sep=crcr]{1	0\\
1	0\\
};
\addplot [color=mycolor1,dotted,line width=1.0pt,forget plot]
  table[row sep=crcr]{0.951056516295154	0.309016994374947\\
0.906577591518048	0.290693092532446\\
0.867277205719189	0.267124444629623\\
0.833330533437629	0.239553777474139\\
0.804679893747523	0.20903820245934\\
0.781122098933164	0.176433510213537\\
0.762382187756933	0.142402376999381\\
0.748171074803624	0.107437628337747\\
0.738227708485823	0.0718935522249012\\
0.732347931289196	0.0360204899377399\\
0.730402691048646	2.81010850277162e-17\\
};
\addplot [color=mycolor1,dotted,line width=1.0pt,forget plot]
  table[row sep=crcr]{0.809016994374947	0.587785252292473\\
0.737380455396588	0.527071687397996\\
0.6808142826414	0.463341883835339\\
0.637053765657314	0.399254954339046\\
0.603812761314093	0.336417677088311\\
0.579022949915482	0.275632227640288\\
0.560948163233974	0.217130071437151\\
0.54821711318997	0.160763451735608\\
0.539811466724715	0.106147624627789\\
0.535036016768209	0.0527590625798542\\
0.533488091091103	4.10502162512614e-17\\
};
\addplot [color=mycolor1,dotted,line width=1.0pt,forget plot]
  table[row sep=crcr]{0.587785252292473	0.809016994374947\\
0.515276498489902	0.692182785870847\\
0.46668476526979	0.583707991451876\\
0.435233321876477	0.485319980094308\\
0.415551842123536	0.396928474901319\\
0.403656860517901	0.317461475735791\\
0.396737049613855	0.245416450708029\\
0.392888142823216	0.179177710929467\\
0.39087245229983	0.11717008156476\\
0.389932112762627	0.0579102497954412\\
0.389661137375347	4.49747826270753e-17\\
};
\addplot [color=mycolor1,dotted,line width=1.0pt,forget plot]
  table[row sep=crcr]{0.309016994374947	0.951056516295154\\
0.265925372344309	0.777304721760365\\
0.248822386132444	0.630899544522142\\
0.24643298177389	0.508693744238057\\
0.251412997268255	0.406266573115132\\
0.25929445161488	0.31919477108011\\
0.267517373913267	0.243597429511063\\
0.274697115780397	0.1762665508339\\
0.280129101393366	0.114599409879343\\
0.283480020554887	0.0564559973822998\\
0.284609543336029	4.37996030135249e-17\\
};
\addplot [color=mycolor1,dotted,line width=1.0pt,forget plot]
  table[row sep=crcr]{6.12323399573677e-17	1\\
0.0151248701748503	0.781989711398728\\
0.0472692933177679	0.613631335769719\\
0.0835006201485716	0.482943980920436\\
0.117381749765263	0.379469463911326\\
0.146223972383669	0.295078319831894\\
0.169261627793672	0.223809451176491\\
0.186582776182006	0.161390341419992\\
0.198560105942714	0.104739935929793\\
0.205572433929556	0.0515565221197431\\
0.207879576350762	3.99891848849262e-17\\
};
\addplot [color=mycolor1,dotted,line width=1.0pt,forget plot]
  table[row sep=crcr]{-0.309016994374947	0.951056516295154\\
-0.213607139159912	0.713331044437031\\
-0.122920349149865	0.544815253953639\\
-0.0461074386071066	0.422454854218942\\
0.0151311193047769	0.329888717873058\\
0.0621570724668225	0.256291005261778\\
0.0971710522581798	0.194731597161215\\
0.122256440677152	0.140793596164787\\
0.13905244595299	0.0915971142347777\\
0.148693455532156	0.0451621321066958\\
0.151835801980649	3.50498499033503e-17\\
};
\addplot [color=mycolor1,dotted,line width=1.0pt,forget plot]
  table[row sep=crcr]{-0.587785252292473	0.809016994374948\\
-0.401011853057454	0.584595820351374\\
-0.252139489074835	0.439670821081765\\
-0.139623392550337	0.340999317931598\\
-0.0567836371213516	0.268617800427267\\
0.0033339412143336	0.21116115844769\\
0.0463512370945915	0.162297289886265\\
0.0763422025660035	0.118472638203443\\
0.0960671266193367	0.0776165020305757\\
0.1072685824301	0.0384304051397518\\
0.110901278364195	2.98672554719679e-17\\
};
\addplot [color=mycolor1,dotted,line width=1.0pt,forget plot]
  table[row sep=crcr]{-0.809016994374948	0.587785252292473\\
-0.533486326814499	0.413410095118228\\
-0.336121655437603	0.313963860155743\\
-0.198040110920963	0.250717832404618\\
-0.101844033235305	0.204281393673556\\
-0.0346516892466227	0.165530866251108\\
0.0122258376810533	0.130333089269644\\
0.0443886084751889	0.0968398262452624\\
0.065339288702761	0.0642052594194206\\
0.0771736424133687	0.032008294596762\\
0.0810025921579431	2.49315700239574e-17\\
};
\addplot [color=mycolor1,dotted,line width=1.0pt,forget plot]
  table[row sep=crcr]{-0.951056516295154	0.309016994374948\\
-0.60382220830535	0.219707538176049\\
-0.375378071887508	0.182507388795924\\
-0.225093275108467	0.161489568357552\\
-0.124654461172013	0.143091836517116\\
-0.0562723920172722	0.12318609851568\\
-0.00923898083522721	0.101104614081101\\
0.0228060316514889	0.0772217637055013\\
0.0436193292019494	0.0520955750986265\\
0.0553650029180358	0.0262210407420432\\
0.0591645112940776	2.0486346567262e-17\\
};
\addplot [color=mycolor1,dotted,line width=1.0pt,forget plot]
  table[row sep=crcr]{-1	1.22464679914735e-16\\
-0.61127914703566	0.0236549857259488\\
-0.374309030147768	0.0580118391989452\\
-0.226262535142082	0.0806522438077526\\
-0.130218598863194	0.0890855793127953\\
-0.0656897647351535	0.0862950481802363\\
-0.0214411717925588	0.0757647040434823\\
0.00876629006412298	0.0602253159022079\\
0.0284556614934045	0.0415943455493057\\
0.039601950618638	0.0211971994741971\\
0.0432139182637723	1.66258696249815e-17\\
};
\addplot [color=mycolor1,dotted,line width=1.0pt,forget plot]
  table[row sep=crcr]{1	-0\\
1	-0\\
};
\addplot [color=mycolor1,dotted,line width=1.0pt,forget plot]
  table[row sep=crcr]{0.951056516295154	-0.309016994374947\\
0.906577591518048	-0.290693092532446\\
0.867277205719189	-0.267124444629623\\
0.833330533437629	-0.239553777474139\\
0.804679893747523	-0.20903820245934\\
0.781122098933164	-0.176433510213537\\
0.762382187756933	-0.142402376999381\\
0.748171074803624	-0.107437628337747\\
0.738227708485823	-0.0718935522249012\\
0.732347931289196	-0.0360204899377399\\
0.730402691048646	-2.81010850277162e-17\\
};
\addplot [color=mycolor1,dotted,line width=1.0pt,forget plot]
  table[row sep=crcr]{0.809016994374947	-0.587785252292473\\
0.737380455396588	-0.527071687397996\\
0.6808142826414	-0.463341883835339\\
0.637053765657314	-0.399254954339046\\
0.603812761314093	-0.336417677088311\\
0.579022949915482	-0.275632227640288\\
0.560948163233974	-0.217130071437151\\
0.54821711318997	-0.160763451735608\\
0.539811466724715	-0.106147624627789\\
0.535036016768209	-0.0527590625798542\\
0.533488091091103	-4.10502162512614e-17\\
};
\addplot [color=mycolor1,dotted,line width=1.0pt,forget plot]
  table[row sep=crcr]{0.587785252292473	-0.809016994374947\\
0.515276498489902	-0.692182785870847\\
0.46668476526979	-0.583707991451876\\
0.435233321876477	-0.485319980094308\\
0.415551842123536	-0.396928474901319\\
0.403656860517901	-0.317461475735791\\
0.396737049613855	-0.245416450708029\\
0.392888142823216	-0.179177710929467\\
0.39087245229983	-0.11717008156476\\
0.389932112762627	-0.0579102497954412\\
0.389661137375347	-4.49747826270753e-17\\
};
\addplot [color=mycolor1,dotted,line width=1.0pt,forget plot]
  table[row sep=crcr]{0.309016994374947	-0.951056516295154\\
0.265925372344309	-0.777304721760365\\
0.248822386132444	-0.630899544522142\\
0.24643298177389	-0.508693744238057\\
0.251412997268255	-0.406266573115132\\
0.25929445161488	-0.31919477108011\\
0.267517373913267	-0.243597429511063\\
0.274697115780397	-0.1762665508339\\
0.280129101393366	-0.114599409879343\\
0.283480020554887	-0.0564559973822998\\
0.284609543336029	-4.37996030135249e-17\\
};
\addplot [color=mycolor1,dotted,line width=1.0pt,forget plot]
  table[row sep=crcr]{6.12323399573677e-17	-1\\
0.0151248701748503	-0.781989711398728\\
0.0472692933177679	-0.613631335769719\\
0.0835006201485716	-0.482943980920436\\
0.117381749765263	-0.379469463911326\\
0.146223972383669	-0.295078319831894\\
0.169261627793672	-0.223809451176491\\
0.186582776182006	-0.161390341419992\\
0.198560105942714	-0.104739935929793\\
0.205572433929556	-0.0515565221197431\\
0.207879576350762	-3.99891848849262e-17\\
};
\addplot [color=mycolor1,dotted,line width=1.0pt,forget plot]
  table[row sep=crcr]{-0.309016994374947	-0.951056516295154\\
-0.213607139159912	-0.713331044437031\\
-0.122920349149865	-0.544815253953639\\
-0.0461074386071066	-0.422454854218942\\
0.0151311193047769	-0.329888717873058\\
0.0621570724668225	-0.256291005261778\\
0.0971710522581798	-0.194731597161215\\
0.122256440677152	-0.140793596164787\\
0.13905244595299	-0.0915971142347777\\
0.148693455532156	-0.0451621321066958\\
0.151835801980649	-3.50498499033503e-17\\
};
\addplot [color=mycolor1,dotted,line width=1.0pt,forget plot]
  table[row sep=crcr]{-0.587785252292473	-0.809016994374948\\
-0.401011853057454	-0.584595820351374\\
-0.252139489074835	-0.439670821081765\\
-0.139623392550337	-0.340999317931598\\
-0.0567836371213516	-0.268617800427267\\
0.0033339412143336	-0.21116115844769\\
0.0463512370945915	-0.162297289886265\\
0.0763422025660035	-0.118472638203443\\
0.0960671266193367	-0.0776165020305757\\
0.1072685824301	-0.0384304051397518\\
0.110901278364195	-2.98672554719679e-17\\
};
\addplot [color=mycolor1,dotted,line width=1.0pt,forget plot]
  table[row sep=crcr]{-0.809016994374948	-0.587785252292473\\
-0.533486326814499	-0.413410095118228\\
-0.336121655437603	-0.313963860155743\\
-0.198040110920963	-0.250717832404618\\
-0.101844033235305	-0.204281393673556\\
-0.0346516892466227	-0.165530866251108\\
0.0122258376810533	-0.130333089269644\\
0.0443886084751889	-0.0968398262452624\\
0.065339288702761	-0.0642052594194206\\
0.0771736424133687	-0.032008294596762\\
0.0810025921579431	-2.49315700239574e-17\\
};
\addplot [color=mycolor1,dotted,line width=1.0pt,forget plot]
  table[row sep=crcr]{-0.951056516295154	-0.309016994374948\\
-0.60382220830535	-0.219707538176049\\
-0.375378071887508	-0.182507388795924\\
-0.225093275108467	-0.161489568357552\\
-0.124654461172013	-0.143091836517116\\
-0.0562723920172722	-0.12318609851568\\
-0.00923898083522721	-0.101104614081101\\
0.0228060316514889	-0.0772217637055013\\
0.0436193292019494	-0.0520955750986265\\
0.0553650029180358	-0.0262210407420432\\
0.0591645112940776	-2.0486346567262e-17\\
};
\addplot [color=mycolor1,dotted,line width=1.0pt,forget plot]
  table[row sep=crcr]{-1	-1.22464679914735e-16\\
-0.61127914703566	-0.0236549857259488\\
-0.374309030147768	-0.0580118391989452\\
-0.226262535142082	-0.0806522438077526\\
-0.130218598863194	-0.0890855793127953\\
-0.0656897647351535	-0.0862950481802363\\
-0.0214411717925588	-0.0757647040434823\\
0.00876629006412298	-0.0602253159022079\\
0.0284556614934045	-0.0415943455493057\\
0.039601950618638	-0.0211971994741971\\
0.0432139182637723	-1.66258696249815e-17\\
};
\addplot [color=black!50!mycolor1,line width=1.5pt,mark size=5.0pt,only marks,mark=x,mark options={solid},forget plot]
  table[row sep=crcr]{0	0\\
0.676084684146394	0.471252951045488\\
0.676084684146394	-0.471252951045488\\
0.476716764894897	0\\
};
\addplot [color=black,line width=1.5pt,mark size=4.0pt,only marks,mark=o,mark options={solid},forget plot]
  table[row sep=crcr]{0.500166366019944	0.288659057024714\\
0.500166366019944	-0.288659057024714\\
};
\end{axis}
\end{tikzpicture}%}
    \end{figure}
    \end{small}
    \vfill
  }

  \frame{
    \frametitle{\insertsection}
    \vfill
    Margem de ganho para $G(z)$, $G_o(z)$ e $\Delta(z)$ e diagrama de pólos e zeros para $\Delta(z)$ para a tensão do capacitor $v_C$ como variável controlada na malha interna
    \vfill
  }

  \frame{
    \frametitle{\insertsection}
    \vfill
    \begin{small}
    \begin{figure}[htb]
      \centering
      \raisebox{-0.5\height}{
        \def\svgwidth{0.5\textwidth}
        % This file was created by matlab2tikz v0.4.7 running on MATLAB 7.14.
% Copyright (c) 2008--2014, Nico Schlömer <nico.schloemer@gmail.com>
% All rights reserved.
% Minimal pgfplots version: 1.3
% 
% The latest updates can be retrieved from
%   http://www.mathworks.com/matlabcentral/fileexchange/22022-matlab2tikz
% where you can also make suggestions and rate matlab2tikz.
% 
%
% defining custom colors
\definecolor{mycolor1}{rgb}{0.66667,0.66667,0.66667}%
%
\begin{tikzpicture}

\begin{axis}[%
width=0.8\textwidth,
height=0.461611624834875\textwidth,
scale only axis,
xmode=log,
xmin=10,
xmax=31622.7766016838,
xminorticks=true,
xlabel={Frequência (rad/s)},
xmajorgrids,
xminorgrids,
ymin=-60,
ymax=20,
ylabel={Magnitude (dB)},
ymajorgrids,
legend style={draw=black,fill=white,legend cell align=left}
]
\addplot [color=mycolor1,solid,line width=1.5pt]
  table[row sep=crcr]{10	-27.220021747748\\
10.0461810465159	-27.2200215920221\\
10.0925753619375	-27.2200214348545\\
10.139183931163	-27.220021276232\\
10.1860077436388	-27.220021116141\\
10.2330477933808	-27.220020954568\\
10.2803050789954	-27.2200207914992\\
10.3277806037004	-27.2200206269208\\
10.375475375347	-27.2200204608188\\
10.4233904064403	-27.220020293179\\
10.4715267141616	-27.2200201239874\\
10.5198853203895	-27.2200199532294\\
10.5684672517218	-27.2200197808907\\
10.6172735394972	-27.2200196069565\\
10.6663052198171	-27.2200194314121\\
10.715563333568	-27.2200192542426\\
10.7650489264432	-27.2200190754329\\
10.814763048965	-27.2200188949679\\
10.8647067565072	-27.2200187128322\\
10.9148811093176	-27.2200185290104\\
10.9652871725401	-27.2200183434868\\
11.0159260162376	-27.2200181562458\\
11.0667987154147	-27.2200179672713\\
11.1179063500406	-27.2200177765474\\
11.1692500050717	-27.2200175840579\\
11.2208307704748	-27.2200173897863\\
11.2726497412507	-27.2200171937163\\
11.3247080174565	-27.2200169958312\\
11.3770067042298	-27.2200167961141\\
11.4295469118117	-27.2200165945482\\
11.4823297555707	-27.2200163911162\\
11.535356356026	-27.220016185801\\
11.5886278388715	-27.220015978585\\
11.6421453349997	-27.2200157694507\\
11.6959099805258	-27.2200155583803\\
11.7499229168114	-27.2200153453559\\
11.8041852904894	-27.2200151303595\\
11.8586982534876	-27.2200149133727\\
11.9134629630538	-27.2200146943771\\
11.96848058178	-27.2200144733542\\
12.0237522776272	-27.2200142502851\\
12.0792792239501	-27.2200140251509\\
12.135062599522	-27.2200137979326\\
12.1911035885602	-27.2200135686107\\
12.2474033807505	-27.2200133371659\\
12.3039631712731	-27.2200131035785\\
12.3607841608273	-27.2200128678286\\
12.4178675556577	-27.2200126298963\\
12.4752145675793	-27.2200123897613\\
12.5328264140034	-27.2200121474032\\
12.5907043179635	-27.2200119028014\\
12.6488495081411	-27.2200116559353\\
12.7072632188919	-27.2200114067837\\
12.765946690272	-27.2200111553257\\
12.8249011680643	-27.2200109015397\\
12.8841279038047	-27.2200106454043\\
12.9436281548089	-27.2200103868977\\
13.0034031841991	-27.2200101259979\\
13.0634542609305	-27.2200098626829\\
13.1237826598188	-27.2200095969301\\
13.1843896615665	-27.2200093287172\\
13.2452765527909	-27.2200090580213\\
13.306444626051	-27.2200087848193\\
13.3678951798747	-27.2200085090882\\
13.4296295187868	-27.2200082308045\\
13.4916489533366	-27.2200079499445\\
13.5539548001256	-27.2200076664845\\
13.6165483818355	-27.2200073804003\\
13.6794310272563	-27.2200070916676\\
13.7426040713143	-27.220006800262\\
13.806068855101	-27.2200065061587\\
13.8698267259009	-27.2200062093327\\
13.9338790372205	-27.2200059097588\\
13.998227148817	-27.2200056074116\\
14.062872426727	-27.2200053022653\\
14.1278162432955	-27.2200049942942\\
14.1930599772055	-27.2200046834719\\
14.2586050135065	-27.2200043697722\\
14.3244527436445	-27.2200040531685\\
14.3906045654914	-27.2200037336337\\
14.4570618833745	-27.2200034111408\\
14.5238261081064	-27.2200030856624\\
14.5908986570152	-27.2200027571708\\
14.658280953974	-27.2200024256382\\
14.7259744294318	-27.2200020910365\\
14.7939805204436	-27.2200017533371\\
14.8623006707006	-27.2200014125114\\
14.9309363305612	-27.2200010685305\\
14.999888957082	-27.2200007213652\\
15.069160014048	-27.2200003709859\\
15.1387509720044	-27.220000017363\\
15.2086633082875	-27.2199996604664\\
15.2788985070559	-27.2199993002658\\
15.3494580593225	-27.2199989367306\\
15.4203434629857	-27.2199985698299\\
15.4915562228612	-27.2199981995326\\
15.5630978507143	-27.2199978258073\\
15.6349698652918	-27.2199974486221\\
15.7071737923542	-27.2199970679451\\
15.779711164708	-27.219996683744\\
15.8525835222385	-27.2199962959861\\
15.9257924119422	-27.2199959046385\\
15.9993393879601	-27.2199955096679\\
16.07322601161	-27.2199951110409\\
16.1474538514202	-27.2199947087235\\
16.2220244831628	-27.2199943026817\\
16.2969394898867	-27.2199938928808\\
16.3722004619516	-27.2199934792862\\
16.4478089970617	-27.2199930618627\\
16.5237667002994	-27.2199926405749\\
16.6000751841599	-27.2199922153869\\
16.6767360685846	-27.2199917862627\\
16.7537509809962	-27.2199913531659\\
16.8311215563331	-27.2199909160596\\
16.9088494370839	-27.2199904749068\\
16.9869362733223	-27.2199900296699\\
17.0653837227424	-27.2199895803112\\
17.1441934506935	-27.2199891267925\\
17.2233671302159	-27.2199886690754\\
17.302906442076	-27.2199882071208\\
17.3828130748022	-27.2199877408897\\
17.4630887247206	-27.2199872703424\\
17.5437350959914	-27.2199867954389\\
17.6247539006444	-27.219986316139\\
17.7061468586161	-27.2199858324018\\
17.7879156977856	-27.2199853441865\\
17.8700621540116	-27.2199848514514\\
17.9525879711692	-27.2199843541548\\
18.035494901187	-27.2199838522544\\
18.1187847040838	-27.2199833457076\\
18.2024591480069	-27.2199828344713\\
18.2865200092687	-27.2199823185023\\
18.3709690723849	-27.2199817977566\\
18.4558081301122	-27.21998127219\\
18.5410389834868	-27.2199807417579\\
18.6266634418617	-27.2199802064153\\
18.7126833229461	-27.2199796661167\\
18.7991004528435	-27.2199791208162\\
18.8859166660905	-27.2199785704676\\
18.9731338056957	-27.219978015024\\
19.060753723179	-27.2199774544383\\
19.1487782786108	-27.2199768886629\\
19.2372093406515	-27.2199763176498\\
19.3260487865912	-27.2199757413504\\
19.4152985023894	-27.2199751597159\\
19.5049603827153	-27.2199745726969\\
19.5950363309877	-27.2199739802434\\
19.6855282594159	-27.2199733823052\\
19.7764380890397	-27.2199727788315\\
19.8677677497706	-27.2199721697711\\
19.9595191804325	-27.2199715550722\\
20.0516943288031	-27.2199709346827\\
20.1442951516552	-27.2199703085498\\
20.2373236147981	-27.2199696766204\\
20.3307816931193	-27.2199690388409\\
20.4246713706267	-27.219968395157\\
20.5189946404906	-27.2199677455141\\
20.6137535050858	-27.219967089857\\
20.7089499760343	-27.2199664281302\\
20.8045860742482	-27.2199657602772\\
20.900663829972	-27.2199650862416\\
20.9971852828265	-27.2199644059659\\
21.0941524818514	-27.2199637193925\\
21.1915674855492	-27.2199630264631\\
21.2894323619287	-27.2199623271187\\
21.387749188549	-27.2199616213001\\
21.4865200525637	-27.2199609089473\\
21.5857470507649	-27.2199601899997\\
21.685432289628	-27.2199594643964\\
21.7855778853565	-27.2199587320756\\
21.8861859639264	-27.2199579929753\\
21.987258661132	-27.2199572470327\\
22.0887981226306	-27.2199564941844\\
22.1908065039887	-27.2199557343665\\
22.2932859707273	-27.2199549675144\\
22.396238698368	-27.2199541935631\\
22.499666872479	-27.2199534124468\\
22.603572688722	-27.2199526240992\\
22.7079583528983	-27.2199518284534\\
22.8128260809959	-27.2199510254417\\
22.9181780992365	-27.219950214996\\
23.0240166441225	-27.2199493970475\\
23.130343962485	-27.2199485715267\\
23.237162311531	-27.2199477383634\\
23.3444739588916	-27.2199468974871\\
23.4522811826701	-27.2199460488261\\
23.5605862714901	-27.2199451923086\\
23.6693915245447	-27.2199443278616\\
23.7786992516445	-27.2199434554119\\
23.8885117732672	-27.2199425748854\\
23.9988314206069	-27.2199416862072\\
24.1096605356231	-27.219940789302\\
24.2210014710909	-27.2199398840934\\
24.3328565906507	-27.2199389705047\\
24.4452282688584	-27.2199380484583\\
24.558118891236	-27.2199371178759\\
24.6715308543219	-27.2199361786785\\
24.7854665657221	-27.2199352307863\\
24.899928444161	-27.2199342741187\\
25.0149189195332	-27.2199333085947\\
25.1304404329547	-27.2199323341321\\
25.2464954368146	-27.2199313506482\\
25.3630863948276	-27.2199303580596\\
25.4802157820863	-27.2199293562818\\
25.597886085113	-27.2199283452299\\
25.7160998019135	-27.219927324818\\
25.8348594420295	-27.2199262949594\\
25.9541675265918	-27.2199252555667\\
26.0740265883745	-27.2199242065516\\
26.194439171848	-27.2199231478251\\
26.3154078332332	-27.2199220792971\\
26.4369351405564	-27.219921000877\\
26.5590236737027	-27.2199199124733\\
26.6816760244719	-27.2199188139934\\
26.8048947966327	-27.2199177053441\\
26.9286826059784	-27.2199165864312\\
27.0530420803822	-27.2199154571597\\
27.1779758598533	-27.2199143174338\\
27.3034865965925	-27.2199131671566\\
27.4295769550488	-27.2199120062305\\
27.556249611976	-27.2199108345568\\
27.6835072564894	-27.2199096520361\\
27.8113525901229	-27.219908458568\\
27.9397883268864	-27.219907254051\\
28.0688171933232	-27.2199060383829\\
28.1984419285683	-27.2199048114605\\
28.3286652844062	-27.2199035731796\\
28.4594900253294	-27.2199023234351\\
28.5909189285972	-27.2199010621207\\
28.7229547842946	-27.2198997891294\\
28.8556003953913	-27.219898504353\\
28.9888585778017	-27.2198972076825\\
29.1227321604441	-27.2198958990078\\
29.2572239853012	-27.2198945782177\\
29.3923369074803	-27.2198932452001\\
29.5280737952738	-27.2198918998417\\
29.6644375302202	-27.2198905420283\\
29.8014310071653	-27.2198891716447\\
29.9390571343235	-27.2198877885743\\
30.0773188333397	-27.2198863926998\\
30.2162190393513	-27.2198849839026\\
30.3557607010504	-27.2198835620631\\
30.4959467807464	-27.2198821270606\\
30.6367802544292	-27.2198806787731\\
30.7782641118319	-27.2198792170778\\
30.9204013564946	-27.2198777418504\\
31.063195005828	-27.2198762529656\\
31.2066480911777	-27.2198747502971\\
31.350763657888	-27.2198732337172\\
31.4955447653674	-27.2198717030972\\
31.6409944871527	-27.219870158307\\
31.7871159109747	-27.2198685992155\\
31.9339121388238	-27.2198670256902\\
32.0813862870155	-27.2198654375976\\
32.229541486257	-27.2198638348028\\
32.3783808817133	-27.2198622171695\\
32.5279076330741	-27.2198605845606\\
32.6781249146208	-27.2198589368373\\
32.8290359152942	-27.2198572738597\\
32.9806438387618	-27.2198555954866\\
33.132951903486	-27.2198539015755\\
33.2859633427924	-27.2198521919824\\
33.4396814049384	-27.2198504665622\\
33.5941093531822	-27.2198487251684\\
33.7492504658521	-27.2198469676531\\
33.9051080364161	-27.2198451938671\\
34.0616853735517	-27.2198434036596\\
34.2189858012163	-27.2198415968788\\
34.3770126587175	-27.219839773371\\
34.5357693007845	-27.2198379329815\\
34.6952590976386	-27.2198360755541\\
34.8554854350656	-27.2198342009308\\
35.0164517144866	-27.2198323089526\\
35.1781613530314	-27.2198303994588\\
35.3406177836102	-27.2198284722871\\
35.5038244549868	-27.219826527274\\
35.6677848318515	-27.2198245642543\\
35.8325023948954	-27.2198225830612\\
35.9979806408833	-27.2198205835265\\
36.1642230827288	-27.2198185654804\\
36.3312332495683	-27.2198165287515\\
36.4990146868361	-27.2198144731668\\
36.6675709563398	-27.2198123985519\\
36.8369056363357	-27.2198103047304\\
37.007022321605	-27.2198081915245\\
37.1779246235299	-27.2198060587549\\
37.3496161701703	-27.2198039062403\\
37.5221006063408	-27.219801733798\\
37.6953815936883	-27.2197995412435\\
37.8694628107696	-27.2197973283905\\
38.0443479531292	-27.2197950950512\\
38.2200407333782	-27.2197928410359\\
38.3965448812729	-27.2197905661531\\
38.5738641437941	-27.2197882702097\\
38.7520022852263	-27.2197859530106\\
38.9309630872381	-27.2197836143591\\
39.1107503489621	-27.2197812540566\\
39.2913678870758	-27.2197788719026\\
39.4728195358824	-27.2197764676948\\
39.6551091473924	-27.219774041229\\
39.8382405914052	-27.2197715922992\\
40.0222177555915	-27.2197691206974\\
40.2070445455755	-27.2197666262136\\
40.3927248850181	-27.219764108636\\
40.5792627157	-27.2197615677509\\
40.7666619976054	-27.2197590033424\\
40.9549267090063	-27.2197564151927\\
41.1440608465467	-27.2197538030821\\
41.3340684253274	-27.2197511667886\\
41.5249534789915	-27.2197485060884\\
41.7167200598098	-27.2197458207556\\
41.9093722387671	-27.219743110562\\
42.1029141056481	-27.2197403752775\\
42.2973497691249	-27.2197376146699\\
42.4926833568435	-27.2197348285045\\
42.6889190155123	-27.2197320165449\\
42.8860609109892	-27.2197291785522\\
43.0841132283705	-27.2197263142854\\
43.2830801720801	-27.2197234235012\\
43.4829659659579	-27.2197205059541\\
43.6837748533501	-27.2197175613963\\
43.8855110971994	-27.2197145895778\\
44.0881789801347	-27.2197115902461\\
44.2917828045629	-27.2197085631466\\
44.4963268927599	-27.2197055080221\\
44.701815586962	-27.2197024246131\\
44.9082532494586	-27.2196993126579\\
45.1156442626847	-27.219696171892\\
45.3239930293136	-27.2196930020489\\
45.5333039723509	-27.2196898028591\\
45.7435815352278	-27.2196865740511\\
45.954830181896	-27.2196833153506\\
46.167054396922	-27.2196800264808\\
46.3802586855826	-27.2196767071625\\
46.5944475739604	-27.2196733571136\\
46.8096256090399	-27.2196699760498\\
47.0257973588042	-27.2196665636838\\
47.2429674123316	-27.2196631197258\\
47.4611403798933	-27.2196596438833\\
47.6803208930514	-27.2196561358612\\
47.9005136047569	-27.2196525953614\\
48.1217231894485	-27.2196490220833\\
48.3439543431522	-27.2196454157233\\
48.5672117835806	-27.2196417759753\\
48.791500250233	-27.21963810253\\
49.0168245044966	-27.2196343950755\\
49.243189329747	-27.2196306532969\\
49.4705995314498	-27.2196268768764\\
49.6990599372628	-27.2196230654932\\
49.9285753971387	-27.2196192188236\\
50.1591507834275	-27.219615336541\\
50.3907909909802	-27.2196114183156\\
50.6235009372529	-27.2196074638145\\
50.8572855624109	-27.2196034727021\\
51.0921498294339	-27.2195994446391\\
51.3280987242209	-27.2195953792836\\
51.5651372556965	-27.2195912762903\\
51.8032704559168	-27.2195871353107\\
52.0425033801768	-27.219582955993\\
52.2828411071172	-27.2195787379823\\
52.5242887388322	-27.2195744809203\\
52.7668514009784	-27.2195701844455\\
53.010534242883	-27.219565848193\\
53.2553424376533	-27.2195614717944\\
53.5012811822866	-27.219557054878\\
53.7483556977805	-27.2195525970687\\
53.9965712292437	-27.2195480979879\\
54.2459330460073	-27.2195435572534\\
54.4964464417369	-27.2195389744796\\
54.7481167345445	-27.2195343492772\\
55.0009492671021	-27.2195296812533\\
55.2549494067543	-27.2195249700116\\
55.510122545633	-27.2195202151517\\
55.7664741007712	-27.2195154162699\\
56.0240095142187	-27.2195105729586\\
56.282734253157	-27.2195056848064\\
56.5426538100157	-27.219500751398\\
56.8037737025889	-27.2194957723145\\
57.0660994741527	-27.219490747133\\
57.3296366935823	-27.2194856754266\\
57.5943909554708	-27.2194805567645\\
57.8603678802477	-27.219475390712\\
58.1275731142982	-27.2194701768303\\
58.3960123300829	-27.2194649146765\\
58.6656912262587	-27.2194596038036\\
58.9366155277994	-27.2194542437606\\
59.2087909861172	-27.2194488340921\\
59.4822233791852	-27.2194433743387\\
59.7569185116595	-27.2194378640367\\
60.0328822150028	-27.2194323027179\\
60.3101203476082	-27.21942668991\\
60.5886387949234	-27.2194210251362\\
60.8684434695757	-27.2194153079155\\
61.1495403114975	-27.219409537762\\
61.4319352880526	-27.2194037141858\\
61.7156343941624	-27.2193978366921\\
62.0006436524339	-27.2193919047818\\
62.2869691132868	-27.2193859179509\\
62.5746168550822	-27.2193798756909\\
62.8635929842521	-27.2193737774885\\
63.1539036354283	-27.2193676228259\\
63.445554971573	-27.2193614111801\\
63.7385531841099	-27.2193551420235\\
64.032904493055	-27.2193488148236\\
64.3286151471491	-27.219342429043\\
64.6256914239905	-27.2193359841392\\
64.9241396301678	-27.2193294795648\\
65.2239661013943	-27.2193229147672\\
65.5251772026422	-27.2193162891888\\
65.827779328278	-27.2193096022668\\
66.1317789021976	-27.2193028534331\\
66.4371823779638	-27.2192960421145\\
66.7439962389419	-27.2192891677324\\
67.0522269984386	-27.2192822297028\\
67.3618811998395	-27.2192752274364\\
67.6729654167483	-27.2192681603383\\
67.9854862531262	-27.2192610278083\\
68.2994503434323	-27.2192538292403\\
68.6148643527643	-27.219246564023\\
68.931734977	-27.2192392315392\\
69.2500689429393	-27.2192318311659\\
69.5698730084476	-27.2192243622746\\
69.8911539625984	-27.2192168242307\\
70.213918625818	-27.219209216394\\
70.5381738500302	-27.2192015381181\\
70.8639265188016	-27.2191937887507\\
71.191183547488	-27.2191859676336\\
71.5199518833808	-27.2191780741023\\
71.8502385058548	-27.2191701074864\\
72.1820504265165	-27.2191620671089\\
72.5153946893524	-27.2191539522869\\
72.8502783708792	-27.219145762331\\
73.1867085802933	-27.2191374965455\\
73.5246924596224	-27.2191291542282\\
73.8642371838769	-27.2191207346702\\
74.2053499612018	-27.2191122371565\\
74.5480380330304	-27.2191036609651\\
74.8923086742377	-27.2190950053674\\
75.2381691932944	-27.2190862696281\\
75.585626932423	-27.219077453005\\
75.9346892677529	-27.2190685547492\\
76.2853636094773	-27.2190595741047\\
76.6376574020103	-27.2190505103085\\
76.9915781241455	-27.2190413625906\\
77.3471332892138	-27.2190321301739\\
77.7043304452438	-27.219022812274\\
78.0631771751216	-27.2190134080993\\
78.4236810967518	-27.2190039168507\\
78.7858498632195	-27.2189943377221\\
79.1496911629522	-27.2189846698995\\
79.5152127198837	-27.2189749125616\\
79.8824222936175	-27.2189650648795\\
80.2513276795919	-27.2189551260164\\
80.6219367092452	-27.2189450951279\\
80.9942572501823	-27.2189349713619\\
81.3682972063414	-27.2189247538582\\
81.7440645181619	-27.2189144417488\\
82.121567162753	-27.2189040341574\\
82.5008131540631	-27.2188935301999\\
82.8818105430498	-27.2188829289838\\
83.2645674178507	-27.2188722296085\\
83.6490919039556	-27.2188614311648\\
84.0353921643786	-27.2188505327352\\
84.423476399831	-27.2188395333939\\
84.8133528488963	-27.2188284322063\\
85.2050297882047	-27.2188172282291\\
85.5985155326084	-27.2188059205104\\
85.9938184353586	-27.2187945080894\\
86.3909468882829	-27.2187829899965\\
86.7899093219629	-27.2187713652531\\
87.1907142059136	-27.2187596328713\\
87.5933700487633	-27.2187477918543\\
87.9978853984339	-27.2187358411961\\
88.4042688423224	-27.2187237798811\\
88.8125290074835	-27.2187116068846\\
89.2226745608124	-27.2186993211722\\
89.6347142092288	-27.2186869217\\
90.0486566998624	-27.2186744074145\\
90.4645108202374	-27.2186617772521\\
90.88228539846	-27.2186490301399\\
91.3019893034057	-27.2186361649945\\
91.7236314449071	-27.2186231807229\\
92.1472207739435	-27.2186100762217\\
92.5727662828307	-27.2185968503773\\
93.0002770054119	-27.218583502066\\
93.4297620172496	-27.2185700301534\\
93.8612304358183	-27.2185564334948\\
94.2946914206979	-27.2185427109348\\
94.730154173768	-27.2185288613073\\
95.1676279394036	-27.2185148834354\\
95.6071220046713	-27.2185007761313\\
96.0486456995261	-27.2184865381961\\
96.4922083970099	-27.2184721684201\\
96.9378195134503	-27.2184576655819\\
97.3854885086602	-27.2184430284493\\
97.8352248861393	-27.2184282557783\\
98.2870381932753	-27.2184133463135\\
98.7409380215466	-27.2183982987879\\
99.1969340067261	-27.2183831119227\\
99.655035829086	-27.2183677844273\\
100.115253213603	-27.2183523149991\\
100.577595930163	-27.2183367023235\\
101.042073793774	-27.2183209450737\\
101.508696664767	-27.2183050419104\\
101.977474449012	-27.2182889914822\\
102.448417098122	-27.218272792425\\
102.921534609671	-27.218256443362\\
103.3968370274	-27.2182399429039\\
103.874334441436	-27.2182232896481\\
104.354036988501	-27.2182064821795\\
104.83595485213	-27.2181895190696\\
105.320098262886	-27.2181723988765\\
105.80647749858	-27.2181551201453\\
106.295102884484	-27.2181376814075\\
106.785984793556	-27.2181200811808\\
107.279133646656	-27.2181023179693\\
107.774559912768	-27.2180843902633\\
108.272274109224	-27.218066296539\\
108.772286801926	-27.2180480352586\\
109.27460860557	-27.2180296048698\\
109.779250183872	-27.2180110038062\\
110.286222249794	-27.2179922304868\\
110.795535565772	-27.2179732833157\\
111.307200943943	-27.2179541606825\\
111.821229246378	-27.2179348609619\\
112.337631385307	-27.2179153825132\\
112.856418323356	-27.2178957236808\\
113.377601073777	-27.2178758827937\\
113.901190700682	-27.2178558581652\\
114.427198319278	-27.2178356480933\\
114.955635096105	-27.21781525086\\
115.486512249268	-27.2177946647314\\
116.019841048683	-27.2177738879575\\
116.555632816306	-27.2177529187722\\
117.093898926384	-27.2177317553929\\
117.634650805689	-27.2177103960206\\
118.177899933763	-27.2176888388394\\
118.723657843162	-27.2176670820167\\
119.271936119701	-27.2176451237031\\
119.822746402699	-27.2176229620318\\
120.376100385228	-27.2176005951187\\
120.932009814357	-27.2175780210623\\
121.490486491407	-27.2175552379435\\
122.051542272196	-27.2175322438253\\
122.615189067297	-27.2175090367529\\
123.181438842284	-27.2174856147532\\
123.750303617991	-27.2174619758349\\
124.321795470765	-27.2174381179882\\
124.895926532723	-27.2174140391848\\
125.472708992008	-27.2173897373774\\
126.052155093052	-27.2173652104998\\
126.63427713683	-27.2173404564665\\
127.219087481126	-27.217315473173\\
127.806598540793	-27.217290258495\\
128.396822788018	-27.2172648102885\\
128.989772752585	-27.2172391263898\\
129.585461022141	-27.217213204615\\
130.183900242466	-27.2171870427599\\
130.785103117737	-27.2171606385999\\
131.389082410804	-27.21713398989\\
131.995850943453	-27.217107094364\\
132.605421596686	-27.2170799497349\\
133.217807310988	-27.2170525536944\\
133.833021086605	-27.217024903913\\
134.451075983821	-27.2169969980393\\
135.071985123233	-27.2169688337002\\
135.69576168603	-27.2169404085006\\
136.322418914273	-27.2169117200233\\
136.951970111177	-27.2168827658285\\
137.584428641392	-27.2168535434539\\
138.219807931287	-27.2168240504142\\
138.858121469236	-27.2167942842011\\
139.499382805904	-27.2167642422833\\
140.143605554534	-27.2167339221056\\
140.790803391236	-27.2167033210893\\
141.440990055278	-27.2166724366319\\
142.094179349378	-27.2166412661064\\
142.750385139995	-27.2166098068619\\
143.409621357626	-27.2165780562225\\
144.0719019971	-27.2165460114876\\
144.737241117876	-27.2165136699317\\
145.405652844341	-27.2164810288038\\
146.077151366109	-27.2164480853274\\
146.751750938323	-27.2164148367005\\
147.42946588196	-27.2163812800947\\
148.110310584131	-27.2163474126556\\
148.794299498388	-27.2163132315024\\
149.481447145032	-27.2162787337272\\
150.171768111418	-27.2162439163955\\
150.865277052271	-27.2162087765455\\
151.561988689989	-27.2161733111876\\
152.261917814963	-27.2161375173049\\
152.965079285884	-27.2161013918522\\
153.671488030065	-27.2160649317561\\
154.381159043753	-27.2160281339148\\
155.094107392451	-27.2159909951975\\
155.810348211234	-27.2159535124446\\
156.529896705074	-27.2159156824669\\
157.25276814916	-27.2158775020459\\
157.978977889225	-27.2158389679329\\
158.708541341869	-27.2158000768492\\
159.441473994887	-27.2157608254858\\
160.177791407599	-27.2157212105027\\
160.917509211179	-27.2156812285292\\
161.66064310899	-27.2156408761631\\
162.40720887691	-27.2156001499707\\
163.157222363676	-27.2155590464863\\
163.910699491214	-27.2155175622122\\
164.66765625498	-27.2154756936181\\
165.428108724297	-27.2154334371411\\
166.192073042701	-27.215390789185\\
166.959565428276	-27.2153477461203\\
167.730602174008	-27.2153043042838\\
168.505199648122	-27.2152604599783\\
169.283374294433	-27.2152162094723\\
170.065142632699	-27.2151715489995\\
170.850521258965	-27.2151264747589\\
171.639526845917	-27.2150809829139\\
172.432176143241	-27.2150350695925\\
173.228485977972	-27.2149887308866\\
174.028473254854	-27.2149419628519\\
174.832154956701	-27.2148947615074\\
175.639548144754	-27.2148471228351\\
176.450669959044	-27.2147990427798\\
177.265537618758	-27.2147505172485\\
178.084168422602	-27.2147015421103\\
178.906579749169	-27.2146521131958\\
179.732789057308	-27.214602226297\\
180.562813886497	-27.2145518771668\\
181.39667185721	-27.2145010615186\\
182.234380671297	-27.2144497750259\\
183.075958112354	-27.2143980133221\\
183.921422046107	-27.214345772\\
184.770790420785	-27.2142930466116\\
185.624081267505	-27.2142398326672\\
186.481312700654	-27.2141861256359\\
187.342502918271	-27.2141319209442\\
188.207670202438	-27.2140772139765\\
189.076832919665	-27.214022000074\\
189.950009521279	-27.2139662745348\\
190.827218543818	-27.2139100326133\\
191.708478609426	-27.2138532695196\\
192.593808426241	-27.2137959804196\\
193.483226788801	-27.2137381604339\\
194.376752578439	-27.2136798046382\\
195.274404763682	-27.213620908062\\
196.176202400658	-27.2135614656888\\
197.082164633495	-27.2135014724555\\
197.992310694735	-27.2134409232518\\
198.906659905733	-27.21337981292\\
199.825231677076	-27.2133181362545\\
200.748045508988	-27.2132558880012\\
201.675120991751	-27.2131930628571\\
202.606477806113	-27.21312965547\\
203.542135723711	-27.213065660438\\
204.482114607491	-27.2130010723087\\
205.426434412127	-27.2129358855794\\
206.375115184445	-27.2128700946959\\
207.32817706385	-27.2128036940524\\
208.285640282755	-27.2127366779911\\
209.247525167004	-27.2126690408015\\
210.213852136311	-27.21260077672\\
211.18464170469	-27.2125318799293\\
212.159914480891	-27.2124623445582\\
213.139691168835	-27.2123921646807\\
214.123992568061	-27.2123213343157\\
215.112839574156	-27.2122498474266\\
216.10625317921	-27.2121776979204\\
217.104254472254	-27.2121048796476\\
218.106864639713	-27.2120313864015\\
219.114104965849	-27.2119572119175\\
220.12599683322	-27.2118823498729\\
221.142561723131	-27.2118067938859\\
222.163821216089	-27.2117305375156\\
223.189796992262	-27.211653574261\\
224.220510831939	-27.2115758975606\\
225.255984615994	-27.2114975007918\\
226.296240326348	-27.2114183772706\\
227.341300046436	-27.2113385202504\\
228.391185961678	-27.2112579229222\\
229.44592035995	-27.2111765784132\\
230.505525632052	-27.211094479787\\
231.570024272191	-27.2110116200424\\
232.639438878451	-27.2109279921131\\
233.713792153278	-27.210843588867\\
234.793106903962	-27.2107584031055\\
235.877406043117	-27.2106724275631\\
236.966712589169	-27.2105856549065\\
238.061049666849	-27.2104980777343\\
239.160440507677	-27.210409688576\\
240.264908450462	-27.2103204798915\\
241.374476941791	-27.2102304440707\\
242.48916953653	-27.2101395734323\\
243.609009898327	-27.2100478602237\\
244.734021800108	-27.2099552966199\\
245.864229124585	-27.2098618747232\\
246.999655864764	-27.209767586562\\
248.140326124455	-27.2096724240908\\
249.286264118777	-27.2095763791888\\
250.437494174681	-27.2094794436597\\
251.594040731461	-27.2093816092309\\
252.755928341275	-27.2092828675524\\
253.923181669665	-27.2091832101967\\
255.09582549608	-27.2090826286575\\
256.273884714404	-27.2089811143494\\
257.457384333485	-27.2088786586069\\
258.646349477661	-27.2087752526835\\
259.840805387301	-27.2086708877515\\
261.040777419332	-27.2085655549007\\
262.246291047787	-27.2084592451379\\
263.457371864337	-27.2083519493858\\
264.674045578839	-27.2082436584828\\
265.896338019882	-27.2081343631817\\
267.124275135332	-27.2080240541488\\
268.357882992886	-27.2079127219638\\
269.597187780627	-27.2078003571182\\
270.842215807572	-27.2076869500149\\
272.092993504239	-27.2075724909672\\
273.349547423206	-27.2074569701981\\
274.611904239671	-27.2073403778394\\
275.880090752022	-27.2072227039306\\
277.154133882405	-27.2071039384184\\
278.434060677294	-27.2069840711558\\
279.719898308069	-27.2068630919009\\
281.011674071587	-27.2067409903161\\
282.309415390768	-27.2066177559675\\
283.613149815172	-27.2064933783239\\
284.922905021585	-27.2063678467553\\
286.238708814609	-27.206241150533\\
287.560589127251	-27.2061132788278\\
288.888574021513	-27.2059842207093\\
290.222691688993	-27.2058539651451\\
291.562970451479	-27.205722501\\
292.909438761552	-27.2055898170345\\
294.262125203191	-27.205455901904\\
295.621058492378	-27.2053207441583\\
296.98626747771	-27.2051843322398\\
298.357781141006	-27.2050466544832\\
299.735628597932	-27.2049076991141\\
301.119839098607	-27.2047674542479\\
302.510442028234	-27.2046259078892\\
303.907466907719	-27.2044830479303\\
305.310943394298	-27.2043388621502\\
306.720901282168	-27.2041933382139\\
308.137370503119	-27.204046463671\\
309.560381127168	-27.2038982259544\\
310.989963363199	-27.2037486123799\\
312.426147559604	-27.2035976101443\\
313.868964204927	-27.2034452063249\\
315.318443928511	-27.203291387878\\
316.774617501149	-27.2031361416379\\
318.237515835737	-27.2029794543157\\
319.707169987928	-27.2028213124981\\
321.183611156796	-27.2026617026464\\
322.666870685493	-27.2025006110951\\
324.156980061919	-27.2023380240508\\
325.653970919389	-27.202173927591\\
327.1578750373	-27.2020083076629\\
328.668724341813	-27.201841150082\\
330.186550906528	-27.2016724405311\\
331.711386953162	-27.2015021645589\\
333.243264852235	-27.2013303075787\\
334.78221712376	-27.2011568548672\\
336.328276437929	-27.2009817915632\\
337.881475615808	-27.2008051026662\\
339.441847630035	-27.2006267730352\\
341.009425605519	-27.2004467873874\\
342.584242820144	-27.2002651302965\\
344.166332705473	-27.200081786192\\
345.75572884746	-27.199896739357\\
347.352464987164	-27.1997099739276\\
348.956575021463	-27.1995214738911\\
350.568093003772	-27.1993312230844\\
352.187053144771	-27.1991392051931\\
353.813489813129	-27.1989454037497\\
355.447437536229	-27.1987498021321\\
357.088931000911	-27.1985523835625\\
358.738005054198	-27.1983531311053\\
360.39469470404	-27.1981520276664\\
362.059035120061	-27.197949055991\\
363.731061634299	-27.1977441986624\\
365.410809741959	-27.1975374381004\\
367.09831510217	-27.1973287565597\\
368.793613538734	-27.1971181361285\\
370.496741040893	-27.1969055587266\\
372.207733764093	-27.1966910061042\\
373.926628030746	-27.1964744598399\\
375.653460331008	-27.1962559013393\\
377.388267323548	-27.1960353118332\\
379.13108583633	-27.1958126723761\\
380.881952867393	-27.1955879638445\\
382.640905585636	-27.1953611669349\\
384.407981331609	-27.1951322621625\\
386.183217618305	-27.1949012298594\\
387.966652131954	-27.1946680501725\\
389.758322732826	-27.194432703062\\
391.558267456034	-27.1941951682998\\
393.366524512341	-27.1939554254672\\
395.183132288971	-27.1937134539536\\
397.008129350424	-27.1934692329544\\
398.841554439296	-27.193222741469\\
400.683446477099	-27.1929739582993\\
402.533844565089	-27.1927228620476\\
404.392787985098	-27.1924694311145\\
406.260316200361	-27.1922136436975\\
408.136468856361	-27.1919554777885\\
410.021285781671	-27.191694911172\\
411.914806988789	-27.1914319214234\\
413.817072675003	-27.1911664859066\\
415.72812322323	-27.1908985817722\\
417.647999202884	-27.1906281859554\\
419.576741370729	-27.1903552751738\\
421.514390671752	-27.1900798259257\\
423.460988240025	-27.1898018144874\\
425.416575399583	-27.1895212169117\\
427.381193665299	-27.1892380090252\\
429.354884743766	-27.1889521664267\\
431.337690534184	-27.1886636644843\\
433.329653129246	-27.1883724783338\\
435.330814816033	-27.1880785828763\\
437.341218076915	-27.1877819527756\\
439.360905590448	-27.1874825624566\\
441.389920232282	-27.1871803861021\\
443.428305076071	-27.1868753976514\\
445.476103394389	-27.1865675707973\\
447.533358659646	-27.186256878984\\
449.600114545014	-27.1859432954045\\
451.67641492535	-27.1856267929985\\
453.762303878129	-27.1853073444497\\
455.857825684385	-27.1849849221834\\
457.963024829641	-27.184659498364\\
460.077946004862	-27.1843310448924\\
462.202634107401	-27.1839995334038\\
464.33713424195	-27.1836649352645\\
466.481491721497	-27.18332722157\\
468.635752068297	-27.1829863631419\\
470.799961014824	-27.1826423305253\\
472.974164504755	-27.1822950939864\\
475.158408693936	-27.1819446235096\\
477.352739951367	-27.1815908887946\\
479.557204860185	-27.1812338592539\\
481.771850218653	-27.1808735040102\\
483.996723041153	-27.180509791893\\
486.231870559183	-27.1801426914362\\
488.477340222363	-27.179772170875\\
490.73317969944	-27.1793981981434\\
492.999436879299	-27.1790207408707\\
495.276159871982	-27.1786397663789\\
497.563397009708	-27.1782552416797\\
499.861196847899	-27.1778671334712\\
502.169608166212	-27.1774754081354\\
504.48867996957	-27.1770800317345\\
506.818461489212	-27.1766809700081\\
509.159002183727	-27.17627818837\\
511.51035174011	-27.1758716519052\\
513.872560074817	-27.1754613253663\\
516.245677334823	-27.1750471731708\\
518.629753898685	-27.174629159397\\
521.024840377617	-27.1742072477817\\
523.430987616559	-27.1737814017161\\
525.848246695256	-27.1733515842428\\
528.27666892935	-27.172917758052\\
530.716305871459	-27.1724798854786\\
533.167209312278	-27.1720379284983\\
535.629431281677	-27.1715918487243\\
538.103024049808	-27.1711416074035\\
540.588040128206	-27.1706871654133\\
543.084532270915	-27.1702284832574\\
545.592553475602	-27.1697655210628\\
548.11215698468	-27.1692982385755\\
550.643396286443	-27.1688265951573\\
553.186325116201	-27.1683505497814\\
555.740997457415	-27.1678700610291\\
558.307467542852	-27.1673850870859\\
560.885789855729	-27.1668955857372\\
563.476019130871	-27.1664015143647\\
566.078210355879	-27.1659028299424\\
568.692418772287	-27.1653994890324\\
571.318699876742	-27.164891447781\\
573.957109422183	-27.1643786619145\\
576.607703419019	-27.1638610867352\\
579.27053813632	-27.1633386771169\\
581.945670103016	-27.162811387501\\
584.633156109091	-27.1622791718919\\
587.333053206791	-27.1617419838531\\
590.045418711838	-27.1611997765024\\
592.77031020464	-27.1606525025075\\
595.507785531519	-27.1601001140819\\
598.25790280594	-27.1595425629802\\
601.020720409738	-27.1589798004932\\
603.796296994364	-27.158411777444\\
606.584691482126	-27.1578384441827\\
609.385963067442	-27.1572597505819\\
612.200171218097	-27.156675646032\\
615.027375676503	-27.1560860794365\\
617.867636460969	-27.155490999207\\
620.721013866976	-27.1548903532581\\
623.587568468455	-27.1542840890028\\
626.467361119072	-27.1536721533474\\
629.360452953524	-27.1530544926861\\
632.266905388836	-27.1524310528964\\
635.186780125658	-27.1518017793335\\
638.120139149584	-27.1511666168252\\
641.067044732464	-27.150525509667\\
644.027559433723	-27.1498784016158\\
647.001746101695	-27.1492252358855\\
649.989667874954	-27.1485659551412\\
652.991388183652	-27.1479005014932\\
656.00697075087	-27.1472288164923\\
659.03647959397	-27.1465508411233\\
662.07997902595	-27.1458665158\\
665.137533656814	-27.145175780359\\
668.209208394941	-27.144478574054\\
671.295068448464	-27.14377483555\\
674.395179326655	-27.1430645029172\\
677.509606841313	-27.1423475136251\\
680.638417108163	-27.1416238045364\\
683.78167654826	-27.1408933119008\\
686.9394518894	-27.1401559713488\\
690.11181016753	-27.1394117178856\\
693.298818728181	-27.1386604858843\\
696.500545227891	-27.1379022090801\\
699.717057635642	-27.1371368205633\\
702.948424234306	-27.1363642527729\\
706.19471362209	-27.1355844374901\\
709.455994713995	-27.1347973058314\\
712.732336743282	-27.1340027882419\\
716.023809262934	-27.1332008144885\\
719.330482147139	-27.1323913136529\\
722.652425592773	-27.1315742141246\\
725.989710120886	-27.1307494435939\\
729.3424065782	-27.1299169290446\\
732.71058613862	-27.1290765967468\\
736.094320304735	-27.1282283722497\\
739.493680909343	-27.127372180374\\
742.908740116972	-27.1265079452046\\
746.339570425412	-27.1256355900828\\
749.786244667258	-27.124755037599\\
753.248836011454	-27.1238662095845\\
756.727417964843	-27.1229690271041\\
760.222064373731	-27.122063410448\\
763.732849425456	-27.1211492791237\\
767.259847649959	-27.120226551848\\
770.803133921369	-27.119295146539\\
774.362783459591	-27.1183549803074\\
777.938871831903	-27.1174059694486\\
781.531474954562	-27.1164480294338\\
785.140669094413	-27.1154810749017\\
788.76653087051	-27.1145050196499\\
792.40913725574	-27.1135197766259\\
796.068565578463	-27.1125252579184\\
799.744893524145	-27.1115213747483\\
803.438199137013	-27.1105080374597\\
807.148560821713	-27.1094851555107\\
810.876057344967	-27.1084526374639\\
814.620767837254	-27.1074103909774\\
818.382771794484	-27.1063583227953\\
822.162149079689	-27.1052963387377\\
825.958979924714	-27.1042243436912\\
829.773344931927	-27.1031422415996\\
833.605325075921	-27.1020499354529\\
837.455001705243	-27.1009473272783\\
841.322456544116	-27.0998343181294\\
845.207771694168	-27.0987108080763\\
849.111029636188	-27.0975766961948\\
853.032313231867	-27.0964318805562\\
856.971705725559	-27.0952762582166\\
860.92929074605	-27.094109725206\\
864.905152308334	-27.0929321765177\\
868.899374815392	-27.0917435060968\\
872.91204305999	-27.0905436068295\\
876.943242226474	-27.0893323705315\\
880.993057892579	-27.088109687937\\
885.061576031251	-27.0868754486866\\
889.148883012464	-27.0856295413159\\
893.255065605058	-27.0843718532438\\
897.380210978585	-27.0831022707604\\
901.52440670515	-27.0818206790147\\
905.687740761275	-27.0805269620027\\
909.870301529772	-27.0792210025548\\
914.07217780161	-27.0779026823231\\
918.293458777803	-27.0765718817691\\
922.534234071309	-27.0752284801506\\
926.794593708924	-27.0738723555086\\
931.074628133199	-27.0725033846544\\
935.374428204358	-27.071121443156\\
939.694085202226	-27.0697264053249\\
944.033690828168	-27.0683181442021\\
948.39333720704	-27.0668965315448\\
952.773116889131	-27.0654614378118\\
957.173122852146	-27.0640127321498\\
961.593448503166	-27.0625502823791\\
966.034187680636	-27.0610739549786\\
970.495434656357	-27.0595836150718\\
974.977284137491	-27.0580791264114\\
979.479831268559	-27.0565603513646\\
984.003171633478	-27.0550271508975\\
988.547401257578	-27.0534793845604\\
993.112616609642	-27.0519169104712\\
997.698914603958	-27.0503395853006\\
1002.30639260238	-27.0487472642552\\
1006.93514841637	-27.0471398010622\\
1011.58528030912	-27.0455170479522\\
1016.2568869976	-27.0438788556431\\
1020.95006765465	-27.0422250733233\\
1025.66492191112	-27.0405555486342\\
1030.40154985798	-27.0388701276536\\
1035.16005204838	-27.0371686548778\\
1039.94052949988	-27.0354509732042\\
1044.74308369654	-27.0337169239133\\
1049.56781659108	-27.0319663466503\\
1054.41483060703	-27.0301990794075\\
1059.28422864096	-27.0284149585049\\
1064.17611406461	-27.0266138185718\\
1069.09059072708	-27.0247954925277\\
1074.02776295708	-27.0229598115632\\
1078.98773556513	-27.0211066051197\\
1083.97061384575	-27.0192357008707\\
1088.97650357974	-27.0173469247008\\
1094.00551103639	-27.0154401006858\\
1099.05774297578	-27.0135150510722\\
1104.13330665098	-27.011571596256\\
1109.2323098104	-27.0096095547622\\
1114.35486070002	-27.0076287432225\\
1119.50106806574	-27.0056289763546\\
1124.67104115564	-27.0036100669396\\
1129.86488972231	-27.0015718257999\\
1135.0827240252	-26.9995140617768\\
1140.32465483296	-26.9974365817077\\
1145.59079342577	-26.9953391904027\\
1150.8812515977	-26.9932216906213\\
1156.19614165913	-26.9910838830489\\
1161.53557643908	-26.9889255662723\\
1166.89966928762	-26.9867465367556\\
1172.28853407829	-26.9845465888154\\
1177.70228521052	-26.982325514596\\
1183.14103761204	-26.9800831040435\\
1188.60490674132	-26.9778191448807\\
1194.09400859005	-26.9755334225805\\
1199.60845968555	-26.9732257203398\\
1205.14837709331	-26.9708958190527\\
1210.71387841942	-26.9685434972834\\
1216.30508181309	-26.9661685312385\\
1221.92210596917	-26.9637706947394\\
1227.56507013062	-26.9613497591936\\
1233.23409409112	-26.9589054935669\\
1238.92929819754	-26.9564376643532\\
1244.65080335254	-26.9539460355461\\
1250.3987310171	-26.9514303686087\\
1256.17320321316	-26.9488904224429\\
1261.97434252612	-26.9463259533596\\
1267.80227210752	-26.9437367150468\\
1273.65711567764	-26.9411224585386\\
1279.53899752808	-26.9384829321831\\
1285.44804252445	-26.9358178816096\\
1291.38437610901	-26.9331270496965\\
1297.34812430331	-26.9304101765372\\
1303.33941371088	-26.9276669994068\\
1309.35837151994	-26.9248972527276\\
1315.40512550606	-26.9221006680345\\
1321.47980403488	-26.9192769739395\\
1327.58253606487	-26.9164258960963\\
1333.71345115004	-26.9135471571636\\
1339.87267944269	-26.9106404767686\\
1346.06035169616	-26.9077055714698\\
1352.27659926764	-26.9047421547189\\
1358.52155412096	-26.9017499368224\\
1364.79534882933	-26.8987286249028\\
1371.09811657822	-26.8956779228593\\
1377.42999116818	-26.8925975313271\\
1383.79110701763	-26.8894871476373\\
1390.18159916578	-26.8863464657755\\
1396.60160327544	-26.8831751763397\\
1403.05125563594	-26.8799729664983\\
1409.53069316601	-26.8767395199465\\
1416.04005341668	-26.8734745168629\\
1422.5794745742	-26.8701776338648\\
1429.14909546299	-26.8668485439638\\
1435.74905554856	-26.8634869165191\\
1442.3794949405	-26.8600924171923\\
1449.04055439544	-26.8566647078992\\
1455.73237532004	-26.8532034467629\\
1462.45509977397	-26.8497082880646\\
1469.20887047298	-26.8461788821949\\
1475.99383079187	-26.8426148756037\\
1482.81012476756	-26.8390159107491\\
1489.65789710218	-26.8353816260465\\
1496.53729316606	-26.8317116558159\\
1503.44845900091	-26.8280056302289\\
1510.39154132284	-26.824263175255\\
1517.36668752555	-26.8204839126064\\
1524.37404568338	-26.8166674596832\\
1531.41376455451	-26.8128134295163\\
1538.4859935841	-26.8089214307104\\
1545.59088290748	-26.8049910673859\\
1552.72858335329	-26.8010219391195\\
1559.89924644673	-26.7970136408847\\
1567.10302441275	-26.7929657629903\\
1574.34007017931	-26.788877891019\\
1581.61053738059	-26.7847496057641\\
1588.91458036027	-26.7805804831659\\
1596.25235417481	-26.776370094247\\
1603.62401459674	-26.7721180050457\\
1611.02971811795	-26.76782377655\\
1618.46962195303	-26.7634869646288\\
1625.94388404263	-26.7591071199632\\
1633.45266305675	-26.7546837879762\\
1640.99611839816	-26.7502165087613\\
1648.57441020578	-26.7457048170099\\
1656.18769935804	-26.7411482419381\\
1663.83614747635	-26.7365463072112\\
1671.51991692849	-26.731898530868\\
1679.23917083208	-26.7272044252436\\
1686.99407305804	-26.7224634968902\\
1694.78478823403	-26.7176752464979\\
1702.61148174801	-26.7128391688132\\
1710.47431975172	-26.7079547525564\\
1718.37346916419	-26.7030214803376\\
1726.30909767531	-26.6980388285719\\
1734.28137374936	-26.6930062673919\\
1742.29046662864	-26.6879232605604\\
1750.33654633699	-26.6827892653798\\
1758.41978368348	-26.6776037326017\\
1766.54035026596	-26.6723661063338\\
1774.69841847474	-26.6670758239458\\
1782.89416149627	-26.6617323159732\\
1791.12775331676	-26.6563350060203\\
1799.39936872594	-26.6508833106606\\
1807.70918332073	-26.6453766393361\\
1816.05737350894	-26.6398143942544\\
1824.44411651309	-26.634195970285\\
1832.86959037413	-26.6285207548522\\
1841.33397395519	-26.6227881278278\\
1849.83744694544	-26.6169974614208\\
1858.38018986386	-26.6111481200656\\
1866.9623840631	-26.6052394603083\\
1875.58421173328	-26.5992708306909\\
1884.24585590593	-26.5932415716331\\
1892.94750045783	-26.587151015313\\
1901.68933011491	-26.5809984855445\\
1910.47153045619	-26.5747832976533\\
1919.29428791772	-26.5685047583504\\
1928.15778979652	-26.5621621656033\\
1937.06222425457	-26.5557548085049\\
1946.00778032282	-26.5492819671405\\
1954.99464790516	-26.5427429124513\\
1964.02301778248	-26.5361369060967\\
1973.09308161673	-26.5294632003131\\
1982.20503195496	-26.5227210377708\\
1991.35906223344	-26.5159096514277\\
2000.55536678172	-26.5090282643807\\
2009.79414082682	-26.5020760897141\\
2019.07558049731	-26.4950523303456\\
2028.39988282751	-26.4879561788688\\
2037.76724576168	-26.4807868173936\\
2047.17786815819	-26.4735434173825\\
2056.63194979376	-26.4662251394848\\
2066.12969136771	-26.4588311333673\\
2075.6712945062	-26.4513605375417\\
2085.25696176653	-26.4438124791888\\
2094.88689664142	-26.4361860739798\\
2104.56130356336	-26.4284804258928\\
2114.2803879089	-26.4206946270278\\
2124.04435600306	-26.4128277574163\\
2133.8534151237	-26.4048788848282\\
2143.70777350589	-26.396847064575\\
2153.60764034637	-26.3887313393083\\
2163.55322580795	-26.3805307388156\\
2173.54474102401	-26.372244279811\\
2183.58239810298	-26.3638709657221\\
2193.66641013278	-26.3554097864733\\
2203.79699118545	-26.3468597182641\\
2213.97435632161	-26.3382197233427\\
2224.19872159503	-26.3294887497764\\
2234.47030405729	-26.3206657312158\\
2244.78932176229	-26.3117495866555\\
2255.15599377096	-26.3027392201891\\
2265.57054015585	-26.2936335207598\\
2276.03318200585	-26.2844313619056\\
2286.54414143084	-26.2751316014995\\
2297.10364156645	-26.2657330814841\\
2307.71190657875	-26.2562346276011\\
2318.36916166905	-26.2466350491146\\
2329.07563307865	-26.2369331385294\\
2339.83154809368	-26.2271276713032\\
2350.63713504986	-26.2172174055519\\
2361.49262333744	-26.2072010817504\\
2372.39824340596	-26.1970774224254\\
2383.35422676926	-26.1868451318427\\
2394.36080601029	-26.1765028956877\\
2405.4182147861	-26.1660493807387\\
2416.52668783283	-26.1554832345338\\
2427.6864609706	-26.1448030850296\\
2438.8977711086	-26.134007540254\\
2450.16085625011	-26.1230951879502\\
2461.4759554975	-26.1120645952136\\
2472.84330905736	-26.1009143081203\\
2484.26315824557	-26.0896428513479\\
2495.73574549243	-26.0782487277878\\
2507.26131434783	-26.0667304181489\\
2518.84010948637	-26.0550863805518\\
2530.4723767126	-26.0433150501156\\
2542.15836296621	-26.0314148385334\\
2553.8983163273	-26.01938413364\\
2565.69248602162	-26.0072212989685\\
2577.54112242586	-25.9949246732975\\
2589.444477073	-25.9824925701883\\
2601.4028026576	-25.9699232775103\\
2613.41635304121	-25.9572150569567\\
2625.48538325773	-25.9443661435484\\
2637.61014951883	-25.9313747451265\\
2649.7909092194	-25.9182390418327\\
2662.02792094301	-25.9049571855783\\
2674.32144446738	-25.8915272994998\\
2686.67174076992	-25.8779474774021\\
2699.07907203326	-25.8642157831884\\
2711.54370165082	-25.8503302502767\\
2724.0658942324	-25.8362888810018\\
2736.64591560979	-25.8220896460037\\
2749.28403284242	-25.8077304836002\\
2761.98051422303	-25.7932092991454\\
2774.73562928336	-25.7785239643715\\
2787.54964879989	-25.7636723167153\\
2800.42284479955	-25.7486521586275\\
2813.35549056553	-25.7334612568655\\
2826.34786064309	-25.7180973417681\\
2839.40023084533	-25.7025581065121\\
2852.51287825912	-25.6868412063511\\
2865.68608125093	-25.6709442578336\\
2878.92011947275	-25.654864838003\\
2892.21527386804	-25.6386004835758\\
2905.57182667769	-25.6221486900999\\
2918.99006144599	-25.6055069110902\\
2932.4702630267	-25.588672557143\\
2946.01271758903	-25.5716429950267\\
2959.61771262377	-25.5544155467488\\
2973.28553694936	-25.5369874885989\\
2987.01648071805	-25.5193560501663\\
3000.81083542203	-25.5015184133313\\
3014.66889389963	-25.4834717112302\\
3028.59095034155	-25.4652130271921\\
3042.57730029708	-25.4467393936478\\
3056.6282406804	-25.4280477910087\\
3070.74406977686	-25.4091351465157\\
3084.92508724934	-25.3899983330561\\
3099.17159414457	-25.3706341679489\\
3113.48389289956	-25.3510394116961\\
3127.86228734801	-25.3312107666994\\
3142.30708272674	-25.3111448759419\\
3156.8185856822	-25.2908383216316\\
3171.39710427697	-25.2702876238083\\
3186.04294799627	-25.2494892389101\\
3200.75642775457	-25.2284395582993\\
3215.53785590219	-25.2071349067466\\
3230.38754623189	-25.1855715408717\\
3245.30581398558	-25.1637456475381\\
3260.29297586097	-25.1416533422029\\
3275.34935001834	-25.1192906672168\\
3290.47525608724	-25.0966535900752\\
3305.67101517332	-25.0737380016177\\
3320.93694986511	-25.050539714174\\
3336.27338424092	-25.0270544596543\\
3351.68064387565	-25.0032778875834\\
3367.15905584778	-24.9792055630742\\
3382.70894874623	-24.9548329647411\\
3398.3306526774	-24.9301554825494\\
3414.02449927217	-24.9051684155981\\
3429.7908216929	-24.879866969836\\
3445.62995464054	-24.8542462557055\\
3461.54223436172	-24.8283012857138\\
3477.5279986559	-24.8020269719282\\
3493.58758688252	-24.7754181233923\\
3509.72133996824	-24.7484694434603\\
3525.92960041413	-24.7211755270465\\
3542.21271230298	-24.6935308577869\\
3558.57102130658	-24.6655298051096\\
3575.00487469309	-24.6371666212094\\
3591.51462133436	-24.6084354379255\\
3608.1006117134	-24.5793302635152\\
3624.76319793175	-24.5498449793229\\
3641.50273371703	-24.5199733363378\\
3658.31957443038	-24.4897089516374\\
3675.21407707406	-24.4590453047119\\
3692.18660029898	-24.4279757336647\\
3709.23750441235	-24.3964934312841\\
3726.36715138533	-24.3645914409816\\
3743.57590486067	-24.3322626525909\\
3760.86413016048	-24.2994997980216\\
3778.23219429398	-24.2662954467633\\
3795.68046596523	-24.2326420012328\\
3813.20931558105	-24.1985316919575\\
3830.81911525882	-24.16395657259\\
3848.51023883439	-24.128908514746\\
3866.28306187004	-24.0933792026576\\
3884.13796166242	-24.0573601276358\\
3902.07531725059	-24.0208425823331\\
3920.09550942403	-23.9838176547984\\
3938.19892073078	-23.946276222315\\
3956.38593548549	-23.9082089450136\\
3974.65693977764	-23.8696062592494\\
3993.0123214797	-23.8304583707343\\
4011.45247025537	-23.7907552474135\\
4029.97777756789	-23.7504866120757\\
4048.58863668828	-23.7096419346848\\
4067.28544270374	-23.6682104244219\\
4086.06859252603	-23.6261810214256\\
4104.93848489988	-23.5835423882153\\
4123.89552041149	-23.5402829007867\\
4142.94010149697	-23.496390639363\\
4162.07263245094	-23.4518533787881\\
4181.29351943512	-23.4066585785451\\
4200.60317048688	-23.3607933723854\\
4220.00199552799	-23.3142445575494\\
4239.49040637325	-23.266998583562\\
4259.06881673929	-23.2190415405852\\
4278.73764225331	-23.1703591473058\\
4298.49730046193	-23.120936738341\\
4318.34821084004	-23.0707592511385\\
4338.2907947997	-23.0198112123504\\
4358.32547569911	-22.9680767236571\\
4378.45267885158	-22.9155394470191\\
4398.67283153454	-22.8621825893296\\
4418.98636299868	-22.8079888864442\\
4439.39370447695	-22.7529405865607\\
4459.89528919383	-22.6970194329211\\
4480.49155237446	-22.6402066458084\\
4501.18293125388	-22.5824829038079\\
4521.96986508636	-22.5238283243036\\
4542.85279515466	-22.4642224431802\\
4563.83216477945	-22.4036441936967\\
4584.90841932869	-22.3420718845021\\
4606.0820062271	-22.2794831767603\\
4627.35337496566	-22.2158550603506\\
4648.72297711113	-22.1511638291128\\
4670.19126631568	-22.085385055104\\
4691.75869832646	-22.0184935618353\\
4713.42573099534	-21.9504633964577\\
4735.19282428857	-21.8812678008685\\
4757.06044029659	-21.8108791817102\\
4779.02904324381	-21.7392690792382\\
4801.09909949849	-21.6664081350355\\
4823.27107758263	-21.5922660585583\\
4845.54544818189	-21.5168115925013\\
4867.92268415562	-21.4400124769793\\
4890.4032605469	-21.361835412529\\
4912.98765459258	-21.2822460219463\\
4935.67634573345	-21.201208810988\\
4958.46981562442	-21.1186871279798\\
4981.36854814472	-21.0346431223941\\
5004.37302940819	-20.9490377024812\\
5027.48374777359	-20.8618304920659\\
5050.70119385497	-20.7729797866525\\
5074.02586053209	-20.6824425090204\\
5097.4582429609	-20.5901741645373\\
5120.998838584	-20.4961287964734\\
5144.64814714125	-20.4002589416602\\
5168.40667068035	-20.3025155869211\\
5192.27491356753	-20.2028481267847\\
5216.2533824982	-20.1012043231061\\
5240.34258650778	-19.9975302673464\\
5264.54303698246	-19.8917703464145\\
5288.85524767004	-19.7838672131552\\
5313.27973469088	-19.6737617627831\\
5337.81701654885	-19.5613931168151\\
5362.46761414231	-19.446698616358\\
5387.23205077517	-19.3296138269626\\
5412.11085216805	-19.2100725576848\\
5437.10454646936	-19.0880068974974\\
5462.21366426659	-18.9633472727928\\
5487.43873859752	-18.8360225304342\\
5512.78030496154	-18.7059600516524\\
5538.23890133107	-18.573085903092\\
5563.81506816292	-18.4373250325008\\
5589.50934840979	-18.298601517972\\
5615.32228753178	-18.1568388813245\\
5641.254433508	-18.011960478199\\
5667.30633684818	-17.8638899798031\\
5693.47855060436	-17.7125519640258\\
5719.77163038263	-17.5578726369361\\
5746.18613435493	-17.3997807095559\\
5772.72262327089	-17.2382084593557\\
5799.38166046975	-17.0730930112446\\
5826.16381189231	-16.904377879033\\
5853.06964609293	-16.7320148155086\\
5880.09973425162	-16.5559660274878\\
5907.25465018618	-16.3762068215151\\
5934.53497036433	-16.1927287562764\\
5961.94127391598	-16.0055433891592\\
5989.47414264556	-15.8146867164584\\
6017.13416104428	-15.6202244189818\\
6044.92191630263	-15.4222580363906\\
6072.83799832281	-15.2209322031367\\
6100.88299973121	-15.0164430842266\\
6129.05751589107	-14.8090481471735\\
6157.36214491506	-14.5990773930027\\
6185.79748767801	-14.3869461379944\\
6214.36414782964	-14.1731693809279\\
6243.06273180741	-13.9583776976175\\
6271.89384884933	-13.7433344630096\\
6300.85811100697	-13.5289539969905\\
6329.95613315842	-13.3163199494567\\
6359.18853302131	-13.1067028726841\\
6388.55593116599	-12.9015754732914\\
6418.05895102864	-12.7026235083418\\
6447.69821892457	-12.5117497352518\\
6477.47436406142	-12.3310678289823\\
6507.38801855264	-12.1628828785513\\
6537.43981743082	-12.0096551538317\\
6567.63039866118	-11.8739445070347\\
6597.96040315515	-11.7583342328294\\
6628.43047478397	-11.6653355418186\\
6659.0412603923	-11.5972768799528\\
6689.79340981204	-11.5561857353643\\
6720.68757587607	-11.5436736022197\\
6751.7244144321	-11.5608365396528\\
6782.90458435663	-11.6081834936323\\
6814.22874756894	-11.6856018891169\\
6845.69756904508	-11.7923652621898\\
6877.31171683206	-11.9271818629781\\
6909.07186206199	-12.0882775713433\\
6940.97867896634	-12.2735023933392\\
6973.03284489025	-12.4804480045122\\
7005.23504030693	-12.7065643363614\\
7037.58594883204	-12.9492655241402\\
7070.0862572383	-13.2060187956624\\
7102.73665547	-13.4744132386746\\
7135.53783665763	-13.7522082258125\\
7168.49049713269	-14.0373632957134\\
7201.59533644237	-14.3280524490772\\
7234.85305736446	-14.6226662540428\\
7268.26436592224	-14.9198050738927\\
7301.82997139948	-15.2182663398705\\
7335.55058635552	-15.5170282617528\\
7369.42692664033	-15.8152318161895\\
7403.45971140979	-16.1121623474849\\
7437.64966314091	-16.4072316906497\\
7471.99750764715	-16.6999613909812\\
7506.50397409388	-16.9899673427265\\
7541.16979501382	-17.2769459893672\\
7575.99570632259	-17.5606621055086\\
7610.98244733438	-17.8409381016407\\
7646.13076077758	-18.1176447464101\\
7681.44139281058	-18.3906931770224\\
7716.91509303763	-18.6600280596654\\
7752.55261452471	-18.9256217630008\\
7788.35471381553	-19.1874694149276\\
7824.32215094763	-19.4455847232854\\
7860.45568946846	-19.6999964531114\\
7896.7560964516	-19.9507454653351\\
7933.22414251308	-20.1978822336538\\
7969.86060182773	-20.4414647673666\\
8006.66625214554	-20.6815568779624\\
8043.6418748083	-20.9182267361714\\
8080.78825476606	-21.1515456740237\\
8118.10618059391	-21.3815871932702\\
8155.5964445086	-21.6084261473966\\
8193.25984238547	-21.8321380695016\\
8231.09717377527	-22.0527986226111\\
8269.10924192116	-22.2704831526576\\
8307.29685377578	-22.4852663274583\\
8345.66082001834	-22.6972218476493\\
8384.20195507185	-22.9064222177511\\
8422.92107712042	-23.1129385674155\\
8461.81900812665	-23.3168405144788\\
8500.89657384898	-23.5181960627783\\
8540.15460385935	-23.7170715288094\\
8579.59393156072	-23.9135314922438\\
8619.21539420481	-24.1076387661231\\
8659.01983290983	-24.2994543832129\\
8699.0080926784	-24.4890375955638\\
8739.18102241541	-24.6764458848029\\
8779.5394749461	-24.8617349810762\\
8820.08430703417	-25.0449588889026\\
8860.81637939989	-25.2261699184809\\
8901.73655673847	-25.4054187212327\\
8942.84570773836	-25.5827543285658\\
8984.14470509972	-25.7582241930101\\
9025.63442555289	-25.9318742310255\\
9067.31574987707	-26.1037488668987\\
9109.18956291901	-26.2738910772493\\
9151.25675361173	-26.4423424357503\\
9193.51821499347	-26.6091431577422\\
9235.97484422661	-26.7743321444777\\
9278.62754261669	-26.9379470267883\\
9321.47721563161	-27.1000242080053\\
9364.5247729208	-27.2605989060018\\
9407.77112833455	-27.4197051942581\\
9451.21719994339	-27.577376041871\\
9494.86391005764	-27.7336433524546\\
9538.71218524688	-27.8885380018932\\
9582.76295635973	-28.0420898749244\\
9627.01715854358	-28.1943279005403\\
9671.47573126437	-28.3452800862065\\
9716.13961832666	-28.4949735509045\\
9761.00976789355	-28.6434345570114\\
9806.08713250686	-28.7906885410342\\
9851.37266910737	-28.9367601432222\\
9896.86733905512	-29.0816732360824\\
9942.57210814977	-29.2254509518259\\
9988.48794665118	-29.3681157087755\\
10034.6158293	-29.5096892367649\\
10080.9567353382	-29.6501926015623\\
10127.5116485301	-29.7896462283498\\
10174.2815571832	-29.9280699242924\\
10221.267454169	-30.065482900227\\
10268.4703369442	-30.2019037915057\\
10315.891207572	-30.337350678024\\
10363.531072743	-30.4718411034648\\
10411.3909437969	-30.6053920937891\\
10459.4718367439	-30.7380201750031\\
10507.7747722864	-30.8697413902302\\
10556.3007758401	-31.0005713161159\\
10605.0508775566	-31.1305250785927\\
10654.0261123446	-31.2596173680316\\
10703.2275198921	-31.387862453805\\
10752.6561446888	-31.5152741982851\\
10802.3130360475	-31.6418660703017\\
10852.1992481272	-31.7676511580821\\
10902.315839955	-31.8926421816935\\
10952.6638754485	-32.0168515050106\\
11003.244423439	-32.1402911472266\\
11054.0585576935	-32.2629727939273\\
11105.1073569377	-32.384907807747\\
11156.3919048792	-32.506107238623\\
11207.9132902301	-32.626581833666\\
11259.6726067303	-32.7463420466622\\
11311.6709531708	-32.8653980472229\\
11363.9094334169	-32.983759729595\\
11416.3891564316	-33.1014367211484\\
11469.1112362992	-33.218438390552\\
11522.0767922492	-33.3347738556523\\
11575.2869486794	-33.4504519910657\\
11628.7428351806	-33.5654814354981\\
11682.4455865599	-33.6798705988005\\
11736.3963428651	-33.7936276687736\\
11790.596249409	-33.9067606177303\\
11845.0464567934	-34.0192772088268\\
11899.7481209338	-34.1311850021699\\
11954.7024030838	-34.2424913607125\\
12009.9104698599	-34.3532034559421\\
12065.3734932659	-34.4633282733744\\
12121.0926507183	-34.5728726178556\\
12177.069125071	-34.681843118685\\
12233.3041046402	-34.7902462345617\\
12289.7987832301	-34.898088258365\\
12346.554360158	-35.0053753217718\\
12403.5720402798	-35.1121133997209\\
12460.8530340153	-35.2183083147273\\
12518.3985573745	-35.3239657410533\\
12576.2098319827	-35.4290912087422\\
12634.2880851072	-35.5336901075191\\
12692.6345496825	-35.6377676905644\\
12751.2504643373	-35.7413290781644\\
12810.1370734203	-35.8443792612431\\
12869.2956270265	-35.9469231047814\\
12928.7273810244	-36.0489653511256\\
12988.4335970818	-36.1505106231907\\
13048.4155426933	-36.2515634275622\\
13108.6744912069	-36.3521281574989\\
13169.2117218509	-36.4522090958423\\
13230.0285197614	-36.5518104178341\\
13291.1261760091	-36.6509361938459\\
13352.5059876274	-36.7495903920243\\
13414.1692576393	-36.8477768808542\\
13476.1172950852	-36.9454994316423\\
13538.351415051	-37.0427617209258\\
13600.8729386957	-37.1395673328062\\
13663.6831932795	-37.2359197612123\\
13726.7835121922	-37.3318224120956\\
13790.1752349813	-37.4272786055576\\
13853.8597073802	-37.522291577914\\
13917.8382813373	-37.6168644836971\\
13982.1123150444	-37.7110003975974\\
14046.6831729655	-37.8047023163481\\
14111.552225866	-37.897973160553\\
14176.7208508414	-37.9908157764602\\
14242.190431347	-38.0832329376832\\
14307.9623572268	-38.1752273468713\\
14374.0380247435	-38.2668016373312\\
14440.4188366077	-38.3579583745998\\
14507.1062020079	-38.4487000579716\\
14574.1015366405	-38.5390291219812\\
14641.4062627396	-38.6289479378424\\
14709.0218091073	-38.7184588148448\\
14776.9496111443	-38.8075640017104\\
14845.1911108798	-38.8962656879097\\
14913.7477570027	-38.9845660049398\\
14982.6210048919	-39.0724670275654\\
15051.8123166476	-39.1599707750233\\
15121.323161122	-39.2470792121916\\
15191.1550139505	-39.3337942507254\\
15261.3093575835	-39.4201177501584\\
15331.7876813171	-39.506051518973\\
15402.5914813253	-39.5915973156386\\
15473.7222606918	-39.6767568496197\\
15545.1815294413	-39.761531782354\\
15616.9708045722	-39.8459237282019\\
15689.0916100885	-39.929934255368\\
15761.5454770323	-40.0135648867949\\
15834.333943516	-40.0968171010305\\
15907.4585547554	-40.1796923330695\\
15980.9208631021	-40.2621919751695\\
16054.7224280766	-40.3443173776425\\
16128.8648164017	-40.4260698496223\\
16203.3496020352	-40.5074506598085\\
16278.1783662036	-40.5884610371876\\
16353.352697436	-40.6691021717319\\
16428.8741915971	-40.7493752150765\\
16504.7444519217	-40.8292812811751\\
16580.9650890484	-40.908821446935\\
16657.537721054	-40.9879967528318\\
16734.4639734876	-41.066808203504\\
16811.7454794054	-41.1452567683288\\
16889.3838794052	-41.2233433819782\\
16967.3808216612	-41.301068944957\\
17045.7379619589	-41.3784343241226\\
17124.4569637308	-41.4554403531865\\
17203.539498091	-41.5320878331993\\
17282.9872438709	-41.6083775330174\\
17362.8018876552	-41.6843101897539\\
17442.9851238172	-41.7598865092128\\
17523.5386545551	-41.8351071663069\\
17604.464189928	-41.9099728054604\\
17685.7634478922	-41.984484040995\\
17767.4381543379	-42.0586414575017\\
17849.4900431252	-42.1324456101972\\
17931.9208561219	-42.2058970252661\\
18014.7323432394	-42.2789962001874\\
18097.9262624709	-42.3517436040482\\
18181.5043799277	-42.4241396778423\\
18265.4684698776	-42.4961848347561\\
18349.8203147818	-42.5678794604394\\
18434.5617053333	-42.6392239132643\\
18519.6944404947	-42.7102185245704\\
18605.2203275364	-42.7808635988963\\
18691.1411820748	-42.8511594141995\\
18777.4588281113	-42.9211062220626\\
18864.1750980704	-42.9907042478876\\
18951.2918328392	-43.0599536910775\\
19038.810881806	-43.128854725206\\
19126.7341029	-43.1974074981748\\
19215.0633626303	-43.2656121323594\\
19303.8005361259	-43.3334687247424\\
19392.9475071751	-43.4009773470357\\
19482.506168266	-43.4681380457911\\
19572.4784206263	-43.5349508424988\\
19662.8661742637	-43.6014157336753\\
19753.6713480067	-43.6675326909394\\
19844.8958695449	-43.733301661077\\
19936.5416754703	-43.7987225660954\\
20028.6107113184	-43.8637953032656\\
20121.1049316092	-43.9285197451543\\
20214.026299889	-43.9928957396445\\
20307.3767887718	-44.0569231099456\\
20401.1583799817	-44.1206016545921\\
20495.373064394	-44.1839311474319\\
20590.0228420787	-44.2469113376041\\
20685.1097223421	-44.3095419495048\\
20780.6357237695	-44.371822682744\\
20876.6028742684	-44.4337532120898\\
20973.0132111115	-44.4953331874039\\
21069.8687809795	-44.5565622335643\\
21167.1716400053	-44.6174399503793\\
21264.923853817	-44.6779659124894\\
21363.127497582	-44.7381396692587\\
21461.7846560511	-44.7979607446562\\
21560.8974236026	-44.8574286371258\\
21660.467904287	-44.9165428194453\\
21760.4982118713	-44.975302738575\\
21860.9904698845	-45.0337078154956\\
21961.9468116618	-45.0917574450339\\
22063.3693803907	-45.1494509956794\\
22165.260329156	-45.206787809388\\
22267.6218209858	-45.2637672013761\\
22370.4560288971	-45.3203884599022\\
22473.7651359423	-45.376650846038\\
22577.5513352553	-45.4325535934281\\
22681.8168300982	-45.4880959080376\\
22786.5638339077	-45.5432769678887\\
22891.7945703429	-45.5980959227857\\
22997.5112733313	-45.6525518940273\\
23103.7161871177	-45.706643974108\\
23210.4115663104	-45.7603712264069\\
23317.5996759301	-45.8137326848639\\
23425.2827914574	-45.8667273536445\\
23533.4631988815	-45.919354206791\\
23642.1431947482	-45.9716121878618\\
23751.3250862094	-46.0235002095575\\
23861.0111910714	-46.0750171533334\\
23971.2038378446	-46.1261618690001\\
24081.9053657923	-46.1769331743088\\
24193.1181249812	-46.2273298545247\\
24304.8444763306	-46.277350661985\\
24417.0867916629	-46.3269943156439\\
24529.8474537537	-46.3762595006024\\
24643.1288563827	-46.4251448676238\\
24756.933404384	-46.4736490326341\\
24871.2635136979	-46.5217705762078\\
24986.1216114214	-46.569508043037\\
25101.5101358603	-46.616859941386\\
25217.4315365806	-46.6638247425288\\
25333.8882744609	-46.7104008801709\\
25450.882821744	-46.7565867498536\\
25568.4176620901	-46.8023807083417\\
25686.4952906292	-46.8477810729935\\
25805.1182140139	-46.8927861211127\\
25924.2889504728	-46.9373940892822\\
26044.0100298642	-46.9816031726793\\
26164.2839937291	-47.0254115243713\\
26285.113395346	-47.0688172545917\\
26406.5007997846	-47.1118184299964\\
26528.4487839603	-47.1544130728992\\
26650.959936689	-47.1965991604863\\
26774.0368587419	-47.2383746240087\\
26897.6821629011	-47.2797373479533\\
27021.8984740145	-47.3206851691905\\
27146.6884290521	-47.3612158760987\\
27272.0546771616	-47.4013272076648\\
27397.9998797246	-47.4410168525603\\
27524.5267104134	-47.4802824481918\\
27651.6378552475	-47.5191215797258\\
27779.3360126509	-47.5575317790864\\
27907.623893509	-47.5955105239264\\
28036.5042212264	-47.6330552365693\\
28165.9797317848	-47.6701632829228\\
28296.0531738006	-47.7068319713621\\
28426.7273085842	-47.7430585515831\\
28558.0049101974	-47.7788402134229\\
28689.8887655133	-47.8141740856492\\
28822.3816742749	-47.8490572347145\\
28955.4864491548	-47.8834866634774\\
29089.2059158146	-47.9174593098866\\
29223.5429129655	-47.9509720456298\\
29358.5002924277	-47.9840216747436\\
29494.0809191919	-48.0166049321849\\
29630.2876714792	-48.0487184823617\\
29767.1234408028	-48.0803589176229\\
29904.5911320291	-48.1115227567045\\
30042.6936634399	-48.1422064431318\\
30181.4339667932	-48.1724063435765\\
30320.8149873869	-48.2021187461656\\
30460.8396841202	-48.2313398587418\\
30601.5110295567	-48.2600658070741\\
30742.8320099879	-48.288292633016\\
30884.8056254962	-48.3160162926089\\
31027.4348900188	-48.3432326541314\\
31170.7228314113	-48.36993749609\\
31314.6724915125	-48.3961265051505\\
31459.2869262085	-48.4217952740073\\
31604.5692054981	-48.4469392991901\\
31750.5224135574	-48.4715539788029\\
};
\addlegendentry{$\Delta\text{(z)}$};

\addplot [color=black!50!mycolor1,dashed,line width=1.5pt]
  table[row sep=crcr]{10	16.1788004247686\\
10.0461810465159	16.1387804170903\\
10.0925753619375	16.0987604094335\\
10.139183931163	16.0587404017984\\
10.1860077436388	16.0187203941853\\
10.2330477933808	15.9787003865944\\
10.2803050789954	15.9386803790257\\
10.3277806037004	15.8986603714797\\
10.375475375347	15.8586403639564\\
10.4233904064403	15.8186203564561\\
10.4715267141616	15.7786003489789\\
10.5198853203895	15.7385803415252\\
10.5684672517218	15.6985603340951\\
10.6172735394972	15.6585403266888\\
10.6663052198171	15.6185203193066\\
10.715563333568	15.5785003119486\\
10.7650489264432	15.5384803046152\\
10.814763048965	15.4984602973065\\
10.8647067565072	15.4584402900227\\
10.9148811093176	15.4184202827642\\
10.9652871725401	15.3784002755311\\
11.0159260162376	15.3383802683236\\
11.0667987154147	15.298360261142\\
11.1179063500406	15.2583402539866\\
11.1692500050717	15.2183202468576\\
11.2208307704748	15.1783002397551\\
11.2726497412507	15.1382802326796\\
11.3247080174565	15.0982602256312\\
11.3770067042298	15.0582402186101\\
11.4295469118117	15.0182202116167\\
11.4823297555707	14.9782002046511\\
11.535356356026	14.9381801977137\\
11.5886278388715	14.8981601908046\\
11.6421453349997	14.8581401839243\\
11.6959099805258	14.8181201770729\\
11.7499229168114	14.7781001702506\\
11.8041852904894	14.7380801634578\\
11.8586982534876	14.6980601566948\\
11.9134629630538	14.6580401499618\\
11.96848058178	14.618020143259\\
12.0237522776272	14.5780001365869\\
12.0792792239501	14.5379801299456\\
12.135062599522	14.4979601233354\\
12.1911035885602	14.4579401167566\\
12.2474033807505	14.4179201102096\\
12.3039631712731	14.3779001036946\\
12.3607841608273	14.3378800972119\\
12.4178675556577	14.2978600907618\\
12.4752145675793	14.2578400843446\\
12.5328264140034	14.2178200779606\\
12.5907043179635	14.1778000716102\\
12.6488495081411	14.1377800652936\\
12.7072632188919	14.0977600590111\\
12.765946690272	14.0577400527631\\
12.8249011680643	14.0177200465498\\
12.8841279038047	13.9777000403717\\
12.9436281548089	13.937680034229\\
13.0034031841991	13.8976600281221\\
13.0634542609305	13.8576400220512\\
13.1237826598188	13.8176200160168\\
13.1843896615665	13.7776000100191\\
13.2452765527909	13.7375800040585\\
13.306444626051	13.6975599981354\\
13.3678951798747	13.65753999225\\
13.4296295187868	13.6175199864028\\
13.4916489533366	13.5774999805941\\
13.5539548001256	13.5374799748242\\
13.6165483818355	13.4974599690935\\
13.6794310272563	13.4574399634024\\
13.7426040713143	13.4174199577512\\
13.806068855101	13.3773999521404\\
13.8698267259009	13.3373799465702\\
13.9338790372205	13.2973599410411\\
13.998227148817	13.2573399355534\\
14.062872426727	13.2173199301075\\
14.1278162432955	13.1772999247038\\
14.1930599772055	13.1372799193428\\
14.2586050135065	13.0972599140247\\
14.3244527436445	13.05723990875\\
14.3906045654914	13.0172199035191\\
14.4570618833745	12.9771998983324\\
14.5238261081064	12.9371798931903\\
14.5908986570152	12.8971598880932\\
14.658280953974	12.8571398830416\\
14.7259744294318	12.8171198780358\\
14.7939805204436	12.7770998730763\\
14.8623006707006	12.7370798681635\\
14.9309363305612	12.6970598632979\\
14.999888957082	12.6570398584798\\
15.069160014048	12.6170198537097\\
15.1387509720044	12.5769998489881\\
15.2086633082875	12.5369798443155\\
15.2788985070559	12.4969598396921\\
15.3494580593225	12.4569398351186\\
15.4203434629857	12.4169198305954\\
15.4915562228612	12.3768998261229\\
15.5630978507143	12.3368798217017\\
15.6349698652918	12.2968598173321\\
15.7071737923542	12.2568398130148\\
15.779711164708	12.21681980875\\
15.8525835222385	12.1767998045384\\
15.9257924119422	12.1367798003805\\
15.9993393879601	12.0967597962766\\
16.07322601161	12.0567397922274\\
16.1474538514202	12.0167197882334\\
16.2220244831628	11.9766997842949\\
16.2969394898867	11.9366797804127\\
16.3722004619516	11.8966597765871\\
16.4478089970617	11.8566397728187\\
16.5237667002994	11.8166197691081\\
16.6000751841599	11.7765997654557\\
16.6767360685846	11.7365797618622\\
16.7537509809962	11.696559758328\\
16.8311215563331	11.6565397548537\\
16.9088494370839	11.6165197514398\\
16.9869362733223	11.576499748087\\
17.0653837227424	11.5364797447958\\
17.1441934506935	11.4964597415667\\
17.2233671302159	11.4564397384003\\
17.302906442076	11.4164197352973\\
17.3828130748022	11.3763997322581\\
17.4630887247206	11.3363797292835\\
17.5437350959914	11.2963597263739\\
17.6247539006444	11.25633972353\\
17.7061468586161	11.2163197207524\\
17.7879156977856	11.1762997180417\\
17.8700621540116	11.1362797153985\\
17.9525879711692	11.0962597128235\\
18.035494901187	11.0562397103173\\
18.1187847040838	11.0162197078805\\
18.2024591480069	10.9761997055137\\
18.2865200092687	10.9361797032177\\
18.3709690723849	10.896159700993\\
18.4558081301122	10.8561396988404\\
18.5410389834868	10.8161196967604\\
18.6266634418617	10.7760996947538\\
18.7126833229461	10.7360796928213\\
18.7991004528435	10.6960596909635\\
18.8859166660905	10.6560396891811\\
18.9731338056957	10.6160196874749\\
19.060753723179	10.5759996858454\\
19.1487782786108	10.5359796842935\\
19.2372093406515	10.4959596828199\\
19.3260487865912	10.4559396814252\\
19.4152985023894	10.4159196801103\\
19.5049603827153	10.3758996788757\\
19.5950363309877	10.3358796777224\\
19.6855282594159	10.2958596766511\\
19.7764380890397	10.2558396756624\\
19.8677677497706	10.2158196747572\\
19.9595191804325	10.1757996739363\\
20.0516943288031	10.1357796732003\\
20.1442951516552	10.0957596725502\\
20.2373236147981	10.0557396719867\\
20.3307816931193	10.0157196715106\\
20.4246713706267	9.97569967112266\\
20.5189946404906	9.93567967082379\\
20.6137535050858	9.89565967061477\\
20.7089499760343	9.85563967049644\\
20.8045860742482	9.81561967046964\\
20.900663829972	9.77559967053521\\
20.9971852828265	9.73557967069401\\
21.0941524818514	9.6955596709469\\
21.1915674855492	9.65553967129475\\
21.2894323619287	9.61551967173845\\
21.387749188549	9.57549967227887\\
21.4865200525637	9.53547967291692\\
21.5857470507649	9.4954596736535\\
21.685432289628	9.45543967448952\\
21.7855778853565	9.4154196754259\\
21.8861859639264	9.37539967646357\\
21.987258661132	9.33537967760347\\
22.0887981226306	9.29535967884654\\
22.1908065039887	9.25533968019374\\
22.2932859707273	9.21531968164603\\
22.396238698368	9.17529968320439\\
22.499666872479	9.13527968486981\\
22.603572688722	9.09525968664325\\
22.7079583528983	9.05523968852574\\
22.8128260809959	9.01521969051828\\
22.9181780992365	8.97519969262189\\
23.0240166441225	8.93517969483759\\
23.130343962485	8.89515969716642\\
23.237162311531	8.85513969960943\\
23.3444739588916	8.81511970216769\\
23.4522811826701	8.77509970484225\\
23.5605862714901	8.73507970763418\\
23.6693915245447	8.69505971054459\\
23.7786992516445	8.65503971357457\\
23.8885117732672	8.61501971672521\\
23.9988314206069	8.57499971999764\\
24.1096605356231	8.53497972339299\\
24.2210014710909	8.49495972691239\\
24.3328565906507	8.454939730557\\
24.4452282688584	8.41491973432797\\
24.558118891236	8.37489973822647\\
24.6715308543219	8.33487974225369\\
24.7854665657221	8.2948597464108\\
24.899928444161	8.25483975069903\\
25.0149189195332	8.21481975511956\\
25.1304404329547	8.17479975967365\\
25.2464954368146	8.13477976436252\\
25.3630863948276	8.09475976918742\\
25.4802157820863	8.0547397741496\\
25.597886085113	8.01471977925035\\
25.7160998019135	7.97469978449094\\
25.8348594420295	7.93467978987266\\
25.9541675265918	7.89465979539683\\
26.0740265883745	7.85463980106476\\
26.194439171848	7.81461980687777\\
26.3154078332332	7.77459981283724\\
26.4369351405564	7.73457981894448\\
26.5590236737027	7.69455982520089\\
26.6816760244719	7.65453983160785\\
26.8048947966327	7.61451983816673\\
26.9286826059784	7.57449984487895\\
27.0530420803822	7.53447985174594\\
27.1779758598533	7.49445985876911\\
27.3034865965925	7.45443986594993\\
27.4295769550488	7.41441987328985\\
27.556249611976	7.37439988079033\\
27.6835072564894	7.33437988845288\\
27.8113525901229	7.29435989627898\\
27.9397883268864	7.25433990427016\\
28.0688171933232	7.21431991242793\\
28.1984419285683	7.17429992075385\\
28.3286652844062	7.13427992924946\\
28.4594900253294	7.09425993791635\\
28.5909189285972	7.05423994675609\\
28.7229547842946	7.01421995577029\\
28.8556003953913	6.97419996496056\\
28.9888585778017	6.93417997432853\\
29.1227321604441	6.89415998387585\\
29.2572239853012	6.85413999360417\\
29.3923369074803	6.81412000351517\\
29.5280737952738	6.77410001361055\\
29.6644375302202	6.73408002389201\\
29.8014310071653	6.69406003436126\\
29.9390571343235	6.65404004502007\\
30.0773188333397	6.61402005587016\\
30.2162190393513	6.57400006691333\\
30.3557607010504	6.53398007815135\\
30.4959467807464	6.49396008958602\\
30.6367802544292	6.45394010121918\\
30.7782641118319	6.41392011305265\\
30.9204013564946	6.37390012508829\\
31.063195005828	6.33388013732798\\
31.2066480911777	6.29386014977359\\
31.350763657888	6.25384016242705\\
31.4955447653674	6.21382017529026\\
31.6409944871527	6.17380018836517\\
31.7871159109747	6.13378020165376\\
31.9339121388238	6.09376021515797\\
32.0813862870155	6.05374022887983\\
32.229541486257	6.01372024282135\\
32.3783808817133	5.97370025698454\\
32.5279076330741	5.93368027137148\\
32.6781249146208	5.89366028598422\\
32.8290359152942	5.85364030082487\\
32.9806438387618	5.81362031589552\\
33.132951903486	5.77360033119831\\
33.2859633427924	5.73358034673539\\
33.4396814049384	5.69356036250893\\
33.5941093531822	5.65354037852111\\
33.7492504658521	5.61352039477415\\
33.9051080364161	5.57350041127027\\
34.0616853735517	5.53348042801173\\
34.2189858012163	5.4934604450008\\
34.3770126587175	5.45344046223975\\
34.5357693007845	5.41342047973093\\
34.6952590976386	5.37340049747666\\
34.8554854350656	5.33338051547928\\
35.0164517144866	5.29336053374119\\
35.1781613530314	5.25334055226478\\
35.3406177836102	5.21332057105247\\
35.5038244549868	5.17330059010671\\
35.6677848318515	5.13328060942997\\
35.8325023948954	5.09326062902474\\
35.9979806408833	5.05324064889352\\
36.1642230827288	5.01322066903887\\
36.3312332495683	4.97320068946334\\
36.4990146868361	4.9331807101695\\
36.6675709563398	4.89316073115998\\
36.8369056363357	4.8531407524374\\
37.007022321605	4.81312077400441\\
37.1779246235299	4.77310079586371\\
37.3496161701703	4.73308081801798\\
37.5221006063408	4.69306084046998\\
37.6953815936883	4.65304086322245\\
37.8694628107696	4.61302088627817\\
38.0443479531292	4.57300090963995\\
38.2200407333782	4.53298093331063\\
38.3965448812729	4.49296095729306\\
38.5738641437941	4.45294098159012\\
38.7520022852263	4.41292100620475\\
38.9309630872381	4.37290103113987\\
39.1107503489621	4.33288105639844\\
39.2913678870758	4.29286108198347\\
39.4728195358824	4.25284110789797\\
39.6551091473924	4.212821134145\\
39.8382405914052	4.17280116072764\\
40.0222177555915	4.13278118764899\\
40.2070445455755	4.09276121491218\\
40.3927248850181	4.05274124252038\\
40.5792627157	4.0127212704768\\
40.7666619976054	3.97270129878464\\
40.9549267090063	3.93268132744716\\
41.1440608465467	3.89266135646765\\
41.3340684253274	3.85264138584942\\
41.5249534789915	3.81262141559581\\
41.7167200598098	3.77260144571021\\
41.9093722387671	3.732581476196\\
42.1029141056481	3.69256150705665\\
42.2973497691249	3.65254153829561\\
42.4926833568435	3.6125215699164\\
42.6889190155123	3.57250160192252\\
42.8860609109892	3.53248163431757\\
43.0841132283705	3.49246166710515\\
43.2830801720801	3.45244170028886\\
43.4829659659579	3.41242173387241\\
43.6837748533501	3.37240176785948\\
43.8855110971994	3.33238180225381\\
44.0881789801347	3.29236183705916\\
44.2917828045629	3.25234187227935\\
44.4963268927599	3.21232190791822\\
44.701815586962	3.17230194397963\\
44.9082532494586	3.1322819804675\\
45.1156442626847	3.09226201738579\\
45.3239930293136	3.05224205473846\\
45.5333039723509	3.01222209252956\\
45.7435815352278	2.97220213076312\\
45.954830181896	2.93218216944326\\
46.167054396922	2.8921622085741\\
46.3802586855826	2.85214224815982\\
46.5944475739604	2.81212228820463\\
46.8096256090399	2.77210232871278\\
47.0257973588042	2.73208236968855\\
47.2429674123316	2.69206241113628\\
47.4611403798933	2.65204245306033\\
47.6803208930514	2.61202249546513\\
47.9005136047569	2.5720025383551\\
48.1217231894485	2.53198258173475\\
48.3439543431522	2.49196262560862\\
48.5672117835806	2.45194266998126\\
48.791500250233	2.41192271485732\\
49.0168245044966	2.37190276024144\\
49.243189329747	2.33188280613832\\
49.4705995314498	2.29186285255272\\
49.6990599372628	2.25184289948942\\
49.9285753971387	2.21182294695326\\
50.1591507834275	2.17180299494912\\
50.3907909909802	2.13178304348193\\
50.6235009372529	2.09176309255665\\
50.8572855624109	2.05174314217831\\
51.0921498294339	2.01172319235196\\
51.3280987242209	1.97170324308272\\
51.5651372556965	1.93168329437574\\
51.8032704559168	1.89166334623623\\
52.0425033801768	1.85164339866945\\
52.2828411071172	1.81162345168068\\
52.5242887388322	1.7716035052753\\
52.7668514009784	1.73158355945869\\
53.010534242883	1.6915636142363\\
53.2553424376533	1.65154366961364\\
53.5012811822866	1.61152372559627\\
53.7483556977805	1.57150378218977\\
53.9965712292437	1.53148383939982\\
54.2459330460073	1.49146389723211\\
54.4964464417369	1.4514439556924\\
54.7481167345445	1.41142401478652\\
55.0009492671021	1.37140407452033\\
55.2549494067543	1.33138413489974\\
55.510122545633	1.29136419593074\\
55.7664741007712	1.25134425761936\\
56.0240095142187	1.21132431997169\\
56.282734253157	1.17130438299386\\
56.5426538100157	1.13128444669209\\
56.8037737025889	1.09126451107263\\
57.0660994741527	1.05124457614179\\
57.3296366935823	1.01122464190596\\
57.5943909554708	0.971204708371564\\
57.8603678802477	0.931184775545096\\
58.1275731142982	0.891164843433117\\
58.3960123300829	0.851144912042232\\
58.6656912262587	0.811124981379119\\
58.9366155277994	0.771105051450519\\
59.2087909861172	0.731085122263228\\
59.4822233791852	0.691065193824111\\
59.7569185116595	0.651045266140094\\
60.0328822150028	0.611025339218169\\
60.3101203476082	0.571005413065381\\
60.5886387949234	0.530985487688868\\
60.8684434695757	0.490965563095805\\
61.1495403114975	0.450945639293434\\
61.4319352880526	0.410925716289101\\
61.7156343941624	0.370905794090179\\
62.0006436524339	0.330885872704122\\
62.2869691132868	0.290865952138455\\
62.5746168550822	0.250846032400783\\
62.8635929842521	0.210826113498761\\
63.1539036354283	0.170806195440126\\
63.445554971573	0.130786278232696\\
63.7385531841099	0.090766361884336\\
64.032904493055	0.0507464464030075\\
64.3286151471491	0.0107265317967402\\
64.6256914239905	-0.0292933819263752\\
64.9241396301678	-0.0693132947581566\\
65.2239661013943	-0.109333206690349\\
65.5251772026422	-0.149353117714632\\
65.827779328278	-0.189373027822596\\
66.1317789021976	-0.229392937005754\\
66.4371823779638	-0.269412845255557\\
66.7439962389419	-0.309432752563353\\
67.0522269984386	-0.349452658920429\\
67.3618811998395	-0.389472564317972\\
67.6729654167483	-0.429492468747119\\
67.9854862531262	-0.469512372198881\\
68.2994503434323	-0.509532274664228\\
68.6148643527643	-0.549552176134022\\
68.931734977	-0.589572076599048\\
69.2500689429393	-0.629591976050001\\
69.5698730084476	-0.669611874477497\\
69.8911539625984	-0.709631771872063\\
70.213918625818	-0.749651668224126\\
70.5381738500302	-0.789671563524046\\
70.8639265188016	-0.829691457762077\\
71.191183547488	-0.86971135092839\\
71.5199518833808	-0.909731243013068\\
71.8502385058548	-0.949751134006092\\
72.1820504265165	-0.989771023897354\\
72.5153946893524	-1.02979091267666\\
72.8502783708792	-1.06981080033372\\
73.1867085802933	-1.10983068685813\\
73.5246924596224	-1.14985057223942\\
73.8642371838769	-1.189870456467\\
74.2053499612018	-1.22989033953019\\
74.5480380330304	-1.2699102214182\\
74.8923086742377	-1.30993010212018\\
75.2381691932944	-1.34994998162512\\
75.585626932423	-1.38996985992195\\
75.9346892677529	-1.42998973699949\\
76.2853636094773	-1.47000961284645\\
76.6376574020103	-1.51002948745143\\
76.9915781241455	-1.55004936080294\\
77.3471332892138	-1.59006923288938\\
77.7043304452438	-1.63008910369902\\
78.0631771751216	-1.67010897322007\\
78.4236810967518	-1.71012884144057\\
78.7858498632195	-1.7501487083485\\
79.1496911629522	-1.79016857393171\\
79.5152127198837	-1.83018843817792\\
79.8824222936175	-1.87020830107476\\
80.2513276795919	-1.91022816260974\\
80.6219367092452	-1.95024802277026\\
80.9942572501823	-1.99026788154358\\
81.3682972063414	-2.03028773891687\\
81.7440645181619	-2.07030759487717\\
82.121567162753	-2.11032744941139\\
82.5008131540631	-2.15034730250633\\
82.8818105430498	-2.19036715414867\\
83.2645674178507	-2.23038700432496\\
83.6490919039556	-2.27040685302163\\
84.0353921643786	-2.31042670022498\\
84.423476399831	-2.35044654592119\\
84.8133528488963	-2.3904663900963\\
85.2050297882047	-2.43048623273624\\
85.5985155326084	-2.47050607382678\\
85.9938184353586	-2.51052591335359\\
86.3909468882829	-2.55054575130219\\
86.7899093219629	-2.59056558765798\\
87.1907142059136	-2.63058542240619\\
87.5933700487633	-2.67060525553196\\
87.9978853984339	-2.71062508702027\\
88.4042688423224	-2.75064491685594\\
88.8125290074835	-2.79066474502369\\
89.2226745608124	-2.83068457150808\\
89.6347142092288	-2.87070439629351\\
90.0486566998624	-2.91072421936426\\
90.4645108202374	-2.95074404070447\\
90.88228539846	-2.99076386029809\\
91.3019893034057	-3.03078367812899\\
91.7236314449071	-3.07080349418081\\
92.1472207739435	-3.11082330843711\\
92.5727662828307	-3.15084312088126\\
93.0002770054119	-3.19086293149647\\
93.4297620172496	-3.23088274026583\\
93.8612304358183	-3.27090254717223\\
94.2946914206979	-3.31092235219845\\
94.730154173768	-3.35094215532706\\
95.1676279394036	-3.3909619565405\\
95.6071220046713	-3.43098175582106\\
96.0486456995261	-3.47100155315081\\
96.4922083970099	-3.51102134851172\\
96.9378195134503	-3.55104114188554\\
97.3854885086602	-3.5910609332539\\
97.8352248861393	-3.63108072259821\\
98.2870381932753	-3.67110050989975\\
98.7409380215466	-3.71112029513961\\
99.1969340067261	-3.75114007829868\\
99.655035829086	-3.79115985935773\\
100.115253213603	-3.83117963829729\\
100.577595930163	-3.87119941509775\\
101.042073793774	-3.91121918973931\\
101.508696664767	-3.95123896220199\\
101.977474449012	-3.9912587324656\\
102.448417098122	-4.03127850050979\\
102.921534609671	-4.07129826631404\\
103.3968370274	-4.11131802985757\\
103.874334441436	-4.15133779111947\\
104.354036988501	-4.19135755007863\\
104.83595485213	-4.23137730671371\\
105.320098262886	-4.27139706100321\\
105.80647749858	-4.31141681292541\\
106.295102884484	-4.3514365624584\\
106.785984793556	-4.39145630958005\\
107.279133646656	-4.43147605426806\\
107.774559912768	-4.47149579649987\\
108.272274109224	-4.51151553625276\\
108.772286801926	-4.55153527350379\\
109.27460860557	-4.59155500822977\\
109.779250183872	-4.63157474040734\\
110.286222249794	-4.67159447001292\\
110.795535565772	-4.71161419702269\\
111.307200943943	-4.75163392141261\\
111.821229246378	-4.79165364315844\\
112.337631385307	-4.83167336223569\\
112.856418323356	-4.87169307861966\\
113.377601073777	-4.91171279228543\\
113.901190700682	-4.95173250320781\\
114.427198319278	-4.99175221136143\\
114.955635096105	-5.03177191672064\\
115.486512249268	-5.07179161925957\\
116.019841048683	-5.11181131895214\\
116.555632816306	-5.15183101577196\\
117.093898926384	-5.19185070969244\\
117.634650805689	-5.23187040068678\\
118.177899933763	-5.27189008872785\\
118.723657843162	-5.31190977378831\\
119.271936119701	-5.3519294558406\\
119.822746402699	-5.39194913485684\\
120.376100385228	-5.43196881080892\\
120.932009814357	-5.47198848366851\\
121.490486491407	-5.51200815340696\\
122.051542272196	-5.55202781999536\\
122.615189067297	-5.59204748340457\\
123.181438842284	-5.63206714360515\\
123.750303617991	-5.67208680056739\\
124.321795470765	-5.71210645426134\\
124.895926532723	-5.7521261046567\\
125.472708992008	-5.79214575172296\\
126.052155093052	-5.8321653954293\\
126.63427713683	-5.87218503574461\\
127.219087481126	-5.91220467263749\\
127.806598540793	-5.95222430607628\\
128.396822788018	-5.99224393602897\\
128.989772752585	-6.03226356246331\\
129.585461022141	-6.07228318534673\\
130.183900242466	-6.11230280464635\\
130.785103117737	-6.15232242032899\\
131.389082410804	-6.19234203236118\\
131.995850943453	-6.2323616407091\\
132.605421596686	-6.27238124533866\\
133.217807310988	-6.31240084621545\\
133.833021086605	-6.35242044330469\\
134.451075983821	-6.39244003657135\\
135.071985123233	-6.43245962598004\\
135.69576168603	-6.47247921149502\\
136.322418914273	-6.51249879308027\\
136.951970111177	-6.5525183706994\\
137.584428641392	-6.59253794431569\\
138.219807931287	-6.63255751389209\\
138.858121469236	-6.6725770793912\\
139.499382805904	-6.71259664077526\\
140.143605554534	-6.75261619800619\\
140.790803391236	-6.79263575104555\\
141.440990055278	-6.83265529985451\\
142.094179349378	-6.87267484439393\\
142.750385139995	-6.91269438462427\\
143.409621357626	-6.95271392050565\\
144.0719019971	-6.99273345199779\\
144.737241117876	-7.03275297906009\\
145.405652844341	-7.07277250165149\\
146.077151366109	-7.11279201973065\\
146.751750938323	-7.15281153325578\\
147.42946588196	-7.19283104218471\\
148.110310584131	-7.2328505464749\\
148.794299498388	-7.27287004608341\\
149.481447145032	-7.31288954096688\\
150.171768111418	-7.3529090310816\\
150.865277052271	-7.3929285163834\\
151.561988689989	-7.43294799682771\\
152.261917814963	-7.4729674723696\\
152.965079285884	-7.51298694296366\\
153.671488030065	-7.55300640856408\\
154.381159043753	-7.59302586912466\\
155.094107392451	-7.63304532459871\\
155.810348211234	-7.67306477493915\\
156.529896705074	-7.71308422009848\\
157.25276814916	-7.7531036600287\\
157.978977889225	-7.79312309468142\\
158.708541341869	-7.83314252400779\\
159.441473994887	-7.87316194795848\\
160.177791407599	-7.91318136648373\\
160.917509211179	-7.95320077953333\\
161.66064310899	-7.99322018705657\\
162.40720887691	-8.0332395890023\\
163.157222363676	-8.07325898531889\\
163.910699491214	-8.11327837595421\\
164.66765625498	-8.15329776085567\\
165.428108724297	-8.19331713997021\\
166.192073042701	-8.23333651324422\\
166.959565428276	-8.27335588062366\\
167.730602174008	-8.31337524205396\\
168.505199648122	-8.35339459748002\\
169.283374294433	-8.39341394684628\\
170.065142632699	-8.43343329009664\\
170.850521258965	-8.47345262717446\\
171.639526845917	-8.51347195802262\\
172.432176143241	-8.55349128258344\\
173.228485977972	-8.59351060079871\\
174.028473254854	-8.63352991260968\\
174.832154956701	-8.67354921795709\\
175.639548144754	-8.71356851678107\\
176.450669959044	-8.75358780902125\\
177.265537618758	-8.79360709461667\\
178.084168422602	-8.83362637350581\\
178.906579749169	-8.8736456456266\\
179.732789057308	-8.91366491091638\\
180.562813886497	-8.95368416931189\\
181.39667185721	-8.99370342074933\\
182.234380671297	-9.03372266516429\\
183.075958112354	-9.07374190249172\\
183.921422046107	-9.11376113266605\\
184.770790420785	-9.15378035562104\\
185.624081267505	-9.19379957128985\\
186.481312700654	-9.23381877960504\\
187.342502918271	-9.27383798049853\\
188.207670202438	-9.3138571739016\\
189.076832919665	-9.35387635974493\\
189.950009521279	-9.39389553795852\\
190.827218543818	-9.43391470847171\\
191.708478609426	-9.47393387121328\\
192.593808426241	-9.51395302611122\\
193.483226788801	-9.55397217309295\\
194.376752578439	-9.59399131208518\\
195.274404763682	-9.63401044301394\\
196.176202400658	-9.67402956580457\\
197.082164633495	-9.71404868038177\\
197.992310694735	-9.75406778666946\\
198.906659905733	-9.79408688459091\\
199.825231677076	-9.83410597406869\\
200.748045508988	-9.87412505502461\\
201.675120991751	-9.91414412737979\\
202.606477806113	-9.9541631910546\\
203.542135723711	-9.99418224596867\\
204.482114607491	-10.0342012920409\\
205.426434412127	-10.0742203291895\\
206.375115184445	-10.1142393573317\\
207.32817706385	-10.1542583763843\\
208.285640282755	-10.1942773862631\\
209.247525167004	-10.2342963868831\\
210.213852136311	-10.2743153781587\\
211.18464170469	-10.3143343600032\\
212.159914480891	-10.3543533323295\\
213.139691168835	-10.3943722950494\\
214.123992568061	-10.434391248074\\
215.112839574156	-10.4744101913135\\
216.10625317921	-10.5144291246773\\
217.104254472254	-10.554448048074\\
218.106864639713	-10.5944669614113\\
219.114104965849	-10.6344858645961\\
220.12599683322	-10.6745047575345\\
221.142561723131	-10.7145236401315\\
222.163821216089	-10.7545425122914\\
223.189796992262	-10.7945613739177\\
224.220510831939	-10.8345802249127\\
225.255984615994	-10.874599065178\\
226.296240326348	-10.9146178946144\\
227.341300046436	-10.9546367131216\\
228.391185961678	-10.9946555205984\\
229.44592035995	-11.0346743169426\\
230.505525632052	-11.0746931020513\\
231.570024272191	-11.1147118758204\\
232.639438878451	-11.1547306381449\\
233.713792153278	-11.1947493889189\\
234.793106903962	-11.2347681280355\\
235.877406043117	-11.2747868553866\\
236.966712589169	-11.3148055708635\\
238.061049666849	-11.3548242743562\\
239.160440507677	-11.3948429657537\\
240.264908450462	-11.434861644944\\
241.374476941791	-11.4748803118143\\
242.48916953653	-11.5148989662503\\
243.609009898327	-11.554917608137\\
244.734021800108	-11.5949362373582\\
245.864229124585	-11.6349548537966\\
246.999655864764	-11.674973457334\\
248.140326124455	-11.7149920478509\\
249.286264118777	-11.7550106252267\\
250.437494174681	-11.7950291893398\\
251.594040731461	-11.8350477400674\\
252.755928341275	-11.8750662772855\\
253.923181669665	-11.9150848008693\\
255.09582549608	-11.9551033106923\\
256.273884714404	-11.9951218066272\\
257.457384333485	-12.0351402885454\\
258.646349477661	-12.0751587563172\\
259.840805387301	-12.1151772098117\\
261.040777419332	-12.1551956488965\\
262.246291047787	-12.1952140734384\\
263.457371864337	-12.2352324833027\\
264.674045578839	-12.2752508783535\\
265.896338019882	-12.3152692584537\\
267.124275135332	-12.3552876234649\\
268.357882992886	-12.3953059732473\\
269.597187780627	-12.4353243076601\\
270.842215807572	-12.4753426265608\\
272.092993504239	-12.515360929806\\
273.349547423206	-12.5553792172507\\
274.611904239671	-12.5953974887485\\
275.880090752022	-12.6354157441519\\
277.154133882405	-12.6754339833119\\
278.434060677294	-12.7154522060781\\
279.719898308069	-12.7554704122987\\
281.011674071587	-12.7954886018205\\
282.309415390768	-12.835506774489\\
283.613149815172	-12.8755249301481\\
284.922905021585	-12.9155430686404\\
286.238708814609	-12.9555611898069\\
287.560589127251	-12.9955792934873\\
288.888574021513	-13.0355973795197\\
290.222691688993	-13.0756154477406\\
291.562970451479	-13.1156334979852\\
292.909438761552	-13.1556515300871\\
294.262125203191	-13.1956695438783\\
295.621058492378	-13.2356875391892\\
296.98626747771	-13.2757055158489\\
298.357781141006	-13.3157234736846\\
299.735628597932	-13.3557414125221\\
301.119839098607	-13.3957593321855\\
302.510442028234	-13.4357772324973\\
303.907466907719	-13.4757951132783\\
305.310943394298	-13.5158129743477\\
306.720901282168	-13.5558308155231\\
308.137370503119	-13.5958486366202\\
309.560381127168	-13.6358664374532\\
310.989963363199	-13.6758842178345\\
312.426147559604	-13.7159019775747\\
313.868964204927	-13.7559197164828\\
315.318443928511	-13.7959374343658\\
316.774617501149	-13.8359551310293\\
318.237515835737	-13.8759728062766\\
319.707169987928	-13.9159904599096\\
321.183611156796	-13.9560080917281\\
322.666870685493	-13.9960257015302\\
324.156980061919	-14.0360432891121\\
325.653970919389	-14.076060854268\\
327.1578750373	-14.1160783967904\\
328.668724341813	-14.1560959164697\\
330.186550906528	-14.1961134130944\\
331.711386953162	-14.2361308864511\\
333.243264852235	-14.2761483363244\\
334.78221712376	-14.3161657624969\\
336.328276437929	-14.3561831647491\\
337.881475615808	-14.3962005428597\\
339.441847630035	-14.4362178966051\\
341.009425605519	-14.4762352257597\\
342.584242820144	-14.5162525300959\\
344.166332705473	-14.556269809384\\
345.75572884746	-14.596287063392\\
347.352464987164	-14.6363042918859\\
348.956575021463	-14.6763214946295\\
350.568093003772	-14.7163386713844\\
352.187053144771	-14.75635582191\\
353.813489813129	-14.7963729459635\\
355.447437536229	-14.8363900432998\\
357.088931000911	-14.8764071136715\\
358.738005054198	-14.9164241568292\\
360.39469470404	-14.9564411725207\\
362.059035120061	-14.9964581604918\\
363.731061634299	-15.0364751204859\\
365.410809741959	-15.076492052244\\
367.09831510217	-15.1165089555046\\
368.793613538734	-15.1565258300041\\
370.496741040893	-15.1965426754759\\
372.207733764093	-15.2365594916516\\
373.926628030746	-15.2765762782598\\
375.653460331008	-15.3165930350268\\
377.388267323548	-15.3566097616763\\
379.13108583633	-15.3966264579295\\
380.881952867393	-15.4366431235051\\
382.640905585636	-15.4766597581189\\
384.407981331609	-15.5166763614844\\
386.183217618305	-15.5566929333123\\
387.966652131954	-15.5967094733106\\
389.758322732826	-15.6367259811846\\
391.558267456034	-15.6767424566369\\
393.366524512341	-15.7167588993674\\
395.183132288971	-15.7567753090731\\
397.008129350424	-15.7967916854483\\
398.841554439296	-15.8368080281845\\
400.683446477099	-15.8768243369701\\
402.533844565089	-15.9168406114909\\
404.392787985098	-15.9568568514297\\
406.260316200361	-15.9968730564664\\
408.136468856361	-16.0368892262777\\
410.021285781671	-16.0769053605375\\
411.914806988789	-16.1169214589169\\
413.817072675003	-16.1569375210835\\
415.72812322323	-16.1969535467021\\
417.647999202884	-16.2369695354343\\
419.576741370729	-16.2769854869388\\
421.514390671752	-16.3170014008707\\
423.460988240025	-16.3570172768824\\
425.416575399583	-16.3970331146226\\
427.381193665299	-16.4370489137371\\
429.354884743766	-16.4770646738683\\
431.337690534184	-16.5170803946554\\
433.329653129246	-16.557096075734\\
435.330814816033	-16.5971117167365\\
437.341218076915	-16.6371273172919\\
439.360905590448	-16.6771428770257\\
441.389920232282	-16.7171583955601\\
443.428305076071	-16.7571738725137\\
445.476103394389	-16.7971893075013\\
447.533358659646	-16.8372047001346\\
449.600114545014	-16.8772200500214\\
451.67641492535	-16.917235356766\\
453.762303878129	-16.9572506199689\\
455.857825684385	-16.9972658392271\\
457.963024829641	-17.0372810141338\\
460.077946004862	-17.0772961442782\\
462.202634107401	-17.117311229246\\
464.33713424195	-17.157326268619\\
466.481491721497	-17.197341261975\\
468.635752068297	-17.2373562088881\\
470.799961014824	-17.2773711089282\\
472.974164504755	-17.3173859616614\\
475.158408693936	-17.3574007666497\\
477.352739951367	-17.3974155234511\\
479.557204860185	-17.4374302316195\\
481.771850218653	-17.4774448907047\\
483.996723041153	-17.5174595002521\\
486.231870559183	-17.5574740598033\\
488.477340222363	-17.5974885688954\\
490.73317969944	-17.6375030270611\\
492.999436879299	-17.677517433829\\
495.276159871982	-17.7175317887233\\
497.563397009708	-17.7575460912637\\
499.861196847899	-17.7975603409655\\
502.169608166212	-17.8375745373396\\
504.48867996957	-17.8775886798921\\
506.818461489212	-17.917602768125\\
509.159002183727	-17.9576168015351\\
511.51035174011	-17.9976307796151\\
513.872560074817	-18.0376447018525\\
516.245677334823	-18.0776585677306\\
518.629753898685	-18.1176723767274\\
521.024840377617	-18.1576861283163\\
523.430987616559	-18.1976998219659\\
525.848246695256	-18.2377134571397\\
528.27666892935	-18.2777270332964\\
530.716305871459	-18.3177405498896\\
533.167209312278	-18.3577540063678\\
535.629431281677	-18.3977674021746\\
538.103024049808	-18.437780736748\\
540.588040128206	-18.4777940095214\\
543.084532270915	-18.5178072199225\\
545.592553475602	-18.5578203673738\\
548.11215698468	-18.5978334512926\\
550.643396286443	-18.6378464710907\\
553.186325116201	-18.6778594261744\\
555.740997457415	-18.7178723159445\\
558.307467542852	-18.7578851397965\\
560.885789855729	-18.79789789712\\
563.476019130871	-18.8379105872991\\
566.078210355879	-18.8779232097121\\
568.692418772287	-18.9179357637317\\
571.318699876742	-18.9579482487246\\
573.957109422183	-18.9979606640518\\
576.607703419019	-19.0379730090683\\
579.27053813632	-19.0779852831232\\
581.945670103016	-19.1179974855594\\
584.633156109091	-19.1580096157139\\
587.333053206791	-19.1980216729175\\
590.045418711838	-19.2380336564949\\
592.77031020464	-19.2780455657642\\
595.507785531519	-19.3180574000377\\
598.25790280594	-19.3580691586209\\
601.020720409738	-19.3980808408131\\
603.796296994364	-19.438092445907\\
606.584691482126	-19.4781039731889\\
609.385963067442	-19.5181154219382\\
612.200171218097	-19.5581267914281\\
615.027375676503	-19.5981380809245\\
617.867636460969	-19.638149289687\\
620.721013866976	-19.678160416968\\
623.587568468455	-19.7181714620132\\
626.467361119072	-19.7581824240613\\
629.360452953524	-19.7981933023437\\
632.266905388836	-19.838204096085\\
635.186780125658	-19.8782148045024\\
638.120139149584	-19.9182254268059\\
641.067044732464	-19.9582359621984\\
644.027559433723	-19.9982464098751\\
647.001746101695	-20.038256769024\\
649.989667874954	-20.0782670388253\\
652.991388183652	-20.118277218452\\
656.00697075087	-20.158287307069\\
659.03647959397	-20.1982973038338\\
662.07997902595	-20.2383072078961\\
665.137533656814	-20.2783170183974\\
668.209208394941	-20.3183267344716\\
671.295068448464	-20.3583363552444\\
674.395179326655	-20.3983458798334\\
677.509606841313	-20.4383553073482\\
680.638417108163	-20.47836463689\\
683.78167654826	-20.5183738675517\\
686.9394518894	-20.5583829984179\\
690.11181016753	-20.5983920285644\\
693.298818728181	-20.6384009570589\\
696.500545227891	-20.6784097829602\\
699.717057635642	-20.7184185053184\\
702.948424234306	-20.7584271231748\\
706.19471362209	-20.798435635562\\
709.455994713995	-20.8384440415033\\
712.732336743282	-20.8784523400134\\
716.023809262934	-20.9184605300974\\
719.330482147139	-20.9584686107516\\
722.652425592773	-20.9984765809628\\
725.989710120886	-21.0384844397083\\
729.3424065782	-21.0784921859562\\
732.71058613862	-21.1184998186649\\
736.094320304735	-21.1585073367832\\
739.493680909343	-21.1985147392502\\
742.908740116972	-21.2385220249949\\
746.339570425412	-21.2785291929368\\
749.786244667258	-21.3185362419852\\
753.248836011454	-21.3585431710391\\
756.727417964843	-21.3985499789877\\
760.222064373731	-21.4385566647097\\
763.732849425456	-21.4785632270733\\
767.259847649959	-21.5185696649364\\
770.803133921369	-21.5585759771464\\
774.362783459591	-21.5985821625398\\
777.938871831903	-21.6385882199423\\
781.531474954562	-21.6785941481691\\
785.140669094413	-21.7185999460241\\
788.76653087051	-21.7586056123001\\
792.40913725574	-21.7986111457789\\
796.068565578463	-21.838616545231\\
799.744893524145	-21.8786218094154\\
803.438199137013	-21.9186269370796\\
807.148560821713	-21.9586319269597\\
810.876057344967	-21.99863677778\\
814.620767837254	-22.0386414882528\\
818.382771794484	-22.0786460570787\\
822.162149079689	-22.1186504829461\\
825.958979924714	-22.1586547645316\\
829.773344931927	-22.198658900499\\
833.605325075921	-22.2386628895002\\
837.455001705243	-22.2786667301744\\
841.322456544116	-22.3186704211483\\
845.207771694168	-22.3586739610356\\
849.111029636188	-22.3986773484376\\
853.032313231867	-22.4386805819424\\
856.971705725559	-22.4786836601249\\
860.92929074605	-22.5186865815471\\
864.905152308334	-22.5586893447575\\
868.899374815392	-22.5986919482912\\
872.91204305999	-22.6386943906697\\
876.943242226474	-22.6786966704008\\
880.993057892579	-22.7186987859786\\
885.061576031251	-22.7587007358832\\
889.148883012464	-22.7987025185804\\
893.255065605058	-22.8387041325222\\
897.380210978585	-22.878705576146\\
901.52440670515	-22.9187068478748\\
905.687740761275	-22.9587079461168\\
909.870301529772	-22.9987088692659\\
914.07217780161	-23.0387096157006\\
918.293458777803	-23.0787101837849\\
922.534234071309	-23.1187105718672\\
926.794593708924	-23.1587107782808\\
931.074628133199	-23.1987108013437\\
935.374428204358	-23.2387106393581\\
939.694085202226	-23.2787102906106\\
944.033690828168	-23.3187097533718\\
948.39333720704	-23.3587090258964\\
952.773116889131	-23.398708106423\\
957.173122852146	-23.4387069931737\\
961.593448503166	-23.4787056843544\\
966.034187680636	-23.5187041781539\\
970.495434656357	-23.5587024727449\\
974.977284137491	-23.5987005662825\\
979.479831268559	-23.6386984569053\\
984.003171633478	-23.6786961427342\\
988.547401257578	-23.718693621873\\
993.112616609642	-23.7586908924077\\
997.698914603958	-23.7986879524069\\
1002.30639260238	-23.8386847999209\\
1006.93514841637	-23.8786814329823\\
1011.58528030912	-23.9186778496053\\
1016.2568869976	-23.9586740477857\\
1020.95006765465	-23.9986700255008\\
1025.66492191112	-24.0386657807091\\
1030.40154985798	-24.0786613113503\\
1035.16005204838	-24.1186566153449\\
1039.94052949988	-24.1586516905942\\
1044.74308369654	-24.1986465349801\\
1049.56781659108	-24.2386411463647\\
1054.41483060703	-24.2786355225906\\
1059.28422864096	-24.3186296614802\\
1064.17611406461	-24.3586235608358\\
1069.09059072708	-24.3986172184392\\
1074.02776295708	-24.4386106320519\\
1078.98773556513	-24.4786037994144\\
1083.97061384575	-24.5185967182466\\
1088.97650357974	-24.558589386247\\
1094.00551103639	-24.598581801093\\
1099.05774297578	-24.6385739604401\\
1104.13330665098	-24.6785658619226\\
1109.2323098104	-24.7185575031525\\
1114.35486070002	-24.7585488817198\\
1119.50106806574	-24.7985399951923\\
1124.67104115564	-24.8385308411152\\
1129.86488972231	-24.8785214170107\\
1135.0827240252	-24.9185117203785\\
1140.32465483296	-24.9585017486947\\
1145.59079342577	-24.9984914994125\\
1150.8812515977	-25.038480969961\\
1156.19614165913	-25.0784701577458\\
1161.53557643908	-25.1184590601485\\
1166.89966928762	-25.1584476745262\\
1172.28853407829	-25.1984359982117\\
1177.70228521052	-25.2384240285131\\
1183.14103761204	-25.2784117627135\\
1188.60490674132	-25.3183991980708\\
1194.09400859005	-25.3583863318175\\
1199.60845968555	-25.3983731611607\\
1205.14837709331	-25.4383596832812\\
1210.71387841942	-25.4783458953341\\
1216.30508181309	-25.5183317944478\\
1221.92210596917	-25.5583173777244\\
1227.56507013062	-25.598302642239\\
1233.23409409112	-25.6382875850396\\
1238.92929819754	-25.6782722031468\\
1244.65080335254	-25.7182564935537\\
1250.3987310171	-25.7582404532256\\
1256.17320321316	-25.7982240790995\\
1261.97434252612	-25.8382073680843\\
1267.80227210752	-25.8781903170598\\
1273.65711567764	-25.9181729228774\\
1279.53899752808	-25.958155182359\\
1285.44804252445	-25.9981370922973\\
1291.38437610901	-26.038118649455\\
1297.34812430331	-26.078099850565\\
1303.33941371088	-26.1180806923299\\
1309.35837151994	-26.1580611714218\\
1315.40512550606	-26.1980412844819\\
1321.47980403488	-26.2380210281202\\
1327.58253606487	-26.2780003989154\\
1333.71345115004	-26.3179793934145\\
1339.87267944269	-26.3579580081326\\
1346.06035169616	-26.3979362395522\\
1352.27659926764	-26.4379140841236\\
1358.52155412096	-26.4778915382639\\
1364.79534882933	-26.5178685983572\\
1371.09811657822	-26.557845260754\\
1377.42999116818	-26.597821521771\\
1383.79110701763	-26.6377973776907\\
1390.18159916578	-26.6777728247613\\
1396.60160327544	-26.7177478591963\\
1403.05125563594	-26.7577224771738\\
1409.53069316601	-26.7976966748367\\
1416.04005341668	-26.8376704482922\\
1422.5794745742	-26.8776437936113\\
1429.14909546299	-26.9176167068287\\
1435.74905554856	-26.9575891839423\\
1442.3794949405	-26.9975612209129\\
1449.04055439544	-27.0375328136639\\
1455.73237532004	-27.077503958081\\
1462.45509977397	-27.1174746500116\\
1469.20887047298	-27.1574448852647\\
1475.99383079187	-27.1974146596106\\
1482.81012476756	-27.2373839687801\\
1489.65789710218	-27.2773528084648\\
1496.53729316606	-27.3173211743161\\
1503.44845900091	-27.3572890619453\\
1510.39154132284	-27.3972564669227\\
1517.36668752555	-27.437223384778\\
1524.37404568338	-27.4771898109992\\
1531.41376455451	-27.5171557410324\\
1538.4859935841	-27.5571211702817\\
1545.59088290748	-27.5970860941086\\
1552.72858335329	-27.6370505078314\\
1559.89924644673	-27.6770144067253\\
1567.10302441275	-27.7169777860215\\
1574.34007017931	-27.7569406409071\\
1581.61053738059	-27.7969029665245\\
1588.91458036027	-27.8368647579713\\
1596.25235417481	-27.8768260102993\\
1603.62401459674	-27.9167867185147\\
1611.02971811795	-27.9567468775775\\
1618.46962195303	-27.9967064824006\\
1625.94388404263	-28.0366655278502\\
1633.45266305675	-28.0766240087446\\
1640.99611839816	-28.1165819198542\\
1648.57441020578	-28.156539255901\\
1656.18769935804	-28.1964960115578\\
1663.83614747635	-28.2364521814484\\
1671.51991692849	-28.2764077601464\\
1679.23917083208	-28.3163627421754\\
1686.99407305804	-28.356317122008\\
1694.78478823403	-28.3962708940658\\
1702.61148174801	-28.4362240527184\\
1710.47431975172	-28.4761765922835\\
1718.37346916419	-28.5161285070259\\
1726.30909767531	-28.5560797911574\\
1734.28137374936	-28.596030438836\\
1742.29046662864	-28.6359804441657\\
1750.33654633699	-28.6759298011956\\
1758.41978368348	-28.7158785039199\\
1766.54035026596	-28.755826546277\\
1774.69841847474	-28.7957739221491\\
1782.89416149627	-28.8357206253616\\
1791.12775331676	-28.8756666496828\\
1799.39936872594	-28.9156119888233\\
1807.70918332073	-28.955556636435\\
1816.05737350894	-28.9955005861114\\
1824.44411651309	-29.0354438313863\\
1832.86959037413	-29.0753863657338\\
1841.33397395519	-29.1153281825671\\
1849.83744694544	-29.1552692752387\\
1858.38018986386	-29.1952096370394\\
1866.9623840631	-29.2351492611976\\
1875.58421173328	-29.2750881408791\\
1884.24585590593	-29.3150262691863\\
1892.94750045783	-29.3549636391575\\
1901.68933011491	-29.3949002437666\\
1910.47153045619	-29.4348360759222\\
1919.29428791772	-29.4747711284674\\
1928.15778979652	-29.5147053941787\\
1937.06222425457	-29.5546388657657\\
1946.00778032282	-29.5945715358704\\
1954.99464790516	-29.6345033970664\\
1964.02301778248	-29.6744344418587\\
1973.09308161673	-29.7143646626826\\
1982.20503195496	-29.7542940519033\\
1991.35906223344	-29.7942226018152\\
2000.55536678172	-29.8341503046412\\
2009.79414082682	-29.8740771525322\\
2019.07558049731	-29.9140031375661\\
2028.39988282751	-29.9539282517475\\
2037.76724576168	-29.9938524870067\\
2047.17786815819	-30.0337758351992\\
2056.63194979376	-30.0736982881051\\
2066.12969136771	-30.113619837428\\
2075.6712945062	-30.1535404747947\\
2085.25696176653	-30.1934601917543\\
2094.88689664142	-30.2333789797775\\
2104.56130356336	-30.273296830256\\
2114.2803879089	-30.3132137345014\\
2124.04435600306	-30.3531296837449\\
2133.8534151237	-30.3930446691363\\
2143.70777350589	-30.4329586817432\\
2153.60764034637	-30.4728717125506\\
2163.55322580795	-30.5127837524595\\
2173.54474102401	-30.5526947922867\\
2183.58239810298	-30.5926048227639\\
2193.66641013278	-30.6325138345366\\
2203.79699118545	-30.6724218181635\\
2213.97435632161	-30.7123287641159\\
2224.19872159503	-30.7522346627766\\
2234.47030405729	-30.792139504439\\
2244.78932176229	-30.8320432793068\\
2255.15599377096	-30.8719459774923\\
2265.57054015585	-30.9118475890165\\
2276.03318200585	-30.9517481038076\\
2286.54414143084	-30.9916475117004\\
2297.10364156645	-31.0315458024352\\
2307.71190657875	-31.0714429656574\\
2318.36916166905	-31.1113389909159\\
2329.07563307865	-31.151233867663\\
2339.83154809368	-31.1911275852529\\
2350.63713504986	-31.2310201329411\\
2361.49262333744	-31.2709114998832\\
2372.39824340596	-31.3108016751343\\
2383.35422676926	-31.350690647648\\
2394.36080601029	-31.3905784062751\\
2405.4182147861	-31.4304649397632\\
2416.52668783283	-31.4703502367552\\
2427.6864609706	-31.5102342857887\\
2438.8977711086	-31.550117075295\\
2450.16085625011	-31.5899985935978\\
2461.4759554975	-31.6298788289125\\
2472.84330905736	-31.6697577693452\\
2484.26315824557	-31.7096354028914\\
2495.73574549243	-31.7495117174353\\
2507.26131434783	-31.7893867007486\\
2518.84010948637	-31.8292603404893\\
2530.4723767126	-31.8691326242009\\
2542.15836296621	-31.9090035393114\\
2553.8983163273	-31.9488730731319\\
2565.69248602162	-31.9887412128557\\
2577.54112242586	-32.0286079455572\\
2589.444477073	-32.0684732581908\\
2601.4028026576	-32.1083371375897\\
2613.41635304121	-32.1481995704649\\
2625.48538325773	-32.1880605434038\\
2637.61014951883	-32.2279200428696\\
2649.7909092194	-32.2677780551994\\
2662.02792094301	-32.3076345666036\\
2674.32144446738	-32.3474895631645\\
2686.67174076992	-32.3873430308351\\
2699.07907203326	-32.427194955438\\
2711.54370165082	-32.4670453226641\\
2724.0658942324	-32.5068941180712\\
2736.64591560979	-32.5467413270832\\
2749.28403284242	-32.5865869349884\\
2761.98051422303	-32.6264309269386\\
2774.73562928336	-32.6662732879475\\
2787.54964879989	-32.7061140028896\\
2800.42284479955	-32.745953056499\\
2813.35549056553	-32.7857904333679\\
2826.34786064309	-32.8256261179452\\
2839.40023084533	-32.8654600945354\\
2852.51287825912	-32.9052923472973\\
2865.68608125093	-32.9451228602424\\
2878.92011947275	-32.9849516172334\\
2892.21527386804	-33.0247786019834\\
2905.57182667769	-33.0646037980539\\
2918.99006144599	-33.1044271888537\\
2932.4702630267	-33.1442487576375\\
2946.01271758903	-33.1840684875042\\
2959.61771262377	-33.2238863613957\\
2973.28553694936	-33.2637023620952\\
2987.01648071805	-33.3035164722261\\
3000.81083542203	-33.3433286742502\\
3014.66889389963	-33.3831389504662\\
3028.59095034155	-33.4229472830081\\
3042.57730029708	-33.4627536538442\\
3056.6282406804	-33.5025580447748\\
3070.74406977686	-33.5423604374312\\
3084.92508724934	-33.5821608132737\\
3099.17159414457	-33.6219591535904\\
3113.48389289956	-33.6617554394954\\
3127.86228734801	-33.701549651927\\
3142.30708272674	-33.7413417716464\\
3156.8185856822	-33.7811317792359\\
3171.39710427697	-33.8209196550972\\
3186.04294799627	-33.8607053794495\\
3200.75642775457	-33.9004889323283\\
3215.53785590219	-33.9402702935834\\
3230.38754623189	-33.9800494428769\\
3245.30581398558	-34.0198263596819\\
3260.29297586097	-34.0596010232807\\
3275.34935001834	-34.0993734127625\\
3290.47525608724	-34.1391435070223\\
3305.67101517332	-34.1789112847584\\
3320.93694986511	-34.2186767244713\\
3336.27338424092	-34.2584398044611\\
3351.68064387565	-34.2982005028263\\
3367.15905584778	-34.3379587974612\\
3382.70894874623	-34.3777146660547\\
3398.3306526774	-34.417468086088\\
3414.02449927217	-34.4572190348326\\
3429.7908216929	-34.4969674893487\\
3445.62995464054	-34.5367134264826\\
3461.54223436172	-34.5764568228654\\
3477.5279986559	-34.6161976549106\\
3493.58758688252	-34.655935898812\\
3509.72133996824	-34.6956715305419\\
3525.92960041413	-34.7354045258488\\
3542.21271230298	-34.7751348602555\\
3558.57102130658	-34.8148625090568\\
3575.00487469309	-34.8545874473175\\
3591.51462133436	-34.8943096498701\\
3608.1006117134	-34.934029091313\\
3624.76319793175	-34.9737457460077\\
3641.50273371703	-35.013459588077\\
3658.31957443038	-35.0531705914028\\
3675.21407707406	-35.0928787296236\\
3692.18660029898	-35.1325839761323\\
3709.23750441235	-35.1722863040741\\
3726.36715138533	-35.2119856863438\\
3743.57590486067	-35.2516820955838\\
3760.86413016048	-35.2913755041815\\
3778.23219429398	-35.331065884267\\
3795.68046596523	-35.370753207711\\
3813.20931558105	-35.4104374461215\\
3830.81911525882	-35.4501185708424\\
3848.51023883439	-35.4897965529501\\
3866.28306187004	-35.5294713632518\\
3884.13796166242	-35.5691429722824\\
3902.07531725059	-35.6088113503019\\
3920.09550942403	-35.6484764672934\\
3938.19892073078	-35.6881382929599\\
3956.38593548549	-35.727796796722\\
3974.65693977764	-35.7674519477153\\
3993.0123214797	-35.8071037147873\\
4011.45247025537	-35.8467520664952\\
4029.97777756789	-35.8863969711027\\
4048.58863668828	-35.9260383965779\\
4067.28544270374	-35.9656763105896\\
4086.06859252603	-36.0053106805055\\
4104.93848489988	-36.0449414733885\\
4123.89552041149	-36.0845686559944\\
4142.94010149697	-36.1241921947688\\
4162.07263245094	-36.163812055844\\
4181.29351943512	-36.2034282050366\\
4200.60317048688	-36.2430406078438\\
4220.00199552799	-36.2826492294412\\
4239.49040637325	-36.322254034679\\
4259.06881673929	-36.3618549880794\\
4278.73764225331	-36.4014520538334\\
4298.49730046193	-36.4410451957978\\
4318.34821084004	-36.4806343774919\\
4338.2907947997	-36.5202195620941\\
4358.32547569911	-36.5598007124394\\
4378.45267885158	-36.5993777910152\\
4398.67283153454	-36.638950759959\\
4418.98636299868	-36.6785195810545\\
4439.39370447695	-36.7180842157282\\
4459.89528919383	-36.7576446250466\\
4480.49155237446	-36.7972007697121\\
4501.18293125388	-36.8367526100601\\
4521.96986508636	-36.8763001060555\\
4542.85279515466	-36.9158432172887\\
4563.83216477945	-36.9553819029726\\
4584.90841932869	-36.9949161219391\\
4606.0820062271	-37.0344458326349\\
4627.35337496566	-37.0739709931184\\
4648.72297711113	-37.1134915610559\\
4670.19126631568	-37.1530074937179\\
4691.75869832646	-37.1925187479751\\
4713.42573099534	-37.2320252802951\\
4735.19282428857	-37.2715270467383\\
4757.06044029659	-37.311024002954\\
4779.02904324381	-37.3505161041767\\
4801.09909949849	-37.390003305222\\
4823.27107758263	-37.4294855604829\\
4845.54544818189	-37.4689628239254\\
4867.92268415562	-37.508435049085\\
4890.4032605469	-37.547902189062\\
4912.98765459258	-37.5873641965179\\
4935.67634573345	-37.626821023671\\
4958.46981562442	-37.6662726222923\\
4981.36854814472	-37.7057189437011\\
5004.37302940819	-37.7451599387611\\
5027.48374777359	-37.7845955578756\\
5050.70119385497	-37.8240257509834\\
5074.02586053209	-37.8634504675544\\
5097.4582429609	-37.9028696565852\\
5120.998838584	-37.9422832665945\\
5144.64814714125	-37.9816912456186\\
5168.40667068035	-38.0210935412067\\
5192.27491356753	-38.0604901004165\\
5216.2533824982	-38.0998808698097\\
5240.34258650778	-38.1392657954466\\
5264.54303698246	-38.1786448228822\\
5288.85524767004	-38.2180178971606\\
5313.27973469088	-38.2573849628108\\
5337.81701654885	-38.2967459638415\\
5362.46761414231	-38.3361008437361\\
5387.23205077517	-38.375449545448\\
5412.11085216805	-38.414792011395\\
5437.10454646936	-38.454128183455\\
5462.21366426659	-38.4934580029602\\
5487.43873859752	-38.5327814106922\\
5512.78030496154	-38.5720983468766\\
5538.23890133107	-38.6114087511782\\
5563.81506816292	-38.6507125626947\\
5589.50934840979	-38.6900097199524\\
5615.32228753178	-38.7293001609\\
5641.254433508	-38.7685838229034\\
5667.30633684818	-38.8078606427399\\
5693.47855060436	-38.8471305565931\\
5719.77163038263	-38.8863935000467\\
5746.18613435493	-38.9256494080789\\
5772.72262327089	-38.9648982150569\\
5799.38166046975	-39.0041398547309\\
5826.16381189231	-39.0433742602281\\
5853.06964609293	-39.0826013640468\\
5880.09973425162	-39.1218210980505\\
5907.25465018618	-39.1610333934618\\
5934.53497036433	-39.2002381808562\\
5961.94127391598	-39.239435390156\\
5989.47414264556	-39.2786249506241\\
6017.13416104428	-39.3178067908574\\
6044.92191630263	-39.3569808387809\\
6072.83799832281	-39.3961470216407\\
6100.88299973121	-39.4353052659981\\
6129.05751589107	-39.4744554977226\\
6157.36214491506	-39.5135976419855\\
6185.79748767801	-39.5527316232531\\
6214.36414782964	-39.59185736528\\
6243.06273180741	-39.6309747911022\\
6271.89384884933	-39.6700838230304\\
6300.85811100697	-39.7091843826427\\
6329.95613315842	-39.7482763907779\\
6359.18853302131	-39.7873597675283\\
6388.55593116599	-39.8264344322325\\
6418.05895102864	-39.8655003034681\\
6447.69821892457	-39.9045572990446\\
6477.47436406142	-39.9436053359956\\
6507.38801855264	-39.9826443305718\\
6537.43981743082	-40.0216741982333\\
6567.63039866118	-40.060694853642\\
6597.96040315515	-40.0997062106536\\
6628.43047478397	-40.1387081823105\\
6659.0412603923	-40.1777006808333\\
6689.79340981204	-40.2166836176134\\
6720.68757587607	-40.2556569032044\\
6751.7244144321	-40.2946204473149\\
6782.90458435663	-40.3335741587994\\
6814.22874756894	-40.3725179456507\\
6845.69756904508	-40.4114517149913\\
6877.31171683206	-40.4503753730652\\
6909.07186206199	-40.4892888252291\\
6940.97867896634	-40.5281919759442\\
6973.03284489025	-40.5670847287671\\
7005.23504030693	-40.6059669863414\\
7037.58594883204	-40.6448386503889\\
7070.0862572383	-40.6836996217003\\
7102.73665547	-40.7225498001266\\
7135.53783665763	-40.7613890845696\\
7168.49049713269	-40.800217372973\\
7201.59533644237	-40.8390345623132\\
7234.85305736446	-40.8778405485894\\
7268.26436592224	-40.9166352268147\\
7301.82997139948	-40.9554184910061\\
7335.55058635552	-40.994190234175\\
7369.42692664033	-41.0329503483175\\
7403.45971140979	-41.0716987244044\\
7437.64966314091	-41.1104352523713\\
7471.99750764715	-41.1491598211084\\
7506.50397409388	-41.1878723184505\\
7541.16979501382	-41.2265726311666\\
7575.99570632259	-41.2652606449495\\
7610.98244733438	-41.3039362444055\\
7646.13076077758	-41.3425993130436\\
7681.44139281058	-41.3812497332645\\
7716.91509303763	-41.4198873863508\\
7752.55261452471	-41.4585121524548\\
7788.35471381553	-41.4971239105882\\
7824.32215094763	-41.5357225386111\\
7860.45568946846	-41.57430791322\\
7896.7560964516	-41.6128799099372\\
7933.22414251308	-41.6514384030986\\
7969.86060182773	-41.6899832658428\\
8006.66625214554	-41.7285143700986\\
8043.6418748083	-41.7670315865738\\
8080.78825476606	-41.8055347847428\\
8118.10618059391	-41.8440238328346\\
8155.5964445086	-41.8824985978206\\
8193.25984238547	-41.9209589454024\\
8231.09717377527	-41.9594047399988\\
8269.10924192116	-41.9978358447338\\
8307.29685377578	-42.0362521214234\\
8345.66082001834	-42.0746534305629\\
8384.20195507185	-42.1130396313139\\
8422.92107712042	-42.1514105814911\\
8461.81900812665	-42.189766137549\\
8500.89657384898	-42.2281061545684\\
8540.15460385935	-42.2664304862431\\
8579.59393156072	-42.3047389848658\\
8619.21539420481	-42.3430315013146\\
8659.01983290983	-42.3813078850388\\
8699.0080926784	-42.4195679840445\\
8739.18102241541	-42.4578116448811\\
8779.5394749461	-42.496038712626\\
8820.08430703417	-42.5342490308704\\
8860.81637939989	-42.5724424417045\\
8901.73655673847	-42.6106187857025\\
8942.84570773836	-42.6487779019078\\
8984.14470509972	-42.6869196278173\\
9025.63442555289	-42.7250437993665\\
9067.31574987707	-42.7631502509137\\
9109.18956291901	-42.8012388152242\\
9151.25675361173	-42.8393093234545\\
9193.51821499347	-42.8773616051364\\
9235.97484422661	-42.9153954881606\\
9278.62754261669	-42.9534107987603\\
9321.47721563161	-42.9914073614946\\
9364.5247729208	-43.029384999232\\
9407.77112833455	-43.0673435331331\\
9451.21719994339	-43.1052827826337\\
9494.86391005764	-43.1432025654275\\
9538.71218524688	-43.1811026974485\\
9582.76295635973	-43.2189829928535\\
9627.01715854358	-43.2568432640041\\
9671.47573126437	-43.2946833214486\\
9716.13961832666	-43.3325029739039\\
9761.00976789355	-43.3703020282371\\
9806.08713250686	-43.4080802894465\\
9851.37266910737	-43.4458375606431\\
9896.86733905512	-43.4835736430313\\
9942.57210814977	-43.5212883358901\\
9988.48794665118	-43.5589814365528\\
10034.6158293	-43.5966527403882\\
10080.9567353382	-43.6343020407799\\
10127.5116485301	-43.671929129107\\
10174.2815571832	-43.7095337947229\\
10221.267454169	-43.7471158249355\\
10268.4703369442	-43.7846750049859\\
10315.891207572	-43.8222111180275\\
10363.531072743	-43.859723945105\\
10411.3909437969	-43.8972132651324\\
10459.4718367439	-43.9346788548719\\
10507.7747722864	-43.9721204889115\\
10556.3007758401	-44.0095379396428\\
10605.0508775566	-44.0469309772388\\
10654.0261123446	-44.0842993696309\\
10703.2275198921	-44.1216428824864\\
10752.6561446888	-44.1589612791847\\
10802.3130360475	-44.1962543207941\\
10852.1992481272	-44.2335217660482\\
10902.315839955	-44.2707633713217\\
10952.6638754485	-44.3079788906062\\
11003.244423439	-44.3451680754857\\
11054.0585576935	-44.3823306751116\\
11105.1073569377	-44.4194664361778\\
11156.3919048792	-44.4565751028953\\
11207.9132902301	-44.4936564169663\\
11259.6726067303	-44.5307101175585\\
11311.6709531708	-44.5677359412787\\
11363.9094334169	-44.6047336221465\\
11416.3891564316	-44.6417028915668\\
11469.1112362992	-44.6786434783033\\
11522.0767922492	-44.7155551084508\\
11575.2869486794	-44.7524375054073\\
11628.7428351806	-44.789290389846\\
11682.4455865599	-44.8261134796868\\
11736.3963428651	-44.8629064900675\\
11790.596249409	-44.8996691333149\\
11845.0464567934	-44.9364011189152\\
11899.7481209338	-44.9731021534842\\
11954.7024030838	-45.0097719407373\\
12009.9104698599	-45.046410181459\\
12065.3734932659	-45.0830165734719\\
12121.0926507183	-45.119590811606\\
12177.069125071	-45.1561325876664\\
12233.3041046402	-45.1926415904021\\
12289.7987832301	-45.2291175054734\\
12346.554360158	-45.2655600154191\\
12403.5720402798	-45.3019687996238\\
12460.8530340153	-45.3383435342841\\
12518.3985573745	-45.3746838923748\\
12576.2098319827	-45.410989543615\\
12634.2880851072	-45.4472601544332\\
12692.6345496825	-45.483495387932\\
12751.2504643373	-45.5196949038531\\
12810.1370734203	-45.5558583585411\\
12869.2956270265	-45.5919854049073\\
12928.7273810244	-45.6280756923929\\
12988.4335970818	-45.6641288669315\\
13048.4155426933	-45.7001445709121\\
13108.6744912069	-45.7361224431403\\
13169.2117218509	-45.7720621188001\\
13230.0285197614	-45.807963229415\\
13291.1261760091	-45.8438254028078\\
13352.5059876274	-45.8796482630616\\
13414.1692576393	-45.9154314304784\\
13476.1172950852	-45.9511745215387\\
13538.351415051	-45.9868771488599\\
13600.8729386957	-46.0225389211541\\
13663.6831932795	-46.0581594431857\\
13726.7835121922	-46.0937383157286\\
13790.1752349813	-46.1292751355223\\
13853.8597073802	-46.1647694952281\\
13917.8382813373	-46.2002209833842\\
13982.1123150444	-46.2356291843605\\
14046.6831729655	-46.2709936783131\\
14111.552225866	-46.3063140411377\\
14176.7208508414	-46.3415898444229\\
14242.190431347	-46.3768206554024\\
14307.9623572268	-46.4120060369071\\
14374.0380247435	-46.4471455473163\\
14440.4188366077	-46.4822387405086\\
14507.1062020079	-46.5172851658114\\
14574.1015366405	-46.5522843679511\\
14641.4062627396	-46.587235887001\\
14709.0218091073	-46.62213925833\\
14776.9496111443	-46.6569940125499\\
14845.1911108798	-46.6917996754621\\
14913.7477570027	-46.7265557680037\\
14982.6210048919	-46.7612618061931\\
15051.8123166476	-46.795917301074\\
15121.323161122	-46.8305217586605\\
15191.1550139505	-46.8650746798791\\
15261.3093575835	-46.8995755605123\\
15331.7876813171	-46.9340238911393\\
15402.5914813253	-46.968419157078\\
15473.7222606918	-47.0027608383244\\
15545.1815294413	-47.0370484094924\\
15616.9708045722	-47.0712813397524\\
15689.0916100885	-47.105459092769\\
15761.5454770323	-47.139581126638\\
15834.333943516	-47.1736468938226\\
15907.4585547554	-47.2076558410887\\
15980.9208631021	-47.241607409439\\
16054.7224280766	-47.2755010340471\\
16128.8648164017	-47.3093361441897\\
16203.3496020352	-47.3431121631785\\
16278.1783662036	-47.3768285082909\\
16353.352697436	-47.4104845907\\
16428.8741915971	-47.4440798154036\\
16504.7444519217	-47.4776135811517\\
16580.9650890484	-47.511085280374\\
16657.537721054	-47.5444942991055\\
16734.4639734876	-47.5778400169115\\
16811.7454794054	-47.6111218068116\\
16889.3838794052	-47.6443390352023\\
16967.3808216612	-47.6774910617791\\
17045.7379619589	-47.710577239457\\
17124.4569637308	-47.7435969142899\\
17203.539498091	-47.7765494253893\\
17282.9872438709	-47.8094341048415\\
17362.8018876552	-47.8422502776237\\
17442.9851238172	-47.8749972615187\\
17523.5386545551	-47.9076743670293\\
17604.464189928	-47.9402808972899\\
17685.7634478922	-47.9728161479785\\
17767.4381543379	-48.0052794072262\\
17849.4900431252	-48.037669955526\\
17931.9208561219	-48.0699870656405\\
18014.7323432394	-48.1022300025072\\
18097.9262624709	-48.1343980231438\\
18181.5043799277	-48.1664903765513\\
18265.4684698776	-48.1985063036155\\
18349.8203147818	-48.2304450370079\\
18434.5617053333	-48.2623058010844\\
18519.6944404947	-48.2940878117826\\
18605.2203275364	-48.3257902765181\\
18691.1411820748	-48.3574123940788\\
18777.4588281113	-48.3889533545172\\
18864.1750980704	-48.4204123390424\\
18951.2918328392	-48.451788519909\\
19038.810881806	-48.4830810603055\\
19126.7341029	-48.5142891142401\\
19215.0633626303	-48.5454118264256\\
19303.8005361259	-48.5764483321617\\
19392.9475071751	-48.6073977572164\\
19482.506168266	-48.6382592177046\\
19572.4784206263	-48.6690318199657\\
19662.8661742637	-48.6997146604389\\
19753.6713480067	-48.7303068255367\\
19844.8958695449	-48.7608073915159\\
19936.5416754703	-48.7912154243478\\
20028.6107113184	-48.8215299795849\\
20121.1049316092	-48.8517501022267\\
20214.026299889	-48.8818748265824\\
20307.3767887718	-48.9119031761326\\
20401.1583799817	-48.9418341633875\\
20495.373064394	-48.9716667897437\\
20590.0228420787	-49.0014000453386\\
20685.1097223421	-49.031032908902\\
20780.6357237695	-49.060564347606\\
20876.6028742684	-49.0899933169117\\
20973.0132111115	-49.119318760414\\
21069.8687809795	-49.1485396096839\\
21167.1716400053	-49.1776547841074\\
21264.923853817	-49.2066631907231\\
21363.127497582	-49.2355637240555\\
21461.7846560511	-49.2643552659469\\
21560.8974236026	-49.2930366853861\\
21660.467904287	-49.3216068383336\\
21760.4982118713	-49.3500645675448\\
21860.9904698845	-49.3784087023898\\
21961.9468116618	-49.4066380586698\\
22063.3693803907	-49.4347514384312\\
22165.260329156	-49.4627476297756\\
22267.6218209858	-49.4906254066675\\
22370.4560288971	-49.5183835287377\\
22473.7651359423	-49.5460207410841\\
22577.5513352553	-49.5735357740688\\
22681.8168300982	-49.6009273431115\\
22786.5638339077	-49.6281941484792\\
22891.7945703429	-49.655334875073\\
22997.5112733313	-49.6823481922099\\
23103.7161871177	-49.7092327534016\\
23210.4115663104	-49.7359871961294\\
23317.5996759301	-49.7626101416145\\
23425.2827914574	-49.7891001945847\\
23533.4631988815	-49.8154559430366\\
23642.1431947482	-49.8416759579937\\
23751.3250862094	-49.8677587932597\\
23861.0111910714	-49.8937029851681\\
23971.2038378446	-49.9195070523263\\
24081.9053657923	-49.9451694953554\\
24193.1181249812	-49.9706887966253\\
24304.8444763306	-49.9960634199849\\
24417.0867916629	-50.0212918104867\\
24529.8474537537	-50.0463723941068\\
24643.1288563827	-50.0713035774599\\
24756.933404384	-50.0960837475076\\
24871.2635136979	-50.1207112712627\\
24986.1216114214	-50.1451844954872\\
25101.5101358603	-50.1695017463839\\
25217.4315365806	-50.1936613292832\\
25333.8882744609	-50.2176615283227\\
25450.882821744	-50.2415006061218\\
25568.4176620901	-50.2651768034488\\
25686.4952906292	-50.2886883388819\\
25805.1182140139	-50.3120334084636\\
25924.2889504728	-50.3352101853482\\
26044.0100298642	-50.3582168194424\\
26164.2839937291	-50.381051437038\\
26285.113395346	-50.4037121404382\\
26406.5007997846	-50.4261970075758\\
26528.4487839603	-50.4485040916235\\
26650.959936689	-50.4706314205968\\
26774.0368587419	-50.4925769969479\\
26897.6821629011	-50.5143387971522\\
27021.8984740145	-50.5359147712854\\
27146.6884290521	-50.5573028425923\\
27272.0546771616	-50.5785009070463\\
27397.9998797246	-50.5995068328997\\
27524.5267104134	-50.6203184602246\\
27651.6378552475	-50.6409336004439\\
27779.3360126509	-50.661350035852\\
27907.623893509	-50.6815655191259\\
28036.5042212264	-50.7015777728254\\
28165.9797317848	-50.7213844888817\\
28296.0531738006	-50.7409833280762\\
28426.7273085842	-50.7603719195069\\
28558.0049101974	-50.7795478600434\\
28689.8887655133	-50.7985087137694\\
28822.3816742749	-50.8172520114134\\
28955.4864491548	-50.8357752497664\\
29089.2059158146	-50.8540758910858\\
29223.5429129655	-50.872151362487\\
29358.5002924277	-50.8899990553203\\
29494.0809191919	-50.9076163245334\\
29630.2876714792	-50.92500048802\\
29767.1234408028	-50.9421488259528\\
29904.5911320291	-50.9590585801005\\
30042.6936634399	-50.97572695313\\
30181.4339667932	-50.9921511078909\\
30320.8149873869	-51.008328166684\\
30460.8396841202	-51.0242552105119\\
30601.5110295567	-51.0399292783113\\
30742.8320099879	-51.0553473661681\\
30884.8056254962	-51.0705064265118\\
31027.4348900188	-51.0854033672913\\
31170.7228314113	-51.1000350511305\\
31314.6724915125	-51.1143982944626\\
31459.2869262085	-51.1284898666433\\
31604.5692054981	-51.1423064890415\\
31750.5224135574	-51.1558448341077\\
};
\addlegendentry{$\text{G}_\text{o}\text{(z)}$};

\addplot [color=black,solid,line width=1.5pt]
  table[row sep=crcr]{10	16.2533228248387\\
10.0461810465159	16.2133054555528\\
10.0925753619375	16.173288110712\\
10.139183931163	16.1332707905425\\
10.1860077436388	16.0932534952728\\
10.2330477933808	16.0532362251332\\
10.2803050789954	16.0132189803565\\
10.3277806037004	15.9732017611774\\
10.375475375347	15.9331845678328\\
10.4233904064403	15.8931674005618\\
10.4715267141616	15.8531502596059\\
10.5198853203895	15.8131331452085\\
10.5684672517218	15.7731160576156\\
10.6172735394972	15.7330989970752\\
10.6663052198171	15.6930819638377\\
10.715563333568	15.6530649581559\\
10.7650489264432	15.6130479802849\\
10.814763048965	15.573031030482\\
10.8647067565072	15.533014109007\\
10.9148811093176	15.4929972161222\\
10.9652871725401	15.4529803520922\\
11.0159260162376	15.4129635171841\\
11.0667987154147	15.3729467116674\\
11.1179063500406	15.3329299358142\\
11.1692500050717	15.2929131898991\\
11.2208307704748	15.2528964741992\\
11.2726497412507	15.2128797889941\\
11.3247080174565	15.1728631345662\\
11.3770067042298	15.1328465112002\\
11.4295469118117	15.0928299191838\\
11.4823297555707	15.052813358807\\
11.535356356026	15.0127968303628\\
11.5886278388715	14.9727803341467\\
11.6421453349997	14.932763870457\\
11.6959099805258	14.8927474395949\\
11.7499229168114	14.852731041864\\
11.8041852904894	14.8127146775712\\
11.8586982534876	14.7726983470258\\
11.9134629630538	14.7326820505404\\
11.96848058178	14.69266578843\\
12.0237522776272	14.652649561013\\
12.0792792239501	14.6126333686104\\
12.135062599522	14.5726172115463\\
12.1911035885602	14.5326010901478\\
12.2474033807505	14.492585004745\\
12.3039631712731	14.4525689556712\\
12.3607841608273	14.4125529432624\\
12.4178675556577	14.3725369678582\\
12.4752145675793	14.3325210298011\\
12.5328264140034	14.2925051294366\\
12.5907043179635	14.2524892671137\\
12.6488495081411	14.2124734431845\\
12.7072632188919	14.1724576580043\\
12.765946690272	14.1324419119318\\
12.8249011680643	14.092426205329\\
12.8841279038047	14.0524105385612\\
12.9436281548089	14.0123949119971\\
13.0034031841991	13.9723793260088\\
13.0634542609305	13.9323637809719\\
13.1237826598188	13.8923482772653\\
13.1843896615665	13.8523328152717\\
13.2452765527909	13.8123173953771\\
13.306444626051	13.7723020179711\\
13.3678951798747	13.7322866834471\\
13.4296295187868	13.6922713922019\\
13.4916489533366	13.6522561446361\\
13.5539548001256	13.6122409411539\\
13.6165483818355	13.5722257821634\\
13.6794310272563	13.5322106680765\\
13.7426040713143	13.4921955993085\\
13.806068855101	13.4521805762791\\
13.8698267259009	13.4121655994116\\
13.9338790372205	13.3721506691331\\
13.998227148817	13.332135785875\\
14.062872426727	13.2921209500723\\
14.1278162432955	13.2521061621644\\
14.1930599772055	13.2120914225945\\
14.2586050135065	13.1720767318099\\
14.3244527436445	13.1320620902622\\
14.3906045654914	13.0920474984072\\
14.4570618833745	13.0520329567046\\
14.5238261081064	13.0120184656188\\
14.5908986570152	12.9720040256181\\
14.658280953974	12.9319896371754\\
14.7259744294318	12.8919753007678\\
14.7939805204436	12.8519610168769\\
14.8623006707006	12.8119467859889\\
14.9309363305612	12.7719326085941\\
14.999888957082	12.7319184851878\\
15.069160014048	12.6919044162696\\
15.1387509720044	12.6518904023437\\
15.2086633082875	12.6118764439192\\
15.2788985070559	12.5718625415097\\
15.3494580593225	12.5318486956335\\
15.4203434629857	12.491834906814\\
15.4915562228612	12.4518211755792\\
15.5630978507143	12.411807502462\\
15.6349698652918	12.3717938880002\\
15.7071737923542	12.3317803327368\\
15.779711164708	12.2917668372196\\
15.8525835222385	12.2517534020016\\
15.9257924119422	12.2117400276408\\
15.9993393879601	12.1717267147003\\
16.07322601161	12.1317134637487\\
16.1474538514202	12.0917002753597\\
16.2220244831628	12.0516871501122\\
16.2969394898867	12.0116740885905\\
16.3722004619516	11.9716610913845\\
16.4478089970617	11.9316481590894\\
16.5237667002994	11.8916352923058\\
16.6000751841599	11.8516224916401\\
16.6767360685846	11.8116097577042\\
16.7537509809962	11.7715970911155\\
16.8311215563331	11.7315844924974\\
16.9088494370839	11.6915719624788\\
16.9869362733223	11.6515595016947\\
17.0653837227424	11.6115471107858\\
17.1441934506935	11.5715347903986\\
17.2233671302159	11.5315225411859\\
17.302906442076	11.4915103638062\\
17.3828130748022	11.4514982589245\\
17.4630887247206	11.4114862272114\\
17.5437350959914	11.3714742693443\\
17.6247539006444	11.3314623860064\\
17.7061468586161	11.2914505778876\\
17.7879156977856	11.2514388456837\\
17.8700621540116	11.2114271900975\\
17.9525879711692	11.1714156118378\\
18.035494901187	11.1314041116203\\
18.1187847040838	11.0913926901672\\
18.2024591480069	11.0513813482073\\
18.2865200092687	11.0113700864763\\
18.3709690723849	10.9713589057166\\
18.4558081301122	10.9313478066775\\
18.5410389834868	10.8913367901152\\
18.6266634418617	10.8513258567929\\
18.7126833229461	10.811315007481\\
18.7991004528435	10.7713042429568\\
18.8859166660905	10.7312935640049\\
18.9731338056957	10.6912829714172\\
19.060753723179	10.6512724659929\\
19.1487782786108	10.6112620485385\\
19.2372093406515	10.571251719868\\
19.3260487865912	10.531241480803\\
19.4152985023894	10.4912313321727\\
19.5049603827153	10.4512212748139\\
19.5950363309877	10.4112113095712\\
19.6855282594159	10.3712014372969\\
19.7764380890397	10.3311916588513\\
19.8677677497706	10.2911819751027\\
19.9595191804325	10.2511723869272\\
20.0516943288031	10.2111628952092\\
20.1442951516552	10.1711535008413\\
20.2373236147981	10.1311442047243\\
20.3307816931193	10.0911350077672\\
20.4246713706267	10.0511259108876\\
20.5189946404906	10.0111169150115\\
20.6137535050858	9.97110802107349\\
20.7089499760343	9.93109923001678\\
20.8045860742482	9.89109054279329\\
20.900663829972	9.8510819603638\\
20.9971852828265	9.81107348369792\\
21.0941524818514	9.77106511377425\\
21.1915674855492	9.73105685158045\\
21.2894323619287	9.6910486981133\\
21.387749188549	9.65104065437877\\
21.4865200525637	9.6110327213922\\
21.5857470507649	9.57102490017826\\
21.685432289628	9.53101719177111\\
21.7855778853565	9.49100959721451\\
21.8861859639264	9.4510021175618\\
21.987258661132	9.41099475387614\\
22.0887981226306	9.37098750723046\\
22.1908065039887	9.33098037870764\\
22.2932859707273	9.29097336940056\\
22.396238698368	9.25096648041222\\
22.499666872479	9.21095971285582\\
22.603572688722	9.17095306785483\\
22.7079583528983	9.13094654654312\\
22.8128260809959	9.09094015006507\\
22.9181780992365	9.05093387957558\\
23.0240166441225	9.01092773624029\\
23.130343962485	8.9709217212356\\
23.237162311531	8.93091583574877\\
23.3444739588916	8.89091008097805\\
23.4522811826701	8.85090445813276\\
23.5605862714901	8.8108989684334\\
23.6693915245447	8.77089361311179\\
23.7786992516445	8.73088839341107\\
23.8885117732672	8.69088331058594\\
23.9988314206069	8.65087836590268\\
24.1096605356231	8.61087356063923\\
24.2210014710909	8.57086889608541\\
24.3328565906507	8.53086437354293\\
24.4452282688584	8.49085999432555\\
24.558118891236	8.45085575975915\\
24.6715308543219	8.41085167118189\\
24.7854665657221	8.3708477299443\\
24.899928444161	8.33084393740936\\
25.0149189195332	8.29084029495272\\
25.1304404329547	8.25083680396266\\
25.2464954368146	8.21083346584036\\
25.3630863948276	8.17083028199992\\
25.4802157820863	8.13082725386852\\
25.597886085113	8.09082438288652\\
25.7160998019135	8.05082167050763\\
25.8348594420295	8.01081911819894\\
25.9541675265918	7.97081672744115\\
26.0740265883745	7.93081449972863\\
26.194439171848	7.89081243656954\\
26.3154078332332	7.85081053948602\\
26.4369351405564	7.81080881001424\\
26.5590236737027	7.77080724970458\\
26.6816760244719	7.73080586012178\\
26.8048947966327	7.69080464284497\\
26.9286826059784	7.65080359946795\\
27.0530420803822	7.61080273159922\\
27.1779758598533	7.57080204086211\\
27.3034865965925	7.53080152889502\\
27.4295769550488	7.49080119735143\\
27.556249611976	7.45080104790014\\
27.6835072564894	7.41080108222535\\
27.8113525901229	7.37080130202682\\
27.9397883268864	7.33080170902005\\
28.0688171933232	7.29080230493632\\
28.1984419285683	7.25080309152302\\
28.3286652844062	7.21080407054357\\
28.4594900253294	7.17080524377776\\
28.5909189285972	7.13080661302179\\
28.7229547842946	7.09080818008848\\
28.8556003953913	7.05080994680737\\
28.9888585778017	7.01081191502493\\
29.1227321604441	6.97081408660466\\
29.2572239853012	6.93081646342733\\
29.3923369074803	6.89081904739101\\
29.5280737952738	6.8508218404114\\
29.6644375302202	6.81082484442181\\
29.8014310071653	6.77082806137345\\
29.9390571343235	6.73083149323555\\
30.0773188333397	6.69083514199555\\
30.2162190393513	6.65083900965922\\
30.3557607010504	6.61084309825088\\
30.4959467807464	6.57084740981352\\
30.6367802544292	6.53085194640905\\
30.7782641118319	6.49085671011838\\
30.9204013564946	6.45086170304166\\
31.063195005828	6.41086692729844\\
31.2066480911777	6.37087238502785\\
31.350763657888	6.33087807838877\\
31.4955447653674	6.29088400956001\\
31.6409944871527	6.25089018074053\\
31.7871159109747	6.21089659414957\\
31.9339121388238	6.17090325202688\\
32.0813862870155	6.13091015663289\\
32.229541486257	6.09091731024894\\
32.3783808817133	6.05092471517736\\
32.5279076330741	6.01093237374181\\
32.6781249146208	5.97094028828739\\
32.8290359152942	5.93094846118084\\
32.9806438387618	5.89095689481076\\
33.132951903486	5.85096559158782\\
33.2859633427924	5.81097455394494\\
33.4396814049384	5.77098378433748\\
33.5941093531822	5.73099328524353\\
33.7492504658521	5.69100305916398\\
33.9051080364161	5.65101310862289\\
34.0616853735517	5.61102343616757\\
34.2189858012163	5.57103404436885\\
34.3770126587175	5.53104493582133\\
34.5357693007845	5.49105611314352\\
34.6952590976386	5.45106757897814\\
34.8554854350656	5.41107933599226\\
35.0164517144866	5.37109138687763\\
35.1781613530314	5.33110373435078\\
35.3406177836102	5.29111638115336\\
35.5038244549868	5.25112933005232\\
35.6677848318515	5.21114258384014\\
35.8325023948954	5.17115614533508\\
35.9979806408833	5.13117001738139\\
36.1642230827288	5.09118420284961\\
36.3312332495683	5.05119870463674\\
36.4990146868361	5.01121352566651\\
36.6675709563398	4.97122866888967\\
36.8369056363357	4.93124413728416\\
37.007022321605	4.89125993385542\\
37.1779246235299	4.85127606163662\\
37.3496161701703	4.81129252368892\\
37.5221006063408	4.7713093231017\\
37.6953815936883	4.73132646299292\\
37.8694628107696	4.69134394650918\\
38.0443479531292	4.65136177682624\\
38.2200407333782	4.6113799571491\\
38.3965448812729	4.57139849071231\\
38.5738641437941	4.53141738078029\\
38.7520022852263	4.49143663064757\\
38.9309630872381	4.45145624363906\\
39.1107503489621	4.41147622311037\\
39.2913678870758	4.371496572448\\
39.4728195358824	4.33151729506976\\
39.6551091473924	4.29153839442494\\
39.8382405914052	4.25155987399465\\
40.0222177555915	4.21158173729211\\
40.2070445455755	4.17160398786293\\
40.3927248850181	4.13162662928545\\
40.5792627157	4.09164966517096\\
40.7666619976054	4.05167309916409\\
40.9549267090063	4.01169693494307\\
41.1440608465467	3.97172117622004\\
41.3340684253274	3.93174582674136\\
41.5249534789915	3.89177089028796\\
41.7167200598098	3.85179637067558\\
41.9093722387671	3.8118222717552\\
42.1029141056481	3.77184859741325\\
42.2973497691249	3.73187535157204\\
42.4926833568435	3.69190253818998\\
42.6889190155123	3.651930161262\\
42.8860609109892	3.61195822481987\\
43.0841132283705	3.57198673293248\\
43.2830801720801	3.53201568970625\\
43.4829659659579	3.49204509928546\\
43.6837748533501	3.45207496585255\\
43.8855110971994	3.41210529362851\\
44.0881789801347	3.37213608687325\\
44.2917828045629	3.33216734988594\\
44.4963268927599	3.29219908700533\\
44.701815586962	3.25223130261017\\
44.9082532494586	3.21226400111955\\
45.1156442626847	3.17229718699329\\
45.3239930293136	3.13233086473228\\
45.5333039723509	3.09236503887888\\
45.7435815352278	3.05239971401727\\
45.954830181896	3.01243489477393\\
46.167054396922	2.97247058581787\\
46.3802586855826	2.93250679186117\\
46.5944475739604	2.89254351765927\\
46.8096256090399	2.85258076801145\\
47.0257973588042	2.81261854776113\\
47.2429674123316	2.77265686179638\\
47.4611403798933	2.73269571505027\\
47.6803208930514	2.6927351125013\\
47.9005136047569	2.65277505917379\\
48.1217231894485	2.61281556013836\\
48.3439543431522	2.57285662051227\\
48.5672117835806	2.53289824545991\\
48.791500250233	2.49294044019325\\
49.0168245044966	2.45298320997217\\
49.243189329747	2.413026560105\\
49.4705995314498	2.37307049594894\\
49.6990599372628	2.3331150229105\\
49.9285753971387	2.29316014644593\\
50.1591507834275	2.25320587206169\\
50.3907909909802	2.21325220531494\\
50.6235009372529	2.17329915181395\\
50.8572855624109	2.13334671721862\\
51.0921498294339	2.09339490724091\\
51.3280987242209	2.05344372764532\\
51.5651372556965	2.0134931842494\\
51.8032704559168	1.97354328292425\\
52.0425033801768	1.93359402959489\\
52.2828411071172	1.89364543024093\\
52.5242887388322	1.85369749089694\\
52.7668514009784	1.81375021765297\\
53.010534242883	1.77380361665512\\
53.2553424376533	1.73385769410599\\
53.5012811822866	1.69391245626522\\
53.7483556977805	1.65396790944999\\
53.9965712292437	1.61402406003557\\
54.2459330460073	1.57408091445588\\
54.4964464417369	1.53413847920394\\
54.7481167345445	1.49419676083247\\
55.0009492671021	1.45425576595443\\
55.2549494067543	1.41431550124356\\
55.510122545633	1.37437597343494\\
55.7664741007712	1.33443718932554\\
56.0240095142187	1.29449915577479\\
56.282734253157	1.25456187970515\\
56.5426538100157	1.2146253681027\\
56.8037737025889	1.1746896280177\\
57.0660994741527	1.13475466656515\\
57.3296366935823	1.09482049092546\\
57.5943909554708	1.05488710834496\\
57.8603678802477	1.01495452613656\\
58.1275731142982	0.975022751680309\\
58.3960123300829	0.935091792424056\\
58.6656912262587	0.895161655884042\\
58.9366155277994	0.855232349645501\\
59.2087909861172	0.815303881363345\\
59.4822233791852	0.775376258762724\\
59.7569185116595	0.735449489639747\\
60.0328822150028	0.69552358186207\\
60.3101203476082	0.655598543369546\\
60.5886387949234	0.615674382174938\\
60.8684434695757	0.575751106364505\\
61.1495403114975	0.53582872409873\\
61.4319352880526	0.495907243612991\\
61.7156343941624	0.455986673218189\\
62.0006436524339	0.416067021301501\\
62.2869691132868	0.376148296327019\\
62.5746168550822	0.336230506836511\\
62.8635929842521	0.296313661450037\\
63.1539036354283	0.256397768866763\\
63.445554971573	0.216482837865608\\
63.7385531841099	0.17656887730598\\
64.032904493055	0.136655896128525\\
64.3286151471491	0.0967439033558602\\
64.6256914239905	0.0568329080933098\\
64.9241396301678	0.0169229195296711\\
65.2239661013943	-0.0229860530620616\\
65.5251772026422	-0.0628940003238974\\
65.827779328278	-0.102800912812097\\
66.1317789021976	-0.14270678099638\\
66.4371823779638	-0.182611595259174\\
66.7439962389419	-0.222515345894758\\
67.0522269984386	-0.262418023108549\\
67.3618811998395	-0.30231961701622\\
67.6729654167483	-0.342220117642952\\
67.9854862531262	-0.382119514922545\\
68.2994503434323	-0.422017798696652\\
68.6148643527643	-0.461914958713924\\
68.931734977	-0.501810984629152\\
69.2500689429393	-0.54170586600241\\
69.5698730084476	-0.581599592298275\\
69.8911539625984	-0.621492152884852\\
70.213918625818	-0.661383537032985\\
70.5381738500302	-0.701273733915336\\
70.8639265188016	-0.741162732605522\\
71.191183547488	-0.781050522077186\\
71.5199518833808	-0.820937091203135\\
71.8502385058548	-0.860822428754407\\
72.1820504265165	-0.900706523399334\\
72.5153946893524	-0.940589363702674\\
72.8502783708792	-0.980470938124608\\
73.1867085802933	-1.02035123501981\\
73.5246924596224	-1.06023024263655\\
73.8642371838769	-1.10010794911568\\
74.2053499612018	-1.13998434248965\\
74.5480380330304	-1.17985941068159\\
74.8923086742377	-1.21973314150431\\
75.2381691932944	-1.25960552265923\\
75.585626932423	-1.2994765417355\\
75.9346892677529	-1.33934618620889\\
76.2853636094773	-1.37921444344081\\
76.6376574020103	-1.41908130067728\\
76.9915781241455	-1.45894674504788\\
77.3471332892138	-1.49881076356472\\
77.7043304452438	-1.53867334312134\\
78.0631771751216	-1.57853447049172\\
78.4236810967518	-1.61839413232913\\
78.7858498632195	-1.65825231516508\\
79.1496911629522	-1.69810900540822\\
79.5152127198837	-1.73796418934325\\
79.8824222936175	-1.77781785312978\\
80.2513276795919	-1.81766998280121\\
80.6219367092452	-1.85752056426364\\
80.9942572501823	-1.89736958329467\\
81.3682972063414	-1.93721702554229\\
81.7440645181619	-1.97706287652369\\
82.121567162753	-2.01690712162409\\
82.5008131540631	-2.05674974609558\\
82.8818105430498	-2.09659073505592\\
83.2645674178507	-2.1364300734873\\
83.6490919039556	-2.17626774623518\\
84.0353921643786	-2.21610373800704\\
84.423476399831	-2.25593803337113\\
84.8133528488963	-2.29577061675525\\
85.2050297882047	-2.33560147244551\\
85.5985155326084	-2.37543058458498\\
85.9938184353586	-2.41525793717252\\
86.3909468882829	-2.45508351406144\\
86.7899093219629	-2.49490729895818\\
87.1907142059136	-2.53472927542103\\
87.5933700487633	-2.57454942685879\\
87.9978853984339	-2.61436773652947\\
88.4042688423224	-2.65418418753888\\
88.8125290074835	-2.69399876283936\\
89.2226745608124	-2.73381144522833\\
89.6347142092288	-2.77362221734696\\
90.0486566998624	-2.81343106167874\\
90.4645108202374	-2.85323796054815\\
90.88228539846	-2.89304289611915\\
91.3019893034057	-2.93284585039381\\
91.7236314449071	-2.97264680521085\\
92.1472207739435	-3.01244574224422\\
92.5727662828307	-3.05224264300158\\
93.0002770054119	-3.09203748882288\\
93.4297620172496	-3.13183026087882\\
93.8612304358183	-3.1716209401694\\
94.2946914206979	-3.21140950752236\\
94.730154173768	-3.25119594359166\\
95.1676279394036	-3.29098022885597\\
95.6071220046713	-3.33076234361712\\
96.0486456995261	-3.37054226799844\\
96.4922083970099	-3.41031998194332\\
96.9378195134503	-3.45009546521352\\
97.3854885086602	-3.48986869738759\\
97.8352248861393	-3.52963965785923\\
98.2870381932753	-3.56940832583574\\
98.7409380215466	-3.60917468033626\\
99.1969340067261	-3.64893870019016\\
99.655035829086	-3.68870036403538\\
100.115253213603	-3.72845965031674\\
100.577595930163	-3.76821653728418\\
101.042073793774	-3.80797100299112\\
101.508696664767	-3.84772302529269\\
101.977474449012	-3.88747258184397\\
102.448417098122	-3.92721965009825\\
102.921534609671	-3.96696420730529\\
103.3968370274	-4.00670623050943\\
103.874334441436	-4.04644569654788\\
104.354036988501	-4.08618258204889\\
104.83595485213	-4.12591686342982\\
105.320098262886	-4.16564851689545\\
105.80647749858	-4.20537751843596\\
106.295102884484	-4.24510384382515\\
106.785984793556	-4.2848274686185\\
107.279133646656	-4.32454836815126\\
107.774559912768	-4.36426651753654\\
108.272274109224	-4.40398189166334\\
108.772286801926	-4.44369446519465\\
109.27460860557	-4.48340421256537\\
109.779250183872	-4.52311110798045\\
110.286222249794	-4.56281512541274\\
110.795535565772	-4.6025162386011\\
111.307200943943	-4.64221442104826\\
111.821229246378	-4.68190964601882\\
112.337631385307	-4.72160188653713\\
112.856418323356	-4.76129111538528\\
113.377601073777	-4.80097730510088\\
113.901190700682	-4.84066042797503\\
114.427198319278	-4.88034045605011\\
114.955635096105	-4.92001736111769\\
115.486512249268	-4.95969111471626\\
116.019841048683	-4.99936168812917\\
116.555632816306	-5.03902905238228\\
117.093898926384	-5.07869317824177\\
117.634650805689	-5.11835403621199\\
118.177899933763	-5.15801159653306\\
118.723657843162	-5.19766582917862\\
119.271936119701	-5.23731670385365\\
119.822746402699	-5.27696418999193\\
120.376100385228	-5.31660825675389\\
120.932009814357	-5.3562488730242\\
121.490486491407	-5.39588600740933\\
122.051542272196	-5.43551962823522\\
122.615189067297	-5.47514970354488\\
123.181438842284	-5.5147762010959\\
123.750303617991	-5.55439908835804\\
124.321795470765	-5.59401833251074\\
124.895926532723	-5.63363390044061\\
125.472708992008	-5.67324575873901\\
126.052155093052	-5.71285387369941\\
126.63427713683	-5.75245821131487\\
127.219087481126	-5.79205873727553\\
127.806598540793	-5.83165541696597\\
128.396822788018	-5.87124821546258\\
128.989772752585	-5.910837097531\\
129.585461022141	-5.9504220276234\\
130.183900242466	-5.99000296987585\\
130.785103117737	-6.02957988810564\\
131.389082410804	-6.06915274580852\\
131.995850943453	-6.10872150615597\\
132.605421596686	-6.14828613199256\\
133.217807310988	-6.18784658583301\\
133.833021086605	-6.22740282985951\\
134.451075983821	-6.26695482591884\\
135.071985123233	-6.30650253551959\\
135.69576168603	-6.34604591982923\\
136.322418914273	-6.38558493967133\\
136.951970111177	-6.42511955552253\\
137.584428641392	-6.46464972750972\\
138.219807931287	-6.50417541540702\\
138.858121469236	-6.54369657863289\\
139.499382805904	-6.58321317624707\\
140.143605554534	-6.62272516694757\\
140.790803391236	-6.6622325090677\\
141.440990055278	-6.70173516057294\\
142.094179349378	-6.74123307905789\\
142.750385139995	-6.78072622174318\\
143.409621357626	-6.82021454547228\\
144.0719019971	-6.85969800670845\\
144.737241117876	-6.89917656153151\\
145.405652844341	-6.93865016563459\\
146.077151366109	-6.97811877432109\\
146.751750938323	-7.01758234250118\\
147.42946588196	-7.05704082468875\\
148.110310584131	-7.09649417499806\\
148.794299498388	-7.13594234714036\\
149.481447145032	-7.17538529442062\\
150.171768111418	-7.21482296973418\\
150.865277052271	-7.2542553255633\\
151.561988689989	-7.29368231397381\\
152.261917814963	-7.33310388661166\\
152.965079285884	-7.37251999469944\\
153.671488030065	-7.41193058903293\\
154.381159043753	-7.45133561997761\\
155.094107392451	-7.49073503746503\\
155.810348211234	-7.53012879098937\\
156.529896705074	-7.56951682960383\\
157.25276814916	-7.60889910191696\\
157.978977889225	-7.64827555608912\\
158.708541341869	-7.68764613982877\\
159.441473994887	-7.72701080038878\\
160.177791407599	-7.76636948456279\\
160.917509211179	-7.8057221386814\\
161.66064310899	-7.84506870860846\\
162.40720887691	-7.8844091397373\\
163.157222363676	-7.92374337698688\\
163.910699491214	-7.96307136479797\\
164.66765625498	-8.00239304712931\\
165.428108724297	-8.04170836745374\\
166.192073042701	-8.08101726875422\\
166.959565428276	-8.12031969351997\\
167.730602174008	-8.15961558374249\\
168.505199648122	-8.19890488091151\\
169.283374294433	-8.23818752601106\\
170.065142632699	-8.27746345951543\\
170.850521258965	-8.31673262138498\\
171.639526845917	-8.35599495106222\\
172.432176143241	-8.39525038746757\\
173.228485977972	-8.43449886899526\\
174.028473254854	-8.47374033350914\\
174.832154956701	-8.51297471833853\\
175.639548144754	-8.5522019602739\\
176.450669959044	-8.59142199556273\\
177.265537618758	-8.63063475990517\\
178.084168422602	-8.66984018844973\\
178.906579749169	-8.70903821578897\\
179.732789057308	-8.74822877595516\\
180.562813886497	-8.78741180241583\\
181.39667185721	-8.82658722806946\\
182.234380671297	-8.86575498524097\\
183.075958112354	-8.90491500567722\\
183.921422046107	-8.94406722054265\\
184.770790420785	-8.98321156041466\\
185.624081267505	-9.02234795527904\\
186.481312700654	-9.06147633452554\\
187.342502918271	-9.10059662694313\\
188.207670202438	-9.1397087607154\\
189.076832919665	-9.17881266341602\\
189.950009521279	-9.21790826200392\\
190.827218543818	-9.25699548281865\\
191.708478609426	-9.29607425157569\\
192.593808426241	-9.33514449336161\\
193.483226788801	-9.37420613262933\\
194.376752578439	-9.41325909319332\\
195.274404763682	-9.45230329822475\\
196.176202400658	-9.4913386702466\\
197.082164633495	-9.53036513112885\\
197.992310694735	-9.56938260208345\\
198.906659905733	-9.60839100365948\\
199.825231677076	-9.64739025573813\\
200.748045508988	-9.68638027752773\\
201.675120991751	-9.72536098755871\\
202.606477806113	-9.76433230367855\\
203.542135723711	-9.80329414304674\\
204.482114607491	-9.84224642212967\\
205.426434412127	-9.88118905669548\\
206.375115184445	-9.92012196180893\\
207.32817706385	-9.95904505182627\\
208.285640282755	-9.99795824038997\\
209.247525167004	-10.0368614404235\\
210.213852136311	-10.0757545641262\\
211.18464170469	-10.1146375229678\\
212.159914480891	-10.1535102276833\\
213.139691168835	-10.1923725882676\\
214.123992568061	-10.23122451397\\
215.112839574156	-10.2700659132892\\
216.10625317921	-10.3088966939676\\
217.104254472254	-10.3477167629857\\
218.106864639713	-10.3865260265572\\
219.114104965849	-10.4253243901231\\
220.12599683322	-10.4641117583464\\
221.142561723131	-10.5028880351066\\
222.163821216089	-10.5416531234939\\
223.189796992262	-10.5804069258041\\
224.220510831939	-10.6191493435325\\
225.255984615994	-10.6578802773688\\
226.296240326348	-10.6965996271911\\
227.341300046436	-10.7353072920605\\
228.391185961678	-10.7740031702151\\
229.44592035995	-10.8126871590648\\
230.505525632052	-10.8513591551852\\
231.570024272191	-10.890019054312\\
232.639438878451	-10.9286667513354\\
233.713792153278	-10.9673021402938\\
234.793106903962	-11.0059251143688\\
235.877406043117	-11.0445355658788\\
236.966712589169	-11.0831333862734\\
238.061049666849	-11.1217184661275\\
239.160440507677	-11.1602906951355\\
240.264908450462	-11.1988499621052\\
241.374476941791	-11.2373961549523\\
242.48916953653	-11.2759291606942\\
243.609009898327	-11.3144488654439\\
244.734021800108	-11.3529551544047\\
245.864229124585	-11.3914479118635\\
246.999655864764	-11.4299270211852\\
248.140326124455	-11.4683923648067\\
249.286264118777	-11.5068438242309\\
250.437494174681	-11.5452812800204\\
251.594040731461	-11.5837046117921\\
252.755928341275	-11.6221136982103\\
253.923181669665	-11.6605084169813\\
255.09582549608	-11.6988886448471\\
256.273884714404	-11.7372542575794\\
257.457384333485	-11.7756051299734\\
258.646349477661	-11.8139411358416\\
259.840805387301	-11.852262148008\\
261.040777419332	-11.8905680383019\\
262.246291047787	-11.9288586775514\\
263.457371864337	-11.9671339355777\\
264.674045578839	-12.0053936811889\\
265.896338019882	-12.0436377821737\\
267.124275135332	-12.0818661052951\\
268.357882992886	-12.1200785162848\\
269.597187780627	-12.1582748798365\\
270.842215807572	-12.1964550595998\\
272.092993504239	-12.2346189181743\\
273.349547423206	-12.2727663171033\\
274.611904239671	-12.3108971168675\\
275.880090752022	-12.3490111768789\\
277.154133882405	-12.3871083554749\\
278.434060677294	-12.4251885099116\\
279.719898308069	-12.4632514963579\\
281.011674071587	-12.5012971698896\\
282.309415390768	-12.5393253844828\\
283.613149815172	-12.5773359930079\\
284.922905021585	-12.6153288472235\\
286.238708814609	-12.6533037977703\\
287.560589127251	-12.6912606941649\\
288.888574021513	-12.7291993847933\\
290.222691688993	-12.7671197169055\\
291.562970451479	-12.8050215366089\\
292.909438761552	-12.8429046888623\\
294.262125203191	-12.8807690174696\\
295.621058492378	-12.9186143650742\\
296.98626747771	-12.9564405731526\\
298.357781141006	-12.9942474820084\\
299.735628597932	-13.0320349307662\\
301.119839098607	-13.0698027573659\\
302.510442028234	-13.1075507985561\\
303.907466907719	-13.1452788898889\\
305.310943394298	-13.1829868657132\\
306.720901282168	-13.2206745591693\\
308.137370503119	-13.2583418021827\\
309.560381127168	-13.2959884254585\\
310.989963363199	-13.3336142584749\\
312.426147559604	-13.3712191294783\\
313.868964204927	-13.4088028654767\\
315.318443928511	-13.4463652922343\\
316.774617501149	-13.4839062342656\\
318.237515835737	-13.5214255148299\\
319.707169987928	-13.5589229559252\\
321.183611156796	-13.5963983782831\\
322.666870685493	-13.6338516013629\\
324.156980061919	-13.6712824433457\\
325.653970919389	-13.7086907211296\\
327.1578750373	-13.7460762503235\\
328.668724341813	-13.7834388452421\\
330.186550906528	-13.8207783189004\\
331.711386953162	-13.858094483008\\
333.243264852235	-13.8953871479643\\
334.78221712376	-13.9326561228528\\
336.328276437929	-13.9699012154362\\
337.881475615808	-14.007122232151\\
339.441847630035	-14.0443189781025\\
341.009425605519	-14.0814912570595\\
342.584242820144	-14.1186388714498\\
344.166332705473	-14.1557616223547\\
345.75572884746	-14.1928593095044\\
347.352464987164	-14.2299317312731\\
348.956575021463	-14.2669786846743\\
350.568093003772	-14.3039999653559\\
352.187053144771	-14.3409953675957\\
353.813489813129	-14.377964684297\\
355.447437536229	-14.4149077069837\\
357.088931000911	-14.4518242257962\\
358.738005054198	-14.4887140294869\\
360.39469470404	-14.525576905416\\
362.059035120061	-14.5624126395471\\
363.731061634299	-14.5992210164434\\
365.410809741959	-14.6360018192634\\
367.09831510217	-14.6727548297571\\
368.793613538734	-14.7094798282617\\
370.496741040893	-14.7461765936985\\
372.207733764093	-14.7828449035689\\
373.926628030746	-14.8194845339504\\
375.653460331008	-14.8560952594937\\
377.388267323548	-14.8926768534189\\
379.13108583633	-14.9292290875123\\
380.881952867393	-14.9657517321231\\
382.640905585636	-15.0022445561605\\
384.407981331609	-15.0387073270903\\
386.183217618305	-15.0751398109324\\
387.966652131954	-15.1115417722576\\
389.758322732826	-15.1479129741852\\
391.558267456034	-15.1842531783803\\
393.366524512341	-15.2205621450515\\
395.183132288971	-15.2568396329481\\
397.008129350424	-15.2930853993583\\
398.841554439296	-15.329299200107\\
400.683446477099	-15.3654807895538\\
402.533844565089	-15.4016299205909\\
404.392787985098	-15.4377463446416\\
406.260316200361	-15.473829811659\\
408.136468856361	-15.5098800701236\\
410.021285781671	-15.5458968670431\\
411.914806988789	-15.5818799479503\\
413.817072675003	-15.6178290569026\\
415.72812322323	-15.6537439364807\\
417.647999202884	-15.6896243277881\\
419.576741370729	-15.7254699704504\\
421.514390671752	-15.7612806026147\\
423.460988240025	-15.7970559609496\\
425.416575399583	-15.8327957806444\\
427.381193665299	-15.8684997954099\\
429.354884743766	-15.9041677374779\\
431.337690534184	-15.9397993376019\\
433.329653129246	-15.9753943250573\\
435.330814816033	-16.0109524276422\\
437.341218076915	-16.0464733716781\\
439.360905590448	-16.081956882011\\
441.389920232282	-16.1174026820125\\
443.428305076071	-16.1528104935811\\
445.476103394389	-16.1881800371437\\
447.533358659646	-16.2235110316573\\
449.600114545014	-16.258803194611\\
451.67641492535	-16.2940562420276\\
453.762303878129	-16.3292698884667\\
455.857825684385	-16.3644438470261\\
457.963024829641	-16.3995778293453\\
460.077946004862	-16.4346715456078\\
462.202634107401	-16.4697247045443\\
464.33713424195	-16.504737013436\\
466.481491721497	-16.5397081781179\\
468.635752068297	-16.5746379029825\\
470.799961014824	-16.6095258909836\\
472.974164504755	-16.6443718436402\\
475.158408693936	-16.6791754610411\\
477.352739951367	-16.7139364418493\\
479.557204860185	-16.7486544833063\\
481.771850218653	-16.7833292812376\\
483.996723041153	-16.8179605300573\\
486.231870559183	-16.8525479227738\\
488.477340222363	-16.8870911509954\\
490.73317969944	-16.9215899049357\\
492.999436879299	-16.9560438734202\\
495.276159871982	-16.9904527438923\\
497.563397009708	-17.0248162024197\\
499.861196847899	-17.0591339337012\\
502.169608166212	-17.093405621074\\
504.48867996957	-17.1276309465204\\
506.818461489212	-17.1618095906759\\
509.159002183727	-17.1959412328362\\
511.51035174011	-17.2300255509658\\
513.872560074817	-17.2640622217058\\
516.245677334823	-17.2980509203825\\
518.629753898685	-17.3319913210161\\
521.024840377617	-17.3658830963297\\
523.430987616559	-17.3997259177587\\
525.848246695256	-17.4335194554597\\
528.27666892935	-17.4672633783208\\
530.716305871459	-17.5009573539716\\
533.167209312278	-17.5346010487931\\
535.629431281677	-17.5681941279282\\
538.103024049808	-17.6017362552932\\
540.588040128206	-17.635227093588\\
543.084532270915	-17.6686663043079\\
545.592553475602	-17.7020535477551\\
548.11215698468	-17.7353884830505\\
550.643396286443	-17.7686707681458\\
553.186325116201	-17.8019000598363\\
555.740997457415	-17.835076013773\\
558.307467542852	-17.8681982844759\\
560.885789855729	-17.9012665253473\\
563.476019130871	-17.934280388685\\
566.078210355879	-17.9672395256964\\
568.692418772287	-18.0001435865124\\
571.318699876742	-18.0329922202017\\
573.957109422183	-18.0657850747857\\
576.607703419019	-18.0985217972528\\
579.27053813632	-18.1312020335745\\
581.945670103016	-18.1638254287198\\
584.633156109091	-18.196391626672\\
587.333053206791	-18.2289002704436\\
590.045418711838	-18.2613510020937\\
592.77031020464	-18.2937434627437\\
595.507785531519	-18.3260772925947\\
598.25790280594	-18.3583521309443\\
601.020720409738	-18.3905676162044\\
603.796296994364	-18.4227233859184\\
606.584691482126	-18.4548190767795\\
609.385963067442	-18.4868543246489\\
612.200171218097	-18.5188287645742\\
615.027375676503	-18.5507420308082\\
617.867636460969	-18.5825937568283\\
620.721013866976	-18.6143835753553\\
623.587568468455	-18.6461111183735\\
626.467361119072	-18.6777760171502\\
629.360452953524	-18.7093779022561\\
632.266905388836	-18.7409164035858\\
635.186780125658	-18.772391150378\\
638.120139149584	-18.803801771237\\
641.067044732464	-18.8351478941536\\
644.027559433723	-18.8664291465266\\
647.001746101695	-18.8976451551847\\
649.989667874954	-18.9287955464082\\
652.991388183652	-18.9598799459514\\
656.00697075087	-18.990897979065\\
659.03647959397	-19.021849270519\\
662.07997902595	-19.0527334446255\\
665.137533656814	-19.0835501252619\\
668.209208394941	-19.1142989358943\\
671.295068448464	-19.1449794996014\\
674.395179326655	-19.1755914390984\\
677.509606841313	-19.2061343767607\\
680.638417108163	-19.2366079346489\\
683.78167654826	-19.267011734533\\
686.9394518894	-19.297345397917\\
690.11181016753	-19.3276085460645\\
693.298818728181	-19.3578008000235\\
696.500545227891	-19.387921780652\\
699.717057635642	-19.4179711086432\\
702.948424234306	-19.447948404552\\
706.19471362209	-19.4778532888205\\
709.455994713995	-19.5076853818043\\
712.732336743282	-19.5374443037989\\
716.023809262934	-19.5671296750664\\
719.330482147139	-19.5967411158617\\
722.652425592773	-19.62627824646\\
725.989710120886	-19.6557406871833\\
729.3424065782	-19.6851280584279\\
732.71058613862	-19.7144399806917\\
736.094320304735	-19.7436760746012\\
739.493680909343	-19.7728359609399\\
742.908740116972	-19.8019192606752\\
746.339570425412	-19.8309255949869\\
749.786244667258	-19.8598545852945\\
753.248836011454	-19.8887058532858\\
756.727417964843	-19.9174790209448\\
760.222064373731	-19.94617371058\\
763.732849425456	-19.9747895448525\\
767.259847649959	-20.0033261468048\\
770.803133921369	-20.0317831398889\\
774.362783459591	-20.060160147995\\
777.938871831903	-20.08845679548\\
781.531474954562	-20.1166727071964\\
785.140669094413	-20.1448075085204\\
788.76653087051	-20.172860825381\\
792.40913725574	-20.2008322842886\\
796.068565578463	-20.2287215123638\\
799.744893524145	-20.2565281373659\\
803.438199137013	-20.2842517877219\\
807.148560821713	-20.3118920925547\\
810.876057344967	-20.3394486817125\\
814.620767837254	-20.3669211857966\\
818.382771794484	-20.3943092361906\\
822.162149079689	-20.4216124650887\\
825.958979924714	-20.4488305055241\\
829.773344931927	-20.4759629913976\\
833.605325075921	-20.5030095575056\\
837.455001705243	-20.5299698395686\\
841.322456544116	-20.5568434742592\\
845.207771694168	-20.58363009923\\
849.111029636188	-20.6103293531415\\
853.032313231867	-20.6369408756899\\
856.971705725559	-20.6634643076348\\
860.92929074605	-20.6898992908264\\
864.905152308334	-20.7162454682327\\
868.899374815392	-20.7425024839669\\
872.91204305999	-20.768669983314\\
876.943242226474	-20.7947476127577\\
880.993057892579	-20.8207350200067\\
885.061576031251	-20.846631854021\\
889.148883012464	-20.872437765038\\
893.255065605058	-20.8981524045982\\
897.380210978585	-20.9237754255707\\
901.52440670515	-20.9493064821785\\
905.687740761275	-20.9747452300233\\
909.870301529772	-21.0000913261106\\
914.07217780161	-21.0253444288734\\
918.293458777803	-21.0505041981968\\
922.534234071309	-21.0755702954414\\
926.794593708924	-21.100542383467\\
931.074628133199	-21.1254201266551\\
935.374428204358	-21.1502031909322\\
939.694085202226	-21.1748912437915\\
944.033690828168	-21.1994839543152\\
948.39333720704	-21.2239809931958\\
952.773116889131	-21.248382032757\\
957.173122852146	-21.272686746975\\
961.593448503166	-21.2968948114979\\
966.034187680636	-21.321005903666\\
970.495434656357	-21.3450197025308\\
974.977284137491	-21.3689358888743\\
979.479831268559	-21.3927541452266\\
984.003171633478	-21.4164741558844\\
988.547401257578	-21.4400956069279\\
993.112616609642	-21.4636181862381\\
997.698914603958	-21.4870415835123\\
1002.30639260238	-21.5103654902807\\
1006.93514841637	-21.5335895999208\\
1011.58528030912	-21.5567136076726\\
1016.2568869976	-21.5797372106522\\
1020.95006765465	-21.6026601078652\\
1025.66492191112	-21.62548200022\\
1030.40154985798	-21.6482025905392\\
1035.16005204838	-21.6708215835719\\
1039.94052949988	-21.6933386860042\\
1044.74308369654	-21.7157536064694\\
1049.56781659108	-21.7380660555578\\
1054.41483060703	-21.7602757458255\\
1059.28422864096	-21.7823823918026\\
1064.17611406461	-21.8043857100007\\
1069.09059072708	-21.8262854189198\\
1074.02776295708	-21.8480812390542\\
1078.98773556513	-21.8697728928979\\
1083.97061384575	-21.8913601049494\\
1088.97650357974	-21.9128426017151\\
1094.00551103639	-21.9342201117126\\
1099.05774297578	-21.955492365473\\
1104.13330665098	-21.9766590955425\\
1109.2323098104	-21.9977200364826\\
1114.35486070002	-22.0186749248704\\
1119.50106806574	-22.0395234992977\\
1124.67104115564	-22.0602655003687\\
1129.86488972231	-22.080900670698\\
1135.0827240252	-22.1014287549066\\
1140.32465483296	-22.1218494996178\\
1145.59079342577	-22.1421626534519\\
1150.8812515977	-22.16236796702\\
1156.19614165913	-22.1824651929171\\
1161.53557643908	-22.2024540857143\\
1166.89966928762	-22.2223344019496\\
1172.28853407829	-22.2421059001185\\
1177.70228521052	-22.2617683406631\\
1183.14103761204	-22.2813214859607\\
1188.60490674132	-22.3007651003108\\
1194.09400859005	-22.320098949922\\
1199.60845968555	-22.3393228028974\\
1205.14837709331	-22.3584364292187\\
1210.71387841942	-22.3774396007304\\
1216.30508181309	-22.3963320911218\\
1221.92210596917	-22.4151136759087\\
1227.56507013062	-22.4337841324142\\
1233.23409409112	-22.4523432397475\\
1238.92929819754	-22.4707907787833\\
1244.65080335254	-22.4891265321383\\
1250.3987310171	-22.5073502841485\\
1256.17320321316	-22.5254618208437\\
1261.97434252612	-22.5434609299225\\
1267.80227210752	-22.5613474007251\\
1273.65711567764	-22.5791210242056\\
1279.53899752808	-22.596781592903\\
1285.44804252445	-22.6143289009113\\
1291.38437610901	-22.6317627438484\\
1297.34812430331	-22.6490829188239\\
1303.33941371088	-22.6662892244056\\
1309.35837151994	-22.6833814605852\\
1315.40512550606	-22.700359428743\\
1321.47980403488	-22.7172229316108\\
1327.58253606487	-22.7339717732346\\
1333.71345115004	-22.7506057589351\\
1339.87267944269	-22.7671246952682\\
1346.06035169616	-22.7835283899831\\
1352.27659926764	-22.7998166519808\\
1358.52155412096	-22.8159892912697\\
1364.79534882933	-22.8320461189214\\
1371.09811657822	-22.8479869470245\\
1377.42999116818	-22.8638115886379\\
1383.79110701763	-22.8795198577422\\
1390.18159916578	-22.8951115691902\\
1396.60160327544	-22.9105865386567\\
1403.05125563594	-22.9259445825863\\
1409.53069316601	-22.9411855181402\\
1416.04005341668	-22.9563091631425\\
1422.5794745742	-22.9713153360239\\
1429.14909546299	-22.9862038557658\\
1435.74905554856	-23.0009745418417\\
1442.3794949405	-23.0156272141583\\
1449.04055439544	-23.0301616929951\\
1455.73237532004	-23.0445777989425\\
1462.45509977397	-23.058875352839\\
1469.20887047298	-23.0730541757071\\
1475.99383079187	-23.0871140886875\\
1482.81012476756	-23.1010549129725\\
1489.65789710218	-23.1148764697382\\
1496.53729316606	-23.1285785800748\\
1503.44845900091	-23.1421610649161\\
1510.39154132284	-23.1556237449677\\
1517.36668752555	-23.1689664406337\\
1524.37404568338	-23.1821889719421\\
1531.41376455451	-23.195291158469\\
1538.4859935841	-23.2082728192614\\
1545.59088290748	-23.2211337727585\\
1552.72858335329	-23.233873836712\\
1559.89924644673	-23.2464928281049\\
1567.10302441275	-23.2589905630689\\
1574.34007017931	-23.2713668568\\
1581.61053738059	-23.2836215234738\\
1588.91458036027	-23.2957543761584\\
1596.25235417481	-23.3077652267262\\
1603.62401459674	-23.3196538857647\\
1611.02971811795	-23.3314201624851\\
1618.46962195303	-23.3430638646302\\
1625.94388404263	-23.3545847983806\\
1633.45266305675	-23.3659827682593\\
1640.99611839816	-23.3772575770349\\
1648.57441020578	-23.3884090256235\\
1656.18769935804	-23.3994369129892\\
1663.83614747635	-23.4103410360424\\
1671.51991692849	-23.4211211895376\\
1679.23917083208	-23.4317771659689\\
1686.99407305804	-23.4423087554644\\
1694.78478823403	-23.4527157456787\\
1702.61148174801	-23.4629979216841\\
1710.47431975172	-23.4731550658602\\
1718.37346916419	-23.4831869577818\\
1726.30909767531	-23.4930933741052\\
1734.28137374936	-23.5028740884528\\
1742.29046662864	-23.5125288712964\\
1750.33654633699	-23.5220574898381\\
1758.41978368348	-23.5314597078904\\
1766.54035026596	-23.5407352857541\\
1774.69841847474	-23.5498839800944\\
1782.89416149627	-23.5589055438153\\
1791.12775331676	-23.5677997259327\\
1799.39936872594	-23.5765662714449\\
1807.70918332073	-23.5852049212018\\
1816.05737350894	-23.5937154117723\\
1824.44411651309	-23.6020974753097\\
1832.86959037413	-23.6103508394149\\
1841.33397395519	-23.6184752269982\\
1849.83744694544	-23.6264703561388\\
1858.38018986386	-23.6343359399426\\
1866.9623840631	-23.6420716863976\\
1875.58421173328	-23.6496772982275\\
1884.24585590593	-23.6571524727434\\
1892.94750045783	-23.6644969016931\\
1901.68933011491	-23.6717102711081\\
1910.47153045619	-23.6787922611493\\
1919.29428791772	-23.6857425459491\\
1928.15778979652	-23.6925607934525\\
1937.06222425457	-23.6992466652554\\
1946.00778032282	-23.7057998164404\\
1954.99464790516	-23.7122198954106\\
1964.02301778248	-23.7185065437207\\
1973.09308161673	-23.7246593959057\\
1982.20503195496	-23.7306780793074\\
1991.35906223344	-23.7365622138976\\
2000.55536678172	-23.7423114120997\\
2009.79414082682	-23.7479252786069\\
2019.07558049731	-23.7534034101978\\
2028.39988282751	-23.7587453955497\\
2037.76724576168	-23.7639508150483\\
2047.17786815819	-23.7690192405954\\
2056.63194979376	-23.773950235413\\
2066.12969136771	-23.7787433538448\\
2075.6712945062	-23.7833981411544\\
2085.25696176653	-23.7879141333205\\
2094.88689664142	-23.7922908568289\\
2104.56130356336	-23.7965278284613\\
2114.2803879089	-23.8006245550806\\
2124.04435600306	-23.8045805334133\\
2133.8534151237	-23.8083952498273\\
2143.70777350589	-23.8120681801077\\
2153.60764034637	-23.8155987892275\\
2163.55322580795	-23.8189865311159\\
2173.54474102401	-23.8222308484214\\
2183.58239810298	-23.8253311722724\\
2193.66641013278	-23.8282869220328\\
2203.79699118545	-23.8310975050542\\
2213.97435632161	-23.8337623164232\\
2224.19872159503	-23.8362807387054\\
2234.47030405729	-23.8386521416841\\
2244.78932176229	-23.8408758820951\\
2255.15599377096	-23.8429513033566\\
2265.57054015585	-23.8448777352949\\
2276.03318200585	-23.8466544938645\\
2286.54414143084	-23.8482808808641\\
2297.10364156645	-23.8497561836467\\
2307.71190657875	-23.8510796748255\\
2318.36916166905	-23.8522506119736\\
2329.07563307865	-23.8532682373188\\
2339.83154809368	-23.8541317774327\\
2350.63713504986	-23.8548404429138\\
2361.49262333744	-23.8553934280651\\
2372.39824340596	-23.8557899105658\\
2383.35422676926	-23.856029051136\\
2394.36080601029	-23.8561099931962\\
2405.4182147861	-23.8560318625195\\
2416.52668783283	-23.855793766877\\
2427.6864609706	-23.8553947956776\\
2438.8977711086	-23.8548340195988\\
2450.16085625011	-23.8541104902122\\
2461.4759554975	-23.8532232396005\\
2472.84330905736	-23.8521712799669\\
2484.26315824557	-23.8509536032371\\
2495.73574549243	-23.8495691806533\\
2507.26131434783	-23.8480169623594\\
2518.84010948637	-23.8462958769784\\
2530.4723767126	-23.8444048311806\\
2542.15836296621	-23.8423427092429\\
2553.8983163273	-23.8401083725994\\
2565.69248602162	-23.8377006593821\\
2577.54112242586	-23.8351183839517\\
2589.444477073	-23.8323603364191\\
2601.4028026576	-23.8294252821561\\
2613.41635304121	-23.8263119612956\\
2625.48538325773	-23.8230190882214\\
2637.61014951883	-23.8195453510459\\
2649.7909092194	-23.8158894110776\\
2662.02792094301	-23.8120499022756\\
2674.32144446738	-23.8080254306928\\
2686.67174076992	-23.8038145739059\\
2699.07907203326	-23.7994158804333\\
2711.54370165082	-23.7948278691393\\
2724.0658942324	-23.7900490286244\\
2736.64591560979	-23.7850778166025\\
2749.28403284242	-23.7799126592625\\
2761.98051422303	-23.7745519506161\\
2774.73562928336	-23.7689940518295\\
2787.54964879989	-23.76323729054\\
2800.42284479955	-23.7572799601556\\
2813.35549056553	-23.7511203191386\\
2826.34786064309	-23.7447565902717\\
2839.40023084533	-23.7381869599062\\
2852.51287825912	-23.7314095771919\\
2865.68608125093	-23.7244225532886\\
2878.92011947275	-23.7172239605577\\
2892.21527386804	-23.7098118317337\\
2905.57182667769	-23.7021841590757\\
2918.99006144599	-23.6943388934968\\
2932.4702630267	-23.6862739436721\\
2946.01271758903	-23.6779871751237\\
2959.61771262377	-23.6694764092828\\
2973.28553694936	-23.6607394225272\\
2987.01648071805	-23.651773945194\\
3000.81083542203	-23.6425776605671\\
3014.66889389963	-23.6331482038377\\
3028.59095034155	-23.6234831610378\\
3042.57730029708	-23.6135800679451\\
3056.6282406804	-23.6034364089592\\
3070.74406977686	-23.5930496159477\\
3084.92508724934	-23.5824170670607\\
3099.17159414457	-23.5715360855136\\
3113.48389289956	-23.5604039383371\\
3127.86228734801	-23.5490178350922\\
3142.30708272674	-23.53737492655\\
3156.8185856822	-23.5254723033357\\
3171.39710427697	-23.5133069945336\\
3186.04294799627	-23.5008759662537\\
3200.75642775457	-23.4881761201577\\
3215.53785590219	-23.4752042919437\\
3230.38754623189	-23.4619572497863\\
3245.30581398558	-23.4484316927337\\
3260.29297586097	-23.4346242490572\\
3275.34935001834	-23.4205314745537\\
3290.47525608724	-23.4061498507978\\
3305.67101517332	-23.3914757833434\\
3320.93694986511	-23.3765055998716\\
3336.27338424092	-23.3612355482841\\
3351.68064387565	-23.3456617947392\\
3367.15905584778	-23.3297804216289\\
3382.70894874623	-23.3135874254955\\
3398.3306526774	-23.2970787148837\\
3414.02449927217	-23.2802501081281\\
3429.7908216929	-23.2630973310724\\
3445.62995464054	-23.245616014718\\
3461.54223436172	-23.2278016927991\\
3477.5279986559	-23.2096497992831\\
3493.58758688252	-23.1911556657911\\
3509.72133996824	-23.1723145189375\\
3525.92960041413	-23.1531214775851\\
3542.21271230298	-23.1335715500116\\
3558.57102130658	-23.113659630986\\
3575.00487469309	-23.0933804987492\\
3591.51462133436	-23.0727288118976\\
3608.1006117134	-23.0516991061629\\
3624.76319793175	-23.0302857910881\\
3641.50273371703	-23.0084831465913\\
3658.31957443038	-22.9862853194162\\
3675.21407707406	-22.9636863194636\\
3692.18660029898	-22.9406800159991\\
3709.23750441235	-22.917260133732\\
3726.36715138533	-22.893420248761\\
3743.57590486067	-22.8691537843811\\
3760.86413016048	-22.8444540067448\\
3778.23219429398	-22.8193140203748\\
3795.68046596523	-22.793726763518\\
3813.20931558105	-22.7676850033387\\
3830.81911525882	-22.7411813309403\\
3848.51023883439	-22.7142081562125\\
3866.28306187004	-22.6867577024926\\
3884.13796166242	-22.6588220010373\\
3902.07531725059	-22.6303928852936\\
3920.09550942403	-22.6014619849629\\
3938.19892073078	-22.5720207198481\\
3956.38593548549	-22.5420602934758\\
3974.65693977764	-22.5115716864823\\
3993.0123214797	-22.4805456497555\\
4011.45247025537	-22.4489726973207\\
4029.97777756789	-22.4168430989604\\
4048.58863668828	-22.3841468725555\\
4067.28544270374	-22.3508737761371\\
4086.06859252603	-22.3170132996359\\
4104.93848489988	-22.2825546563161\\
4123.89552041149	-22.2474867738803\\
4142.94010149697	-22.2117982852304\\
4162.07263245094	-22.1754775188708\\
4181.29351943512	-22.1385124889372\\
4200.60317048688	-22.100890884835\\
4220.00199552799	-22.0626000604704\\
4239.49040637325	-22.0236270230562\\
4259.06881673929	-21.9839584214734\\
4278.73764225331	-21.9435805341704\\
4298.49730046193	-21.9024792565774\\
4318.34821084004	-21.8606400880171\\
4338.2907947997	-21.8180481180884\\
4358.32547569911	-21.7746880125011\\
4378.45267885158	-21.7305439983364\\
4398.67283153454	-21.6855998487104\\
4418.98636299868	-21.6398388668137\\
4439.39370447695	-21.593243869301\\
4459.89528919383	-21.5457971690029\\
4480.49155237446	-21.4974805569327\\
4501.18293125388	-21.4482752835573\\
4521.96986508636	-21.3981620393038\\
4542.85279515466	-21.3471209342707\\
4563.83216477945	-21.2951314771114\\
4584.90841932869	-21.2421725530594\\
4606.0820062271	-21.1882224010625\\
4627.35337496566	-21.1332585899931\\
4648.72297711113	-21.0772579939023\\
4670.19126631568	-21.020196766286\\
4691.75869832646	-20.9620503133306\\
4713.42573099534	-20.9027932661081\\
4735.19282428857	-20.8423994516914\\
4757.06044029659	-20.780841863162\\
4779.02904324381	-20.7180926284861\\
4801.09909949849	-20.6541229782382\\
4823.27107758263	-20.5889032121546\\
4845.54544818189	-20.5224026645079\\
4867.92268415562	-20.4545896682958\\
4890.4032605469	-20.3854315182511\\
4912.98765459258	-20.3148944326872\\
4935.67634573345	-20.2429435142071\\
4958.46981562442	-20.169542709319\\
4981.36854814472	-20.0946547670214\\
5004.37302940819	-20.0182411964415\\
5027.48374777359	-19.9402622236386\\
5050.70119385497	-19.8606767477163\\
5074.02586053209	-19.7794422964238\\
5097.4582429609	-19.6965149814766\\
5120.998838584	-19.6118494538757\\
5144.64814714125	-19.5253988595736\\
5168.40667068035	-19.4371147959095\\
5192.27491356753	-19.3469472693284\\
5216.2533824982	-19.2548446550083\\
5240.34258650778	-19.1607536591441\\
5264.54303698246	-19.064619284794\\
5288.85524767004	-18.966384802373\\
5313.27973469088	-18.8659917260902\\
5337.81701654885	-18.7633797978861\\
5362.46761414231	-18.6584869807232\\
5387.23205077517	-18.5512494634441\\
5412.11085216805	-18.4416016798358\\
5437.10454646936	-18.3294763450439\\
5462.21366426659	-18.2148045130789\\
5487.43873859752	-18.0975156598689\\
5512.78030496154	-17.9775377971586\\
5538.23890133107	-17.8547976235576\\
5563.81506816292	-17.7292207202309\\
5589.50934840979	-17.6007318001423\\
5615.32228753178	-17.4692550214353\\
5641.254433508	-17.3347143775301\\
5667.30633684818	-17.1970341788684\\
5693.47855060436	-17.0561396440278\\
5719.77163038263	-16.9119576212198\\
5746.18613435493	-16.7644174650619\\
5772.72262327089	-16.6134520980717\\
5799.38166046975	-16.4589992916558\\
5826.16381189231	-16.3010032075699\\
5853.06964609293	-16.1394162479925\\
5880.09973425162	-15.9742012705741\\
5907.25465018618	-15.8053342341327\\
5934.53497036433	-15.6328073510638\\
5961.94127391598	-15.4566328338969\\
5989.47414264556	-15.2768473354965\\
6017.13416104428	-15.093517194663\\
6044.92191630263	-14.9067446104684\\
6072.83799832281	-14.7166748781874\\
6100.88299973121	-14.5235048250556\\
6129.05751589107	-14.327492582215\\
6157.36214491506	-14.1289688157126\\
6185.79748767801	-13.928349508236\\
6214.36414782964	-13.7261503263494\\
6243.06273180741	-13.5230025150216\\
6271.89384884933	-13.3196701197142\\
6300.85811100697	-13.1170681321808\\
6329.95613315842	-12.9162808755267\\
6359.18853302131	-12.7185795765688\\
6388.55593116599	-12.5254376177882\\
6418.05895102864	-12.3385414334205\\
6447.69821892457	-12.1597944593538\\
6477.47436406142	-11.991311050307\\
6507.38801855264	-11.8353969763304\\
6537.43981743082	-11.6945131895904\\
6567.63039866118	-11.5712202258393\\
6597.96040315515	-11.4681020645208\\
6628.43047478397	-11.3876706022314\\
6659.0412603923	-11.3322549721192\\
6689.79340981204	-11.3038833507023\\
6720.68757587607	-11.3041679217055\\
6751.7244144321	-11.3342054349759\\
6782.90458435663	-11.3945055283343\\
6814.22874756894	-11.4849563197116\\
6845.69756904508	-11.6048320392658\\
6877.31171683206	-11.7528416322817\\
6909.07186206199	-11.9272116748432\\
6940.97867896634	-12.1257928702699\\
6973.03284489025	-12.3461775923979\\
7005.23504030693	-12.5858164720184\\
7037.58594883204	-12.8421243446575\\
7070.0862572383	-13.1125691393605\\
7102.73665547	-13.3947406460413\\
7135.53783665763	-13.6863989404152\\
7168.49049713269	-13.9855042650877\\
7201.59533644237	-14.2902313255907\\
7234.85305736446	-14.5989713957338\\
7268.26436592224	-14.9103255452837\\
7301.82997139948	-15.2230919127546\\
7335.55058635552	-15.5362494159537\\
7369.42692664033	-15.8489397402939\\
7403.45971140979	-16.1604489395467\\
7437.64966314091	-16.4701895588657\\
7471.99750764715	-16.7776838543371\\
7506.50397409388	-17.0825484316134\\
7541.16979501382	-17.3844804461674\\
7575.99570632259	-17.6832453851502\\
7610.98244733438	-17.9786663721205\\
7646.13076077758	-18.2706148892841\\
7681.44139281058	-18.5590027878632\\
7716.91509303763	-18.8437754484863\\
7752.55261452471	-19.1249059546454\\
7788.35471381553	-19.4023901494246\\
7824.32215094763	-19.6762424561681\\
7860.45568946846	-19.9464923557005\\
7896.7560964516	-20.2131814249857\\
7933.22414251308	-20.4763608539643\\
7969.86060182773	-20.7360893683499\\
8006.66625214554	-20.9924314961786\\
8043.6418748083	-21.245456124821\\
8080.78825476606	-21.4952353030011\\
8118.10618059391	-21.7418432491767\\
8155.5964445086	-21.9853555335121\\
8193.25984238547	-22.2258484057151\\
8231.09717377527	-22.4633982453079\\
8269.10924192116	-22.6980811145644\\
8307.29685377578	-22.9299723974441\\
8345.66082001834	-23.1591465104822\\
8384.20195507185	-23.3856766738103\\
8422.92107712042	-23.6096347323578\\
8461.81900812665	-23.8310910188586\\
8500.89657384898	-24.0501142516212\\
8540.15460385935	-24.2667714611382\\
8579.59393156072	-24.4811279405569\\
8619.21539420481	-24.6932472158234\\
8659.01983290983	-24.9031910319884\\
8699.0080926784	-25.1110193527178\\
8739.18102241541	-25.3167903705336\\
8779.5394749461	-25.5205605257055\\
8820.08430703417	-25.7223845320532\\
8860.81637939989	-25.9223154082006\\
8901.73655673847	-26.1204045130665\\
8942.84570773836	-26.3167015845734\\
8984.14470509972	-26.5112547807303\\
9025.63442555289	-26.7041107223854\\
9067.31574987707	-26.8953145370671\\
9109.18956291901	-27.084909903435\\
9151.25675361173	-27.2729390959428\\
9193.51821499347	-27.4594430293963\\
9235.97484422661	-27.6444613031396\\
9278.62754261669	-27.8280322446637\\
9321.47721563161	-28.010192952468\\
9364.5247729208	-28.1909793380439\\
9407.77112833455	-28.370426166879\\
9451.21719994339	-28.5485670984063\\
9494.86391005764	-28.7254347248435\\
9538.71218524688	-28.9010606088841\\
9582.76295635973	-29.0754753202183\\
9627.01715854358	-29.248708470871\\
9671.47573126437	-29.4207887493574\\
9716.13961832666	-29.5917439536612\\
9761.00976789355	-29.7616010230497\\
9806.08713250686	-29.9303860687427\\
9851.37266910737	-30.0981244034596\\
9896.86733905512	-30.264840569868\\
9942.57210814977	-30.4305583679634\\
9988.48794665118	-30.5953008814092\\
10034.6158293	-30.7590905028685\\
10080.9567353382	-30.9219489583585\\
10127.5116485301	-31.0838973306614\\
10174.2815571832	-31.2449560818233\\
10221.267454169	-31.4051450747735\\
10268.4703369442	-31.564483594097\\
10315.891207572	-31.7229903659917\\
10363.531072743	-31.88068357744\\
10411.3909437969	-32.0375808946283\\
10459.4718367439	-32.1936994806406\\
10507.7747722864	-32.3490560124575\\
10556.3007758401	-32.5036666972867\\
10605.0508775566	-32.6575472882548\\
10654.0261123446	-32.8107130994826\\
10703.2275198921	-32.9631790205733\\
10752.6561446888	-33.114959530535\\
10802.3130360475	-33.2660687111619\\
10852.1992481272	-33.4165202598969\\
10902.315839955	-33.5663275021974\\
10952.6638754485	-33.715503403424\\
11003.244423439	-33.8640605802733\\
11054.0585576935	-34.0120113117729\\
11105.1073569377	-34.1593675498579\\
11156.3919048792	-34.3061409295448\\
11207.9132902301	-34.4523427787216\\
11259.6726067303	-34.5979841275686\\
11311.6709531708	-34.7430757176258\\
11363.9094334169	-34.8876280105218\\
11416.3891564316	-35.0316511963777\\
11469.1112362992	-35.1751552019\\
11522.0767922492	-35.3181496981747\\
11575.2869486794	-35.4606441081757\\
11628.7428351806	-35.6026476139985\\
11682.4455865599	-35.7441691638305\\
11736.3963428651	-35.8852174786692\\
11790.596249409	-36.0258010587972\\
11845.0464567934	-36.1659281900261\\
11899.7481209338	-36.3056069497155\\
11954.7024030838	-36.4448452125794\\
12009.9104698599	-36.5836506562855\\
12065.3734932659	-36.7220307668577\\
12121.0926507183	-36.8599928438887\\
12177.069125071	-36.9975440055697\\
12233.3041046402	-37.1346911935458\\
12289.7987832301	-37.2714411776021\\
12346.554360158	-37.4078005601885\\
12403.5720402798	-37.5437757807884\\
12460.8530340153	-37.6793731201376\\
12518.3985573745	-37.8145987042998\\
12576.2098319827	-37.9494585086023\\
12634.2880851072	-38.0839583614392\\
12692.6345496825	-38.218103947946\\
12751.2504643373	-38.3519008135499\\
12810.1370734203	-38.4853543674014\\
12869.2956270265	-38.6184698856913\\
12928.7273810244	-38.7512525148561\\
12988.4335970818	-38.8837072746779\\
13048.4155426933	-39.015839061281\\
13108.6744912069	-39.1476526500282\\
13169.2117218509	-39.2791526983235\\
13230.0285197614	-39.4103437483205\\
13291.1261760091	-39.5412302295426\\
13352.5059876274	-39.6718164614169\\
13414.1692576393	-39.8021066557247\\
13476.1172950852	-39.9321049189721\\
13538.351415051	-40.0618152546818\\
13600.8729386957	-40.1912415656116\\
13663.6831932795	-40.3203876558983\\
13726.7835121922	-40.4492572331325\\
13790.1752349813	-40.5778539103653\\
13853.8597073802	-40.7061812080482\\
13917.8382813373	-40.8342425559103\\
13982.1123150444	-40.9620412947729\\
14046.6831729655	-41.0895806783046\\
14111.552225866	-41.2168638747179\\
14176.7208508414	-41.34389396841\\
14242.190431347	-41.4706739615477\\
14307.9623572268	-41.5972067756006\\
14374.0380247435	-41.7234952528216\\
14440.4188366077	-41.8495421576779\\
14507.1062020079	-41.9753501782326\\
14574.1015366405	-42.1009219274794\\
14641.4062627396	-42.2262599446305\\
14709.0218091073	-42.3513666963606\\
14776.9496111443	-42.4762445780054\\
14845.1911108798	-42.6008959147195\\
14913.7477570027	-42.7253229625903\\
14982.6210048919	-42.8495279097133\\
15051.8123166476	-42.9735128772266\\
15121.323161122	-43.0972799203065\\
15191.1550139505	-43.2208310291258\\
15261.3093575835	-43.3441681297747\\
15331.7876813171	-43.4672930851451\\
15402.5914813253	-43.5902076957801\\
15473.7222606918	-43.7129137006871\\
15545.1815294413	-43.8354127781186\\
15616.9708045722	-43.957706546318\\
15689.0916100885	-44.0797965642326\\
15761.5454770323	-44.2016843321951\\
15834.333943516	-44.3233712925714\\
15907.4585547554	-44.4448588303788\\
15980.9208631021	-44.5661482738717\\
16054.7224280766	-44.687240895098\\
16128.8648164017	-44.8081379104241\\
16203.3496020352	-44.9288404810312\\
16278.1783662036	-45.0493497133809\\
16353.352697436	-45.1696666596526\\
16428.8741915971	-45.2897923181518\\
16504.7444519217	-45.4097276336891\\
16580.9650890484	-45.5294734979314\\
16657.537721054	-45.6490307497246\\
16734.4639734876	-45.768400175388\\
16811.7454794054	-45.8875825089809\\
16889.3838794052	-46.0065784325412\\
16967.3808216612	-46.1253885762963\\
17045.7379619589	-46.2440135188465\\
17124.4569637308	-46.36245378732\\
17203.539498091	-46.4807098575006\\
17282.9872438709	-46.5987821539279\\
17362.8018876552	-46.716671049969\\
17442.9851238172	-46.8343768678631\\
17523.5386545551	-46.9518998787378\\
17604.464189928	-47.0692403025971\\
17685.7634478922	-47.1863983082811\\
17767.4381543379	-47.3033740133981\\
17849.4900431252	-47.4201674842268\\
17931.9208561219	-47.5367787355908\\
18014.7323432394	-47.653207730703\\
18097.9262624709	-47.7694543809817\\
18181.5043799277	-47.8855185458353\\
18265.4684698776	-48.0014000324191\\
18349.8203147818	-48.1170985953595\\
18434.5617053333	-48.2326139364493\\
18519.6944404947	-48.3479457043105\\
18605.2203275364	-48.4630934940257\\
18691.1411820748	-48.5780568467378\\
18777.4588281113	-48.6928352492163\\
18864.1750980704	-48.8074281333914\\
18951.2918328392	-48.9218348758534\\
19038.810881806	-49.036054797318\\
19126.7341029	-49.1500871620573\\
19215.0633626303	-49.2639311772944\\
19303.8005361259	-49.3775859925619\\
19392.9475071751	-49.4910506990236\\
19482.506168266	-49.6043243287587\\
19572.4784206263	-49.7174058540066\\
19662.8661742637	-49.8302941863733\\
19753.6713480067	-49.9429881759971\\
19844.8958695449	-50.0554866106735\\
19936.5416754703	-50.1677882149381\\
20028.6107113184	-50.2798916491066\\
20121.1049316092	-50.391795508271\\
20214.026299889	-50.5034983212508\\
20307.3767887718	-50.6149985494981\\
20401.1583799817	-50.7262945859568\\
20495.373064394	-50.8373847538721\\
20590.0228420787	-50.9482673055518\\
20685.1097223421	-51.0589404210771\\
20780.6357237695	-51.1694022069614\\
20876.6028742684	-51.279650694756\\
20973.0132111115	-51.3896838396026\\
21069.8687809795	-51.4994995187288\\
21167.1716400053	-51.6090955298881\\
21264.923853817	-51.7184695897407\\
21363.127497582	-51.8276193321745\\
21461.7846560511	-51.9365423065658\\
21560.8974236026	-52.0452359759759\\
21660.467904287	-52.1536977152847\\
21760.4982118713	-52.2619248092576\\
21860.9904698845	-52.3699144505455\\
21961.9468116618	-52.4776637376148\\
22063.3693803907	-52.5851696726079\\
22165.260329156	-52.6924291591302\\
22267.6218209858	-52.7994389999632\\
22370.4560288971	-52.9061958947017\\
22473.7651359423	-53.0126964373132\\
22577.5513352553	-53.1189371136174\\
22681.8168300982	-53.2249142986853\\
22786.5638339077	-53.3306242541536\\
22891.7945703429	-53.4360631254555\\
22997.5112733313	-53.5412269389635\\
23103.7161871177	-53.6461115990441\\
23210.4115663104	-53.7507128850211\\
23317.5996759301	-53.855026448047\\
23425.2827914574	-53.9590478078801\\
23533.4631988815	-54.0627723495643\\
23642.1431947482	-54.1661953200123\\
23751.3250862094	-54.2693118244876\\
23861.0111910714	-54.3721168229859\\
23971.2038378446	-54.474605126512\\
24081.9053657923	-54.5767713932531\\
24193.1181249812	-54.6786101246441\\
24304.8444763306	-54.7801156613254\\
24417.0867916629	-54.8812821789909\\
24529.8474537537	-54.9821036841248\\
24643.1288563827	-55.0825740096263\\
24756.933404384	-55.1826868103202\\
24871.2635136979	-55.2824355583543\\
24986.1216114214	-55.3818135384798\\
25101.5101358603	-55.4808138432164\\
25217.4315365806	-55.5794293679009\\
25333.8882744609	-55.6776528056184\\
25450.882821744	-55.7754766420167\\
25568.4176620901	-55.8728931500033\\
25686.4952906292	-55.9698943843259\\
25805.1182140139	-56.0664721760373\\
25924.2889504728	-56.1626181268447\\
26044.0100298642	-56.2583236033457\\
26164.2839937291	-56.3535797311521\\
26285.113395346	-56.448377388904\\
26406.5007997846	-56.5427072021757\\
26528.4487839603	-56.6365595372779\\
26650.959936689	-56.7299244949578\\
26774.0368587419	-56.8227919040032\\
26897.6821629011	-56.9151513147525\\
27021.8984740145	-57.0069919925182\\
27146.6884290521	-57.0983029109275\\
27272.0546771616	-57.1890727451882\\
27397.9998797246	-57.2792898652857\\
27524.5267104134	-57.3689423291194\\
27651.6378552475	-57.4580178755873\\
27779.3360126509	-57.5465039176278\\
27907.623893509	-57.63438753523\\
28036.5042212264	-57.7216554684224\\
28165.9797317848	-57.8082941102537\\
28296.0531738006	-57.8942894997775\\
28426.7273085842	-57.979627315057\\
28558.0049101974	-58.0642928662029\\
28689.8887655133	-58.1482710884636\\
28822.3816742749	-58.2315465353825\\
28955.4864491548	-58.3141033720439\\
29089.2059158146	-58.3959253684266\\
29223.5429129655	-58.4769958928862\\
29358.5002924277	-58.557297905789\\
29494.0809191919	-58.636813953323\\
29630.2876714792	-58.7155261615087\\
29767.1234408028	-58.7934162304387\\
29904.5911320291	-58.8704654287729\\
30042.6936634399	-58.9466545885196\\
30181.4339667932	-59.0219641001312\\
30320.8149873869	-59.0963739079498\\
30460.8396841202	-59.1698635060319\\
30601.5110295567	-59.2424119343901\\
30742.8320099879	-59.3139977756846\\
30884.8056254962	-59.3845991524036\\
31027.4348900188	-59.4541937245672\\
31170.7228314113	-59.5227586879957\\
31314.6724915125	-59.5902707731802\\
31459.2869262085	-59.6567062447949\\
31604.5692054981	-59.7220409018919\\
31750.5224135574	-59.786250078818\\
};
\addlegendentry{G(z)};

\end{axis}
\end{tikzpicture}%}
    \end{figure}
    \end{small}
    \vfill
  }

  \frame{
    \frametitle{\insertsection}
    \vfill
    \begin{small}
    \begin{figure}[htb]
      \centering
      \raisebox{-0.5\height}{
        \def\svgwidth{0.5\textwidth}
        % This file was created by matlab2tikz v0.4.7 running on MATLAB 8.3.
% Copyright (c) 2008--2014, Nico Schlömer <nico.schloemer@gmail.com>
% All rights reserved.
% Minimal pgfplots version: 1.3
% 
% The latest updates can be retrieved from
%   http://www.mathworks.com/matlabcentral/fileexchange/22022-matlab2tikz
% where you can also make suggestions and rate matlab2tikz.
% 
%
% defining custom colors
\definecolor{mycolor1}{rgb}{0.66667,0.66667,0.66667}%
%
\begin{tikzpicture}

\begin{axis}[%
width=0.8\textwidth,
height=0.461611624834875\textwidth,
scale only axis,
xmin=-1.2,
xmax=4,
xtick={-1,  0,  1,  2,  3,  4},
xlabel={Eixo Real},
ymin=-1.4,
ymax=1.4,
ytick={-1,  0,  1},
ylabel={Eixo Imaginário}
]
\addplot [color=mycolor1,dotted,line width=1.0pt,forget plot]
  table[row sep=crcr]{1	0\\
0.987688340595138	0.156434465040231\\
0.951056516295154	0.309016994374947\\
0.891006524188368	0.453990499739547\\
0.809016994374947	0.587785252292473\\
0.707106781186548	0.707106781186547\\
0.587785252292473	0.809016994374947\\
0.453990499739547	0.891006524188368\\
0.309016994374947	0.951056516295154\\
0.156434465040231	0.987688340595138\\
6.12323399573677e-17	1\\
-0.156434465040231	0.987688340595138\\
-0.309016994374947	0.951056516295154\\
-0.453990499739547	0.891006524188368\\
-0.587785252292473	0.809016994374947\\
-0.707106781186547	0.707106781186548\\
-0.809016994374947	0.587785252292473\\
-0.891006524188368	0.453990499739547\\
-0.951056516295154	0.309016994374948\\
-0.987688340595138	0.156434465040231\\
-1	1.22464679914735e-16\\
};
\addplot [color=mycolor1,dotted,line width=1.0pt,forget plot]
  table[row sep=crcr]{1	0\\
0.972218045703082	0.153984211042097\\
0.921496791140463	0.299412457456957\\
0.849791018366557	0.432990150609548\\
0.759508516401756	0.551815237548133\\
0.653437051516106	0.653437051516106\\
0.534664304371591	0.735902282058355\\
0.406492952796512	0.797787339572243\\
0.272353130096494	0.83821674479613\\
0.135714483753824	0.856867527364047\\
5.22899666167123e-17	0.853959960588124\\
-0.131496350792703	0.830235283991667\\
-0.255686250369537	0.786921363444393\\
-0.369756251949576	0.725687504552447\\
-0.47122811850853	0.648589862732789\\
-0.558009078759514	0.558009078759514\\
-0.628431083783263	0.456581908289041\\
-0.681278445058817	0.347128705949459\\
-0.715803538350409	0.232578668243255\\
-0.731730559897045	0.115894735197653\\
-0.729247614287671	8.9307075662324e-17\\
};
\addplot [color=mycolor1,dotted,line width=1.0pt,forget plot]
  table[row sep=crcr]{1	0\\
0.956521682769877	0.151498151383791\\
0.891982039736211	0.289822533396298\\
0.809292583610082	0.412355167436434\\
0.711634858347723	0.517032989000692\\
0.602364630186427	0.602364630186426\\
0.484917701774751	0.667431957639199\\
0.362719588349683	0.711877274647591\\
0.239101059900595	0.735877395777214\\
0.117221283062812	0.740106053489981\\
4.44354699422903e-17	0.725686295399261\\
-0.109940132237539	0.694134676438339\\
-0.210320240583588	0.647299141978847\\
-0.299240493845688	0.587292536894097\\
-0.375203754387119	0.516423664031337\\
-0.437127756959533	0.437127756959533\\
-0.484346224770267	0.351898130574443\\
-0.516599582217932	0.263220634338226\\
-0.534016141209622	0.173512362385299\\
-0.537084820159711	0.0850658786478001\\
-0.526620599330303	6.44924231334916e-17\\
};
\addplot [color=mycolor1,dotted,line width=1.0pt,forget plot]
  table[row sep=crcr]{1	0\\
0.940082788364644	0.148894486293864\\
0.86158608093073	0.279946287694513\\
0.768279786681378	0.391458103646313\\
0.663960650859636	0.482395649771645\\
0.552351927561387	0.552351927561387\\
0.437014383712816	0.601498696728173\\
0.321269860940431	0.630527604183869\\
0.208138182971344	0.640583459188477\\
0.100287778328637	0.633192112325829\\
3.73630739569174e-17	0.610185303761559\\
-0.0908532273476425	0.573624701779294\\
-0.170819110593797	0.525727164509449\\
-0.238861981883349	0.468793035010043\\
-0.294350736089166	0.40513903143051\\
-0.337037028702328	0.337037028702328\\
-0.367025664975262	0.266659754473976\\
-0.384738677688982	0.196034147687371\\
-0.390874612629551	0.127002860400749\\
-0.386364521398959	0.0611941284830315\\
-0.372326104926586	4.55967972637346e-17\\
};
\addplot [color=mycolor1,dotted,line width=1.0pt,forget plot]
  table[row sep=crcr]{1	0\\
0.922246029428501	0.146069421212559\\
0.829201462983264	0.269423887468404\\
0.725373165529273	0.369596088217499\\
0.614985835074999	0.446813363302878\\
0.501902475185001	0.501902475185001\\
0.38956496084428	0.536190168955128\\
0.280953750703967	0.551402782692681\\
0.178565395716864	0.549567778706978\\
0.0844061798404073	0.532919645815306\\
3.08495992433718e-17	0.50381218919366\\
-0.0735915548753502	0.464638791061534\\
-0.135739038566523	0.417761804348184\\
-0.186207014322272	0.365451842507638\\
-0.225110018726605	0.309837359892227\\
-0.252864584784672	0.252864584784672\\
-0.270139142845401	0.196267575756099\\
-0.277803443075876	0.141547924204336\\
-0.27687894570066	0.0899634229303457\\
-0.268491413422519	0.0425248622468694\\
-0.253826721980109	3.10848082611005e-17\\
};
\addplot [color=mycolor1,dotted,line width=1.0pt,forget plot]
  table[row sep=crcr]{1	0\\
0.902056570675584	0.142871725087523\\
0.793293726869509	0.257756756757911\\
0.678769666307395	0.345850419328459\\
0.562876391436159	0.408953636388562\\
0.449318435861499	0.449318435861499\\
0.341115747349381	0.469505547439701\\
0.24062663375655	0.472256359314945\\
0.149586755320764	0.460380694230177\\
0.069160331541479	0.436661148025441\\
2.47240337905553e-17	0.403774113610049\\
-0.057687904157153	0.364227092250654\\
-0.104075505986926	0.320311471399148\\
-0.139645446506012	0.274069620358657\\
-0.165124892040398	0.227274916024158\\
-0.18142315316863	0.18142315316863\\
-0.189574186252466	0.137733708524426\\
-0.190684892879101	0.0971588057560207\\
-0.185889806735969	0.0603992595374446\\
-0.176312467991919	0.0279251515651378\\
-0.16303353482158	1.99658496572927e-17\\
};
\addplot [color=mycolor1,dotted,line width=1.0pt,forget plot]
  table[row sep=crcr]{1	0\\
0.877921760602431	0.139049146701748\\
0.751411952540787	0.244148543365328\\
0.625732257688895	0.318826509862098\\
0.505011411193747	0.36691226736026\\
0.392341669548685	0.392341669548685\\
0.289890514921595	0.399000063654162\\
0.199020572856858	0.390599867100522\\
0.120411913496453	0.370589763847805\\
0.0541820486548744	0.34209199176291\\
1.88512313530119e-17	0.307863971328499\\
-0.042808222513439	0.270280479734772\\
-0.075164540154518	0.231332667812908\\
-0.0981551954909503	0.192640417840451\\
-0.112959067758674	0.155474818616317\\
-0.120787864504912	0.120787864504912\\
-0.122837744156894	0.0892468451742257\\
-0.120251626285951	0.0612712639357851\\
-0.114091236431876	0.0370704898842818\\
-0.105317799143806	0.0166807006736036\\
-0.0947802248421549	1.16072298975411e-17\\
};
\addplot [color=mycolor1,dotted,line width=1.0pt,forget plot]
  table[row sep=crcr]{1	0\\
0.84674396664984	0.134111069256013\\
0.698989566160914	0.227115477506975\\
0.561406498364893	0.286050898428465\\
0.437004973478602	0.317502698182059\\
0.327450698344114	0.327450698344114\\
0.233352139914598	0.321181666481713\\
0.154515499860225	0.303253743289057\\
0.0901653834921457	0.27750051639687\\
0.0391311034995635	0.247064063991275\\
1.31311479217367e-17	0.214447919692096\\
-0.0287598398096346	0.181582482159892\\
-0.0487044416812479	0.149896858349689\\
-0.0613430200940395	0.120392455675976\\
-0.06808779312177	0.0937148074575858\\
-0.0702211210616195	0.0702211210616195\\
-0.068876740220244	0.0500418809603846\\
-0.0650321343871793	0.0331355275052096\\
-0.0595094084547913	0.0193357789181307\\
-0.0529823745536039	0.00839158374073751\\
-0.0459879102602678	5.63189470997126e-18\\
};
\addplot [color=mycolor1,dotted,line width=1.0pt,forget plot]
  table[row sep=crcr]{1	0\\
0.801053465278425	0.126874404768154\\
0.625589539649299	0.203266363189334\\
0.475341369738971	0.242198525079547\\
0.350045057714404	0.254322621147605\\
0.24813777530853	0.24813777530853\\
0.167289292234614	0.230253957320141\\
0.104794333468008	0.205670459781722\\
0.0578515468028918	0.178048753195384\\
0.0237523524567972	0.149966451301199\\
7.54043881219821e-18	0.123144711070133\\
-0.0156239119173694	0.0986454975334405\\
-0.0250311775652273	0.0770380431086105\\
-0.0298254673925332	0.0585357756361869\\
-0.0313184856998208	0.0431061974937683\\
-0.0305568546459545	0.0305568546459545\\
-0.0283545570469435	0.0206007915573586\\
-0.0253272293454815	0.012904867916705\\
-0.021925762268643	0.00712411201600239\\
-0.0184675753156993	0.00292497658052826\\
-0.0151646198645466	1.85713031774033e-18\\
};
\addplot [color=mycolor1,dotted,line width=1.0pt,forget plot]
  table[row sep=crcr]{1	0\\
0.714110955679367	0.113104064044896\\
0.497161927717827	0.161537702532642\\
0.336758208677056	0.171586877647116\\
0.221075553032581	0.160620791180475\\
0.139705610200823	0.139705610200823\\
0.0839640345306934	0.115566579097864\\
0.0468885871776745	0.0920240337831981\\
0.0230753615892199	0.0710186604775083\\
0.00844586939409394	0.0533251206797082\\
2.39022368106624e-18	0.0390353150431684\\
-0.00441505277265522	0.0278755461307222\\
-0.00630567510971132	0.0194068724759343\\
-0.00669794782622876	0.0131454627690819\\
-0.0062698831862128	0.00862975386096949\\
-0.0054534525074872	0.0054534525074872\\
-0.00451117782354635	0.00327756254020109\\
-0.00359218715027031	0.00183031077240959\\
-0.00277223578568335	0.000900754009370227\\
-0.00208156288544739	0.000329687172611911\\
-0.00152375582051941	1.86606268828125e-19\\
};
\addplot [color=mycolor1,dotted,line width=1.0pt,forget plot]
  table[row sep=crcr]{1	-0\\
0.987688340595138	-0.156434465040231\\
0.951056516295154	-0.309016994374947\\
0.891006524188368	-0.453990499739547\\
0.809016994374947	-0.587785252292473\\
0.707106781186548	-0.707106781186547\\
0.587785252292473	-0.809016994374947\\
0.453990499739547	-0.891006524188368\\
0.309016994374947	-0.951056516295154\\
0.156434465040231	-0.987688340595138\\
6.12323399573677e-17	-1\\
-0.156434465040231	-0.987688340595138\\
-0.309016994374947	-0.951056516295154\\
-0.453990499739547	-0.891006524188368\\
-0.587785252292473	-0.809016994374947\\
-0.707106781186547	-0.707106781186548\\
-0.809016994374947	-0.587785252292473\\
-0.891006524188368	-0.453990499739547\\
-0.951056516295154	-0.309016994374948\\
-0.987688340595138	-0.156434465040231\\
-1	-1.22464679914735e-16\\
};
\addplot [color=mycolor1,dotted,line width=1.0pt,forget plot]
  table[row sep=crcr]{1	-0\\
0.972218045703082	-0.153984211042097\\
0.921496791140463	-0.299412457456957\\
0.849791018366557	-0.432990150609548\\
0.759508516401756	-0.551815237548133\\
0.653437051516106	-0.653437051516106\\
0.534664304371591	-0.735902282058355\\
0.406492952796512	-0.797787339572243\\
0.272353130096494	-0.83821674479613\\
0.135714483753824	-0.856867527364047\\
5.22899666167123e-17	-0.853959960588124\\
-0.131496350792703	-0.830235283991667\\
-0.255686250369537	-0.786921363444393\\
-0.369756251949576	-0.725687504552447\\
-0.47122811850853	-0.648589862732789\\
-0.558009078759514	-0.558009078759514\\
-0.628431083783263	-0.456581908289041\\
-0.681278445058817	-0.347128705949459\\
-0.715803538350409	-0.232578668243255\\
-0.731730559897045	-0.115894735197653\\
-0.729247614287671	-8.9307075662324e-17\\
};
\addplot [color=mycolor1,dotted,line width=1.0pt,forget plot]
  table[row sep=crcr]{1	-0\\
0.956521682769877	-0.151498151383791\\
0.891982039736211	-0.289822533396298\\
0.809292583610082	-0.412355167436434\\
0.711634858347723	-0.517032989000692\\
0.602364630186427	-0.602364630186426\\
0.484917701774751	-0.667431957639199\\
0.362719588349683	-0.711877274647591\\
0.239101059900595	-0.735877395777214\\
0.117221283062812	-0.740106053489981\\
4.44354699422903e-17	-0.725686295399261\\
-0.109940132237539	-0.694134676438339\\
-0.210320240583588	-0.647299141978847\\
-0.299240493845688	-0.587292536894097\\
-0.375203754387119	-0.516423664031337\\
-0.437127756959533	-0.437127756959533\\
-0.484346224770267	-0.351898130574443\\
-0.516599582217932	-0.263220634338226\\
-0.534016141209622	-0.173512362385299\\
-0.537084820159711	-0.0850658786478001\\
-0.526620599330303	-6.44924231334916e-17\\
};
\addplot [color=mycolor1,dotted,line width=1.0pt,forget plot]
  table[row sep=crcr]{1	-0\\
0.940082788364644	-0.148894486293864\\
0.86158608093073	-0.279946287694513\\
0.768279786681378	-0.391458103646313\\
0.663960650859636	-0.482395649771645\\
0.552351927561387	-0.552351927561387\\
0.437014383712816	-0.601498696728173\\
0.321269860940431	-0.630527604183869\\
0.208138182971344	-0.640583459188477\\
0.100287778328637	-0.633192112325829\\
3.73630739569174e-17	-0.610185303761559\\
-0.0908532273476425	-0.573624701779294\\
-0.170819110593797	-0.525727164509449\\
-0.238861981883349	-0.468793035010043\\
-0.294350736089166	-0.40513903143051\\
-0.337037028702328	-0.337037028702328\\
-0.367025664975262	-0.266659754473976\\
-0.384738677688982	-0.196034147687371\\
-0.390874612629551	-0.127002860400749\\
-0.386364521398959	-0.0611941284830315\\
-0.372326104926586	-4.55967972637346e-17\\
};
\addplot [color=mycolor1,dotted,line width=1.0pt,forget plot]
  table[row sep=crcr]{1	-0\\
0.922246029428501	-0.146069421212559\\
0.829201462983264	-0.269423887468404\\
0.725373165529273	-0.369596088217499\\
0.614985835074999	-0.446813363302878\\
0.501902475185001	-0.501902475185001\\
0.38956496084428	-0.536190168955128\\
0.280953750703967	-0.551402782692681\\
0.178565395716864	-0.549567778706978\\
0.0844061798404073	-0.532919645815306\\
3.08495992433718e-17	-0.50381218919366\\
-0.0735915548753502	-0.464638791061534\\
-0.135739038566523	-0.417761804348184\\
-0.186207014322272	-0.365451842507638\\
-0.225110018726605	-0.309837359892227\\
-0.252864584784672	-0.252864584784672\\
-0.270139142845401	-0.196267575756099\\
-0.277803443075876	-0.141547924204336\\
-0.27687894570066	-0.0899634229303457\\
-0.268491413422519	-0.0425248622468694\\
-0.253826721980109	-3.10848082611005e-17\\
};
\addplot [color=mycolor1,dotted,line width=1.0pt,forget plot]
  table[row sep=crcr]{1	-0\\
0.902056570675584	-0.142871725087523\\
0.793293726869509	-0.257756756757911\\
0.678769666307395	-0.345850419328459\\
0.562876391436159	-0.408953636388562\\
0.449318435861499	-0.449318435861499\\
0.341115747349381	-0.469505547439701\\
0.24062663375655	-0.472256359314945\\
0.149586755320764	-0.460380694230177\\
0.069160331541479	-0.436661148025441\\
2.47240337905553e-17	-0.403774113610049\\
-0.057687904157153	-0.364227092250654\\
-0.104075505986926	-0.320311471399148\\
-0.139645446506012	-0.274069620358657\\
-0.165124892040398	-0.227274916024158\\
-0.18142315316863	-0.18142315316863\\
-0.189574186252466	-0.137733708524426\\
-0.190684892879101	-0.0971588057560207\\
-0.185889806735969	-0.0603992595374446\\
-0.176312467991919	-0.0279251515651378\\
-0.16303353482158	-1.99658496572927e-17\\
};
\addplot [color=mycolor1,dotted,line width=1.0pt,forget plot]
  table[row sep=crcr]{1	-0\\
0.877921760602431	-0.139049146701748\\
0.751411952540787	-0.244148543365328\\
0.625732257688895	-0.318826509862098\\
0.505011411193747	-0.36691226736026\\
0.392341669548685	-0.392341669548685\\
0.289890514921595	-0.399000063654162\\
0.199020572856858	-0.390599867100522\\
0.120411913496453	-0.370589763847805\\
0.0541820486548744	-0.34209199176291\\
1.88512313530119e-17	-0.307863971328499\\
-0.042808222513439	-0.270280479734772\\
-0.075164540154518	-0.231332667812908\\
-0.0981551954909503	-0.192640417840451\\
-0.112959067758674	-0.155474818616317\\
-0.120787864504912	-0.120787864504912\\
-0.122837744156894	-0.0892468451742257\\
-0.120251626285951	-0.0612712639357851\\
-0.114091236431876	-0.0370704898842818\\
-0.105317799143806	-0.0166807006736036\\
-0.0947802248421549	-1.16072298975411e-17\\
};
\addplot [color=mycolor1,dotted,line width=1.0pt,forget plot]
  table[row sep=crcr]{1	-0\\
0.84674396664984	-0.134111069256013\\
0.698989566160914	-0.227115477506975\\
0.561406498364893	-0.286050898428465\\
0.437004973478602	-0.317502698182059\\
0.327450698344114	-0.327450698344114\\
0.233352139914598	-0.321181666481713\\
0.154515499860225	-0.303253743289057\\
0.0901653834921457	-0.27750051639687\\
0.0391311034995635	-0.247064063991275\\
1.31311479217367e-17	-0.214447919692096\\
-0.0287598398096346	-0.181582482159892\\
-0.0487044416812479	-0.149896858349689\\
-0.0613430200940395	-0.120392455675976\\
-0.06808779312177	-0.0937148074575858\\
-0.0702211210616195	-0.0702211210616195\\
-0.068876740220244	-0.0500418809603846\\
-0.0650321343871793	-0.0331355275052096\\
-0.0595094084547913	-0.0193357789181307\\
-0.0529823745536039	-0.00839158374073751\\
-0.0459879102602678	-5.63189470997126e-18\\
};
\addplot [color=mycolor1,dotted,line width=1.0pt,forget plot]
  table[row sep=crcr]{1	-0\\
0.801053465278425	-0.126874404768154\\
0.625589539649299	-0.203266363189334\\
0.475341369738971	-0.242198525079547\\
0.350045057714404	-0.254322621147605\\
0.24813777530853	-0.24813777530853\\
0.167289292234614	-0.230253957320141\\
0.104794333468008	-0.205670459781722\\
0.0578515468028918	-0.178048753195384\\
0.0237523524567972	-0.149966451301199\\
7.54043881219821e-18	-0.123144711070133\\
-0.0156239119173694	-0.0986454975334405\\
-0.0250311775652273	-0.0770380431086105\\
-0.0298254673925332	-0.0585357756361869\\
-0.0313184856998208	-0.0431061974937683\\
-0.0305568546459545	-0.0305568546459545\\
-0.0283545570469435	-0.0206007915573586\\
-0.0253272293454815	-0.012904867916705\\
-0.021925762268643	-0.00712411201600239\\
-0.0184675753156993	-0.00292497658052826\\
-0.0151646198645466	-1.85713031774033e-18\\
};
\addplot [color=mycolor1,dotted,line width=1.0pt,forget plot]
  table[row sep=crcr]{1	-0\\
0.714110955679367	-0.113104064044896\\
0.497161927717827	-0.161537702532642\\
0.336758208677056	-0.171586877647116\\
0.221075553032581	-0.160620791180475\\
0.139705610200823	-0.139705610200823\\
0.0839640345306934	-0.115566579097864\\
0.0468885871776745	-0.0920240337831981\\
0.0230753615892199	-0.0710186604775083\\
0.00844586939409394	-0.0533251206797082\\
2.39022368106624e-18	-0.0390353150431684\\
-0.00441505277265522	-0.0278755461307222\\
-0.00630567510971132	-0.0194068724759343\\
-0.00669794782622876	-0.0131454627690819\\
-0.0062698831862128	-0.00862975386096949\\
-0.0054534525074872	-0.0054534525074872\\
-0.00451117782354635	-0.00327756254020109\\
-0.00359218715027031	-0.00183031077240959\\
-0.00277223578568335	-0.000900754009370227\\
-0.00208156288544739	-0.000329687172611911\\
-0.00152375582051941	-1.86606268828125e-19\\
};
\addplot [color=mycolor1,dotted,line width=1.0pt,forget plot]
  table[row sep=crcr]{1	0\\
1	0\\
};
\addplot [color=mycolor1,dotted,line width=1.0pt,forget plot]
  table[row sep=crcr]{0.951056516295154	0.309016994374947\\
0.906577591518048	0.290693092532446\\
0.867277205719189	0.267124444629623\\
0.833330533437629	0.239553777474139\\
0.804679893747523	0.20903820245934\\
0.781122098933164	0.176433510213537\\
0.762382187756933	0.142402376999381\\
0.748171074803624	0.107437628337747\\
0.738227708485823	0.0718935522249012\\
0.732347931289196	0.0360204899377399\\
0.730402691048646	2.81010850277162e-17\\
};
\addplot [color=mycolor1,dotted,line width=1.0pt,forget plot]
  table[row sep=crcr]{0.809016994374947	0.587785252292473\\
0.737380455396588	0.527071687397996\\
0.6808142826414	0.463341883835339\\
0.637053765657314	0.399254954339046\\
0.603812761314093	0.336417677088311\\
0.579022949915482	0.275632227640288\\
0.560948163233974	0.217130071437151\\
0.54821711318997	0.160763451735608\\
0.539811466724715	0.106147624627789\\
0.535036016768209	0.0527590625798542\\
0.533488091091103	4.10502162512614e-17\\
};
\addplot [color=mycolor1,dotted,line width=1.0pt,forget plot]
  table[row sep=crcr]{0.587785252292473	0.809016994374947\\
0.515276498489902	0.692182785870847\\
0.46668476526979	0.583707991451876\\
0.435233321876477	0.485319980094308\\
0.415551842123536	0.396928474901319\\
0.403656860517901	0.317461475735791\\
0.396737049613855	0.245416450708029\\
0.392888142823216	0.179177710929467\\
0.39087245229983	0.11717008156476\\
0.389932112762627	0.0579102497954412\\
0.389661137375347	4.49747826270753e-17\\
};
\addplot [color=mycolor1,dotted,line width=1.0pt,forget plot]
  table[row sep=crcr]{0.309016994374947	0.951056516295154\\
0.265925372344309	0.777304721760365\\
0.248822386132444	0.630899544522142\\
0.24643298177389	0.508693744238057\\
0.251412997268255	0.406266573115132\\
0.25929445161488	0.31919477108011\\
0.267517373913267	0.243597429511063\\
0.274697115780397	0.1762665508339\\
0.280129101393366	0.114599409879343\\
0.283480020554887	0.0564559973822998\\
0.284609543336029	4.37996030135249e-17\\
};
\addplot [color=mycolor1,dotted,line width=1.0pt,forget plot]
  table[row sep=crcr]{6.12323399573677e-17	1\\
0.0151248701748503	0.781989711398728\\
0.0472692933177679	0.613631335769719\\
0.0835006201485716	0.482943980920436\\
0.117381749765263	0.379469463911326\\
0.146223972383669	0.295078319831894\\
0.169261627793672	0.223809451176491\\
0.186582776182006	0.161390341419992\\
0.198560105942714	0.104739935929793\\
0.205572433929556	0.0515565221197431\\
0.207879576350762	3.99891848849262e-17\\
};
\addplot [color=mycolor1,dotted,line width=1.0pt,forget plot]
  table[row sep=crcr]{-0.309016994374947	0.951056516295154\\
-0.213607139159912	0.713331044437031\\
-0.122920349149865	0.544815253953639\\
-0.0461074386071066	0.422454854218942\\
0.0151311193047769	0.329888717873058\\
0.0621570724668225	0.256291005261778\\
0.0971710522581798	0.194731597161215\\
0.122256440677152	0.140793596164787\\
0.13905244595299	0.0915971142347777\\
0.148693455532156	0.0451621321066958\\
0.151835801980649	3.50498499033503e-17\\
};
\addplot [color=mycolor1,dotted,line width=1.0pt,forget plot]
  table[row sep=crcr]{-0.587785252292473	0.809016994374948\\
-0.401011853057454	0.584595820351374\\
-0.252139489074835	0.439670821081765\\
-0.139623392550337	0.340999317931598\\
-0.0567836371213516	0.268617800427267\\
0.0033339412143336	0.21116115844769\\
0.0463512370945915	0.162297289886265\\
0.0763422025660035	0.118472638203443\\
0.0960671266193367	0.0776165020305757\\
0.1072685824301	0.0384304051397518\\
0.110901278364195	2.98672554719679e-17\\
};
\addplot [color=mycolor1,dotted,line width=1.0pt,forget plot]
  table[row sep=crcr]{-0.809016994374948	0.587785252292473\\
-0.533486326814499	0.413410095118228\\
-0.336121655437603	0.313963860155743\\
-0.198040110920963	0.250717832404618\\
-0.101844033235305	0.204281393673556\\
-0.0346516892466227	0.165530866251108\\
0.0122258376810533	0.130333089269644\\
0.0443886084751889	0.0968398262452624\\
0.065339288702761	0.0642052594194206\\
0.0771736424133687	0.032008294596762\\
0.0810025921579431	2.49315700239574e-17\\
};
\addplot [color=mycolor1,dotted,line width=1.0pt,forget plot]
  table[row sep=crcr]{-0.951056516295154	0.309016994374948\\
-0.60382220830535	0.219707538176049\\
-0.375378071887508	0.182507388795924\\
-0.225093275108467	0.161489568357552\\
-0.124654461172013	0.143091836517116\\
-0.0562723920172722	0.12318609851568\\
-0.00923898083522721	0.101104614081101\\
0.0228060316514889	0.0772217637055013\\
0.0436193292019494	0.0520955750986265\\
0.0553650029180358	0.0262210407420432\\
0.0591645112940776	2.0486346567262e-17\\
};
\addplot [color=mycolor1,dotted,line width=1.0pt,forget plot]
  table[row sep=crcr]{-1	1.22464679914735e-16\\
-0.61127914703566	0.0236549857259488\\
-0.374309030147768	0.0580118391989452\\
-0.226262535142082	0.0806522438077526\\
-0.130218598863194	0.0890855793127953\\
-0.0656897647351535	0.0862950481802363\\
-0.0214411717925588	0.0757647040434823\\
0.00876629006412298	0.0602253159022079\\
0.0284556614934045	0.0415943455493057\\
0.039601950618638	0.0211971994741971\\
0.0432139182637723	1.66258696249815e-17\\
};
\addplot [color=mycolor1,dotted,line width=1.0pt,forget plot]
  table[row sep=crcr]{1	-0\\
1	-0\\
};
\addplot [color=mycolor1,dotted,line width=1.0pt,forget plot]
  table[row sep=crcr]{0.951056516295154	-0.309016994374947\\
0.906577591518048	-0.290693092532446\\
0.867277205719189	-0.267124444629623\\
0.833330533437629	-0.239553777474139\\
0.804679893747523	-0.20903820245934\\
0.781122098933164	-0.176433510213537\\
0.762382187756933	-0.142402376999381\\
0.748171074803624	-0.107437628337747\\
0.738227708485823	-0.0718935522249012\\
0.732347931289196	-0.0360204899377399\\
0.730402691048646	-2.81010850277162e-17\\
};
\addplot [color=mycolor1,dotted,line width=1.0pt,forget plot]
  table[row sep=crcr]{0.809016994374947	-0.587785252292473\\
0.737380455396588	-0.527071687397996\\
0.6808142826414	-0.463341883835339\\
0.637053765657314	-0.399254954339046\\
0.603812761314093	-0.336417677088311\\
0.579022949915482	-0.275632227640288\\
0.560948163233974	-0.217130071437151\\
0.54821711318997	-0.160763451735608\\
0.539811466724715	-0.106147624627789\\
0.535036016768209	-0.0527590625798542\\
0.533488091091103	-4.10502162512614e-17\\
};
\addplot [color=mycolor1,dotted,line width=1.0pt,forget plot]
  table[row sep=crcr]{0.587785252292473	-0.809016994374947\\
0.515276498489902	-0.692182785870847\\
0.46668476526979	-0.583707991451876\\
0.435233321876477	-0.485319980094308\\
0.415551842123536	-0.396928474901319\\
0.403656860517901	-0.317461475735791\\
0.396737049613855	-0.245416450708029\\
0.392888142823216	-0.179177710929467\\
0.39087245229983	-0.11717008156476\\
0.389932112762627	-0.0579102497954412\\
0.389661137375347	-4.49747826270753e-17\\
};
\addplot [color=mycolor1,dotted,line width=1.0pt,forget plot]
  table[row sep=crcr]{0.309016994374947	-0.951056516295154\\
0.265925372344309	-0.777304721760365\\
0.248822386132444	-0.630899544522142\\
0.24643298177389	-0.508693744238057\\
0.251412997268255	-0.406266573115132\\
0.25929445161488	-0.31919477108011\\
0.267517373913267	-0.243597429511063\\
0.274697115780397	-0.1762665508339\\
0.280129101393366	-0.114599409879343\\
0.283480020554887	-0.0564559973822998\\
0.284609543336029	-4.37996030135249e-17\\
};
\addplot [color=mycolor1,dotted,line width=1.0pt,forget plot]
  table[row sep=crcr]{6.12323399573677e-17	-1\\
0.0151248701748503	-0.781989711398728\\
0.0472692933177679	-0.613631335769719\\
0.0835006201485716	-0.482943980920436\\
0.117381749765263	-0.379469463911326\\
0.146223972383669	-0.295078319831894\\
0.169261627793672	-0.223809451176491\\
0.186582776182006	-0.161390341419992\\
0.198560105942714	-0.104739935929793\\
0.205572433929556	-0.0515565221197431\\
0.207879576350762	-3.99891848849262e-17\\
};
\addplot [color=mycolor1,dotted,line width=1.0pt,forget plot]
  table[row sep=crcr]{-0.309016994374947	-0.951056516295154\\
-0.213607139159912	-0.713331044437031\\
-0.122920349149865	-0.544815253953639\\
-0.0461074386071066	-0.422454854218942\\
0.0151311193047769	-0.329888717873058\\
0.0621570724668225	-0.256291005261778\\
0.0971710522581798	-0.194731597161215\\
0.122256440677152	-0.140793596164787\\
0.13905244595299	-0.0915971142347777\\
0.148693455532156	-0.0451621321066958\\
0.151835801980649	-3.50498499033503e-17\\
};
\addplot [color=mycolor1,dotted,line width=1.0pt,forget plot]
  table[row sep=crcr]{-0.587785252292473	-0.809016994374948\\
-0.401011853057454	-0.584595820351374\\
-0.252139489074835	-0.439670821081765\\
-0.139623392550337	-0.340999317931598\\
-0.0567836371213516	-0.268617800427267\\
0.0033339412143336	-0.21116115844769\\
0.0463512370945915	-0.162297289886265\\
0.0763422025660035	-0.118472638203443\\
0.0960671266193367	-0.0776165020305757\\
0.1072685824301	-0.0384304051397518\\
0.110901278364195	-2.98672554719679e-17\\
};
\addplot [color=mycolor1,dotted,line width=1.0pt,forget plot]
  table[row sep=crcr]{-0.809016994374948	-0.587785252292473\\
-0.533486326814499	-0.413410095118228\\
-0.336121655437603	-0.313963860155743\\
-0.198040110920963	-0.250717832404618\\
-0.101844033235305	-0.204281393673556\\
-0.0346516892466227	-0.165530866251108\\
0.0122258376810533	-0.130333089269644\\
0.0443886084751889	-0.0968398262452624\\
0.065339288702761	-0.0642052594194206\\
0.0771736424133687	-0.032008294596762\\
0.0810025921579431	-2.49315700239574e-17\\
};
\addplot [color=mycolor1,dotted,line width=1.0pt,forget plot]
  table[row sep=crcr]{-0.951056516295154	-0.309016994374948\\
-0.60382220830535	-0.219707538176049\\
-0.375378071887508	-0.182507388795924\\
-0.225093275108467	-0.161489568357552\\
-0.124654461172013	-0.143091836517116\\
-0.0562723920172722	-0.12318609851568\\
-0.00923898083522721	-0.101104614081101\\
0.0228060316514889	-0.0772217637055013\\
0.0436193292019494	-0.0520955750986265\\
0.0553650029180358	-0.0262210407420432\\
0.0591645112940776	-2.0486346567262e-17\\
};
\addplot [color=mycolor1,dotted,line width=1.0pt,forget plot]
  table[row sep=crcr]{-1	-1.22464679914735e-16\\
-0.61127914703566	-0.0236549857259488\\
-0.374309030147768	-0.0580118391989452\\
-0.226262535142082	-0.0806522438077526\\
-0.130218598863194	-0.0890855793127953\\
-0.0656897647351535	-0.0862950481802363\\
-0.0214411717925588	-0.0757647040434823\\
0.00876629006412298	-0.0602253159022079\\
0.0284556614934045	-0.0415943455493057\\
0.039601950618638	-0.0211971994741971\\
0.0432139182637723	-1.66258696249815e-17\\
};
\addplot [color=black!50!mycolor1,line width=1.5pt,mark size=5.0pt,only marks,mark=x,mark options={solid},forget plot]
  table[row sep=crcr]{0	0\\
0.809915206392298	0.509761287368869\\
0.809915206392298	-0.509761287368869\\
0.474723347707754	0\\
};
\addplot [color=black,line width=1.5pt,mark size=4.0pt,only marks,mark=o,mark options={solid},forget plot]
  table[row sep=crcr]{3.35287627897953	0\\
0.194909709644537	0\\
};
\end{axis}
\end{tikzpicture}%}
    \end{figure}
    \end{small}
    \vfill
  }

  \frame{
    \frametitle{\insertsection}
    \vfill
    Algoritmo de adaptação paramétrica \cite{ref:TAO}:
    \begin{equation}
      \begin{split}
        \theta(k+1) &= \theta(k) - \mathrm{sgn}(k_p) \gamma_d \frac{\zeta(k) \cdot e_a(k)}{{\bar{m}}^2(k)} \\
        \rho(k+1) &= \rho(k) - \gamma \frac{e_2(k) \cdot e_a(k)}{{\bar{m}}^2(k)} \\
        {\bar{m}}^2 &= m^2(k) + \zeta^T(k) \zeta(k) + {e_2}^2(k) \\
        m^2(k+1) &=  \delta_0(m^2(k) - 1) + u^2(k) + y^2(k) + 1
      \end{split}
      \label{eq:equacoes_algoritmo_adaptativo}
    \end{equation}
    \vfill
  }

  \frame{
    \frametitle{\insertsection}
    \vfill
    Com:
    \begin{equation}
      e_1 = y(k) - y_m(k)
    \end{equation}
    \begin{equation}
      e_a = e_1 + \rho e_2
    \end{equation}
    \begin{equation}
      \zeta(k) = W_m(z) \omega(k)
    \end{equation}
    \begin{equation}
      e_2 = -W_m(z)u(k)+\theta^T(k)\zeta(k)
    \end{equation}
    \vfill
    Com $\gamma$ e $\gamma_d$ ganhos das leis de adaptação e $\delta_0$ é uma constante de projeto
    \vfill
  }

  \frame{
    \frametitle{\insertsection}
    \vfill
    \begin{small}
    \begin{figure}[htb]
      \centering
      \raisebox{-0.5\height}{
        \def\svgwidth{\textwidth}
        \input{./img/diagrama_blocos_adaptativo.pdf_tex}}
    \end{figure}
    \end{small}
    \vfill
  }

%FIM-------------------------------------------------------------------
