%SEÇÃO------------------------------------------------------------------

\section{Modelagem}

  \frame{
    \frametitle{\insertsection}
    \vfill
    Modelagem clássica de sistemas de potência:
    \vfill
    \begin{itemize}
      \item Teoria das Componentes Simétricas \cite{ref:FORTESCUE}
      \vfill
      \item Transformação de Clarke \cite{ref:CLARKE}
        \begin{equation}
          \mathbf{T}_{\alpha \beta 0} = \frac{2}{3} \left[
          \begin{array}{ccc}
            1 & -\frac{1}{2} & -\frac{1}{2} \\[0.3em]
            0 & \phantom{-}\frac{\sqrt{3}}{2} & -\frac{\sqrt{3}}{2} \\[0.3em]
            \frac{1}{2} &  \phantom{-}\frac{1}{2} & \phantom{-}\frac{1}{2}
          \end{array}
          \right]
          \label{eq:alpha_beta_0}
        \end{equation}
        \begin{equation}
          \mathbf{T}_{\alpha \beta 0} \mathbf{x}_{abc} = \mathbf{x}_{\alpha \beta 0}
          \iff
          \mathbf{T}_{\alpha \beta 0}^{-1} \mathbf{x}_{\alpha \beta 0} = \mathbf{x}_{abc}
        \end{equation}
    \end{itemize}
    \vfill
  }

  \frame{
    \frametitle{\insertsection}
    \vfill
    \begin{small}
    \begin{figure}[htb]
      \centering
      \raisebox{-0.5\height}{
        \def\svgwidth{\textwidth}
        \input{./img/LCL_geral.pdf_tex}}
    \end{figure}
    \end{small}
    \vfill
  }

  \frame{
    \frametitle{\insertsection}
    \vfill
    Modelo em função de transferência \cite{ref:XU}:
    \vfill
    \begin{equation}
      \begin{split}
        Z_i & = r_1 + L_1 s \text{,} \\
        Z_C & = \frac{1}{s C} \text{,} \\
        Z_g & = r_2 + r_g + \left( L_2 + L_g \right) s
      \end{split}
    \end{equation}
    \vfill
  }

  \frame{
    \frametitle{\insertsection}
    \vfill
    Modelo em função de transferência:
    \vfill
    \begin{equation}
      \frac{V_C}{U_c} = \frac{Z_C Z_g}{Z_i \left( Z_C + Z_g \right) + Z_C Z_g}
      \label{eq:vc_uc}
    \end{equation}
    %
    \begin{equation}
      \frac{I_C}{U_c} = \frac{Z_g}{Z_i \left( Z_C + Z_g \right) + Z_C Z_g}
      \label{eq:ic_uc}
    \end{equation}
    %
    \begin{equation}
      \frac{I_2}{U_c} = \frac{Z_C}{Z_i \left( Z_C + Z_g \right) + Z_C Z_g}
      \label{eq:i2_uc}
    \end{equation}
    \vfill
  }

  \frame{
    \frametitle{\insertsection}
    \vfill
    Discretizando (\ref{eq:i2_uc}):
    \vfill
    \begin{footnotesize}
      \begin{equation}
        G_d (z) = \frac{I_2}{U_c} = K_1 \frac{1}{z \left( z-1 \right)}
        - \frac{K_1 \sen \left( \omega_n T_s \right) }{\omega_n T_s}
        \frac{z-1}{z \left( z^2 - 2 \cos\left( \omega_n T_s \right) z +1 \right) }
      \end{equation}
    \end{footnotesize}
    \vfill
    Onde $T_s$ é o período de amostragem e
    \vfill
    \begin{equation}
      K_1 = \frac{T_s}{L1 + L2 + Lg}
    \end{equation}
    %
    \begin{equation}
      \omega_n = \sqrt{\frac{ L_1 + L_2 + L_g }{ L_1 C \left( L_2 + L_g \right)}}
    \end{equation}
    \vfill
  }

  \frame{
    \frametitle{\insertsection}
    \vfill
    Para a tensão do capacitor $V_C$ como variável intermediária, discretizando (\ref{eq:vc_uc}):
    \vfill
    \begin{equation}
      G_{id_{vc}}(z) = \frac{V_C}{U_c} = \frac{2 \sen^2 \left( \omega_n \frac{T_s}{2} \right)}{L_1 C \omega_n^2}
        \frac{z+1}{z \left( z^2 - \cos \left( \omega_n T_s \right) z + 1 \right)}
    \end{equation}
    %
    \begin{equation}
      G_{od_{vc}}(z) = \frac{I_2}{V_C} = \frac{G_d}{G_{id_{vc}}}
      \label{eq:god_i2_vc}
    \end{equation}
  }

  \frame{
    \frametitle{\insertsection}
    \vfill
    Para a corrente do capacitor $I_C$ como variável intermediária, discretizando (\ref{eq:ic_uc}):
    \vfill
    \begin{equation}
      G_{id_{ic}}(z) = \frac{I_C}{U_c} = \frac{\sen \left( \omega_n T_s \right)}{\omega_n L_1}
        \frac{z-1}{z \left( z^2 - \cos \left( \omega_n T_s \right) z + 1 \right)}
    \end{equation}
    %
    \begin{equation}
      G_{od_{ic}}(z) = \frac{I_2}{I_C} = \frac{G_d}{G_{id_{ic}}}
      \label{eq:god_i2_ic}
    \end{equation}
  }

  \frame{
    \frametitle{\insertsection}
    \vfill
    \begin{small}
    \begin{figure}[htb]
      \centering
      \raisebox{-0.5\height}{
        \def\svgwidth{\textwidth}
        \input{./img/multiloop_geral.pdf_tex}}
    \end{figure}
    \end{small}
    \vfill
  }

  \frame{
    \frametitle{\insertsection}
    \vfill
    Efeitos da discretização:
    \vfill
    \begin{itemize}
      \item Planta em tempo contínuo
      \vfill
      \item Planta em tempo discreto \cite{ref:ASTROM}
    \end{itemize}
    \vfill
  }

  \frame{
    \frametitle{\insertsection}
    \vfill
    \begin{small}
    \begin{figure}[htb]
      \centering
      \def\svgwidth{0.5\columnwidth}
      \raisebox{-0.5\height}{
        % This file was created by matlab2tikz v0.4.7 running on MATLAB 7.14.
% Copyright (c) 2008--2014, Nico Schlömer <nico.schloemer@gmail.com>
% All rights reserved.
% Minimal pgfplots version: 1.3
% 
% The latest updates can be retrieved from
%   http://www.mathworks.com/matlabcentral/fileexchange/22022-matlab2tikz
% where you can also make suggestions and rate matlab2tikz.
% 
%
% defining custom colors
\definecolor{mycolor1}{rgb}{0.33333,0.33333,0.33333}%
%
\begin{tikzpicture}

\begin{axis}[%
width=0.8\textwidth,
height=0.461611624834875\textwidth,
scale only axis,
xmin=-1,
xmax=1,
xtick={-1,-0.5,0,0.5,1},
xticklabels={{-1},{-0,5},{0},{0,5},{1}},
xlabel={Eixo Real},
xmajorgrids,
ymin=-6000,
ymax=6000,
ytick={-6000,-3000,0,3000,6000},
ylabel={Eixo Imaginário},
ymajorgrids
]
\addplot [color=mycolor1,line width=1.5pt,mark size=5.0pt,only marks,mark=x,mark options={solid},forget plot]
  table[row sep=crcr]{0	0\\
1.13686837721616e-13	5000\\
1.13686837721616e-13	-5000\\
};
\end{axis}
\end{tikzpicture}%}
    \end{figure}
    \end{small}
    \vfill
  }

  \frame{
    \frametitle{\insertsection}
    \vfill
    \begin{small}
    \begin{figure}[htb]
      \centering
      \def\svgwidth{0.6\columnwidth}
      \raisebox{-0.5\height}{
        \input{./img/pzmap_planta_discreta.tex}}
    \end{figure}
    \end{small}
    \vfill
  }


%FIM-------------------------------------------------------------------
