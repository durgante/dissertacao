%SEÇÃO------------------------------------------------------------------

\section{Modelagem}

  \frame{
    \frametitle{\insertsection}
    \vfill
    Modelagem clássica de sistemas de potência:
    \vfill
    \begin{itemize}
      \item Teoria das Componentes Simétricas \cite{ref:FORTESCUE}
      \vfill
      \item Transformação de Clarke \cite{ref:CLARKE}
        \begin{equation}
          \mathbf{T}_{\alpha \beta 0} = \frac{2}{3} \left[
          \begin{array}{ccc}
            1 & -\frac{1}{2} & -\frac{1}{2} \\[0.3em]
            0 & \phantom{-}\frac{\sqrt{3}}{2} & -\frac{\sqrt{3}}{2} \\[0.3em]
            \frac{1}{2} &  \phantom{-}\frac{1}{2} & \phantom{-}\frac{1}{2}
          \end{array}
          \right]
          \label{eq:alpha_beta_0}
        \end{equation}
        \begin{equation}
          \mathbf{T}_{\alpha \beta 0} \mathbf{x}_{abc} = \mathbf{x}_{\alpha \beta 0}
          \iff
          \mathbf{T}_{\alpha \beta 0}^{-1} \mathbf{x}_{\alpha \beta 0} = \mathbf{x}_{abc}
        \end{equation}
    \end{itemize}
    \vfill
  }

  \frame{
    \frametitle{\insertsection}
    \vfill
    \begin{small}
    \begin{figure}[htb]
      \centering
      \raisebox{-0.5\height}{
        \def\svgwidth{\textwidth}
        \input{./img/LCL_geral.pdf_tex}}
    \end{figure}
    \end{small}
    \vfill
  }

  \frame{
    \frametitle{\insertsection}
    \vfill
    Modelo em função de transferência \cite{ref:XU}:
    \vfill
    \begin{equation}
      \begin{split}
        Z_i & = r_1 + L_1 s \text{,} \\
        Z_C & = \frac{1}{s C} \text{,} \\
        Z_g & = r_2 + r_g + \left( L_2 + L_g \right) s
      \end{split}
    \end{equation}
    \vfill
  }

  \frame{
    \frametitle{\insertsection}
    \vfill
    Modelo em função de transferência:
    \vfill
    \begin{equation}
      \frac{V_C}{U_c} = \frac{Z_C Z_g}{Z_i \left( Z_C + Z_g \right) + Z_C Z_g}
      \label{eq:vc_uc}
    \end{equation}
    %
    \begin{equation}
      \frac{I_C}{U_c} = \frac{Z_g}{Z_i \left( Z_C + Z_g \right) + Z_C Z_g}
      \label{eq:ic_uc}
    \end{equation}
    %
    \begin{equation}
      \frac{I_2}{U_c} = \frac{Z_C}{Z_i \left( Z_C + Z_g \right) + Z_C Z_g}
      \label{eq:i2_uc}
    \end{equation}
    \vfill
  }

  \frame{
    \frametitle{\insertsection}
    \vfill
    Discretizando (\ref{eq:i2_uc}):
    \vfill
    \begin{footnotesize}
      \begin{equation}
        G_d (z) = \frac{I_2}{U_c} = K_1 \frac{1}{z \left( z-1 \right)}
        - \frac{K_1 \sen \left( \omega_n T_s \right) }{\omega_n T_s}
        \frac{z-1}{z \left( z^2 - 2 \cos\left( \omega_n T_s \right) z +1 \right) }
      \end{equation}
    \end{footnotesize}
    \vfill
    Onde $T_s$ é o período de amostragem e
    \vfill
    \begin{equation}
      K_1 = \frac{T_s}{L1 + L2 + Lg}
    \end{equation}
    %
    \begin{equation}
      \omega_n = \sqrt{\frac{ L_1 + L_2 + L_g }{ L_1 C \left( L_2 + L_g \right)}}
    \end{equation}
    \vfill
  }

  \frame{
    \frametitle{\insertsection}
    \vfill
    Para a tensão do capacitor $V_C$ como variável intermediária, discretizando (\ref{eq:vc_uc}):
    \vfill
    \begin{equation}
      G_{id_{vc}}(z) = \frac{V_C}{U_c} = \frac{2 \sen^2 \left( \omega_n \frac{T_s}{2} \right)}{L_1 C \omega_n^2}
        \frac{z+1}{z \left( z^2 - \cos \left( \omega_n T_s \right) z + 1 \right)}
    \end{equation}
    %
    \begin{equation}
      G_{od_{vc}}(z) = \frac{I_2}{V_C} = \frac{G_d}{G_{id_{vc}}}
      \label{eq:god_i2_vc}
    \end{equation}
  }

  \frame{
    \frametitle{\insertsection}
    \vfill
    Para a corrente do capacitor $I_C$ como variável intermediária, discretizando (\ref{eq:ic_uc}):
    \vfill
    \begin{equation}
      G_{id_{ic}}(z) = \frac{I_C}{U_c} = \frac{\sen \left( \omega_n T_s \right)}{\omega_n L_1}
        \frac{z-1}{z \left( z^2 - \cos \left( \omega_n T_s \right) z + 1 \right)}
    \end{equation}
    %
    \begin{equation}
      G_{od_{ic}}(z) = \frac{I_2}{I_C} = \frac{G_d}{G_{id_{ic}}}
      \label{eq:god_i2_ic}
    \end{equation}
  }

  \frame{
    \frametitle{\insertsection}
    \vfill
    \begin{small}
    \begin{figure}[htb]
      \centering
      \raisebox{-0.5\height}{
        \def\svgwidth{\textwidth}
        \input{./img/multiloop_geral.pdf_tex}}
    \end{figure}
    \end{small}
    \vfill
  }

  \frame{
    \frametitle{\insertsection}
    \vfill
    Efeitos da discretização:
    \vfill
    \begin{itemize}
      \item Planta em tempo contínuo
      \vfill
      \item Planta em tempo discreto \cite{ref:ASTROM}
    \end{itemize}
    \vfill
  }

  \frame{
    \frametitle{\insertsection}
    \vfill
    \begin{small}
    \begin{figure}[htb]
      \centering
      \def\svgwidth{0.5\columnwidth}
      \raisebox{-0.5\height}{
        % This file was created by matlab2tikz v0.4.7 running on MATLAB 7.14.
% Copyright (c) 2008--2014, Nico Schlömer <nico.schloemer@gmail.com>
% All rights reserved.
% Minimal pgfplots version: 1.3
% 
% The latest updates can be retrieved from
%   http://www.mathworks.com/matlabcentral/fileexchange/22022-matlab2tikz
% where you can also make suggestions and rate matlab2tikz.
% 
%
% defining custom colors
\definecolor{mycolor1}{rgb}{0.33333,0.33333,0.33333}%
%
\begin{tikzpicture}

\begin{axis}[%
width=0.8\textwidth,
height=0.461611624834875\textwidth,
scale only axis,
xmin=-1,
xmax=1,
xtick={  -1, -0.5,    0,  0.5,    1},
xlabel={Eixo Real},
xmajorgrids,
ymin=-6000,
ymax=6000,
ytick={-6000, -3000,     0,  3000,  6000},
ylabel={Eixo Imaginário},
ymajorgrids
]
\addplot [color=mycolor1,line width=1.5pt,mark size=5.0pt,only marks,mark=x,mark options={solid},forget plot]
  table[row sep=crcr]{0	0\\
1.13686837721616e-13	5000\\
1.13686837721616e-13	-5000\\
};
\end{axis}
\end{tikzpicture}%}
    \end{figure}
    \end{small}
    \vfill
  }

  \frame{
    \frametitle{\insertsection}
    \vfill
    \begin{small}
    \begin{figure}[htb]
      \centering
      \def\svgwidth{0.6\columnwidth}
      \raisebox{-0.5\height}{
        % This file was created by matlab2tikz v0.4.7 running on MATLAB 7.14.
% Copyright (c) 2008--2014, Nico Schlömer <nico.schloemer@gmail.com>
% All rights reserved.
% Minimal pgfplots version: 1.3
% 
% The latest updates can be retrieved from
%   http://www.mathworks.com/matlabcentral/fileexchange/22022-matlab2tikz
% where you can also make suggestions and rate matlab2tikz.
% 
%
% defining custom colors
\definecolor{mycolor1}{rgb}{0.66667,0.66667,0.66667}%
%
\begin{tikzpicture}

\begin{axis}[%
width=0.8\textwidth,
height=0.461611624834875\textwidth,
scale only axis,
xmin=-4,
xmax=1.2,
xtick={-4, -3, -2, -1,  0,  1},
xlabel={Eixo Real},
ymin=-1.4,
ymax=1.4,
ytick={-1,  0,  1},
ylabel={Eixo Imaginário}
]
\addplot [color=mycolor1,dotted,line width=1.0pt,forget plot]
  table[row sep=crcr]{1	0\\
0.987688340595138	0.156434465040231\\
0.951056516295154	0.309016994374947\\
0.891006524188368	0.453990499739547\\
0.809016994374947	0.587785252292473\\
0.707106781186548	0.707106781186547\\
0.587785252292473	0.809016994374947\\
0.453990499739547	0.891006524188368\\
0.309016994374947	0.951056516295154\\
0.156434465040231	0.987688340595138\\
6.12323399573677e-17	1\\
-0.156434465040231	0.987688340595138\\
-0.309016994374947	0.951056516295154\\
-0.453990499739547	0.891006524188368\\
-0.587785252292473	0.809016994374947\\
-0.707106781186547	0.707106781186548\\
-0.809016994374947	0.587785252292473\\
-0.891006524188368	0.453990499739547\\
-0.951056516295154	0.309016994374948\\
-0.987688340595138	0.156434465040231\\
-1	1.22464679914735e-16\\
};
\addplot [color=mycolor1,dotted,line width=1.0pt,forget plot]
  table[row sep=crcr]{1	0\\
0.972218045703082	0.153984211042097\\
0.921496791140463	0.299412457456957\\
0.849791018366557	0.432990150609548\\
0.759508516401756	0.551815237548133\\
0.653437051516106	0.653437051516106\\
0.534664304371591	0.735902282058355\\
0.406492952796512	0.797787339572243\\
0.272353130096494	0.83821674479613\\
0.135714483753824	0.856867527364047\\
5.22899666167123e-17	0.853959960588124\\
-0.131496350792703	0.830235283991667\\
-0.255686250369537	0.786921363444393\\
-0.369756251949576	0.725687504552447\\
-0.47122811850853	0.648589862732789\\
-0.558009078759514	0.558009078759514\\
-0.628431083783263	0.456581908289041\\
-0.681278445058817	0.347128705949459\\
-0.715803538350409	0.232578668243255\\
-0.731730559897045	0.115894735197653\\
-0.729247614287671	8.9307075662324e-17\\
};
\addplot [color=mycolor1,dotted,line width=1.0pt,forget plot]
  table[row sep=crcr]{1	0\\
0.956521682769877	0.151498151383791\\
0.891982039736211	0.289822533396298\\
0.809292583610082	0.412355167436434\\
0.711634858347723	0.517032989000692\\
0.602364630186427	0.602364630186426\\
0.484917701774751	0.667431957639199\\
0.362719588349683	0.711877274647591\\
0.239101059900595	0.735877395777214\\
0.117221283062812	0.740106053489981\\
4.44354699422903e-17	0.725686295399261\\
-0.109940132237539	0.694134676438339\\
-0.210320240583588	0.647299141978847\\
-0.299240493845688	0.587292536894097\\
-0.375203754387119	0.516423664031337\\
-0.437127756959533	0.437127756959533\\
-0.484346224770267	0.351898130574443\\
-0.516599582217932	0.263220634338226\\
-0.534016141209622	0.173512362385299\\
-0.537084820159711	0.0850658786478001\\
-0.526620599330303	6.44924231334916e-17\\
};
\addplot [color=mycolor1,dotted,line width=1.0pt,forget plot]
  table[row sep=crcr]{1	0\\
0.940082788364644	0.148894486293864\\
0.86158608093073	0.279946287694513\\
0.768279786681378	0.391458103646313\\
0.663960650859636	0.482395649771645\\
0.552351927561387	0.552351927561387\\
0.437014383712816	0.601498696728173\\
0.321269860940431	0.630527604183869\\
0.208138182971344	0.640583459188477\\
0.100287778328637	0.633192112325829\\
3.73630739569174e-17	0.610185303761559\\
-0.0908532273476425	0.573624701779294\\
-0.170819110593797	0.525727164509449\\
-0.238861981883349	0.468793035010043\\
-0.294350736089166	0.40513903143051\\
-0.337037028702328	0.337037028702328\\
-0.367025664975262	0.266659754473976\\
-0.384738677688982	0.196034147687371\\
-0.390874612629551	0.127002860400749\\
-0.386364521398959	0.0611941284830315\\
-0.372326104926586	4.55967972637346e-17\\
};
\addplot [color=mycolor1,dotted,line width=1.0pt,forget plot]
  table[row sep=crcr]{1	0\\
0.922246029428501	0.146069421212559\\
0.829201462983264	0.269423887468404\\
0.725373165529273	0.369596088217499\\
0.614985835074999	0.446813363302878\\
0.501902475185001	0.501902475185001\\
0.38956496084428	0.536190168955128\\
0.280953750703967	0.551402782692681\\
0.178565395716864	0.549567778706978\\
0.0844061798404073	0.532919645815306\\
3.08495992433718e-17	0.50381218919366\\
-0.0735915548753502	0.464638791061534\\
-0.135739038566523	0.417761804348184\\
-0.186207014322272	0.365451842507638\\
-0.225110018726605	0.309837359892227\\
-0.252864584784672	0.252864584784672\\
-0.270139142845401	0.196267575756099\\
-0.277803443075876	0.141547924204336\\
-0.27687894570066	0.0899634229303457\\
-0.268491413422519	0.0425248622468694\\
-0.253826721980109	3.10848082611005e-17\\
};
\addplot [color=mycolor1,dotted,line width=1.0pt,forget plot]
  table[row sep=crcr]{1	0\\
0.902056570675584	0.142871725087523\\
0.793293726869509	0.257756756757911\\
0.678769666307395	0.345850419328459\\
0.562876391436159	0.408953636388562\\
0.449318435861499	0.449318435861499\\
0.341115747349381	0.469505547439701\\
0.24062663375655	0.472256359314945\\
0.149586755320764	0.460380694230177\\
0.069160331541479	0.436661148025441\\
2.47240337905553e-17	0.403774113610049\\
-0.057687904157153	0.364227092250654\\
-0.104075505986926	0.320311471399148\\
-0.139645446506012	0.274069620358657\\
-0.165124892040398	0.227274916024158\\
-0.18142315316863	0.18142315316863\\
-0.189574186252466	0.137733708524426\\
-0.190684892879101	0.0971588057560207\\
-0.185889806735969	0.0603992595374446\\
-0.176312467991919	0.0279251515651378\\
-0.16303353482158	1.99658496572927e-17\\
};
\addplot [color=mycolor1,dotted,line width=1.0pt,forget plot]
  table[row sep=crcr]{1	0\\
0.877921760602431	0.139049146701748\\
0.751411952540787	0.244148543365328\\
0.625732257688895	0.318826509862098\\
0.505011411193747	0.36691226736026\\
0.392341669548685	0.392341669548685\\
0.289890514921595	0.399000063654162\\
0.199020572856858	0.390599867100522\\
0.120411913496453	0.370589763847805\\
0.0541820486548744	0.34209199176291\\
1.88512313530119e-17	0.307863971328499\\
-0.042808222513439	0.270280479734772\\
-0.075164540154518	0.231332667812908\\
-0.0981551954909503	0.192640417840451\\
-0.112959067758674	0.155474818616317\\
-0.120787864504912	0.120787864504912\\
-0.122837744156894	0.0892468451742257\\
-0.120251626285951	0.0612712639357851\\
-0.114091236431876	0.0370704898842818\\
-0.105317799143806	0.0166807006736036\\
-0.0947802248421549	1.16072298975411e-17\\
};
\addplot [color=mycolor1,dotted,line width=1.0pt,forget plot]
  table[row sep=crcr]{1	0\\
0.84674396664984	0.134111069256013\\
0.698989566160914	0.227115477506975\\
0.561406498364893	0.286050898428465\\
0.437004973478602	0.317502698182059\\
0.327450698344114	0.327450698344114\\
0.233352139914598	0.321181666481713\\
0.154515499860225	0.303253743289057\\
0.0901653834921457	0.27750051639687\\
0.0391311034995635	0.247064063991275\\
1.31311479217367e-17	0.214447919692096\\
-0.0287598398096346	0.181582482159892\\
-0.0487044416812479	0.149896858349689\\
-0.0613430200940395	0.120392455675976\\
-0.06808779312177	0.0937148074575858\\
-0.0702211210616195	0.0702211210616195\\
-0.068876740220244	0.0500418809603846\\
-0.0650321343871793	0.0331355275052096\\
-0.0595094084547913	0.0193357789181307\\
-0.0529823745536039	0.00839158374073751\\
-0.0459879102602678	5.63189470997126e-18\\
};
\addplot [color=mycolor1,dotted,line width=1.0pt,forget plot]
  table[row sep=crcr]{1	0\\
0.801053465278425	0.126874404768154\\
0.625589539649299	0.203266363189334\\
0.475341369738971	0.242198525079547\\
0.350045057714404	0.254322621147605\\
0.24813777530853	0.24813777530853\\
0.167289292234614	0.230253957320141\\
0.104794333468008	0.205670459781722\\
0.0578515468028918	0.178048753195384\\
0.0237523524567972	0.149966451301199\\
7.54043881219821e-18	0.123144711070133\\
-0.0156239119173694	0.0986454975334405\\
-0.0250311775652273	0.0770380431086105\\
-0.0298254673925332	0.0585357756361869\\
-0.0313184856998208	0.0431061974937683\\
-0.0305568546459545	0.0305568546459545\\
-0.0283545570469435	0.0206007915573586\\
-0.0253272293454815	0.012904867916705\\
-0.021925762268643	0.00712411201600239\\
-0.0184675753156993	0.00292497658052826\\
-0.0151646198645466	1.85713031774033e-18\\
};
\addplot [color=mycolor1,dotted,line width=1.0pt,forget plot]
  table[row sep=crcr]{1	0\\
0.714110955679367	0.113104064044896\\
0.497161927717827	0.161537702532642\\
0.336758208677056	0.171586877647116\\
0.221075553032581	0.160620791180475\\
0.139705610200823	0.139705610200823\\
0.0839640345306934	0.115566579097864\\
0.0468885871776745	0.0920240337831981\\
0.0230753615892199	0.0710186604775083\\
0.00844586939409394	0.0533251206797082\\
2.39022368106624e-18	0.0390353150431684\\
-0.00441505277265522	0.0278755461307222\\
-0.00630567510971132	0.0194068724759343\\
-0.00669794782622876	0.0131454627690819\\
-0.0062698831862128	0.00862975386096949\\
-0.0054534525074872	0.0054534525074872\\
-0.00451117782354635	0.00327756254020109\\
-0.00359218715027031	0.00183031077240959\\
-0.00277223578568335	0.000900754009370227\\
-0.00208156288544739	0.000329687172611911\\
-0.00152375582051941	1.86606268828125e-19\\
};
\addplot [color=mycolor1,dotted,line width=1.0pt,forget plot]
  table[row sep=crcr]{1	-0\\
0.987688340595138	-0.156434465040231\\
0.951056516295154	-0.309016994374947\\
0.891006524188368	-0.453990499739547\\
0.809016994374947	-0.587785252292473\\
0.707106781186548	-0.707106781186547\\
0.587785252292473	-0.809016994374947\\
0.453990499739547	-0.891006524188368\\
0.309016994374947	-0.951056516295154\\
0.156434465040231	-0.987688340595138\\
6.12323399573677e-17	-1\\
-0.156434465040231	-0.987688340595138\\
-0.309016994374947	-0.951056516295154\\
-0.453990499739547	-0.891006524188368\\
-0.587785252292473	-0.809016994374947\\
-0.707106781186547	-0.707106781186548\\
-0.809016994374947	-0.587785252292473\\
-0.891006524188368	-0.453990499739547\\
-0.951056516295154	-0.309016994374948\\
-0.987688340595138	-0.156434465040231\\
-1	-1.22464679914735e-16\\
};
\addplot [color=mycolor1,dotted,line width=1.0pt,forget plot]
  table[row sep=crcr]{1	-0\\
0.972218045703082	-0.153984211042097\\
0.921496791140463	-0.299412457456957\\
0.849791018366557	-0.432990150609548\\
0.759508516401756	-0.551815237548133\\
0.653437051516106	-0.653437051516106\\
0.534664304371591	-0.735902282058355\\
0.406492952796512	-0.797787339572243\\
0.272353130096494	-0.83821674479613\\
0.135714483753824	-0.856867527364047\\
5.22899666167123e-17	-0.853959960588124\\
-0.131496350792703	-0.830235283991667\\
-0.255686250369537	-0.786921363444393\\
-0.369756251949576	-0.725687504552447\\
-0.47122811850853	-0.648589862732789\\
-0.558009078759514	-0.558009078759514\\
-0.628431083783263	-0.456581908289041\\
-0.681278445058817	-0.347128705949459\\
-0.715803538350409	-0.232578668243255\\
-0.731730559897045	-0.115894735197653\\
-0.729247614287671	-8.9307075662324e-17\\
};
\addplot [color=mycolor1,dotted,line width=1.0pt,forget plot]
  table[row sep=crcr]{1	-0\\
0.956521682769877	-0.151498151383791\\
0.891982039736211	-0.289822533396298\\
0.809292583610082	-0.412355167436434\\
0.711634858347723	-0.517032989000692\\
0.602364630186427	-0.602364630186426\\
0.484917701774751	-0.667431957639199\\
0.362719588349683	-0.711877274647591\\
0.239101059900595	-0.735877395777214\\
0.117221283062812	-0.740106053489981\\
4.44354699422903e-17	-0.725686295399261\\
-0.109940132237539	-0.694134676438339\\
-0.210320240583588	-0.647299141978847\\
-0.299240493845688	-0.587292536894097\\
-0.375203754387119	-0.516423664031337\\
-0.437127756959533	-0.437127756959533\\
-0.484346224770267	-0.351898130574443\\
-0.516599582217932	-0.263220634338226\\
-0.534016141209622	-0.173512362385299\\
-0.537084820159711	-0.0850658786478001\\
-0.526620599330303	-6.44924231334916e-17\\
};
\addplot [color=mycolor1,dotted,line width=1.0pt,forget plot]
  table[row sep=crcr]{1	-0\\
0.940082788364644	-0.148894486293864\\
0.86158608093073	-0.279946287694513\\
0.768279786681378	-0.391458103646313\\
0.663960650859636	-0.482395649771645\\
0.552351927561387	-0.552351927561387\\
0.437014383712816	-0.601498696728173\\
0.321269860940431	-0.630527604183869\\
0.208138182971344	-0.640583459188477\\
0.100287778328637	-0.633192112325829\\
3.73630739569174e-17	-0.610185303761559\\
-0.0908532273476425	-0.573624701779294\\
-0.170819110593797	-0.525727164509449\\
-0.238861981883349	-0.468793035010043\\
-0.294350736089166	-0.40513903143051\\
-0.337037028702328	-0.337037028702328\\
-0.367025664975262	-0.266659754473976\\
-0.384738677688982	-0.196034147687371\\
-0.390874612629551	-0.127002860400749\\
-0.386364521398959	-0.0611941284830315\\
-0.372326104926586	-4.55967972637346e-17\\
};
\addplot [color=mycolor1,dotted,line width=1.0pt,forget plot]
  table[row sep=crcr]{1	-0\\
0.922246029428501	-0.146069421212559\\
0.829201462983264	-0.269423887468404\\
0.725373165529273	-0.369596088217499\\
0.614985835074999	-0.446813363302878\\
0.501902475185001	-0.501902475185001\\
0.38956496084428	-0.536190168955128\\
0.280953750703967	-0.551402782692681\\
0.178565395716864	-0.549567778706978\\
0.0844061798404073	-0.532919645815306\\
3.08495992433718e-17	-0.50381218919366\\
-0.0735915548753502	-0.464638791061534\\
-0.135739038566523	-0.417761804348184\\
-0.186207014322272	-0.365451842507638\\
-0.225110018726605	-0.309837359892227\\
-0.252864584784672	-0.252864584784672\\
-0.270139142845401	-0.196267575756099\\
-0.277803443075876	-0.141547924204336\\
-0.27687894570066	-0.0899634229303457\\
-0.268491413422519	-0.0425248622468694\\
-0.253826721980109	-3.10848082611005e-17\\
};
\addplot [color=mycolor1,dotted,line width=1.0pt,forget plot]
  table[row sep=crcr]{1	-0\\
0.902056570675584	-0.142871725087523\\
0.793293726869509	-0.257756756757911\\
0.678769666307395	-0.345850419328459\\
0.562876391436159	-0.408953636388562\\
0.449318435861499	-0.449318435861499\\
0.341115747349381	-0.469505547439701\\
0.24062663375655	-0.472256359314945\\
0.149586755320764	-0.460380694230177\\
0.069160331541479	-0.436661148025441\\
2.47240337905553e-17	-0.403774113610049\\
-0.057687904157153	-0.364227092250654\\
-0.104075505986926	-0.320311471399148\\
-0.139645446506012	-0.274069620358657\\
-0.165124892040398	-0.227274916024158\\
-0.18142315316863	-0.18142315316863\\
-0.189574186252466	-0.137733708524426\\
-0.190684892879101	-0.0971588057560207\\
-0.185889806735969	-0.0603992595374446\\
-0.176312467991919	-0.0279251515651378\\
-0.16303353482158	-1.99658496572927e-17\\
};
\addplot [color=mycolor1,dotted,line width=1.0pt,forget plot]
  table[row sep=crcr]{1	-0\\
0.877921760602431	-0.139049146701748\\
0.751411952540787	-0.244148543365328\\
0.625732257688895	-0.318826509862098\\
0.505011411193747	-0.36691226736026\\
0.392341669548685	-0.392341669548685\\
0.289890514921595	-0.399000063654162\\
0.199020572856858	-0.390599867100522\\
0.120411913496453	-0.370589763847805\\
0.0541820486548744	-0.34209199176291\\
1.88512313530119e-17	-0.307863971328499\\
-0.042808222513439	-0.270280479734772\\
-0.075164540154518	-0.231332667812908\\
-0.0981551954909503	-0.192640417840451\\
-0.112959067758674	-0.155474818616317\\
-0.120787864504912	-0.120787864504912\\
-0.122837744156894	-0.0892468451742257\\
-0.120251626285951	-0.0612712639357851\\
-0.114091236431876	-0.0370704898842818\\
-0.105317799143806	-0.0166807006736036\\
-0.0947802248421549	-1.16072298975411e-17\\
};
\addplot [color=mycolor1,dotted,line width=1.0pt,forget plot]
  table[row sep=crcr]{1	-0\\
0.84674396664984	-0.134111069256013\\
0.698989566160914	-0.227115477506975\\
0.561406498364893	-0.286050898428465\\
0.437004973478602	-0.317502698182059\\
0.327450698344114	-0.327450698344114\\
0.233352139914598	-0.321181666481713\\
0.154515499860225	-0.303253743289057\\
0.0901653834921457	-0.27750051639687\\
0.0391311034995635	-0.247064063991275\\
1.31311479217367e-17	-0.214447919692096\\
-0.0287598398096346	-0.181582482159892\\
-0.0487044416812479	-0.149896858349689\\
-0.0613430200940395	-0.120392455675976\\
-0.06808779312177	-0.0937148074575858\\
-0.0702211210616195	-0.0702211210616195\\
-0.068876740220244	-0.0500418809603846\\
-0.0650321343871793	-0.0331355275052096\\
-0.0595094084547913	-0.0193357789181307\\
-0.0529823745536039	-0.00839158374073751\\
-0.0459879102602678	-5.63189470997126e-18\\
};
\addplot [color=mycolor1,dotted,line width=1.0pt,forget plot]
  table[row sep=crcr]{1	-0\\
0.801053465278425	-0.126874404768154\\
0.625589539649299	-0.203266363189334\\
0.475341369738971	-0.242198525079547\\
0.350045057714404	-0.254322621147605\\
0.24813777530853	-0.24813777530853\\
0.167289292234614	-0.230253957320141\\
0.104794333468008	-0.205670459781722\\
0.0578515468028918	-0.178048753195384\\
0.0237523524567972	-0.149966451301199\\
7.54043881219821e-18	-0.123144711070133\\
-0.0156239119173694	-0.0986454975334405\\
-0.0250311775652273	-0.0770380431086105\\
-0.0298254673925332	-0.0585357756361869\\
-0.0313184856998208	-0.0431061974937683\\
-0.0305568546459545	-0.0305568546459545\\
-0.0283545570469435	-0.0206007915573586\\
-0.0253272293454815	-0.012904867916705\\
-0.021925762268643	-0.00712411201600239\\
-0.0184675753156993	-0.00292497658052826\\
-0.0151646198645466	-1.85713031774033e-18\\
};
\addplot [color=mycolor1,dotted,line width=1.0pt,forget plot]
  table[row sep=crcr]{1	-0\\
0.714110955679367	-0.113104064044896\\
0.497161927717827	-0.161537702532642\\
0.336758208677056	-0.171586877647116\\
0.221075553032581	-0.160620791180475\\
0.139705610200823	-0.139705610200823\\
0.0839640345306934	-0.115566579097864\\
0.0468885871776745	-0.0920240337831981\\
0.0230753615892199	-0.0710186604775083\\
0.00844586939409394	-0.0533251206797082\\
2.39022368106624e-18	-0.0390353150431684\\
-0.00441505277265522	-0.0278755461307222\\
-0.00630567510971132	-0.0194068724759343\\
-0.00669794782622876	-0.0131454627690819\\
-0.0062698831862128	-0.00862975386096949\\
-0.0054534525074872	-0.0054534525074872\\
-0.00451117782354635	-0.00327756254020109\\
-0.00359218715027031	-0.00183031077240959\\
-0.00277223578568335	-0.000900754009370227\\
-0.00208156288544739	-0.000329687172611911\\
-0.00152375582051941	-1.86606268828125e-19\\
};
\addplot [color=mycolor1,dotted,line width=1.0pt,forget plot]
  table[row sep=crcr]{1	0\\
1	0\\
};
\addplot [color=mycolor1,dotted,line width=1.0pt,forget plot]
  table[row sep=crcr]{0.951056516295154	0.309016994374947\\
0.906577591518048	0.290693092532446\\
0.867277205719189	0.267124444629623\\
0.833330533437629	0.239553777474139\\
0.804679893747523	0.20903820245934\\
0.781122098933164	0.176433510213537\\
0.762382187756933	0.142402376999381\\
0.748171074803624	0.107437628337747\\
0.738227708485823	0.0718935522249012\\
0.732347931289196	0.0360204899377399\\
0.730402691048646	2.81010850277162e-17\\
};
\addplot [color=mycolor1,dotted,line width=1.0pt,forget plot]
  table[row sep=crcr]{0.809016994374947	0.587785252292473\\
0.737380455396588	0.527071687397996\\
0.6808142826414	0.463341883835339\\
0.637053765657314	0.399254954339046\\
0.603812761314093	0.336417677088311\\
0.579022949915482	0.275632227640288\\
0.560948163233974	0.217130071437151\\
0.54821711318997	0.160763451735608\\
0.539811466724715	0.106147624627789\\
0.535036016768209	0.0527590625798542\\
0.533488091091103	4.10502162512614e-17\\
};
\addplot [color=mycolor1,dotted,line width=1.0pt,forget plot]
  table[row sep=crcr]{0.587785252292473	0.809016994374947\\
0.515276498489902	0.692182785870847\\
0.46668476526979	0.583707991451876\\
0.435233321876477	0.485319980094308\\
0.415551842123536	0.396928474901319\\
0.403656860517901	0.317461475735791\\
0.396737049613855	0.245416450708029\\
0.392888142823216	0.179177710929467\\
0.39087245229983	0.11717008156476\\
0.389932112762627	0.0579102497954412\\
0.389661137375347	4.49747826270753e-17\\
};
\addplot [color=mycolor1,dotted,line width=1.0pt,forget plot]
  table[row sep=crcr]{0.309016994374947	0.951056516295154\\
0.265925372344309	0.777304721760365\\
0.248822386132444	0.630899544522142\\
0.24643298177389	0.508693744238057\\
0.251412997268255	0.406266573115132\\
0.25929445161488	0.31919477108011\\
0.267517373913267	0.243597429511063\\
0.274697115780397	0.1762665508339\\
0.280129101393366	0.114599409879343\\
0.283480020554887	0.0564559973822998\\
0.284609543336029	4.37996030135249e-17\\
};
\addplot [color=mycolor1,dotted,line width=1.0pt,forget plot]
  table[row sep=crcr]{6.12323399573677e-17	1\\
0.0151248701748503	0.781989711398728\\
0.0472692933177679	0.613631335769719\\
0.0835006201485716	0.482943980920436\\
0.117381749765263	0.379469463911326\\
0.146223972383669	0.295078319831894\\
0.169261627793672	0.223809451176491\\
0.186582776182006	0.161390341419992\\
0.198560105942714	0.104739935929793\\
0.205572433929556	0.0515565221197431\\
0.207879576350762	3.99891848849262e-17\\
};
\addplot [color=mycolor1,dotted,line width=1.0pt,forget plot]
  table[row sep=crcr]{-0.309016994374947	0.951056516295154\\
-0.213607139159912	0.713331044437031\\
-0.122920349149865	0.544815253953639\\
-0.0461074386071066	0.422454854218942\\
0.0151311193047769	0.329888717873058\\
0.0621570724668225	0.256291005261778\\
0.0971710522581798	0.194731597161215\\
0.122256440677152	0.140793596164787\\
0.13905244595299	0.0915971142347777\\
0.148693455532156	0.0451621321066958\\
0.151835801980649	3.50498499033503e-17\\
};
\addplot [color=mycolor1,dotted,line width=1.0pt,forget plot]
  table[row sep=crcr]{-0.587785252292473	0.809016994374948\\
-0.401011853057454	0.584595820351374\\
-0.252139489074835	0.439670821081765\\
-0.139623392550337	0.340999317931598\\
-0.0567836371213516	0.268617800427267\\
0.0033339412143336	0.21116115844769\\
0.0463512370945915	0.162297289886265\\
0.0763422025660035	0.118472638203443\\
0.0960671266193367	0.0776165020305757\\
0.1072685824301	0.0384304051397518\\
0.110901278364195	2.98672554719679e-17\\
};
\addplot [color=mycolor1,dotted,line width=1.0pt,forget plot]
  table[row sep=crcr]{-0.809016994374948	0.587785252292473\\
-0.533486326814499	0.413410095118228\\
-0.336121655437603	0.313963860155743\\
-0.198040110920963	0.250717832404618\\
-0.101844033235305	0.204281393673556\\
-0.0346516892466227	0.165530866251108\\
0.0122258376810533	0.130333089269644\\
0.0443886084751889	0.0968398262452624\\
0.065339288702761	0.0642052594194206\\
0.0771736424133687	0.032008294596762\\
0.0810025921579431	2.49315700239574e-17\\
};
\addplot [color=mycolor1,dotted,line width=1.0pt,forget plot]
  table[row sep=crcr]{-0.951056516295154	0.309016994374948\\
-0.60382220830535	0.219707538176049\\
-0.375378071887508	0.182507388795924\\
-0.225093275108467	0.161489568357552\\
-0.124654461172013	0.143091836517116\\
-0.0562723920172722	0.12318609851568\\
-0.00923898083522721	0.101104614081101\\
0.0228060316514889	0.0772217637055013\\
0.0436193292019494	0.0520955750986265\\
0.0553650029180358	0.0262210407420432\\
0.0591645112940776	2.0486346567262e-17\\
};
\addplot [color=mycolor1,dotted,line width=1.0pt,forget plot]
  table[row sep=crcr]{-1	1.22464679914735e-16\\
-0.61127914703566	0.0236549857259488\\
-0.374309030147768	0.0580118391989452\\
-0.226262535142082	0.0806522438077526\\
-0.130218598863194	0.0890855793127953\\
-0.0656897647351535	0.0862950481802363\\
-0.0214411717925588	0.0757647040434823\\
0.00876629006412298	0.0602253159022079\\
0.0284556614934045	0.0415943455493057\\
0.039601950618638	0.0211971994741971\\
0.0432139182637723	1.66258696249815e-17\\
};
\addplot [color=mycolor1,dotted,line width=1.0pt,forget plot]
  table[row sep=crcr]{1	-0\\
1	-0\\
};
\addplot [color=mycolor1,dotted,line width=1.0pt,forget plot]
  table[row sep=crcr]{0.951056516295154	-0.309016994374947\\
0.906577591518048	-0.290693092532446\\
0.867277205719189	-0.267124444629623\\
0.833330533437629	-0.239553777474139\\
0.804679893747523	-0.20903820245934\\
0.781122098933164	-0.176433510213537\\
0.762382187756933	-0.142402376999381\\
0.748171074803624	-0.107437628337747\\
0.738227708485823	-0.0718935522249012\\
0.732347931289196	-0.0360204899377399\\
0.730402691048646	-2.81010850277162e-17\\
};
\addplot [color=mycolor1,dotted,line width=1.0pt,forget plot]
  table[row sep=crcr]{0.809016994374947	-0.587785252292473\\
0.737380455396588	-0.527071687397996\\
0.6808142826414	-0.463341883835339\\
0.637053765657314	-0.399254954339046\\
0.603812761314093	-0.336417677088311\\
0.579022949915482	-0.275632227640288\\
0.560948163233974	-0.217130071437151\\
0.54821711318997	-0.160763451735608\\
0.539811466724715	-0.106147624627789\\
0.535036016768209	-0.0527590625798542\\
0.533488091091103	-4.10502162512614e-17\\
};
\addplot [color=mycolor1,dotted,line width=1.0pt,forget plot]
  table[row sep=crcr]{0.587785252292473	-0.809016994374947\\
0.515276498489902	-0.692182785870847\\
0.46668476526979	-0.583707991451876\\
0.435233321876477	-0.485319980094308\\
0.415551842123536	-0.396928474901319\\
0.403656860517901	-0.317461475735791\\
0.396737049613855	-0.245416450708029\\
0.392888142823216	-0.179177710929467\\
0.39087245229983	-0.11717008156476\\
0.389932112762627	-0.0579102497954412\\
0.389661137375347	-4.49747826270753e-17\\
};
\addplot [color=mycolor1,dotted,line width=1.0pt,forget plot]
  table[row sep=crcr]{0.309016994374947	-0.951056516295154\\
0.265925372344309	-0.777304721760365\\
0.248822386132444	-0.630899544522142\\
0.24643298177389	-0.508693744238057\\
0.251412997268255	-0.406266573115132\\
0.25929445161488	-0.31919477108011\\
0.267517373913267	-0.243597429511063\\
0.274697115780397	-0.1762665508339\\
0.280129101393366	-0.114599409879343\\
0.283480020554887	-0.0564559973822998\\
0.284609543336029	-4.37996030135249e-17\\
};
\addplot [color=mycolor1,dotted,line width=1.0pt,forget plot]
  table[row sep=crcr]{6.12323399573677e-17	-1\\
0.0151248701748503	-0.781989711398728\\
0.0472692933177679	-0.613631335769719\\
0.0835006201485716	-0.482943980920436\\
0.117381749765263	-0.379469463911326\\
0.146223972383669	-0.295078319831894\\
0.169261627793672	-0.223809451176491\\
0.186582776182006	-0.161390341419992\\
0.198560105942714	-0.104739935929793\\
0.205572433929556	-0.0515565221197431\\
0.207879576350762	-3.99891848849262e-17\\
};
\addplot [color=mycolor1,dotted,line width=1.0pt,forget plot]
  table[row sep=crcr]{-0.309016994374947	-0.951056516295154\\
-0.213607139159912	-0.713331044437031\\
-0.122920349149865	-0.544815253953639\\
-0.0461074386071066	-0.422454854218942\\
0.0151311193047769	-0.329888717873058\\
0.0621570724668225	-0.256291005261778\\
0.0971710522581798	-0.194731597161215\\
0.122256440677152	-0.140793596164787\\
0.13905244595299	-0.0915971142347777\\
0.148693455532156	-0.0451621321066958\\
0.151835801980649	-3.50498499033503e-17\\
};
\addplot [color=mycolor1,dotted,line width=1.0pt,forget plot]
  table[row sep=crcr]{-0.587785252292473	-0.809016994374948\\
-0.401011853057454	-0.584595820351374\\
-0.252139489074835	-0.439670821081765\\
-0.139623392550337	-0.340999317931598\\
-0.0567836371213516	-0.268617800427267\\
0.0033339412143336	-0.21116115844769\\
0.0463512370945915	-0.162297289886265\\
0.0763422025660035	-0.118472638203443\\
0.0960671266193367	-0.0776165020305757\\
0.1072685824301	-0.0384304051397518\\
0.110901278364195	-2.98672554719679e-17\\
};
\addplot [color=mycolor1,dotted,line width=1.0pt,forget plot]
  table[row sep=crcr]{-0.809016994374948	-0.587785252292473\\
-0.533486326814499	-0.413410095118228\\
-0.336121655437603	-0.313963860155743\\
-0.198040110920963	-0.250717832404618\\
-0.101844033235305	-0.204281393673556\\
-0.0346516892466227	-0.165530866251108\\
0.0122258376810533	-0.130333089269644\\
0.0443886084751889	-0.0968398262452624\\
0.065339288702761	-0.0642052594194206\\
0.0771736424133687	-0.032008294596762\\
0.0810025921579431	-2.49315700239574e-17\\
};
\addplot [color=mycolor1,dotted,line width=1.0pt,forget plot]
  table[row sep=crcr]{-0.951056516295154	-0.309016994374948\\
-0.60382220830535	-0.219707538176049\\
-0.375378071887508	-0.182507388795924\\
-0.225093275108467	-0.161489568357552\\
-0.124654461172013	-0.143091836517116\\
-0.0562723920172722	-0.12318609851568\\
-0.00923898083522721	-0.101104614081101\\
0.0228060316514889	-0.0772217637055013\\
0.0436193292019494	-0.0520955750986265\\
0.0553650029180358	-0.0262210407420432\\
0.0591645112940776	-2.0486346567262e-17\\
};
\addplot [color=mycolor1,dotted,line width=1.0pt,forget plot]
  table[row sep=crcr]{-1	-1.22464679914735e-16\\
-0.61127914703566	-0.0236549857259488\\
-0.374309030147768	-0.0580118391989452\\
-0.226262535142082	-0.0806522438077526\\
-0.130218598863194	-0.0890855793127953\\
-0.0656897647351535	-0.0862950481802363\\
-0.0214411717925588	-0.0757647040434823\\
0.00876629006412298	-0.0602253159022079\\
0.0284556614934045	-0.0415943455493057\\
0.039601950618638	-0.0211971994741971\\
0.0432139182637723	-1.66258696249815e-17\\
};
\addplot [color=black!50!mycolor1,line width=1.5pt,mark size=5.0pt,only marks,mark=x,mark options={solid},forget plot]
  table[row sep=crcr]{0	0\\
0.999999999999999	0\\
0.914443066593832	0.404714563561125\\
0.914443066593832	-0.404714563561125\\
};
\addplot [color=black,line width=1.5pt,mark size=4.0pt,only marks,mark=o,mark options={solid},forget plot]
  table[row sep=crcr]{-3.6945980719401	0\\
-0.270665436545005	0\\
};
\end{axis}
\end{tikzpicture}%}
    \end{figure}
    \end{small}
    \vfill
  }


%FIM-------------------------------------------------------------------
