\chapter{Procedimento de Projeto do Filtro \textit{LCL}}

	O projeto de um filtro LCL pode ser feito de várias maneiras, dependendo
    do objetivo do projetista. O procedimento de projeto apresentado em~\cite{ref:TANG}
    é muito utilizado, por ser generalizado, simples e em valores \textit{por unidade},
    o que torna simples a escalabilidade do sistema. Os passos deste procedimento são:

    \begin{enumerate}

        \item Definir qual a ordem $k$ mais alta das correntes harmônicas que precisam ser
        compensadas. A frequência de ressonância $\omega_{res}$ deve ser função da frequência
        fundamental nominal $\omega_n$:

        \begin{equation}
            \frac{k \omega_n}{0,3} \leq \omega_{res} \leq \frac{k \omega_n}{0,25}
        \label{eq:wres}
        \end{equation}

        \item A frequência de comutação deve ser pelo menos duas vezes maior que a frequência
        de ressonância. Valores maiores podem ser usados para uma melhor atenuação harmônica,
        mas resultarão em mais perdas.

        \item Valores de impedância, capacitância e indutância base devem ser definidos. Dessa
        forma, a impedância base $Z_b$ é função da tensão nominal $V$ e da potência nominal $P$:

        \begin{equation}
            Z_b = \frac{V^2}{P}
        \end{equation}

        Os valores da capacitância e indutância base são, respectivamente:

        \begin{equation}
            C_b = \frac{1}{\omega_n Z_b}
        \end{equation}

        \begin{equation}
            L_b = \frac{Z_b}{\omega_n}
        \end{equation}

        \item As indutâncias do lado do conversor $L_{ff}$ e da rede $L_{fg}$ devem ser iguais para
        produzir a menor frequência de ressonância possível, e a máxima atenuação de harmônicas de
        comutação. Além disso, é recomendável que o valor total em \textit{por unidade} dos dois
        indutores seja igual ao valor do capacitor do filtro $C_f$. Desta forma:

        \begin{equation}
            L_{ff} = L_{gf} = \frac{1}{4k} L_b
        \end{equation}

        \begin{equation}
            C_f = \frac{1}{2k} C_b
        \label{eq:cf}
        \end{equation}

        \item O valor comercial de capacitor mais próximo ao valor encontrado em (\ref{eq:cf})
        deve ser escolhido, e os valores de indutância ajustados de acordo. A frequência de
        ressonância recalculada com os valores ajustados deve, no entando, estar de acordo com
        (\ref{eq:wres}).

    \end{enumerate}


% \chapter{Cronograma}
% \label{cronograma}
% \addcontentsline{toc}{chapter}{Cronograma}

% \begin{table}[htb]
%    \centering
%    \begin{small}
%        \setlength{\tabcolsep}{3pt}%redefine espaçamento das colunas
%        \begin{tabular}{|L|c|c|c|c|c|c|c|c|}\hline
%             & \multicolumn{8}{c|}{\textbf{Meses}}\\ \cline{2-9}
%            \raisebox{1.5ex}{\textbf{Etapa}} & 02 & 03 & 04 & 05 & 06 & 07 & 08 & 09 \\ \hline
%            Qualificação
%            & \textbullet &   &   &   &   &   &   &   \\ \hline
%            Resultados de simulação do controlador Multi-malhas com malha interna utilizando a tensão
%            do capacitor numa estrutura IMC
%            & \textbullet &   &   &   &   &   &   &   \\ \hline
%            Resultados de simulação do controlador Multi-malhas com malha interna utilizando a corrente
%            do capacitor numa estrutura IMC
%            &   & \textbullet &   &   &   &   &   &   \\ \hline
%            Resultados experimentais
%            &   & \textbullet & \textbullet &   &   &   &   &   \\ \hline
%            Escrita de um artigo científico
%            &   & \textbullet & \textbullet &   &   &   &   &   \\ \hline
%            Comparação das duas técnicas multi-malhas utilizadas, entre si e com outras técnicas clássicas
%            (PI e Deadbeat, por exemplo)
%            &   &   &   & \textbullet & \textbullet &   &   &   \\ \hline
%            Verificação de melhorias da relação Desempenho x Robustez
%            &   &   &   &   & \textbullet &   &   &   \\ \hline
%            Análise dos resultados e redação
%            &   &   &   &   &   & \textbullet & \textbullet &   \\ \hline
%            Defesa
%            &   &   &   &   &   &   &   & \textbullet \\ \hline
%        \end{tabular}
%    \end{small}
% \end{table}
