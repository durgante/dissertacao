%=========================================================================
% Resumo
%=========================================================================

	O controle de conversores eletrônicos de potência tem recebido muita atenção devido às suas inúmeras aplicações. Destacam-se especialmente aplicações em problemas de qualidade de energia, onde é necessário injetar uma corrente na rede elétrica de acordo com uma referência. A conexão de conversores na rede elétrica, no entanto, apresenta diversos desafios, como a existência de incerteza paramétrica na planta e distúrbios advindos da rede. Além disso, inerentemente ao seu funcionamento, conversores eletrônicos de potência geram componentes  harmônicas de comutação que precisam ser filtradas. A tendência atual das estratégias de controle é o relaxamento da exigência clássica de conhecimento completo da planta a ser controlada, buscando robustez com relação às incertezas paramétricas. Este trabalho apresenta uma estratégia de controle capaz de rejeitar distúrbios e apresentar bom desempenho frente a incertezas, utilizando técnicas de controle multimalhas e controle adaptativo. Por fim, são apresentados resultados de simulação, e resultados experimentais que mostram o bom funcionamento do sistema.

 \vspace{\onelineskip}

 \noindent
 Palavras-chave: Controle multimalha. Controle adaptativo. Conversores conectados à rede elétrica. Rejeição de distúrbios. Incerteza paramétrica.
