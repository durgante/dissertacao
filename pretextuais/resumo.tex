%===============================================================================
% Resumo
%===============================================================================

	O controle de conversores tem sido muito explorado devido às inúmeras aplicações
	que possui, dentre as quais está a geração de correntes que serão injetadas na
	rede. A conexão de conversores na rede elétrica, no entanto, apresenta diversos
	desafios, como a existência de incerteza paramétrica na planta, e distúrbios
	advindos da rede. Além disso, inerentemente ao seu funcionamento, inversores
	de tensão geram harmônicas de comutação que precisam ser compensadas. A tendência
	atual das estratégias de controle é o relaxamento da exigência clássica de
	conhecimento completo da planta a ser controlada, buscando robustez com relação
	às incertezas paramétricas. Este trabalho apresenta uma estratégia de controle
	capaz de rejeitar distúrbios e apresentar bom desempenho frente a incertezas,
	utilizando técnicas de Controle Multi-Malhas e Controle Adaptativo.
	São apresentados resultados de simulação, e resultados experimentais que
	demonstram o funcionamento do sistema.

 \vspace{\onelineskip}
    
 \noindent
 \textbf{Palavras-chave}: controle multi-malha. controle adaptativo. conexão de conversores na rede.
 	rejeição de distúrbios. incerteza paramétrica.
 