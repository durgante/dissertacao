%TITULO-------------------------------------------------------------------

%=========================================================================
\chapter{Controle Multimalha}\label{controle}
%=========================================================================

	Em aplicações onde o desempenho é um requisito importante, o controle multimalha, também chamado de controle em cascata, apresenta vantagens significativas. Essa estratégia propõe que uma variável seja criada para detectar a presença de distúrbios antes que esses afetem a variável controlada. Dessa forma, não é necessário esperar que a variável controlada se desvie do ponto fixo para começar uma ação corretiva, como acontece no controle por realimentação simples. Ao perceber a presença do distúrbio, a ação corretiva começa imediatamente, de forma que o desvio sofrido pela variável controlada tende a ser reduzido.

    É necessário, no entanto, que a variável escolhida para ser a \emph{variável intermediária} responda mais rapidamente a variações no distúrbio e na variável manipulada do que a variável controlada. Isso faz sentido devido ao fato de que quanto mais rápido a variável intermediária responder ao distúrbio, mais rápido a ação corretiva será iniciada e menor será o desvio do ponto fixo que a variável controlada sofrerá. De fato, quanto mais rápido a variável intermediária responder, melhor.

    A Fig.~\ref{fig:realimentacao_vs_cascata} apresenta uma comparação de desempenho do controle da temperatura de um reator quando ocorre uma variação de $25\,^{\circ}\mathrm{C}$ na temperatura de entrada do reagente. A linha cheia representa o controle multimalha, e a linha tracejada representa o controle por realimentação simples.

    \begin{figure}[htb]
        \centering{
            \def\svgwidth{0.6\textwidth}
            \input{./img/realimentacao_vs_cascata.pdf_tex}}
        \renewcommand\figurename{Fig.}
        \caption{Comparação de desempenho entre realimentação simples e controle em cascata.}
        \label{fig:realimentacao_vs_cascata}
    \end{figure}

    É possível desenvolver controladores com qualquer número de malhas aninhadas em cascata. É importante observar apenas o fato de que a malha mais externa gera a referência para a malha imediatamente interna em relação a ela. Dessa forma, a variável intermediária escolhida para a malha externa deve responder mais rápido ao distúrbio do que a variável intermediária escolhida para a malha interna.

    Na prática, é comum observar controladores com duas ou três malhas aninhadas. Além do quesito desempenho, outros fatores podem levar à utilização de controladores multimalha. Como é o caso deste trabalho, a separação do modelo em duas partes simplifica a abordagem de controle, devido ao fato de a malha interna fazer o amortecimento ativo da ressonância do filtro, e a malha externa tratar da incerteza paramétrica inerente à rede elétrica.

    O projeto deste controlador divide-se então em uma malha interna e uma malha externa, conforme a seguir.

\section{Malha Interna}

    A modelagem desenvolvida separa a planta em duas partes: $G_{id}$ pertence à malha interna, e $G_{od}$ pertence à malha externa. Esta forma de modelagem permite isolar a incerteza paramétrica em $G_{od}$, de forma que o controlador projetado para a malha interna trata de uma planta com parâmetros conhecidos e projetados. Dessa forma, pode-se utilizar controladores convencionais para realizar o amortecimento ativo.

    É conhecido da literatura que, no caso específico de um filtro LCL, é suficiente para a realização do amortecimento da ressonância do filtro a utilização de um controlador proporcional, quando a variável intermediária escolhida é a corrente do capacitor, e de um controlador proporcional-derivativo, quando a variável intermediária escolhida é a tensão do capacitor~\cite{ref:DANNEHL}. O critério de escolha para a variável intermediária varia de acordo com a aplicação e a topologia~\cite{ref:POH}.

    Levando isso em consideração, neste trabalho um controlador proporcional é projetado para o caso da variável intermediária ser a corrente do capacitor, e um controlador proporcional-derivativo é projetado para o caso da variável intermediária ser a tensão do capacitor. O projeto de ambos leva em consideração os parâmetros listados na Tabela~\ref{tab:sim_parameters}.

    \begin{table}[htb]
        \renewcommand{\arraystretch}{1.35}
        \setlength{\tabcolsep}{1.2mm}
        \caption{Parâmetros do sistema utilizados no projeto.}
        \label{tab:sim_parameters}
        \centering
        \begin{tabular}{l l l l}
            \hline
            \multicolumn{1}{c}{Parâmetro} & \multicolumn{1}{c}{Valor} &
            \multicolumn{1}{c}{Parâmetro} & \multicolumn{1}{c}{Valor} \\
            \hline
            $L_1$ &  $2$mH     &  $L_2$      &  $2$mH    \\
            $C$   &  $40\mu$F  & Frequência de Amostragem ($f_s = 1/T_s$) &  $12$kHz  \\
            \hline
        \end{tabular}
    \end{table}


\subsection{Projeto para $U_p = V_C$}

    Considerando o controlador da malha interna $C_i(z)$ como sendo proporcional-derivativo, tem-se sua expressão:
    %
    \begin{equation}
        C_i(z) = \left( K_P + K_D \right) \frac{z- \frac{K_D}{K_P+K_D}}{z}
    \end{equation}

    Em uma aplicação prática, o atraso de tempo associado à implementação digital limita o ganho do controlador, de forma que se deve projetar o zero visando maximizar o amortecimento.
    %
    \begin{equation}
        z = \frac{K_D}{K_P+K_D} > 1
    \end{equation}

    Essa escolha, no entanto, resulta em um sistema de fase não-mínima, uma característica que viola o requisito principal para o funcionamento do controlador da malha externa. Assim sendo, o zero é projetado em $z=0,9$ e é utilizado um traçado do lugar das raízes para projetar o ganho $K_P+K_D$.

    \begin{figure}[htb]
        \centering{
            \def\svgwidth{0.9\textwidth}
            % This file was created by matlab2tikz v0.4.7 running on MATLAB 7.14.
% Copyright (c) 2008--2014, Nico Schlömer <nico.schloemer@gmail.com>
% All rights reserved.
% Minimal pgfplots version: 1.3
% 
% The latest updates can be retrieved from
%   http://www.mathworks.com/matlabcentral/fileexchange/22022-matlab2tikz
% where you can also make suggestions and rate matlab2tikz.
% 
%
% defining custom colors
\definecolor{mycolor1}{rgb}{0.66667,0.66667,0.66667}%
%
\begin{tikzpicture}

\begin{axis}[%
width=0.8\textwidth,
height=0.461611624834875\textwidth,
scale only axis,
xmin=-1.9,
xmax=1.9,
xtick={-1,  0,  1},
xlabel={Eixo Real},
ymin=-1.1,
ymax=1.1,
ytick={-1,  0,  1},
ylabel={Eixo Imaginário},
scaled y ticks = false,
y tick label style={/pgf/number format/.cd, fixed, fixed zerofill, precision=0},
scaled x ticks = false,
x tick label style={/pgf/number format/.cd, fixed, fixed zerofill, precision=0}
]
\addplot [color=mycolor1,dotted,line width=1.0pt,forget plot]
  table[row sep=crcr]{1	0\\
0.987688340595138	0.156434465040231\\
0.951056516295154	0.309016994374947\\
0.891006524188368	0.453990499739547\\
0.809016994374947	0.587785252292473\\
0.707106781186548	0.707106781186547\\
0.587785252292473	0.809016994374947\\
0.453990499739547	0.891006524188368\\
0.309016994374947	0.951056516295154\\
0.156434465040231	0.987688340595138\\
6.12323399573677e-17	1\\
-0.156434465040231	0.987688340595138\\
-0.309016994374947	0.951056516295154\\
-0.453990499739547	0.891006524188368\\
-0.587785252292473	0.809016994374947\\
-0.707106781186547	0.707106781186548\\
-0.809016994374947	0.587785252292473\\
-0.891006524188368	0.453990499739547\\
-0.951056516295154	0.309016994374948\\
-0.987688340595138	0.156434465040231\\
-1	1.22464679914735e-16\\
};
\addplot [color=mycolor1,dotted,line width=1.0pt,forget plot]
  table[row sep=crcr]{1	0\\
0.972218045703082	0.153984211042097\\
0.921496791140463	0.299412457456957\\
0.849791018366557	0.432990150609548\\
0.759508516401756	0.551815237548133\\
0.653437051516106	0.653437051516106\\
0.534664304371591	0.735902282058355\\
0.406492952796512	0.797787339572243\\
0.272353130096494	0.83821674479613\\
0.135714483753824	0.856867527364047\\
5.22899666167123e-17	0.853959960588124\\
-0.131496350792703	0.830235283991667\\
-0.255686250369537	0.786921363444393\\
-0.369756251949576	0.725687504552447\\
-0.47122811850853	0.648589862732789\\
-0.558009078759514	0.558009078759514\\
-0.628431083783263	0.456581908289041\\
-0.681278445058817	0.347128705949459\\
-0.715803538350409	0.232578668243255\\
-0.731730559897045	0.115894735197653\\
-0.729247614287671	8.9307075662324e-17\\
};
\addplot [color=mycolor1,dotted,line width=1.0pt,forget plot]
  table[row sep=crcr]{1	0\\
0.956521682769877	0.151498151383791\\
0.891982039736211	0.289822533396298\\
0.809292583610082	0.412355167436434\\
0.711634858347723	0.517032989000692\\
0.602364630186427	0.602364630186426\\
0.484917701774751	0.667431957639199\\
0.362719588349683	0.711877274647591\\
0.239101059900595	0.735877395777214\\
0.117221283062812	0.740106053489981\\
4.44354699422903e-17	0.725686295399261\\
-0.109940132237539	0.694134676438339\\
-0.210320240583588	0.647299141978847\\
-0.299240493845688	0.587292536894097\\
-0.375203754387119	0.516423664031337\\
-0.437127756959533	0.437127756959533\\
-0.484346224770267	0.351898130574443\\
-0.516599582217932	0.263220634338226\\
-0.534016141209622	0.173512362385299\\
-0.537084820159711	0.0850658786478001\\
-0.526620599330303	6.44924231334916e-17\\
};
\addplot [color=mycolor1,dotted,line width=1.0pt,forget plot]
  table[row sep=crcr]{1	0\\
0.940082788364644	0.148894486293864\\
0.86158608093073	0.279946287694513\\
0.768279786681378	0.391458103646313\\
0.663960650859636	0.482395649771645\\
0.552351927561387	0.552351927561387\\
0.437014383712816	0.601498696728173\\
0.321269860940431	0.630527604183869\\
0.208138182971344	0.640583459188477\\
0.100287778328637	0.633192112325829\\
3.73630739569174e-17	0.610185303761559\\
-0.0908532273476425	0.573624701779294\\
-0.170819110593797	0.525727164509449\\
-0.238861981883349	0.468793035010043\\
-0.294350736089166	0.40513903143051\\
-0.337037028702328	0.337037028702328\\
-0.367025664975262	0.266659754473976\\
-0.384738677688982	0.196034147687371\\
-0.390874612629551	0.127002860400749\\
-0.386364521398959	0.0611941284830315\\
-0.372326104926586	4.55967972637346e-17\\
};
\addplot [color=mycolor1,dotted,line width=1.0pt,forget plot]
  table[row sep=crcr]{1	0\\
0.922246029428501	0.146069421212559\\
0.829201462983264	0.269423887468404\\
0.725373165529273	0.369596088217499\\
0.614985835074999	0.446813363302878\\
0.501902475185001	0.501902475185001\\
0.38956496084428	0.536190168955128\\
0.280953750703967	0.551402782692681\\
0.178565395716864	0.549567778706978\\
0.0844061798404073	0.532919645815306\\
3.08495992433718e-17	0.50381218919366\\
-0.0735915548753502	0.464638791061534\\
-0.135739038566523	0.417761804348184\\
-0.186207014322272	0.365451842507638\\
-0.225110018726605	0.309837359892227\\
-0.252864584784672	0.252864584784672\\
-0.270139142845401	0.196267575756099\\
-0.277803443075876	0.141547924204336\\
-0.27687894570066	0.0899634229303457\\
-0.268491413422519	0.0425248622468694\\
-0.253826721980109	3.10848082611005e-17\\
};
\addplot [color=mycolor1,dotted,line width=1.0pt,forget plot]
  table[row sep=crcr]{1	0\\
0.902056570675584	0.142871725087523\\
0.793293726869509	0.257756756757911\\
0.678769666307395	0.345850419328459\\
0.562876391436159	0.408953636388562\\
0.449318435861499	0.449318435861499\\
0.341115747349381	0.469505547439701\\
0.24062663375655	0.472256359314945\\
0.149586755320764	0.460380694230177\\
0.069160331541479	0.436661148025441\\
2.47240337905553e-17	0.403774113610049\\
-0.057687904157153	0.364227092250654\\
-0.104075505986926	0.320311471399148\\
-0.139645446506012	0.274069620358657\\
-0.165124892040398	0.227274916024158\\
-0.18142315316863	0.18142315316863\\
-0.189574186252466	0.137733708524426\\
-0.190684892879101	0.0971588057560207\\
-0.185889806735969	0.0603992595374446\\
-0.176312467991919	0.0279251515651378\\
-0.16303353482158	1.99658496572927e-17\\
};
\addplot [color=mycolor1,dotted,line width=1.0pt,forget plot]
  table[row sep=crcr]{1	0\\
0.877921760602431	0.139049146701748\\
0.751411952540787	0.244148543365328\\
0.625732257688895	0.318826509862098\\
0.505011411193747	0.36691226736026\\
0.392341669548685	0.392341669548685\\
0.289890514921595	0.399000063654162\\
0.199020572856858	0.390599867100522\\
0.120411913496453	0.370589763847805\\
0.0541820486548744	0.34209199176291\\
1.88512313530119e-17	0.307863971328499\\
-0.042808222513439	0.270280479734772\\
-0.075164540154518	0.231332667812908\\
-0.0981551954909503	0.192640417840451\\
-0.112959067758674	0.155474818616317\\
-0.120787864504912	0.120787864504912\\
-0.122837744156894	0.0892468451742257\\
-0.120251626285951	0.0612712639357851\\
-0.114091236431876	0.0370704898842818\\
-0.105317799143806	0.0166807006736036\\
-0.0947802248421549	1.16072298975411e-17\\
};
\addplot [color=mycolor1,dotted,line width=1.0pt,forget plot]
  table[row sep=crcr]{1	0\\
0.84674396664984	0.134111069256013\\
0.698989566160914	0.227115477506975\\
0.561406498364893	0.286050898428465\\
0.437004973478602	0.317502698182059\\
0.327450698344114	0.327450698344114\\
0.233352139914598	0.321181666481713\\
0.154515499860225	0.303253743289057\\
0.0901653834921457	0.27750051639687\\
0.0391311034995635	0.247064063991275\\
1.31311479217367e-17	0.214447919692096\\
-0.0287598398096346	0.181582482159892\\
-0.0487044416812479	0.149896858349689\\
-0.0613430200940395	0.120392455675976\\
-0.06808779312177	0.0937148074575858\\
-0.0702211210616195	0.0702211210616195\\
-0.068876740220244	0.0500418809603846\\
-0.0650321343871793	0.0331355275052096\\
-0.0595094084547913	0.0193357789181307\\
-0.0529823745536039	0.00839158374073751\\
-0.0459879102602678	5.63189470997126e-18\\
};
\addplot [color=mycolor1,dotted,line width=1.0pt,forget plot]
  table[row sep=crcr]{1	0\\
0.801053465278425	0.126874404768154\\
0.625589539649299	0.203266363189334\\
0.475341369738971	0.242198525079547\\
0.350045057714404	0.254322621147605\\
0.24813777530853	0.24813777530853\\
0.167289292234614	0.230253957320141\\
0.104794333468008	0.205670459781722\\
0.0578515468028918	0.178048753195384\\
0.0237523524567972	0.149966451301199\\
7.54043881219821e-18	0.123144711070133\\
-0.0156239119173694	0.0986454975334405\\
-0.0250311775652273	0.0770380431086105\\
-0.0298254673925332	0.0585357756361869\\
-0.0313184856998208	0.0431061974937683\\
-0.0305568546459545	0.0305568546459545\\
-0.0283545570469435	0.0206007915573586\\
-0.0253272293454815	0.012904867916705\\
-0.021925762268643	0.00712411201600239\\
-0.0184675753156993	0.00292497658052826\\
-0.0151646198645466	1.85713031774033e-18\\
};
\addplot [color=mycolor1,dotted,line width=1.0pt,forget plot]
  table[row sep=crcr]{1	0\\
0.714110955679367	0.113104064044896\\
0.497161927717827	0.161537702532642\\
0.336758208677056	0.171586877647116\\
0.221075553032581	0.160620791180475\\
0.139705610200823	0.139705610200823\\
0.0839640345306934	0.115566579097864\\
0.0468885871776745	0.0920240337831981\\
0.0230753615892199	0.0710186604775083\\
0.00844586939409394	0.0533251206797082\\
2.39022368106624e-18	0.0390353150431684\\
-0.00441505277265522	0.0278755461307222\\
-0.00630567510971132	0.0194068724759343\\
-0.00669794782622876	0.0131454627690819\\
-0.0062698831862128	0.00862975386096949\\
-0.0054534525074872	0.0054534525074872\\
-0.00451117782354635	0.00327756254020109\\
-0.00359218715027031	0.00183031077240959\\
-0.00277223578568335	0.000900754009370227\\
-0.00208156288544739	0.000329687172611911\\
-0.00152375582051941	1.86606268828125e-19\\
};
\addplot [color=mycolor1,dotted,line width=1.0pt,forget plot]
  table[row sep=crcr]{1	-0\\
0.987688340595138	-0.156434465040231\\
0.951056516295154	-0.309016994374947\\
0.891006524188368	-0.453990499739547\\
0.809016994374947	-0.587785252292473\\
0.707106781186548	-0.707106781186547\\
0.587785252292473	-0.809016994374947\\
0.453990499739547	-0.891006524188368\\
0.309016994374947	-0.951056516295154\\
0.156434465040231	-0.987688340595138\\
6.12323399573677e-17	-1\\
-0.156434465040231	-0.987688340595138\\
-0.309016994374947	-0.951056516295154\\
-0.453990499739547	-0.891006524188368\\
-0.587785252292473	-0.809016994374947\\
-0.707106781186547	-0.707106781186548\\
-0.809016994374947	-0.587785252292473\\
-0.891006524188368	-0.453990499739547\\
-0.951056516295154	-0.309016994374948\\
-0.987688340595138	-0.156434465040231\\
-1	-1.22464679914735e-16\\
};
\addplot [color=mycolor1,dotted,line width=1.0pt,forget plot]
  table[row sep=crcr]{1	-0\\
0.972218045703082	-0.153984211042097\\
0.921496791140463	-0.299412457456957\\
0.849791018366557	-0.432990150609548\\
0.759508516401756	-0.551815237548133\\
0.653437051516106	-0.653437051516106\\
0.534664304371591	-0.735902282058355\\
0.406492952796512	-0.797787339572243\\
0.272353130096494	-0.83821674479613\\
0.135714483753824	-0.856867527364047\\
5.22899666167123e-17	-0.853959960588124\\
-0.131496350792703	-0.830235283991667\\
-0.255686250369537	-0.786921363444393\\
-0.369756251949576	-0.725687504552447\\
-0.47122811850853	-0.648589862732789\\
-0.558009078759514	-0.558009078759514\\
-0.628431083783263	-0.456581908289041\\
-0.681278445058817	-0.347128705949459\\
-0.715803538350409	-0.232578668243255\\
-0.731730559897045	-0.115894735197653\\
-0.729247614287671	-8.9307075662324e-17\\
};
\addplot [color=mycolor1,dotted,line width=1.0pt,forget plot]
  table[row sep=crcr]{1	-0\\
0.956521682769877	-0.151498151383791\\
0.891982039736211	-0.289822533396298\\
0.809292583610082	-0.412355167436434\\
0.711634858347723	-0.517032989000692\\
0.602364630186427	-0.602364630186426\\
0.484917701774751	-0.667431957639199\\
0.362719588349683	-0.711877274647591\\
0.239101059900595	-0.735877395777214\\
0.117221283062812	-0.740106053489981\\
4.44354699422903e-17	-0.725686295399261\\
-0.109940132237539	-0.694134676438339\\
-0.210320240583588	-0.647299141978847\\
-0.299240493845688	-0.587292536894097\\
-0.375203754387119	-0.516423664031337\\
-0.437127756959533	-0.437127756959533\\
-0.484346224770267	-0.351898130574443\\
-0.516599582217932	-0.263220634338226\\
-0.534016141209622	-0.173512362385299\\
-0.537084820159711	-0.0850658786478001\\
-0.526620599330303	-6.44924231334916e-17\\
};
\addplot [color=mycolor1,dotted,line width=1.0pt,forget plot]
  table[row sep=crcr]{1	-0\\
0.940082788364644	-0.148894486293864\\
0.86158608093073	-0.279946287694513\\
0.768279786681378	-0.391458103646313\\
0.663960650859636	-0.482395649771645\\
0.552351927561387	-0.552351927561387\\
0.437014383712816	-0.601498696728173\\
0.321269860940431	-0.630527604183869\\
0.208138182971344	-0.640583459188477\\
0.100287778328637	-0.633192112325829\\
3.73630739569174e-17	-0.610185303761559\\
-0.0908532273476425	-0.573624701779294\\
-0.170819110593797	-0.525727164509449\\
-0.238861981883349	-0.468793035010043\\
-0.294350736089166	-0.40513903143051\\
-0.337037028702328	-0.337037028702328\\
-0.367025664975262	-0.266659754473976\\
-0.384738677688982	-0.196034147687371\\
-0.390874612629551	-0.127002860400749\\
-0.386364521398959	-0.0611941284830315\\
-0.372326104926586	-4.55967972637346e-17\\
};
\addplot [color=mycolor1,dotted,line width=1.0pt,forget plot]
  table[row sep=crcr]{1	-0\\
0.922246029428501	-0.146069421212559\\
0.829201462983264	-0.269423887468404\\
0.725373165529273	-0.369596088217499\\
0.614985835074999	-0.446813363302878\\
0.501902475185001	-0.501902475185001\\
0.38956496084428	-0.536190168955128\\
0.280953750703967	-0.551402782692681\\
0.178565395716864	-0.549567778706978\\
0.0844061798404073	-0.532919645815306\\
3.08495992433718e-17	-0.50381218919366\\
-0.0735915548753502	-0.464638791061534\\
-0.135739038566523	-0.417761804348184\\
-0.186207014322272	-0.365451842507638\\
-0.225110018726605	-0.309837359892227\\
-0.252864584784672	-0.252864584784672\\
-0.270139142845401	-0.196267575756099\\
-0.277803443075876	-0.141547924204336\\
-0.27687894570066	-0.0899634229303457\\
-0.268491413422519	-0.0425248622468694\\
-0.253826721980109	-3.10848082611005e-17\\
};
\addplot [color=mycolor1,dotted,line width=1.0pt,forget plot]
  table[row sep=crcr]{1	-0\\
0.902056570675584	-0.142871725087523\\
0.793293726869509	-0.257756756757911\\
0.678769666307395	-0.345850419328459\\
0.562876391436159	-0.408953636388562\\
0.449318435861499	-0.449318435861499\\
0.341115747349381	-0.469505547439701\\
0.24062663375655	-0.472256359314945\\
0.149586755320764	-0.460380694230177\\
0.069160331541479	-0.436661148025441\\
2.47240337905553e-17	-0.403774113610049\\
-0.057687904157153	-0.364227092250654\\
-0.104075505986926	-0.320311471399148\\
-0.139645446506012	-0.274069620358657\\
-0.165124892040398	-0.227274916024158\\
-0.18142315316863	-0.18142315316863\\
-0.189574186252466	-0.137733708524426\\
-0.190684892879101	-0.0971588057560207\\
-0.185889806735969	-0.0603992595374446\\
-0.176312467991919	-0.0279251515651378\\
-0.16303353482158	-1.99658496572927e-17\\
};
\addplot [color=mycolor1,dotted,line width=1.0pt,forget plot]
  table[row sep=crcr]{1	-0\\
0.877921760602431	-0.139049146701748\\
0.751411952540787	-0.244148543365328\\
0.625732257688895	-0.318826509862098\\
0.505011411193747	-0.36691226736026\\
0.392341669548685	-0.392341669548685\\
0.289890514921595	-0.399000063654162\\
0.199020572856858	-0.390599867100522\\
0.120411913496453	-0.370589763847805\\
0.0541820486548744	-0.34209199176291\\
1.88512313530119e-17	-0.307863971328499\\
-0.042808222513439	-0.270280479734772\\
-0.075164540154518	-0.231332667812908\\
-0.0981551954909503	-0.192640417840451\\
-0.112959067758674	-0.155474818616317\\
-0.120787864504912	-0.120787864504912\\
-0.122837744156894	-0.0892468451742257\\
-0.120251626285951	-0.0612712639357851\\
-0.114091236431876	-0.0370704898842818\\
-0.105317799143806	-0.0166807006736036\\
-0.0947802248421549	-1.16072298975411e-17\\
};
\addplot [color=mycolor1,dotted,line width=1.0pt,forget plot]
  table[row sep=crcr]{1	-0\\
0.84674396664984	-0.134111069256013\\
0.698989566160914	-0.227115477506975\\
0.561406498364893	-0.286050898428465\\
0.437004973478602	-0.317502698182059\\
0.327450698344114	-0.327450698344114\\
0.233352139914598	-0.321181666481713\\
0.154515499860225	-0.303253743289057\\
0.0901653834921457	-0.27750051639687\\
0.0391311034995635	-0.247064063991275\\
1.31311479217367e-17	-0.214447919692096\\
-0.0287598398096346	-0.181582482159892\\
-0.0487044416812479	-0.149896858349689\\
-0.0613430200940395	-0.120392455675976\\
-0.06808779312177	-0.0937148074575858\\
-0.0702211210616195	-0.0702211210616195\\
-0.068876740220244	-0.0500418809603846\\
-0.0650321343871793	-0.0331355275052096\\
-0.0595094084547913	-0.0193357789181307\\
-0.0529823745536039	-0.00839158374073751\\
-0.0459879102602678	-5.63189470997126e-18\\
};
\addplot [color=mycolor1,dotted,line width=1.0pt,forget plot]
  table[row sep=crcr]{1	-0\\
0.801053465278425	-0.126874404768154\\
0.625589539649299	-0.203266363189334\\
0.475341369738971	-0.242198525079547\\
0.350045057714404	-0.254322621147605\\
0.24813777530853	-0.24813777530853\\
0.167289292234614	-0.230253957320141\\
0.104794333468008	-0.205670459781722\\
0.0578515468028918	-0.178048753195384\\
0.0237523524567972	-0.149966451301199\\
7.54043881219821e-18	-0.123144711070133\\
-0.0156239119173694	-0.0986454975334405\\
-0.0250311775652273	-0.0770380431086105\\
-0.0298254673925332	-0.0585357756361869\\
-0.0313184856998208	-0.0431061974937683\\
-0.0305568546459545	-0.0305568546459545\\
-0.0283545570469435	-0.0206007915573586\\
-0.0253272293454815	-0.012904867916705\\
-0.021925762268643	-0.00712411201600239\\
-0.0184675753156993	-0.00292497658052826\\
-0.0151646198645466	-1.85713031774033e-18\\
};
\addplot [color=mycolor1,dotted,line width=1.0pt,forget plot]
  table[row sep=crcr]{1	-0\\
0.714110955679367	-0.113104064044896\\
0.497161927717827	-0.161537702532642\\
0.336758208677056	-0.171586877647116\\
0.221075553032581	-0.160620791180475\\
0.139705610200823	-0.139705610200823\\
0.0839640345306934	-0.115566579097864\\
0.0468885871776745	-0.0920240337831981\\
0.0230753615892199	-0.0710186604775083\\
0.00844586939409394	-0.0533251206797082\\
2.39022368106624e-18	-0.0390353150431684\\
-0.00441505277265522	-0.0278755461307222\\
-0.00630567510971132	-0.0194068724759343\\
-0.00669794782622876	-0.0131454627690819\\
-0.0062698831862128	-0.00862975386096949\\
-0.0054534525074872	-0.0054534525074872\\
-0.00451117782354635	-0.00327756254020109\\
-0.00359218715027031	-0.00183031077240959\\
-0.00277223578568335	-0.000900754009370227\\
-0.00208156288544739	-0.000329687172611911\\
-0.00152375582051941	-1.86606268828125e-19\\
};
\addplot [color=mycolor1,dotted,line width=1.0pt,forget plot]
  table[row sep=crcr]{1	0\\
1	0\\
};
\addplot [color=mycolor1,dotted,line width=1.0pt,forget plot]
  table[row sep=crcr]{0.951056516295154	0.309016994374947\\
0.906577591518048	0.290693092532446\\
0.867277205719189	0.267124444629623\\
0.833330533437629	0.239553777474139\\
0.804679893747523	0.20903820245934\\
0.781122098933164	0.176433510213537\\
0.762382187756933	0.142402376999381\\
0.748171074803624	0.107437628337747\\
0.738227708485823	0.0718935522249012\\
0.732347931289196	0.0360204899377399\\
0.730402691048646	2.81010850277162e-17\\
};
\addplot [color=mycolor1,dotted,line width=1.0pt,forget plot]
  table[row sep=crcr]{0.809016994374947	0.587785252292473\\
0.737380455396588	0.527071687397996\\
0.6808142826414	0.463341883835339\\
0.637053765657314	0.399254954339046\\
0.603812761314093	0.336417677088311\\
0.579022949915482	0.275632227640288\\
0.560948163233974	0.217130071437151\\
0.54821711318997	0.160763451735608\\
0.539811466724715	0.106147624627789\\
0.535036016768209	0.0527590625798542\\
0.533488091091103	4.10502162512614e-17\\
};
\addplot [color=mycolor1,dotted,line width=1.0pt,forget plot]
  table[row sep=crcr]{0.587785252292473	0.809016994374947\\
0.515276498489902	0.692182785870847\\
0.46668476526979	0.583707991451876\\
0.435233321876477	0.485319980094308\\
0.415551842123536	0.396928474901319\\
0.403656860517901	0.317461475735791\\
0.396737049613855	0.245416450708029\\
0.392888142823216	0.179177710929467\\
0.39087245229983	0.11717008156476\\
0.389932112762627	0.0579102497954412\\
0.389661137375347	4.49747826270753e-17\\
};
\addplot [color=mycolor1,dotted,line width=1.0pt,forget plot]
  table[row sep=crcr]{0.309016994374947	0.951056516295154\\
0.265925372344309	0.777304721760365\\
0.248822386132444	0.630899544522142\\
0.24643298177389	0.508693744238057\\
0.251412997268255	0.406266573115132\\
0.25929445161488	0.31919477108011\\
0.267517373913267	0.243597429511063\\
0.274697115780397	0.1762665508339\\
0.280129101393366	0.114599409879343\\
0.283480020554887	0.0564559973822998\\
0.284609543336029	4.37996030135249e-17\\
};
\addplot [color=mycolor1,dotted,line width=1.0pt,forget plot]
  table[row sep=crcr]{6.12323399573677e-17	1\\
0.0151248701748503	0.781989711398728\\
0.0472692933177679	0.613631335769719\\
0.0835006201485716	0.482943980920436\\
0.117381749765263	0.379469463911326\\
0.146223972383669	0.295078319831894\\
0.169261627793672	0.223809451176491\\
0.186582776182006	0.161390341419992\\
0.198560105942714	0.104739935929793\\
0.205572433929556	0.0515565221197431\\
0.207879576350762	3.99891848849262e-17\\
};
\addplot [color=mycolor1,dotted,line width=1.0pt,forget plot]
  table[row sep=crcr]{-0.309016994374947	0.951056516295154\\
-0.213607139159912	0.713331044437031\\
-0.122920349149865	0.544815253953639\\
-0.0461074386071066	0.422454854218942\\
0.0151311193047769	0.329888717873058\\
0.0621570724668225	0.256291005261778\\
0.0971710522581798	0.194731597161215\\
0.122256440677152	0.140793596164787\\
0.13905244595299	0.0915971142347777\\
0.148693455532156	0.0451621321066958\\
0.151835801980649	3.50498499033503e-17\\
};
\addplot [color=mycolor1,dotted,line width=1.0pt,forget plot]
  table[row sep=crcr]{-0.587785252292473	0.809016994374948\\
-0.401011853057454	0.584595820351374\\
-0.252139489074835	0.439670821081765\\
-0.139623392550337	0.340999317931598\\
-0.0567836371213516	0.268617800427267\\
0.0033339412143336	0.21116115844769\\
0.0463512370945915	0.162297289886265\\
0.0763422025660035	0.118472638203443\\
0.0960671266193367	0.0776165020305757\\
0.1072685824301	0.0384304051397518\\
0.110901278364195	2.98672554719679e-17\\
};
\addplot [color=mycolor1,dotted,line width=1.0pt,forget plot]
  table[row sep=crcr]{-0.809016994374948	0.587785252292473\\
-0.533486326814499	0.413410095118228\\
-0.336121655437603	0.313963860155743\\
-0.198040110920963	0.250717832404618\\
-0.101844033235305	0.204281393673556\\
-0.0346516892466227	0.165530866251108\\
0.0122258376810533	0.130333089269644\\
0.0443886084751889	0.0968398262452624\\
0.065339288702761	0.0642052594194206\\
0.0771736424133687	0.032008294596762\\
0.0810025921579431	2.49315700239574e-17\\
};
\addplot [color=mycolor1,dotted,line width=1.0pt,forget plot]
  table[row sep=crcr]{-0.951056516295154	0.309016994374948\\
-0.60382220830535	0.219707538176049\\
-0.375378071887508	0.182507388795924\\
-0.225093275108467	0.161489568357552\\
-0.124654461172013	0.143091836517116\\
-0.0562723920172722	0.12318609851568\\
-0.00923898083522721	0.101104614081101\\
0.0228060316514889	0.0772217637055013\\
0.0436193292019494	0.0520955750986265\\
0.0553650029180358	0.0262210407420432\\
0.0591645112940776	2.0486346567262e-17\\
};
\addplot [color=mycolor1,dotted,line width=1.0pt,forget plot]
  table[row sep=crcr]{-1	1.22464679914735e-16\\
-0.61127914703566	0.0236549857259488\\
-0.374309030147768	0.0580118391989452\\
-0.226262535142082	0.0806522438077526\\
-0.130218598863194	0.0890855793127953\\
-0.0656897647351535	0.0862950481802363\\
-0.0214411717925588	0.0757647040434823\\
0.00876629006412298	0.0602253159022079\\
0.0284556614934045	0.0415943455493057\\
0.039601950618638	0.0211971994741971\\
0.0432139182637723	1.66258696249815e-17\\
};
\addplot [color=mycolor1,dotted,line width=1.0pt,forget plot]
  table[row sep=crcr]{1	-0\\
1	-0\\
};
\addplot [color=mycolor1,dotted,line width=1.0pt,forget plot]
  table[row sep=crcr]{0.951056516295154	-0.309016994374947\\
0.906577591518048	-0.290693092532446\\
0.867277205719189	-0.267124444629623\\
0.833330533437629	-0.239553777474139\\
0.804679893747523	-0.20903820245934\\
0.781122098933164	-0.176433510213537\\
0.762382187756933	-0.142402376999381\\
0.748171074803624	-0.107437628337747\\
0.738227708485823	-0.0718935522249012\\
0.732347931289196	-0.0360204899377399\\
0.730402691048646	-2.81010850277162e-17\\
};
\addplot [color=mycolor1,dotted,line width=1.0pt,forget plot]
  table[row sep=crcr]{0.809016994374947	-0.587785252292473\\
0.737380455396588	-0.527071687397996\\
0.6808142826414	-0.463341883835339\\
0.637053765657314	-0.399254954339046\\
0.603812761314093	-0.336417677088311\\
0.579022949915482	-0.275632227640288\\
0.560948163233974	-0.217130071437151\\
0.54821711318997	-0.160763451735608\\
0.539811466724715	-0.106147624627789\\
0.535036016768209	-0.0527590625798542\\
0.533488091091103	-4.10502162512614e-17\\
};
\addplot [color=mycolor1,dotted,line width=1.0pt,forget plot]
  table[row sep=crcr]{0.587785252292473	-0.809016994374947\\
0.515276498489902	-0.692182785870847\\
0.46668476526979	-0.583707991451876\\
0.435233321876477	-0.485319980094308\\
0.415551842123536	-0.396928474901319\\
0.403656860517901	-0.317461475735791\\
0.396737049613855	-0.245416450708029\\
0.392888142823216	-0.179177710929467\\
0.39087245229983	-0.11717008156476\\
0.389932112762627	-0.0579102497954412\\
0.389661137375347	-4.49747826270753e-17\\
};
\addplot [color=mycolor1,dotted,line width=1.0pt,forget plot]
  table[row sep=crcr]{0.309016994374947	-0.951056516295154\\
0.265925372344309	-0.777304721760365\\
0.248822386132444	-0.630899544522142\\
0.24643298177389	-0.508693744238057\\
0.251412997268255	-0.406266573115132\\
0.25929445161488	-0.31919477108011\\
0.267517373913267	-0.243597429511063\\
0.274697115780397	-0.1762665508339\\
0.280129101393366	-0.114599409879343\\
0.283480020554887	-0.0564559973822998\\
0.284609543336029	-4.37996030135249e-17\\
};
\addplot [color=mycolor1,dotted,line width=1.0pt,forget plot]
  table[row sep=crcr]{6.12323399573677e-17	-1\\
0.0151248701748503	-0.781989711398728\\
0.0472692933177679	-0.613631335769719\\
0.0835006201485716	-0.482943980920436\\
0.117381749765263	-0.379469463911326\\
0.146223972383669	-0.295078319831894\\
0.169261627793672	-0.223809451176491\\
0.186582776182006	-0.161390341419992\\
0.198560105942714	-0.104739935929793\\
0.205572433929556	-0.0515565221197431\\
0.207879576350762	-3.99891848849262e-17\\
};
\addplot [color=mycolor1,dotted,line width=1.0pt,forget plot]
  table[row sep=crcr]{-0.309016994374947	-0.951056516295154\\
-0.213607139159912	-0.713331044437031\\
-0.122920349149865	-0.544815253953639\\
-0.0461074386071066	-0.422454854218942\\
0.0151311193047769	-0.329888717873058\\
0.0621570724668225	-0.256291005261778\\
0.0971710522581798	-0.194731597161215\\
0.122256440677152	-0.140793596164787\\
0.13905244595299	-0.0915971142347777\\
0.148693455532156	-0.0451621321066958\\
0.151835801980649	-3.50498499033503e-17\\
};
\addplot [color=mycolor1,dotted,line width=1.0pt,forget plot]
  table[row sep=crcr]{-0.587785252292473	-0.809016994374948\\
-0.401011853057454	-0.584595820351374\\
-0.252139489074835	-0.439670821081765\\
-0.139623392550337	-0.340999317931598\\
-0.0567836371213516	-0.268617800427267\\
0.0033339412143336	-0.21116115844769\\
0.0463512370945915	-0.162297289886265\\
0.0763422025660035	-0.118472638203443\\
0.0960671266193367	-0.0776165020305757\\
0.1072685824301	-0.0384304051397518\\
0.110901278364195	-2.98672554719679e-17\\
};
\addplot [color=mycolor1,dotted,line width=1.0pt,forget plot]
  table[row sep=crcr]{-0.809016994374948	-0.587785252292473\\
-0.533486326814499	-0.413410095118228\\
-0.336121655437603	-0.313963860155743\\
-0.198040110920963	-0.250717832404618\\
-0.101844033235305	-0.204281393673556\\
-0.0346516892466227	-0.165530866251108\\
0.0122258376810533	-0.130333089269644\\
0.0443886084751889	-0.0968398262452624\\
0.065339288702761	-0.0642052594194206\\
0.0771736424133687	-0.032008294596762\\
0.0810025921579431	-2.49315700239574e-17\\
};
\addplot [color=mycolor1,dotted,line width=1.0pt,forget plot]
  table[row sep=crcr]{-0.951056516295154	-0.309016994374948\\
-0.60382220830535	-0.219707538176049\\
-0.375378071887508	-0.182507388795924\\
-0.225093275108467	-0.161489568357552\\
-0.124654461172013	-0.143091836517116\\
-0.0562723920172722	-0.12318609851568\\
-0.00923898083522721	-0.101104614081101\\
0.0228060316514889	-0.0772217637055013\\
0.0436193292019494	-0.0520955750986265\\
0.0553650029180358	-0.0262210407420432\\
0.0591645112940776	-2.0486346567262e-17\\
};
\addplot [color=mycolor1,dotted,line width=1.0pt,forget plot]
  table[row sep=crcr]{-1	-1.22464679914735e-16\\
-0.61127914703566	-0.0236549857259488\\
-0.374309030147768	-0.0580118391989452\\
-0.226262535142082	-0.0806522438077526\\
-0.130218598863194	-0.0890855793127953\\
-0.0656897647351535	-0.0862950481802363\\
-0.0214411717925588	-0.0757647040434823\\
0.00876629006412298	-0.0602253159022079\\
0.0284556614934045	-0.0415943455493057\\
0.039601950618638	-0.0211971994741971\\
0.0432139182637723	-1.66258696249815e-17\\
};
\addplot [color=mycolor1,line width=1.5pt,mark size=4.0pt,only marks,mark=x,mark options={solid},forget plot]
  table[row sep=crcr]{0	0\\
0.91444306659383	0.404714563561125\\
-0.124253880684544	0\\
0.89755799120607	0.418082106216459\\
-0.168612823306847	0\\
0.880031509804686	0.432694736532765\\
-0.20035214433653	0\\
0.862049033868094	0.448898382097043\\
-0.225714039094344	0\\
0.843968536244412	0.467035231637572\\
-0.247083156478973	0\\
0.826353726809843	0.487334319047914\\
-0.265667627304716	0\\
0.80991520639231	0.509761287368884\\
-0.28217711202824	0\\
0.795322964447454	0.533928890391685\\
-0.297069157418281	0\\
0.78298488576272	0.559181443716326\\
-0.310658437059747	0\\
0.772959683909908	0.584813903534905\\
-0.323171911393944	0\\
0.765041430382214	0.610258212917158\\
-0.334779380319558	0\\
0.758904835963185	0.635145159493358\\
-0.345611626818817	0\\
0.754213373942793	0.659276394587029\\
};
\addplot [color=black!50!mycolor1,line width=1.5pt,mark size=4.0pt,only marks,mark=x,mark options={solid},forget plot]
  table[row sep=crcr]{0	0\\
0.91444306659383	-0.404714563561125\\
0.158024031460067	0\\
0.89755799120607	-0.418082106216459\\
0.237435936885133	0\\
0.880031509804686	-0.432694736532765\\
0.305140209788	0\\
0.862049033868094	-0.448898382097043\\
0.36666309979318	0\\
0.843968536244412	-0.467035231637572\\
0.423261836046949	0\\
0.826353726809843	-0.487334319047914\\
0.474723347707758	0\\
0.80991520639231	-0.509761287368884\\
0.520417316320995	0\\
0.795322964447454	-0.533928890391685\\
0.559985519080503	0\\
0.78298488576272	-0.559181443716326\\
0.593625202427589	0\\
0.772959683909908	-0.584813903534905\\
0.621975183817178	0\\
0.765041430382214	-0.610258212917158\\
0.645855841580849	0\\
0.758904835963185	-0.635145159493358\\
0.666071012120892	0\\
};
\addplot [color=black,line width=1.5pt,mark size=4.0pt,only marks,mark=o,mark options={solid},forget plot]
  table[row sep=crcr]{-1	0\\
0.9	0\\
};
\end{axis}
\end{tikzpicture}%}
        \renewcommand\figurename{Fig.}
        \caption{Lugar das raízes para o controlador proporcional-derivativo.}
        \label{fig:rlocus_vc_2}
    \end{figure}

    A partir da Fig.~\ref{fig:rlocus_vc_2} se pode escolher $K_P+K_D = 3$ para máximo amortecimento, escolha que resulta em $K_P = 0,3$ e $K_D = 2,7$.


\subsection{Projeto para $U_p = I_C$}

    Considerando o controlador da malha interna $C_i(z)$ como sendo do tipo proporcional, sua expressão é dada por:
    %
    \begin{equation}
        C_i(z) = K_P
    \end{equation}

    Tendo como referência a função de transferência de malha fechada $I_C/U$, que pode ser vista na Fig.~\ref{fig:LCL_discreto} fazendo $U_p = I_C$, observa-se que sua equação característica é:
    %
    \begin{equation}
        z^3 - 2 \cos \left( \omega_n T_s \right) z^2 + \left( 1 + K_P K_{id} \right) z - K_P K_{id} = 0
    \end{equation}
    %
    com
    %
    \begin{equation}
        K_{id} = \frac{\sin(\omega_n T_s)}{\omega_n L_1}
    \end{equation}

    Através do critério de estabilidade de Routh-Hurwitz, pode-se projetar o ganho $K_P$
    %
    \begin{equation}
        \overline{K_P} = \frac{2 \cos \left( \omega_n T_s \right) - 1}
            {\sen \left( \omega_n T_s \right)} \omega_n L_1
    \end{equation}
    %
    onde $\overline{K_P}$ representa o limite superior para o valor de $K_P$.

    \begin{figure}[htb]
        \centering{
            \def\svgwidth{\textwidth}
            % This file was created by matlab2tikz v0.4.7 running on MATLAB 7.14.
% Copyright (c) 2008--2014, Nico Schlömer <nico.schloemer@gmail.com>
% All rights reserved.
% Minimal pgfplots version: 1.3
% 
% The latest updates can be retrieved from
%   http://www.mathworks.com/matlabcentral/fileexchange/22022-matlab2tikz
% where you can also make suggestions and rate matlab2tikz.
% 
%
% defining custom colors
\definecolor{mycolor1}{rgb}{0.66667,0.66667,0.66667}%
%
\begin{tikzpicture}

\begin{axis}[%
width=0.8\textwidth,
height=0.461611624834875\textwidth,
scale only axis,
xmin=-1.9,
xmax=1.9,
xtick={-1,  0,  1},
xlabel={Eixo Real},
ymin=-1.1,
ymax=1.1,
ytick={-1,  0,  1},
ylabel={Eixo Imaginário},
scaled y ticks = false,
y tick label style={/pgf/number format/.cd, fixed, fixed zerofill, precision=0},
scaled x ticks = false,
x tick label style={/pgf/number format/.cd, fixed, fixed zerofill, precision=0}
]
\addplot [color=mycolor1,dotted,line width=1.0pt,forget plot]
  table[row sep=crcr]{1	0\\
0.987688340595138	0.156434465040231\\
0.951056516295154	0.309016994374947\\
0.891006524188368	0.453990499739547\\
0.809016994374947	0.587785252292473\\
0.707106781186548	0.707106781186547\\
0.587785252292473	0.809016994374947\\
0.453990499739547	0.891006524188368\\
0.309016994374947	0.951056516295154\\
0.156434465040231	0.987688340595138\\
6.12323399573677e-17	1\\
-0.156434465040231	0.987688340595138\\
-0.309016994374947	0.951056516295154\\
-0.453990499739547	0.891006524188368\\
-0.587785252292473	0.809016994374947\\
-0.707106781186547	0.707106781186548\\
-0.809016994374947	0.587785252292473\\
-0.891006524188368	0.453990499739547\\
-0.951056516295154	0.309016994374948\\
-0.987688340595138	0.156434465040231\\
-1	1.22464679914735e-16\\
};
\addplot [color=mycolor1,dotted,line width=1.0pt,forget plot]
  table[row sep=crcr]{1	0\\
0.972218045703082	0.153984211042097\\
0.921496791140463	0.299412457456957\\
0.849791018366557	0.432990150609548\\
0.759508516401756	0.551815237548133\\
0.653437051516106	0.653437051516106\\
0.534664304371591	0.735902282058355\\
0.406492952796512	0.797787339572243\\
0.272353130096494	0.83821674479613\\
0.135714483753824	0.856867527364047\\
5.22899666167123e-17	0.853959960588124\\
-0.131496350792703	0.830235283991667\\
-0.255686250369537	0.786921363444393\\
-0.369756251949576	0.725687504552447\\
-0.47122811850853	0.648589862732789\\
-0.558009078759514	0.558009078759514\\
-0.628431083783263	0.456581908289041\\
-0.681278445058817	0.347128705949459\\
-0.715803538350409	0.232578668243255\\
-0.731730559897045	0.115894735197653\\
-0.729247614287671	8.9307075662324e-17\\
};
\addplot [color=mycolor1,dotted,line width=1.0pt,forget plot]
  table[row sep=crcr]{1	0\\
0.956521682769877	0.151498151383791\\
0.891982039736211	0.289822533396298\\
0.809292583610082	0.412355167436434\\
0.711634858347723	0.517032989000692\\
0.602364630186427	0.602364630186426\\
0.484917701774751	0.667431957639199\\
0.362719588349683	0.711877274647591\\
0.239101059900595	0.735877395777214\\
0.117221283062812	0.740106053489981\\
4.44354699422903e-17	0.725686295399261\\
-0.109940132237539	0.694134676438339\\
-0.210320240583588	0.647299141978847\\
-0.299240493845688	0.587292536894097\\
-0.375203754387119	0.516423664031337\\
-0.437127756959533	0.437127756959533\\
-0.484346224770267	0.351898130574443\\
-0.516599582217932	0.263220634338226\\
-0.534016141209622	0.173512362385299\\
-0.537084820159711	0.0850658786478001\\
-0.526620599330303	6.44924231334916e-17\\
};
\addplot [color=mycolor1,dotted,line width=1.0pt,forget plot]
  table[row sep=crcr]{1	0\\
0.940082788364644	0.148894486293864\\
0.86158608093073	0.279946287694513\\
0.768279786681378	0.391458103646313\\
0.663960650859636	0.482395649771645\\
0.552351927561387	0.552351927561387\\
0.437014383712816	0.601498696728173\\
0.321269860940431	0.630527604183869\\
0.208138182971344	0.640583459188477\\
0.100287778328637	0.633192112325829\\
3.73630739569174e-17	0.610185303761559\\
-0.0908532273476425	0.573624701779294\\
-0.170819110593797	0.525727164509449\\
-0.238861981883349	0.468793035010043\\
-0.294350736089166	0.40513903143051\\
-0.337037028702328	0.337037028702328\\
-0.367025664975262	0.266659754473976\\
-0.384738677688982	0.196034147687371\\
-0.390874612629551	0.127002860400749\\
-0.386364521398959	0.0611941284830315\\
-0.372326104926586	4.55967972637346e-17\\
};
\addplot [color=mycolor1,dotted,line width=1.0pt,forget plot]
  table[row sep=crcr]{1	0\\
0.922246029428501	0.146069421212559\\
0.829201462983264	0.269423887468404\\
0.725373165529273	0.369596088217499\\
0.614985835074999	0.446813363302878\\
0.501902475185001	0.501902475185001\\
0.38956496084428	0.536190168955128\\
0.280953750703967	0.551402782692681\\
0.178565395716864	0.549567778706978\\
0.0844061798404073	0.532919645815306\\
3.08495992433718e-17	0.50381218919366\\
-0.0735915548753502	0.464638791061534\\
-0.135739038566523	0.417761804348184\\
-0.186207014322272	0.365451842507638\\
-0.225110018726605	0.309837359892227\\
-0.252864584784672	0.252864584784672\\
-0.270139142845401	0.196267575756099\\
-0.277803443075876	0.141547924204336\\
-0.27687894570066	0.0899634229303457\\
-0.268491413422519	0.0425248622468694\\
-0.253826721980109	3.10848082611005e-17\\
};
\addplot [color=mycolor1,dotted,line width=1.0pt,forget plot]
  table[row sep=crcr]{1	0\\
0.902056570675584	0.142871725087523\\
0.793293726869509	0.257756756757911\\
0.678769666307395	0.345850419328459\\
0.562876391436159	0.408953636388562\\
0.449318435861499	0.449318435861499\\
0.341115747349381	0.469505547439701\\
0.24062663375655	0.472256359314945\\
0.149586755320764	0.460380694230177\\
0.069160331541479	0.436661148025441\\
2.47240337905553e-17	0.403774113610049\\
-0.057687904157153	0.364227092250654\\
-0.104075505986926	0.320311471399148\\
-0.139645446506012	0.274069620358657\\
-0.165124892040398	0.227274916024158\\
-0.18142315316863	0.18142315316863\\
-0.189574186252466	0.137733708524426\\
-0.190684892879101	0.0971588057560207\\
-0.185889806735969	0.0603992595374446\\
-0.176312467991919	0.0279251515651378\\
-0.16303353482158	1.99658496572927e-17\\
};
\addplot [color=mycolor1,dotted,line width=1.0pt,forget plot]
  table[row sep=crcr]{1	0\\
0.877921760602431	0.139049146701748\\
0.751411952540787	0.244148543365328\\
0.625732257688895	0.318826509862098\\
0.505011411193747	0.36691226736026\\
0.392341669548685	0.392341669548685\\
0.289890514921595	0.399000063654162\\
0.199020572856858	0.390599867100522\\
0.120411913496453	0.370589763847805\\
0.0541820486548744	0.34209199176291\\
1.88512313530119e-17	0.307863971328499\\
-0.042808222513439	0.270280479734772\\
-0.075164540154518	0.231332667812908\\
-0.0981551954909503	0.192640417840451\\
-0.112959067758674	0.155474818616317\\
-0.120787864504912	0.120787864504912\\
-0.122837744156894	0.0892468451742257\\
-0.120251626285951	0.0612712639357851\\
-0.114091236431876	0.0370704898842818\\
-0.105317799143806	0.0166807006736036\\
-0.0947802248421549	1.16072298975411e-17\\
};
\addplot [color=mycolor1,dotted,line width=1.0pt,forget plot]
  table[row sep=crcr]{1	0\\
0.84674396664984	0.134111069256013\\
0.698989566160914	0.227115477506975\\
0.561406498364893	0.286050898428465\\
0.437004973478602	0.317502698182059\\
0.327450698344114	0.327450698344114\\
0.233352139914598	0.321181666481713\\
0.154515499860225	0.303253743289057\\
0.0901653834921457	0.27750051639687\\
0.0391311034995635	0.247064063991275\\
1.31311479217367e-17	0.214447919692096\\
-0.0287598398096346	0.181582482159892\\
-0.0487044416812479	0.149896858349689\\
-0.0613430200940395	0.120392455675976\\
-0.06808779312177	0.0937148074575858\\
-0.0702211210616195	0.0702211210616195\\
-0.068876740220244	0.0500418809603846\\
-0.0650321343871793	0.0331355275052096\\
-0.0595094084547913	0.0193357789181307\\
-0.0529823745536039	0.00839158374073751\\
-0.0459879102602678	5.63189470997126e-18\\
};
\addplot [color=mycolor1,dotted,line width=1.0pt,forget plot]
  table[row sep=crcr]{1	0\\
0.801053465278425	0.126874404768154\\
0.625589539649299	0.203266363189334\\
0.475341369738971	0.242198525079547\\
0.350045057714404	0.254322621147605\\
0.24813777530853	0.24813777530853\\
0.167289292234614	0.230253957320141\\
0.104794333468008	0.205670459781722\\
0.0578515468028918	0.178048753195384\\
0.0237523524567972	0.149966451301199\\
7.54043881219821e-18	0.123144711070133\\
-0.0156239119173694	0.0986454975334405\\
-0.0250311775652273	0.0770380431086105\\
-0.0298254673925332	0.0585357756361869\\
-0.0313184856998208	0.0431061974937683\\
-0.0305568546459545	0.0305568546459545\\
-0.0283545570469435	0.0206007915573586\\
-0.0253272293454815	0.012904867916705\\
-0.021925762268643	0.00712411201600239\\
-0.0184675753156993	0.00292497658052826\\
-0.0151646198645466	1.85713031774033e-18\\
};
\addplot [color=mycolor1,dotted,line width=1.0pt,forget plot]
  table[row sep=crcr]{1	0\\
0.714110955679367	0.113104064044896\\
0.497161927717827	0.161537702532642\\
0.336758208677056	0.171586877647116\\
0.221075553032581	0.160620791180475\\
0.139705610200823	0.139705610200823\\
0.0839640345306934	0.115566579097864\\
0.0468885871776745	0.0920240337831981\\
0.0230753615892199	0.0710186604775083\\
0.00844586939409394	0.0533251206797082\\
2.39022368106624e-18	0.0390353150431684\\
-0.00441505277265522	0.0278755461307222\\
-0.00630567510971132	0.0194068724759343\\
-0.00669794782622876	0.0131454627690819\\
-0.0062698831862128	0.00862975386096949\\
-0.0054534525074872	0.0054534525074872\\
-0.00451117782354635	0.00327756254020109\\
-0.00359218715027031	0.00183031077240959\\
-0.00277223578568335	0.000900754009370227\\
-0.00208156288544739	0.000329687172611911\\
-0.00152375582051941	1.86606268828125e-19\\
};
\addplot [color=mycolor1,dotted,line width=1.0pt,forget plot]
  table[row sep=crcr]{1	-0\\
0.987688340595138	-0.156434465040231\\
0.951056516295154	-0.309016994374947\\
0.891006524188368	-0.453990499739547\\
0.809016994374947	-0.587785252292473\\
0.707106781186548	-0.707106781186547\\
0.587785252292473	-0.809016994374947\\
0.453990499739547	-0.891006524188368\\
0.309016994374947	-0.951056516295154\\
0.156434465040231	-0.987688340595138\\
6.12323399573677e-17	-1\\
-0.156434465040231	-0.987688340595138\\
-0.309016994374947	-0.951056516295154\\
-0.453990499739547	-0.891006524188368\\
-0.587785252292473	-0.809016994374947\\
-0.707106781186547	-0.707106781186548\\
-0.809016994374947	-0.587785252292473\\
-0.891006524188368	-0.453990499739547\\
-0.951056516295154	-0.309016994374948\\
-0.987688340595138	-0.156434465040231\\
-1	-1.22464679914735e-16\\
};
\addplot [color=mycolor1,dotted,line width=1.0pt,forget plot]
  table[row sep=crcr]{1	-0\\
0.972218045703082	-0.153984211042097\\
0.921496791140463	-0.299412457456957\\
0.849791018366557	-0.432990150609548\\
0.759508516401756	-0.551815237548133\\
0.653437051516106	-0.653437051516106\\
0.534664304371591	-0.735902282058355\\
0.406492952796512	-0.797787339572243\\
0.272353130096494	-0.83821674479613\\
0.135714483753824	-0.856867527364047\\
5.22899666167123e-17	-0.853959960588124\\
-0.131496350792703	-0.830235283991667\\
-0.255686250369537	-0.786921363444393\\
-0.369756251949576	-0.725687504552447\\
-0.47122811850853	-0.648589862732789\\
-0.558009078759514	-0.558009078759514\\
-0.628431083783263	-0.456581908289041\\
-0.681278445058817	-0.347128705949459\\
-0.715803538350409	-0.232578668243255\\
-0.731730559897045	-0.115894735197653\\
-0.729247614287671	-8.9307075662324e-17\\
};
\addplot [color=mycolor1,dotted,line width=1.0pt,forget plot]
  table[row sep=crcr]{1	-0\\
0.956521682769877	-0.151498151383791\\
0.891982039736211	-0.289822533396298\\
0.809292583610082	-0.412355167436434\\
0.711634858347723	-0.517032989000692\\
0.602364630186427	-0.602364630186426\\
0.484917701774751	-0.667431957639199\\
0.362719588349683	-0.711877274647591\\
0.239101059900595	-0.735877395777214\\
0.117221283062812	-0.740106053489981\\
4.44354699422903e-17	-0.725686295399261\\
-0.109940132237539	-0.694134676438339\\
-0.210320240583588	-0.647299141978847\\
-0.299240493845688	-0.587292536894097\\
-0.375203754387119	-0.516423664031337\\
-0.437127756959533	-0.437127756959533\\
-0.484346224770267	-0.351898130574443\\
-0.516599582217932	-0.263220634338226\\
-0.534016141209622	-0.173512362385299\\
-0.537084820159711	-0.0850658786478001\\
-0.526620599330303	-6.44924231334916e-17\\
};
\addplot [color=mycolor1,dotted,line width=1.0pt,forget plot]
  table[row sep=crcr]{1	-0\\
0.940082788364644	-0.148894486293864\\
0.86158608093073	-0.279946287694513\\
0.768279786681378	-0.391458103646313\\
0.663960650859636	-0.482395649771645\\
0.552351927561387	-0.552351927561387\\
0.437014383712816	-0.601498696728173\\
0.321269860940431	-0.630527604183869\\
0.208138182971344	-0.640583459188477\\
0.100287778328637	-0.633192112325829\\
3.73630739569174e-17	-0.610185303761559\\
-0.0908532273476425	-0.573624701779294\\
-0.170819110593797	-0.525727164509449\\
-0.238861981883349	-0.468793035010043\\
-0.294350736089166	-0.40513903143051\\
-0.337037028702328	-0.337037028702328\\
-0.367025664975262	-0.266659754473976\\
-0.384738677688982	-0.196034147687371\\
-0.390874612629551	-0.127002860400749\\
-0.386364521398959	-0.0611941284830315\\
-0.372326104926586	-4.55967972637346e-17\\
};
\addplot [color=mycolor1,dotted,line width=1.0pt,forget plot]
  table[row sep=crcr]{1	-0\\
0.922246029428501	-0.146069421212559\\
0.829201462983264	-0.269423887468404\\
0.725373165529273	-0.369596088217499\\
0.614985835074999	-0.446813363302878\\
0.501902475185001	-0.501902475185001\\
0.38956496084428	-0.536190168955128\\
0.280953750703967	-0.551402782692681\\
0.178565395716864	-0.549567778706978\\
0.0844061798404073	-0.532919645815306\\
3.08495992433718e-17	-0.50381218919366\\
-0.0735915548753502	-0.464638791061534\\
-0.135739038566523	-0.417761804348184\\
-0.186207014322272	-0.365451842507638\\
-0.225110018726605	-0.309837359892227\\
-0.252864584784672	-0.252864584784672\\
-0.270139142845401	-0.196267575756099\\
-0.277803443075876	-0.141547924204336\\
-0.27687894570066	-0.0899634229303457\\
-0.268491413422519	-0.0425248622468694\\
-0.253826721980109	-3.10848082611005e-17\\
};
\addplot [color=mycolor1,dotted,line width=1.0pt,forget plot]
  table[row sep=crcr]{1	-0\\
0.902056570675584	-0.142871725087523\\
0.793293726869509	-0.257756756757911\\
0.678769666307395	-0.345850419328459\\
0.562876391436159	-0.408953636388562\\
0.449318435861499	-0.449318435861499\\
0.341115747349381	-0.469505547439701\\
0.24062663375655	-0.472256359314945\\
0.149586755320764	-0.460380694230177\\
0.069160331541479	-0.436661148025441\\
2.47240337905553e-17	-0.403774113610049\\
-0.057687904157153	-0.364227092250654\\
-0.104075505986926	-0.320311471399148\\
-0.139645446506012	-0.274069620358657\\
-0.165124892040398	-0.227274916024158\\
-0.18142315316863	-0.18142315316863\\
-0.189574186252466	-0.137733708524426\\
-0.190684892879101	-0.0971588057560207\\
-0.185889806735969	-0.0603992595374446\\
-0.176312467991919	-0.0279251515651378\\
-0.16303353482158	-1.99658496572927e-17\\
};
\addplot [color=mycolor1,dotted,line width=1.0pt,forget plot]
  table[row sep=crcr]{1	-0\\
0.877921760602431	-0.139049146701748\\
0.751411952540787	-0.244148543365328\\
0.625732257688895	-0.318826509862098\\
0.505011411193747	-0.36691226736026\\
0.392341669548685	-0.392341669548685\\
0.289890514921595	-0.399000063654162\\
0.199020572856858	-0.390599867100522\\
0.120411913496453	-0.370589763847805\\
0.0541820486548744	-0.34209199176291\\
1.88512313530119e-17	-0.307863971328499\\
-0.042808222513439	-0.270280479734772\\
-0.075164540154518	-0.231332667812908\\
-0.0981551954909503	-0.192640417840451\\
-0.112959067758674	-0.155474818616317\\
-0.120787864504912	-0.120787864504912\\
-0.122837744156894	-0.0892468451742257\\
-0.120251626285951	-0.0612712639357851\\
-0.114091236431876	-0.0370704898842818\\
-0.105317799143806	-0.0166807006736036\\
-0.0947802248421549	-1.16072298975411e-17\\
};
\addplot [color=mycolor1,dotted,line width=1.0pt,forget plot]
  table[row sep=crcr]{1	-0\\
0.84674396664984	-0.134111069256013\\
0.698989566160914	-0.227115477506975\\
0.561406498364893	-0.286050898428465\\
0.437004973478602	-0.317502698182059\\
0.327450698344114	-0.327450698344114\\
0.233352139914598	-0.321181666481713\\
0.154515499860225	-0.303253743289057\\
0.0901653834921457	-0.27750051639687\\
0.0391311034995635	-0.247064063991275\\
1.31311479217367e-17	-0.214447919692096\\
-0.0287598398096346	-0.181582482159892\\
-0.0487044416812479	-0.149896858349689\\
-0.0613430200940395	-0.120392455675976\\
-0.06808779312177	-0.0937148074575858\\
-0.0702211210616195	-0.0702211210616195\\
-0.068876740220244	-0.0500418809603846\\
-0.0650321343871793	-0.0331355275052096\\
-0.0595094084547913	-0.0193357789181307\\
-0.0529823745536039	-0.00839158374073751\\
-0.0459879102602678	-5.63189470997126e-18\\
};
\addplot [color=mycolor1,dotted,line width=1.0pt,forget plot]
  table[row sep=crcr]{1	-0\\
0.801053465278425	-0.126874404768154\\
0.625589539649299	-0.203266363189334\\
0.475341369738971	-0.242198525079547\\
0.350045057714404	-0.254322621147605\\
0.24813777530853	-0.24813777530853\\
0.167289292234614	-0.230253957320141\\
0.104794333468008	-0.205670459781722\\
0.0578515468028918	-0.178048753195384\\
0.0237523524567972	-0.149966451301199\\
7.54043881219821e-18	-0.123144711070133\\
-0.0156239119173694	-0.0986454975334405\\
-0.0250311775652273	-0.0770380431086105\\
-0.0298254673925332	-0.0585357756361869\\
-0.0313184856998208	-0.0431061974937683\\
-0.0305568546459545	-0.0305568546459545\\
-0.0283545570469435	-0.0206007915573586\\
-0.0253272293454815	-0.012904867916705\\
-0.021925762268643	-0.00712411201600239\\
-0.0184675753156993	-0.00292497658052826\\
-0.0151646198645466	-1.85713031774033e-18\\
};
\addplot [color=mycolor1,dotted,line width=1.0pt,forget plot]
  table[row sep=crcr]{1	-0\\
0.714110955679367	-0.113104064044896\\
0.497161927717827	-0.161537702532642\\
0.336758208677056	-0.171586877647116\\
0.221075553032581	-0.160620791180475\\
0.139705610200823	-0.139705610200823\\
0.0839640345306934	-0.115566579097864\\
0.0468885871776745	-0.0920240337831981\\
0.0230753615892199	-0.0710186604775083\\
0.00844586939409394	-0.0533251206797082\\
2.39022368106624e-18	-0.0390353150431684\\
-0.00441505277265522	-0.0278755461307222\\
-0.00630567510971132	-0.0194068724759343\\
-0.00669794782622876	-0.0131454627690819\\
-0.0062698831862128	-0.00862975386096949\\
-0.0054534525074872	-0.0054534525074872\\
-0.00451117782354635	-0.00327756254020109\\
-0.00359218715027031	-0.00183031077240959\\
-0.00277223578568335	-0.000900754009370227\\
-0.00208156288544739	-0.000329687172611911\\
-0.00152375582051941	-1.86606268828125e-19\\
};
\addplot [color=mycolor1,dotted,line width=1.0pt,forget plot]
  table[row sep=crcr]{1	0\\
1	0\\
};
\addplot [color=mycolor1,dotted,line width=1.0pt,forget plot]
  table[row sep=crcr]{0.951056516295154	0.309016994374947\\
0.906577591518048	0.290693092532446\\
0.867277205719189	0.267124444629623\\
0.833330533437629	0.239553777474139\\
0.804679893747523	0.20903820245934\\
0.781122098933164	0.176433510213537\\
0.762382187756933	0.142402376999381\\
0.748171074803624	0.107437628337747\\
0.738227708485823	0.0718935522249012\\
0.732347931289196	0.0360204899377399\\
0.730402691048646	2.81010850277162e-17\\
};
\addplot [color=mycolor1,dotted,line width=1.0pt,forget plot]
  table[row sep=crcr]{0.809016994374947	0.587785252292473\\
0.737380455396588	0.527071687397996\\
0.6808142826414	0.463341883835339\\
0.637053765657314	0.399254954339046\\
0.603812761314093	0.336417677088311\\
0.579022949915482	0.275632227640288\\
0.560948163233974	0.217130071437151\\
0.54821711318997	0.160763451735608\\
0.539811466724715	0.106147624627789\\
0.535036016768209	0.0527590625798542\\
0.533488091091103	4.10502162512614e-17\\
};
\addplot [color=mycolor1,dotted,line width=1.0pt,forget plot]
  table[row sep=crcr]{0.587785252292473	0.809016994374947\\
0.515276498489902	0.692182785870847\\
0.46668476526979	0.583707991451876\\
0.435233321876477	0.485319980094308\\
0.415551842123536	0.396928474901319\\
0.403656860517901	0.317461475735791\\
0.396737049613855	0.245416450708029\\
0.392888142823216	0.179177710929467\\
0.39087245229983	0.11717008156476\\
0.389932112762627	0.0579102497954412\\
0.389661137375347	4.49747826270753e-17\\
};
\addplot [color=mycolor1,dotted,line width=1.0pt,forget plot]
  table[row sep=crcr]{0.309016994374947	0.951056516295154\\
0.265925372344309	0.777304721760365\\
0.248822386132444	0.630899544522142\\
0.24643298177389	0.508693744238057\\
0.251412997268255	0.406266573115132\\
0.25929445161488	0.31919477108011\\
0.267517373913267	0.243597429511063\\
0.274697115780397	0.1762665508339\\
0.280129101393366	0.114599409879343\\
0.283480020554887	0.0564559973822998\\
0.284609543336029	4.37996030135249e-17\\
};
\addplot [color=mycolor1,dotted,line width=1.0pt,forget plot]
  table[row sep=crcr]{6.12323399573677e-17	1\\
0.0151248701748503	0.781989711398728\\
0.0472692933177679	0.613631335769719\\
0.0835006201485716	0.482943980920436\\
0.117381749765263	0.379469463911326\\
0.146223972383669	0.295078319831894\\
0.169261627793672	0.223809451176491\\
0.186582776182006	0.161390341419992\\
0.198560105942714	0.104739935929793\\
0.205572433929556	0.0515565221197431\\
0.207879576350762	3.99891848849262e-17\\
};
\addplot [color=mycolor1,dotted,line width=1.0pt,forget plot]
  table[row sep=crcr]{-0.309016994374947	0.951056516295154\\
-0.213607139159912	0.713331044437031\\
-0.122920349149865	0.544815253953639\\
-0.0461074386071066	0.422454854218942\\
0.0151311193047769	0.329888717873058\\
0.0621570724668225	0.256291005261778\\
0.0971710522581798	0.194731597161215\\
0.122256440677152	0.140793596164787\\
0.13905244595299	0.0915971142347777\\
0.148693455532156	0.0451621321066958\\
0.151835801980649	3.50498499033503e-17\\
};
\addplot [color=mycolor1,dotted,line width=1.0pt,forget plot]
  table[row sep=crcr]{-0.587785252292473	0.809016994374948\\
-0.401011853057454	0.584595820351374\\
-0.252139489074835	0.439670821081765\\
-0.139623392550337	0.340999317931598\\
-0.0567836371213516	0.268617800427267\\
0.0033339412143336	0.21116115844769\\
0.0463512370945915	0.162297289886265\\
0.0763422025660035	0.118472638203443\\
0.0960671266193367	0.0776165020305757\\
0.1072685824301	0.0384304051397518\\
0.110901278364195	2.98672554719679e-17\\
};
\addplot [color=mycolor1,dotted,line width=1.0pt,forget plot]
  table[row sep=crcr]{-0.809016994374948	0.587785252292473\\
-0.533486326814499	0.413410095118228\\
-0.336121655437603	0.313963860155743\\
-0.198040110920963	0.250717832404618\\
-0.101844033235305	0.204281393673556\\
-0.0346516892466227	0.165530866251108\\
0.0122258376810533	0.130333089269644\\
0.0443886084751889	0.0968398262452624\\
0.065339288702761	0.0642052594194206\\
0.0771736424133687	0.032008294596762\\
0.0810025921579431	2.49315700239574e-17\\
};
\addplot [color=mycolor1,dotted,line width=1.0pt,forget plot]
  table[row sep=crcr]{-0.951056516295154	0.309016994374948\\
-0.60382220830535	0.219707538176049\\
-0.375378071887508	0.182507388795924\\
-0.225093275108467	0.161489568357552\\
-0.124654461172013	0.143091836517116\\
-0.0562723920172722	0.12318609851568\\
-0.00923898083522721	0.101104614081101\\
0.0228060316514889	0.0772217637055013\\
0.0436193292019494	0.0520955750986265\\
0.0553650029180358	0.0262210407420432\\
0.0591645112940776	2.0486346567262e-17\\
};
\addplot [color=mycolor1,dotted,line width=1.0pt,forget plot]
  table[row sep=crcr]{-1	1.22464679914735e-16\\
-0.61127914703566	0.0236549857259488\\
-0.374309030147768	0.0580118391989452\\
-0.226262535142082	0.0806522438077526\\
-0.130218598863194	0.0890855793127953\\
-0.0656897647351535	0.0862950481802363\\
-0.0214411717925588	0.0757647040434823\\
0.00876629006412298	0.0602253159022079\\
0.0284556614934045	0.0415943455493057\\
0.039601950618638	0.0211971994741971\\
0.0432139182637723	1.66258696249815e-17\\
};
\addplot [color=mycolor1,dotted,line width=1.0pt,forget plot]
  table[row sep=crcr]{1	-0\\
1	-0\\
};
\addplot [color=mycolor1,dotted,line width=1.0pt,forget plot]
  table[row sep=crcr]{0.951056516295154	-0.309016994374947\\
0.906577591518048	-0.290693092532446\\
0.867277205719189	-0.267124444629623\\
0.833330533437629	-0.239553777474139\\
0.804679893747523	-0.20903820245934\\
0.781122098933164	-0.176433510213537\\
0.762382187756933	-0.142402376999381\\
0.748171074803624	-0.107437628337747\\
0.738227708485823	-0.0718935522249012\\
0.732347931289196	-0.0360204899377399\\
0.730402691048646	-2.81010850277162e-17\\
};
\addplot [color=mycolor1,dotted,line width=1.0pt,forget plot]
  table[row sep=crcr]{0.809016994374947	-0.587785252292473\\
0.737380455396588	-0.527071687397996\\
0.6808142826414	-0.463341883835339\\
0.637053765657314	-0.399254954339046\\
0.603812761314093	-0.336417677088311\\
0.579022949915482	-0.275632227640288\\
0.560948163233974	-0.217130071437151\\
0.54821711318997	-0.160763451735608\\
0.539811466724715	-0.106147624627789\\
0.535036016768209	-0.0527590625798542\\
0.533488091091103	-4.10502162512614e-17\\
};
\addplot [color=mycolor1,dotted,line width=1.0pt,forget plot]
  table[row sep=crcr]{0.587785252292473	-0.809016994374947\\
0.515276498489902	-0.692182785870847\\
0.46668476526979	-0.583707991451876\\
0.435233321876477	-0.485319980094308\\
0.415551842123536	-0.396928474901319\\
0.403656860517901	-0.317461475735791\\
0.396737049613855	-0.245416450708029\\
0.392888142823216	-0.179177710929467\\
0.39087245229983	-0.11717008156476\\
0.389932112762627	-0.0579102497954412\\
0.389661137375347	-4.49747826270753e-17\\
};
\addplot [color=mycolor1,dotted,line width=1.0pt,forget plot]
  table[row sep=crcr]{0.309016994374947	-0.951056516295154\\
0.265925372344309	-0.777304721760365\\
0.248822386132444	-0.630899544522142\\
0.24643298177389	-0.508693744238057\\
0.251412997268255	-0.406266573115132\\
0.25929445161488	-0.31919477108011\\
0.267517373913267	-0.243597429511063\\
0.274697115780397	-0.1762665508339\\
0.280129101393366	-0.114599409879343\\
0.283480020554887	-0.0564559973822998\\
0.284609543336029	-4.37996030135249e-17\\
};
\addplot [color=mycolor1,dotted,line width=1.0pt,forget plot]
  table[row sep=crcr]{6.12323399573677e-17	-1\\
0.0151248701748503	-0.781989711398728\\
0.0472692933177679	-0.613631335769719\\
0.0835006201485716	-0.482943980920436\\
0.117381749765263	-0.379469463911326\\
0.146223972383669	-0.295078319831894\\
0.169261627793672	-0.223809451176491\\
0.186582776182006	-0.161390341419992\\
0.198560105942714	-0.104739935929793\\
0.205572433929556	-0.0515565221197431\\
0.207879576350762	-3.99891848849262e-17\\
};
\addplot [color=mycolor1,dotted,line width=1.0pt,forget plot]
  table[row sep=crcr]{-0.309016994374947	-0.951056516295154\\
-0.213607139159912	-0.713331044437031\\
-0.122920349149865	-0.544815253953639\\
-0.0461074386071066	-0.422454854218942\\
0.0151311193047769	-0.329888717873058\\
0.0621570724668225	-0.256291005261778\\
0.0971710522581798	-0.194731597161215\\
0.122256440677152	-0.140793596164787\\
0.13905244595299	-0.0915971142347777\\
0.148693455532156	-0.0451621321066958\\
0.151835801980649	-3.50498499033503e-17\\
};
\addplot [color=mycolor1,dotted,line width=1.0pt,forget plot]
  table[row sep=crcr]{-0.587785252292473	-0.809016994374948\\
-0.401011853057454	-0.584595820351374\\
-0.252139489074835	-0.439670821081765\\
-0.139623392550337	-0.340999317931598\\
-0.0567836371213516	-0.268617800427267\\
0.0033339412143336	-0.21116115844769\\
0.0463512370945915	-0.162297289886265\\
0.0763422025660035	-0.118472638203443\\
0.0960671266193367	-0.0776165020305757\\
0.1072685824301	-0.0384304051397518\\
0.110901278364195	-2.98672554719679e-17\\
};
\addplot [color=mycolor1,dotted,line width=1.0pt,forget plot]
  table[row sep=crcr]{-0.809016994374948	-0.587785252292473\\
-0.533486326814499	-0.413410095118228\\
-0.336121655437603	-0.313963860155743\\
-0.198040110920963	-0.250717832404618\\
-0.101844033235305	-0.204281393673556\\
-0.0346516892466227	-0.165530866251108\\
0.0122258376810533	-0.130333089269644\\
0.0443886084751889	-0.0968398262452624\\
0.065339288702761	-0.0642052594194206\\
0.0771736424133687	-0.032008294596762\\
0.0810025921579431	-2.49315700239574e-17\\
};
\addplot [color=mycolor1,dotted,line width=1.0pt,forget plot]
  table[row sep=crcr]{-0.951056516295154	-0.309016994374948\\
-0.60382220830535	-0.219707538176049\\
-0.375378071887508	-0.182507388795924\\
-0.225093275108467	-0.161489568357552\\
-0.124654461172013	-0.143091836517116\\
-0.0562723920172722	-0.12318609851568\\
-0.00923898083522721	-0.101104614081101\\
0.0228060316514889	-0.0772217637055013\\
0.0436193292019494	-0.0520955750986265\\
0.0553650029180358	-0.0262210407420432\\
0.0591645112940776	-2.0486346567262e-17\\
};
\addplot [color=mycolor1,dotted,line width=1.0pt,forget plot]
  table[row sep=crcr]{-1	-1.22464679914735e-16\\
-0.61127914703566	-0.0236549857259488\\
-0.374309030147768	-0.0580118391989452\\
-0.226262535142082	-0.0806522438077526\\
-0.130218598863194	-0.0890855793127953\\
-0.0656897647351535	-0.0862950481802363\\
-0.0214411717925588	-0.0757647040434823\\
0.00876629006412298	-0.0602253159022079\\
0.0284556614934045	-0.0415943455493057\\
0.039601950618638	-0.0211971994741971\\
0.0432139182637723	-1.66258696249815e-17\\
};
\addplot [color=lightgray,line width=1.5pt,mark size=4.0pt,only marks,mark=x,mark options={solid},forget plot]
  table[row sep=crcr]{0	0\\
0.0332891817456478	0\\
0.0685951920820552	0\\
0.106261074191997	0\\
0.146718073740713	0\\
0.190507000794354	0\\
0.238288964560797	0\\
0.290802156451388	0\\
0.34863201227429	0\\
0.411482455545339	0\\
0.476716764894891	0\\
0.538701889630014	0\\
0.592009697191867	0\\
0.635111673402557	0\\
0.669445798454785	0\\
0.697088115438512	0\\
0.719764016628818	0\\
0.738726481861229	0\\
0.754859260015905	0\\
0.768789529287501	0\\
0.780970447060097	0\\
0.791736388624927	0\\
0.801339260556346	0\\
0.809972543832126	0\\
0.817787474350972	0\\
0.824904144359498	0\\
0.831419270792714	0\\
0.83741173557266	0\\
0.842946608042182	0\\
0.848078114087544	0\\
0.852851861415261	0\\
0.857306530845705	0\\
0.861475178388775	0\\
0.865386249582643	0\\
};
\addplot [color=gray,line width=1.5pt,mark size=4.0pt,only marks,mark=x,mark options={solid},forget plot]
  table[row sep=crcr]{0.91444306659383	0.404714563561125\\
0.897798475721005	0.408118988555352\\
0.880145470552802	0.411522521388228\\
0.861312529497832	0.414999075403529\\
0.841084029723473	0.418666762346229\\
0.819189566196653	0.422719274912659\\
0.795298584313431	0.427482631171093\\
0.769041988368137	0.433516364723153\\
0.740127060456686	0.441775148924797\\
0.708701838821161	0.453761439128821\\
0.676084684146385	0.471252951045491\\
0.645092121778823	0.494954826602702\\
0.618438217997896	0.523275555707163\\
0.596887229892552	0.553578075862308\\
0.57972016736644	0.583971213409664\\
0.565899008874575	0.613557528499428\\
0.554561058279421	0.642020626188909\\
0.545079825663216	0.669305600011421\\
0.537013436585878	0.695461292765181\\
0.530048301950081	0.720571832020281\\
0.523957843063781	0.744728338733943\\
0.518574872281366	0.768017822317744\\
0.513773436315659	0.790519346317733\\
0.509456794677767	0.812303231185067\\
0.505549329418344	0.833431472307299\\
0.50199099441408	0.853958585180083\\
0.498733431197472	0.873932540554951\\
0.495737198807501	0.893395651952642\\
0.49296976257274	0.912385366558339\\
0.490404009550058	0.930934949262658\\
0.488017135886199	0.949074065677673\\
0.485789801170979	0.966829275718118\\
0.483705477399443	0.984224450514024\\
0.48174994180251	1.00128112467398\\
};
\addplot [color=darkgray,line width=1.5pt,mark size=4.0pt,only marks,mark=x,mark options={solid},forget plot]
  table[row sep=crcr]{0.91444306659383	-0.404714563561125\\
0.897798475721005	-0.408118988555352\\
0.880145470552802	-0.411522521388228\\
0.861312529497832	-0.414999075403529\\
0.841084029723473	-0.418666762346229\\
0.819189566196653	-0.422719274912659\\
0.795298584313431	-0.427482631171093\\
0.769041988368137	-0.433516364723153\\
0.740127060456686	-0.441775148924797\\
0.708701838821161	-0.453761439128821\\
0.676084684146385	-0.471252951045491\\
0.645092121778823	-0.494954826602702\\
0.618438217997896	-0.523275555707163\\
0.596887229892552	-0.553578075862308\\
0.57972016736644	-0.583971213409664\\
0.565899008874575	-0.613557528499428\\
0.554561058279421	-0.642020626188909\\
0.545079825663216	-0.669305600011421\\
0.537013436585878	-0.695461292765181\\
0.530048301950081	-0.720571832020281\\
0.523957843063781	-0.744728338733943\\
0.518574872281366	-0.768017822317744\\
0.513773436315659	-0.790519346317733\\
0.509456794677767	-0.812303231185067\\
0.505549329418344	-0.833431472307299\\
0.50199099441408	-0.853958585180083\\
0.498733431197472	-0.873932540554951\\
0.495737198807501	-0.893395651952642\\
0.49296976257274	-0.912385366558339\\
0.490404009550058	-0.930934949262658\\
0.488017135886199	-0.949074065677673\\
0.485789801170979	-0.966829275718118\\
0.483705477399443	-0.984224450514024\\
};
\addplot [color=black,line width=1.5pt,mark size=4.0pt,only marks,mark=o,mark options={solid},forget plot]
  table[row sep=crcr]{1	0\\
};
\end{axis}
\end{tikzpicture}%}
        \renewcommand\figurename{Fig.}
        \caption{Lugar das raízes para o controlador proporcional.}
        \label{fig:rlocus_ic_2}
    \end{figure}

    A Fig.~\ref{fig:rlocus_ic_2} apresenta o lugar das raízes para o controlador proporcional. O valor de máximo amortecimento obtido é quando $K_P = 8$.


\section{Malha Externa}

    Para este projeto, assume-se um alto desempenho no rastreamento de referência da malha interna. O controlador da malha externa é do tipo adaptativo por modelo de referência (\emph{MRAC}), controla a corrente da rede e gera a referência ${U_p}^*$ para a malha interna.

    O desenvolvimento apresentado nesta seção é realizado em tempo discreto. Assim, o modelo discreto da planta do laço externo, $G_{od}(z)$, da planta do laço interno, $G_{id}(z)$, da planta do distúrbio externo $G_{do}(z)$ e da planta do distúrbio interno $G_{di}(z)$ são dados conforme desenvolvido na Seção~\ref{modelagem}.
    % %
    % \begin{equation}
    %     G_1(z) = \frac{1 - \cos(\omega_0 T_s)}{T_s} \frac{z + 1}{z^2 - 2z
    %         \cos(\omega_0 T_s) + 1} \text{,}
    % \end{equation}
    % %
    % \begin{equation}
    %     F_1(z) = \frac{\text{sen}(\omega_0 T_s)}{C \omega_0} \frac{z - 1}
    %         {z^2 - 2z \cos(\omega_0 T_s) + 1} \text{,}
    % \end{equation}
    % %
    % \begin{equation}
    %     G_2(z) = F_2(z) = \frac{T_s / L_g}{z - 1} \text{,}
    %     \label{eq:g_2_discreta}
    % \end{equation}
    % %
    % com $T_s$ sendo a frequência de amostragem, $z$ o operador de discretização associado à Transformada $\mathcal{Z}$ e $\omega_0 = 1 / \sqrt{L_c C}$.

    Para análise, considere a estrutura da Fig.~\ref{fig:LCL_discreto}.
    %
    \begin{figure}[htb]
        \centering{
            \def\svgwidth{\textwidth}
            \input{./img/multiloop_geral.pdf_tex}}
        \renewcommand\figurename{Fig.}
        \caption{Diagrama de blocos para o filtro LCL em modelo discreto.}
        \label{fig:LCL_discreto}
    \end{figure}

    Em um primeiro momento, desconsidera-se o distúrbio de tensão da rede $V_g$ e projeta-se a ação de controle $U \equiv U_p$ para o caso de parâmetros conhecidos. %De (\ref{eq:g_2_discreta}) obtem-se a equação de diferença
    De (\ref{eq:god_i2_vc}) ou de (\ref{eq:god_i2_ic}) obtem-se a equação de diferença
    %
    \begin{equation}
        I_2 (k + 1) = I_2 (k) + b_p U (k) \text{,}
        \label{eq:diferenca}
    \end{equation}
    %
    com $b_p = T_s / L_g$.

    O desafio do \emph{MRAC} é projetar o controlador de forma que a saída da planta siga assintoticamente a saída de um modelo de referência. Como a planta e o modelo de referência devem ser de mesma ordem, tem-se o modelo de referência
    %
    \begin{equation}
        I_{2m}(k + 1) = a_m I_{2m}(k) + b_m {I_2}^* (k) \text{,}
        \label{eq:modelo_referencia}
    \end{equation}
    %
    com $| a_m | \le 1$ para estabilidade.

    Se a lei de controle for estabelecida como sendo
    %
    \begin{equation}
        U (k) = {\theta_1}^* I_2 (k) + {\theta_2}^* {I_2}^* (k) \text{,}
    \end{equation}
    %
    com
    %
    \begin{equation}
        \begin{split}
            {\theta_1}^* & = \frac{a_m - 1}{b_p} \text{,}\\
            {\theta_2}^* & = \frac{b_m}{b_p} \text{,}
        \end{split}
        \label{eq:ganhos}
    \end{equation}
    %
    então tem-se
    %
    \begin{equation}
        I_2 (k + 1) = a_m I_2 (k) + b_m {I_2}^* (k) \text{,}
    \end{equation}
    %
    o que implica que $I_{2m} = I_2$, casando a planta de malha fechada com o modelo de referência. Entretanto, como o parâmetro $L_g$ é incerto, não se pode calcular os ganhos do controlador dados por (\ref{eq:ganhos}). Para lidar com esta incerteza, a lei de controle é estabelecida como sendo
    %
    \begin{equation}
        U (k) = \theta_1 (k) I_2 (k) + \theta_2 (k) {I_2}^* (k) \text{,}
        \label{eq:controle_incerteza}
    \end{equation}
    %
    onde $\theta_1$ e $\theta_2$ são estimados adaptativamente.

    Para projetar o algoritmo adaptativo, escreve-se a equação de rastreamento do erro. Substituindo (\ref{eq:controle_incerteza}) em (\ref{eq:diferenca}) a malha fechada pode ser escrita como
    %
    \begin{equation}
        \begin{split}
        I_2 (k + 1) &= I_2 (k) + b_p \left( {\theta_1}^* I_2 (k) + {\theta_2}^* {I_2}^*
            (k) \right)\\
            &\qquad {}+ b_p \left( (\theta_1 (k) - {\theta_1}^*) I_2 (k) + ( \theta_2 (k)
            - {\theta_2}^*) {I_2}^* (k) \right) \text{.}
        \end{split}
        \label{eq:malha_fechada}
    \end{equation}

    Utilizando (\ref{eq:modelo_referencia}) e (\ref{eq:malha_fechada}) o erro de rastreamento $e = I_2 - I_{2m}$ é dado por
    %
    \begin{equation}
        e (k+1) = a_m e(k) + b_p \phi^T (k) \omega (k) \text{,}
        \label{eq:erro_rastreamento}
    \end{equation}
    %
    onde $\phi (k) = {\left[ \begin{matrix} \theta_1 (k) - {\theta_1}^* & \theta_2 (k) - {\theta_2}^*
    \end{matrix} \right]}^T$ e
    %
    \begin{equation}
        \omega (k) = {\left[ \begin{matrix} I_2 (k) & {I_2}^* (k) \end{matrix} \right]}^T \text{.}
        \label{eq:omega_k}
    \end{equation}

    Definindo $\zeta (k) = b_m / (z - a_m) \omega (k)$ e utilizando (\ref{eq:erro_rastreamento}) pode-se escrever a função de transferência
    %
    \begin{equation}
        e(k) = \rho^* \left( \frac{b_m}{z - a_m} \left[ \theta^T (k) \omega (k) \right]
            - {\theta^*}^T \zeta (k) \right) \text{,}
        \label{eq:erro_ft}
    \end{equation}
    %
    onde $\rho^* = b_p / b_m$.

    Observa-se que (\ref{eq:erro_ft}) não pode ser usado em uma lei adaptativa para o parâmetro $\theta (k)$ devido ao desconhecimento de $\rho^*$ e $\theta^*$. Para resolver este problema, o erro de estimação é definido como
    %
    \begin{equation}
        \epsilon (k) = e (k) - \rho(k) \left( \frac{b_m}{z - a_m} \left[ \theta^T (k) \omega(k)
            \right] - \theta^T (k) \zeta(k) \right) \text{.}
        \label{eq:erro_estimacao}
    \end{equation}

    Substituindo (\ref{eq:erro_ft}) em (\ref{eq:erro_estimacao}) e adicionando o termo $-\rho^* \theta^T (k) \zeta (k) + \rho^* (k) \theta^T (k) \zeta(k)$, tem-se
    %
    \begin{equation}
        \epsilon(k) = \rho^* \phi^T(k) \zeta(k) + \tilde{\rho}(k) \xi(k) \text{,}
        \label{eq:epsilon_k}
    \end{equation}
    %
    onde $\tilde{\rho}(k) = \rho(k) - \rho^*$ e $\xi(k) = \theta^T(k) \zeta(k) - b_m / (z - a_m)[\theta^T \omega](k)$.

    Da teoria de controle, a função definida positiva
    %
    \begin{equation}
        V = |\rho^*| \phi^T \Gamma^{-1} \phi + \gamma^{-1} {\tilde{\rho}}^2
        \label{eq:funcao_positiva}
    \end{equation}
    %
    que envolve os erros paramétricos, pode ser minimizada definindo as seguintes regras adaptativas para $\theta$ e $\rho$:
    %
    \begin{subequations}
        \begin{align}
            \theta(k + 1) & = \theta(k) - \text{sgn}(\rho^*)\frac{\Gamma \epsilon(k)\zeta(k)}{m^2(k)} \text{,}\\
            \rho(k + 1) & = \rho(k) - \frac{\gamma \epsilon(k) \xi(k)}{m^2(k)} \text{,}
        \end{align}
        \label{eq:regras_adaptativas}
    \end{subequations}
    %
    onde $\text{sgn}(\rho^*)$ denota o sinal do parâmetro fixo $\rho^*$, $0 < \gamma < 2$, $0 < \Gamma = \Gamma^T < 2/\rho_0 I_{dim \theta}$, $\rho_0 \geq |b_p / b_m|$ sendo $I_{dim \theta}$ a matriz identidade de mesma dimensão do vetor $\theta$. Em (\ref{eq:regras_adaptativas}) o sinal de normalização $m$ é dado por
    %
    \begin{equation}
        m(k) = \sqrt{1 + \zeta^T(k) \zeta(k) + \xi^2(k)} \text{.}
    \end{equation}

    É possível provar que utilizando (\ref{eq:epsilon_k}), (\ref{eq:funcao_positiva}) e (\ref{eq:regras_adaptativas}) tem-se
    %
    \begin{equation}
        V(k + 1) - V(k) \leq -c \frac{\epsilon^2(k)}{m^2(k)} \text{,}
    \end{equation}
    %
    com $c > 0$, o que implica na convergência do erro de estimação $\epsilon$ para zero e na convergência de $\theta$ e $\rho$ para um valor limitado, como em \cite{ref:TAO}.

    Como evidenciado pela Fig.~\ref{fig:LCL_discreto}, percebe-se que a corrente da rede $I_2$ está sujeita a um distúrbio exógeno de tensão da rede $V_g$. Para compensar este efeito, pode-se aumentar o vetor (\ref{eq:omega_k}) de forma que
    %
    \begin{equation}
        {U_p}^* = \theta_1(k) I_2(k) + \theta_2(k) {I_2}^*(k)+
            \theta_3(k) \text{sen}(\omega_{g1}t) + \theta_4(k) \cos(\omega_{g1} t) \text{.}
    \end{equation}

    A prova de estabilidade detalhada cobrindo o caso do vetor (\ref{eq:omega_k}) aumentado é conforme consta no Anexo~\ref{provas}.



%FIM----------------------------------------------------------------------
