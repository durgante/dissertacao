%TITULO-------------------------------------------------------------------

%=========================================================================
\chapter{Controle Multimalha}\label{controle}
%=========================================================================

	Em aplicações onde o desempenho é um requisito importante, o controle multimalha, também chamado de controle em cascata, apresenta vantagens significativas. Essa estratégia propõe que uma variável seja criada para detectar a presença de distúrbios antes que esses afetem a variável controlada. Dessa forma, não é necessário esperar que a variável controlada se desvie do ponto fixo para começar uma ação corretiva, como acontece no controle por realimentação simples. Ao perceber a presença do distúrbio, a ação corretiva começa imediatamente, de forma que o desvio sofrido pela variável controlada tende a ser reduzido.

  É necessário, no entanto, que a variável escolhida para ser a \emph{variável intermediária} responda mais rapidamente a variações no distúrbio e na variável manipulada do que a variável controlada. Isso faz sentido devido ao fato de que quanto mais rápido a variável intermediária responder ao distúrbio, mais rápido a ação corretiva será iniciada e menor será o desvio do ponto fixo que a variável controlada sofrerá. De fato, quanto mais rápido a variável intermediária responder, melhor.

  A Fig.~\ref{fig:realimentacao_vs_cascata} apresenta uma comparação de desempenho do controle da temperatura de um reator quando ocorre uma variação de $25\,^{\circ}\mathrm{C}$ na temperatura de entrada do reagente. A linha cheia representa o controle multimalha, e a linha tracejada representa o controle por realimentação simples.

  \begin{figure}[htb]
    \centering{
      \def\svgwidth{0.6\textwidth}
      \input{./img/realimentacao_vs_cascata.pdf_tex}}
    \renewcommand\figurename{Fig.}
    \caption{Comparação de desempenho entre realimentação simples e controle em cascata.}
    \label{fig:realimentacao_vs_cascata}
  \end{figure}

  É possível desenvolver controladores com qualquer número de malhas aninhadas em cascata. É importante observar apenas o fato de que a malha mais externa gera a referência para a malha imediatamente interna em relação a ela. Dessa forma, a variável intermediária escolhida para a malha externa deve responder mais rápido ao distúrbio do que a variável intermediária escolhida para a malha interna.

  Na prática, é comum observar controladores com duas ou três malhas aninhadas. Além do quesito desempenho, outros fatores podem levar à utilização de controladores multimalha. Como é o caso deste trabalho, a separação do modelo em duas partes simplifica a abordagem de controle, devido ao fato de a malha interna fazer o amortecimento ativo da ressonância do filtro, e a malha externa tratar da incerteza paramétrica inerente à rede elétrica.

  O projeto deste controlador divide-se então em uma malha interna e uma malha externa, conforme a seguir.

\section{Malha Interna}

  A modelagem desenvolvida separa a planta em duas partes: $G_{id}$ pertence à malha interna, e $G_{od}$ pertence à malha externa, conforme pode ser visto na Fig.~\ref{fig:LCL_discreto}.

  \begin{figure}[htb]
    \centering{
      \def\svgwidth{\textwidth}
      \input{./img/multiloop_geral.pdf_tex}}
    \renewcommand\figurename{Fig.}
    \caption{Diagrama de blocos para o filtro LCL em modelo discreto.}
    \label{fig:LCL_discreto}
  \end{figure}

  %
  %Esta forma de modelagem permite isolar a incerteza paramétrica em $G_{od}$, de forma que o controlador projetado para a malha interna trata de uma planta com parâmetros conhecidos e projetados.
  %
  Isso permite que controladores diferentes sejam projetados para resolver partes do problema de controle. O controlador da malha interna, por exemplo, será responsável por amortecer a ressonância do filtro LCL sem se preocupar com a incerteza paramétrica devida à rede elétrica. Dessa forma, pode-se utilizar controladores convencionais para realizar o amortecimento ativo.

  É conhecido da literatura que, no caso específico de um filtro LCL, é suficiente para a realização do amortecimento da ressonância do filtro a utilização de um controlador proporcional, quando a variável intermediária escolhida é a corrente do capacitor, e de um controlador proporcional-derivativo, quando a variável intermediária escolhida é a tensão do capacitor~\cite{ref:DANNEHL}. O critério de escolha para a variável intermediária varia de acordo com a aplicação e a topologia~\cite{ref:POH}.

  Levando isso em consideração, neste trabalho um controlador proporcional é projetado para o caso da variável intermediária ser a corrente do capacitor, e um controlador proporcional-derivativo é projetado para o caso da variável intermediária ser a tensão do capacitor.
  %O projeto de ambos leva em consideração os parâmetros listados na Tabela~\ref{tab:sim_parameters}.

  % \begin{table}[htb]
  %   \renewcommand{\arraystretch}{1.35}
  %   \setlength{\tabcolsep}{1.2mm}
  %   \caption{Parâmetros do sistema utilizados no projeto.}
  %   \label{tab:sim_parameters}
  %   \centering
  %   \begin{tabular}{l l l l}
  %     \hline
  %     \multicolumn{1}{c}{Parâmetro} & \multicolumn{1}{c}{Valor} &
  %     \multicolumn{1}{c}{Parâmetro} & \multicolumn{1}{c}{Valor} \\
  %     \hline
  %     $L_1$ &  $2$mH     &  $L_2$      &  $2$mH    \\
  %     $C$   &  $40\mu$F  & Frequência de Amostragem ($f_s = 1/T_s$) &  $12$kHz  \\
  %     \hline
  %   \end{tabular}
  % \end{table}

\subsection{Projeto para tensão do capacitor como variável intermediária}

  Considerando o controlador da malha interna $C_i(z)$ como sendo proporcional-derivativo, tem-se sua expressão:
  %
  \begin{equation}
    C_i(z) = \left( K_P + K_D \right) \frac{z- \frac{K_D}{K_P+K_D}}{z}
  \end{equation}

  Em uma aplicação prática, o atraso de tempo associado à implementação digital limita o ganho do controlador, de forma que se deve projetar o zero visando maximizar o amortecimento.
  %
  \begin{equation}
    z = \frac{K_D}{K_P+K_D} > 1
  \end{equation}

  Essa escolha, no entanto, resulta em um sistema de fase não-mínima, uma característica que viola o requisito principal para o funcionamento do controlador da malha externa. Assim sendo, o zero é projetado em $z=0,9$ e é utilizado um traçado do lugar das raízes para projetar o ganho $K_P+K_D$.

  \begin{figure}[htb]
    \centering{
      \def\svgwidth{0.9\textwidth}
      \input{./img/rlocus_controlador_pd.tex}}
    \renewcommand\figurename{Fig.}
    \caption{Lugar das raízes para o controlador proporcional-derivativo.}
    \label{fig:rlocus_vc_2}
  \end{figure}

  A partir da Fig.~\ref{fig:rlocus_vc_2} se pode escolher $K_P+K_D = 3$ para máximo amortecimento, escolha que resulta em $K_P = 0,3$ e $K_D = 2,7$.


\subsection{Projeto para corrente do capacitor como variável intermediária}

  Considerando o controlador da malha interna $C_i(z)$ como sendo do tipo proporcional, sua expressão é dada por:
  %
  \begin{equation}
    C_i(z) = K_P
  \end{equation}

  Tendo como referência a função de transferência de malha fechada $I_C/U$, que pode ser vista na Fig.~\ref{fig:LCL_discreto} fazendo $U_p = I_C$, observa-se que sua equação característica é:
  %
  \begin{equation}
    z^3 - 2 \cos \left( \omega_n T_s \right) z^2 + \left( 1 + K_P K_{id} \right) z - K_P K_{id} = 0
  \end{equation}
  %
  com
  %
  \begin{equation}
    K_{id} = \frac{\sin(\omega_n T_s)}{\omega_n L_1}
  \end{equation}

  Através do critério de estabilidade de Routh-Hurwitz, pode-se projetar o ganho $K_P$
  %
  \begin{equation}
    \overline{K_P} = \frac{2 \cos \left( \omega_n T_s \right) - 1}
      {\sen \left( \omega_n T_s \right)} \omega_n L_1
  \end{equation}
  %
  onde $\overline{K_P}$ representa o limite superior para o valor de $K_P$.

  \begin{figure}[htb]
    \centering{
      \def\svgwidth{\textwidth}
      \input{./img/rlocus_controlador_p.tex}}
    \renewcommand\figurename{Fig.}
    \caption{Lugar das raízes para o controlador proporcional.}
    \label{fig:rlocus_ic_2}
  \end{figure}

  A Fig.~\ref{fig:rlocus_ic_2} apresenta o lugar das raízes para o controlador proporcional. O valor de máximo amortecimento obtido é quando $K_P = 8$.


\section{Malha Externa}

  Para este projeto, assume-se um alto desempenho no rastreamento de referência da malha interna. O controlador da malha externa, do tipo adaptativo por modelo de referência (do inglês \emph{Model Reference Adaptive Control - MRAC}), controla a corrente da rede e gera a referência ${U_p}^*$ para a malha interna.

  O desenvolvimento apresentado nesta seção é realizado em tempo discreto. Assim, o modelo discreto da planta do laço externo, $G_{od}(z)$, da planta do laço interno, $G_{id}(z)$, da planta do distúrbio externo $G_{do}(z)$ e da planta do distúrbio interno $G_{di}(z)$ são dados conforme desenvolvido na Seção~\ref{modelagem}.
  % %
  % \begin{equation}
  %     G_1(z) = \frac{1 - \cos(\omega_0 T_s)}{T_s} \frac{z + 1}{z^2 - 2z
  %         \cos(\omega_0 T_s) + 1} \text{,}
  % \end{equation}
  % %
  % \begin{equation}
  %     F_1(z) = \frac{\text{sen}(\omega_0 T_s)}{C \omega_0} \frac{z - 1}
  %         {z^2 - 2z \cos(\omega_0 T_s) + 1} \text{,}
  % \end{equation}
  % %
  % \begin{equation}
  %     G_2(z) = F_2(z) = \frac{T_s / L_g}{z - 1} \text{,}
  %     \label{eq:g_2_discreta}
  % \end{equation}
  % %
  % com $T_s$ sendo a frequência de amostragem, $z$ o operador de discretização associado à Transformada $\mathcal{Z}$ e $\omega_0 = 1 / \sqrt{L_c C}$.

  A planta $G(z)$ pode ser escrita de forma a separar a dinâmica não-modelada $\Delta(z)$ da parte conhecida $G_o(z)$. Isto pode ser feito via uma dinâmica não-modelada aditiva ou multiplicativa. Escolheu-se utilizar uma dinâmica não-modelada aditiva devido ao fato de que a planta resultante para este caso tem grau relativo dois, enquanto que para a dinâmica multiplicativa tem grau relativo três. Um grau relativo menor implica em um número menor de ganhos adaptativos.

  Dessa forma, escreve-se
  %
  \begin{equation}
    G(z) = G_o(z) + \Delta(z) \text{,}
    \label{eq:planta_go_delta_aditiva}
  \end{equation}
  %
  e se trabalha apenas com a parte conhecida da planta $G_o(z)$. A dinâmica não-modelada, estável e estritamente própria $\Delta(z)$, será compensada pela característica de robustez do controlador, representada pelo normalizador $m$.

  Para análise, considere a estrutura da Fig.~\ref{fig:LCL_discreto}. Em um primeiro momento, desconsidera-se o distúrbio de tensão da rede $V_g$ e projeta-se a ação de controle $U \equiv U_p$ para o caso de parâmetros conhecidos. %De (\ref{eq:g_2_discreta}) obtem-se a equação de diferença
  De (\ref{eq:god_i2_vc}) ou de (\ref{eq:god_i2_ic}) obtem-se a equação de diferença
  %
  \begin{equation}
    I_2 (k + 1) = I_2 (k) + b_p U (k) \text{,}
    \label{eq:diferenca}
  \end{equation}
  %
  com $b_p = T_s / L_g$.

  O desafio do \emph{MRAC} é projetar o controlador de forma que a saída da planta siga assintoticamente a saída de um modelo de referência. Como a planta e o modelo de referência devem ser de mesma ordem, tem-se o modelo de referência
  %
  \begin{equation}
    I_{2m}(k + 1) = a_m I_{2m}(k) + b_m {I_2}^* (k) \text{,}
    \label{eq:modelo_referencia}
  \end{equation}
  %
  com $| a_m | \le 1$ para estabilidade.

  Se a lei de controle for estabelecida como sendo
  %
  \begin{equation}
    U (k) = {\theta_1}^* I_2 (k) + {\theta_2}^* {I_2}^* (k) \text{,}
  \end{equation}
  %
  com
  %
  \begin{equation}
    \begin{split}
      {\theta_1}^* & = \frac{a_m - 1}{b_p} \text{,}\\
      {\theta_2}^* & = \frac{b_m}{b_p} \text{,}
    \end{split}
    \label{eq:ganhos}
  \end{equation}
  %
  então tem-se
  %
  \begin{equation}
    I_2 (k + 1) = a_m I_2 (k) + b_m {I_2}^* (k) \text{,}
  \end{equation}
  %
  o que implica que $I_{2m} = I_2$, casando a planta de malha fechada com o modelo de referência. Entretanto, como o parâmetro $L_g$ é incerto, não se pode calcular os ganhos do controlador dados por (\ref{eq:ganhos}). Para lidar com esta incerteza, a lei de controle é estabelecida como sendo
  %
  \begin{equation}
    U (k) = \theta_1 (k) I_2 (k) + \theta_2 (k) {I_2}^* (k) \text{,}
    \label{eq:controle_incerteza}
  \end{equation}
  %
  onde $\theta_1$ e $\theta_2$ são estimados adaptativamente.

  Para projetar o algoritmo adaptativo, escreve-se a equação de rastreamento do erro. Substituindo (\ref{eq:controle_incerteza}) em (\ref{eq:diferenca}) a malha fechada pode ser escrita como
  %
  \begin{equation}
    \begin{split}
      I_2 (k + 1) &= I_2 (k) + b_p \left( {\theta_1}^* I_2 (k) + {\theta_2}^* {I_2}^*
      (k) \right)\\
      &\qquad {}+ b_p \left( (\theta_1 (k) - {\theta_1}^*) I_2 (k) + ( \theta_2 (k)
      - {\theta_2}^*) {I_2}^* (k) \right) \text{.}
    \end{split}
    \label{eq:malha_fechada}
  \end{equation}

  Utilizando (\ref{eq:modelo_referencia}) e (\ref{eq:malha_fechada}) o erro de rastreamento $e = I_2 - I_{2m}$ é dado por
  %
  \begin{equation}
    e (k+1) = a_m e(k) + b_p \phi^T (k) \omega (k) \text{,}
    \label{eq:erro_rastreamento}
  \end{equation}
  %
  onde $\phi (k) = {\left[ \begin{matrix} \theta_1 (k) - {\theta_1}^* & \theta_2 (k) - {\theta_2}^*
  \end{matrix} \right]}^T$ e
  %
  \begin{equation}
    \omega (k) = {\left[ \begin{matrix} I_2 (k) & {I_2}^* (k) \end{matrix} \right]}^T \text{.}
    \label{eq:omega_k}
  \end{equation}

  Definindo $\zeta (k) = b_m / (z - a_m) \omega (k)$ e utilizando (\ref{eq:erro_rastreamento}) pode-se escrever a função de transferência
  %
  \begin{equation}
    e(k) = \rho^* \left( \frac{b_m}{z - a_m} \left[ \theta^T (k) \omega (k) \right]
      - {\theta^*}^T \zeta (k) \right) \text{,}
    \label{eq:erro_ft}
  \end{equation}
  %
  onde $\rho^* = b_p / b_m$.

  Observa-se que (\ref{eq:erro_ft}) não pode ser usado em uma lei adaptativa para o parâmetro $\theta (k)$ devido ao desconhecimento de $\rho^*$ e $\theta^*$. Para resolver este problema, o erro de estimação é definido como
  %
  \begin{equation}
    \epsilon (k) = e (k) - \rho(k) \left( \frac{b_m}{z - a_m} \left[ \theta^T (k) \omega(k)
      \right] - \theta^T (k) \zeta(k) \right) \text{.}
    \label{eq:erro_estimacao}
  \end{equation}

  Substituindo (\ref{eq:erro_ft}) em (\ref{eq:erro_estimacao}) e adicionando o termo $-\rho^* \theta^T (k) \zeta (k) + \rho^* (k) \theta^T (k) \zeta(k)$, tem-se
  %
  \begin{equation}
    \epsilon(k) = \rho^* \phi^T(k) \zeta(k) + \tilde{\rho}(k) \xi(k) \text{,}
    \label{eq:epsilon_k}
  \end{equation}
  %
  onde $\tilde{\rho}(k) = \rho(k) - \rho^*$ e $\xi(k) = \theta^T(k) \zeta(k) - b_m / (z - a_m)[\theta^T \omega](k)$.

  Da teoria de controle, a função definida positiva
  %
  \begin{equation}
    V = |\rho^*| \phi^T \Gamma^{-1} \phi + \gamma^{-1} {\tilde{\rho}}^2
    \label{eq:funcao_positiva}
  \end{equation}
  %
  que envolve os erros paramétricos, pode ser minimizada definindo as seguintes regras adaptativas para $\theta$ e $\rho$:
  %
  \begin{subequations}
    \begin{align}
      \theta(k + 1) & = \theta(k) - \text{sgn}(\rho^*)\frac{\Gamma \epsilon(k)\zeta(k)}{m^2(k)} \text{,}\\
      \rho(k + 1) & = \rho(k) - \frac{\gamma \epsilon(k) \xi(k)}{m^2(k)} \text{,}
    \end{align}
    \label{eq:regras_adaptativas}
  \end{subequations}
  %
  onde $\text{sgn}(\rho^*)$ denota o sinal do parâmetro fixo $\rho^*$, $0 < \gamma < 2$, $0 < \Gamma = \Gamma^T < 2/\rho_0 I_{dim \theta}$, $\rho_0 \geq |b_p / b_m|$ sendo $I_{dim \theta}$ a matriz identidade de mesma dimensão do vetor $\theta$. Em (\ref{eq:regras_adaptativas}) o sinal de normalização $m$ é dado por
  %
  \begin{equation}
    m(k) = \sqrt{1 + \zeta^T(k) \zeta(k) + \xi^2(k)} \text{.}
  \end{equation}

  É possível provar que utilizando (\ref{eq:epsilon_k}), (\ref{eq:funcao_positiva}) e (\ref{eq:regras_adaptativas}) tem-se
  %
  \begin{equation}
    V(k + 1) - V(k) \leq -c \frac{\epsilon^2(k)}{m^2(k)} \text{,}
  \end{equation}
  %
  com $c > 0$, o que implica na convergência do erro de estimação $\epsilon$ para zero e na convergência de $\theta$ e $\rho$ para um valor limitado, como em \cite{ref:TAO}.

  Como evidenciado pela Fig.~\ref{fig:LCL_discreto}, percebe-se que a corrente da rede $I_2$ está sujeita a um distúrbio exógeno de tensão da rede $V_g$. Para compensar este efeito, pode-se aumentar o vetor (\ref{eq:omega_k}) de forma que
  %
  \begin{equation}
    {U_p}^* = \theta_1(k) I_2(k) + \theta_2(k) {I_2}^*(k)+
      \theta_3(k) \text{sen}(\omega_{g1}t) + \theta_4(k) \cos(\omega_{g1} t) \text{.}
  \end{equation}

  A prova de estabilidade detalhada cobrindo o caso do vetor (\ref{eq:omega_k}) aumentado é conforme consta no Anexo~\ref{provas}.



%FIM----------------------------------------------------------------------
