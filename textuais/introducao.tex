%TITULO------------------------------------------------------------------------

%==============================================================================
\label{introducao}
%==============================================================================

	%comentar sobre a conexão de conversores de tensão na rede e aplicações de filtragem
	%ativa. Depois, falar de geração distribuída e comentar problemas de harmônicas
	%de comutação
	Diversos fatores têm levado à intensificação no uso de conversores eletrônicos
	de potência nos últimos anos. As inúmeras aplicações que precisam de uma forma
	de conexão com a rede elétrica fazem uso de conversores de potência.
	Novas tecnologias, crise energética e o aumento do efeito estufa são algumas
	dessas aplicações. Aplicações de geração distribuída, como células de energia,
	painéis fotovoltaicos, turbinas eólicas e microturbinas são usadas não só
	para aumentar a energia disponível no sistema, mas também para melhorar sua
	confiabilidade, fornecendo energia aos consumidores mesmo durante uma falta
	na rede~\cite{ref:KARSHENAS}.

	Na maioria desses geradores, a eletricidade está disponível em um estágio
	contínuo, ou é produzida em baixa frequência e, portanto, é convertida
	para um nível contínuo. Inversores de tensão são predominantemente utilizados
	para transferir energia de uma fonte contínua para a rede elétrica.

	Apesar de vastamente utilizados, os inversores de tensão demandam cuidado
	em sua utilização. Isso deve-se ao fato de o inversor de tensão gerar
	inerentemente harmônicas de comutação durante seu funcionamento, o que pode
	ultrapassar os requisitos da rede com relação ao conteúdo harmônico
	das correntes que nela circulam.

	Existem muitas topologias de filtros ativos que podem ser utilizadas para
	resolver este problema~\cite{ref:RIBEIRO}. A mais comum, é a aplicação de
	um filtro L como interface entre a rede e o inversor. Mais recentemente,
	filtros LCL começaram a ser utilizados para esta função~\cite{ref:LINDGREN},
	\cite{ref:TEODORESCU},~\cite{ref:XU}, a fim de obter o
	mesmo resultado de atenuação, porém com componentes fisicamente
	menores~\cite{ref:HOLMES}. Essa redução é muito vantajosa, visto que implica
	em redução no custo do filtro e	nas perdas de operação.

	Por ser de terceira ordem, o filtro LCL apresenta um pico de amplitude em
	sua frequência de ressonância, o que faz com que a estabilidade geral do
	sistema seja reduzida dependendo principalmente de sua frequência de
	ressonância. Dessa forma, é necessário compensar este comportamento.
	Uma maneira de compensar o efeito da ressonância é o amortecimento passivo
	através da inserção	de um resistor em série ou em paralelo com o capacitor
	do filtro~\cite{ref:AHMED}. Embora	esse amortecimento reduza consideravelmente
	o pico de amplitude na frequência de ressonância, ele degrada o desempenho do
	filtro nas altas frequências e, portanto, não é uma solução aceitável para
	aplicações que necessitam do máximo de desempenho~\cite{ref:SHEN}.

	Outra forma de abordar a questão é utilizar uma estratégia de controle que
	compense este comportamento, resultando num amortecimento ativo. Usualmente,
	os controladores são implementados digitalmente em microcontroladores utilizando
	uma frequência de amostragem adequada frente à frequência de ressonância do
	filtro. Isso garante que o controlador seja capaz de atuar sobre e rejeitar
	o distúrbio de amplitude na frequência de ressonância, mas demanda técnicas
	de projeto do filtro e do controlador para atender este requisito.

	Existem múltiplas estratégias de controle para conversores conectados à
	rede elétrica que podem ser classificadas, de maneira geral, como analógicas
	e digitais. Os métodos de controle digital oferecem diversas vantagens sobre
	as técnicas analógicas, como reprogramabilidade, tolerância à variações
	nos componentes, suporte a multiplos modos de operação, melhor eficiência e,
	em geral, melhor desempenho. O controle analógico se limita a estruturas particulares,
	enquanto o controle digital depende apenas dos limites da taxa de amostragem,
	resolução e capacidade computacional~\cite{ref:KIMBALL}.

	Assim sendo, o enfoque recai sobre as técnicas de controle digital. Existem
	muitas técnicas diferentes para o controle de conversores. O controle \textit{PI}
	utiliza compensadores de erro do tipo proporcional-integral para produzir os
	sinais de comando de cada fase. A parte integral do controlador minimiza o
	erro em baixas frequências, enquanto a parte proporcional e a posição do zero
	influenciam na ondulação do sinal. O desafio desta técnica é realizar o
	rastreamento das referências de corrente. Isto é resolvido, em geral,
	utilizando circuitos do tipo PLL (\textit{Phase Locked Loop}) para gerar as
	referências de corrente. O controlador \textit{PI} geralmente é implementado
	em eixos síncronos $dq$, de modo que as	referências senoidais são transformadas
	em sinais constantes. Alternativamente, podem ser utilizados \textit{PI's} em
	eixos estacionários $\alpha \beta$~\cite{ref:KAZMIERKOWSKI}. Em ambos os casos,
	o objetivo é o rastreamento de referências senoidais e a rejeição de distúrbios
	de mesma natureza~\cite{ref:AREERAK}.

	%\comment{Este controlador
	%ainda pode ser implementado em eixos síncronos, sendo utilizado para
	%eliminação de erro da componente fundamental devido à eficiência
	%computacional do método~\cite{ref:AREERAK}.}

	Uma outra técnica é o controle de corrente usando um controlador do tipo
	\textit{Dead-Beat}. Essa é a mais rápida estratégia de controle linear que pode
	ser adotada. Teoricamente, o laço de corrente replica exatamente a corrente
	de referência com dois ciclos de atraso. O controle é baseado no modelo interno
	da carga do conversor, usado para prever o comportamento dinâmico do sistema. O
	controlador, assim sendo, é inerentemente sensível a descasamento de parâmetros
	e de modelo~\cite{ref:MALESANI}.

	Existe ainda o controlador por Histerese. Devido à sua inerente não-linearidade,
	este controlador é capaz de proporcionar a mais rápida resposta dinâmica possível.
	Utilizando esta técnica, é possível atingir o máximo aproveitamento do conversor
	de potência~\cite{ref:YAO}. O limite para a regulação de corrente, na verdade,
	é dado pelo projeto do conversor. O controle de corrente por histerese é estável
	e robusto com relação à variações na carga ou qualquer outro tipo de distúrbios
	dinâmicos~\cite{ref:TENTI}.

	O controle de realimentação é a estratégia de controle mais simples que existe
	para compensar perturbações de um processo. Embora a grande maioria das estratégias
	de controle utilizadas na prática industrial seja controle de realimentação simples,
	essa estratégia apresenta uma desvantagem bastante significativa: é preciso que
	um distúrbio se propague pelo processo, fazendo a variável controlada desviar
	do ponto de operação, para que a realimentação adote uma ação corretiva~\cite{ref:SMITH}.

	Existem aplicações, no entanto, que demandam desempenho superior, devido à alguma
	necessidade específica, dinâmica lenta ou perturbações frequentes. Quando o distúrbio
	é associado à variável controlada ou quando o elemento de controle final apresenta
	comportamento não-linear, o Controle em Multi-Malha melhora significativamente o
	desempenho em relação ao controle com realimentação simples~\cite{ref:KRISHNA}.

	Este tipo de controle pressupõe um conjunto de malhas em cascata, onde as mais
	externas geram as referências para as malhas mais internas. Dessa forma, variáveis
	intermediárias são usadas para reduzir o efeito de algumas dinâmicas no processo.
	Não é mais necessário esperar o distúrbio propagar-se pelo sistema e modificar a
	variável controlada. Uma vez que uma mudança seja detectada em uma variável
	intermediária, a ação corretiva começa imediatamente a ser aplicada na variável
	controlada, reduzindo a magnitude do impacto do distúrbio e consequentemente
	melhorando o desempenho. O único requisito para que isto aconteça é que a malha
	interna seja mais rápida que a malha externa. Quanto mais rápida, melhor, pois
	a velocidade da malha interna implica na velocidade com que mudanças na variável
	intermediária serão detectadas, o que afeta diretamente a redução do impacto
	do distúrbio na variável controlada.

	As técnicas de controle clássicas pressupõe o uso de um modelo interno do sistema
	que deve ser precisamente conhecido. Nas duas últimas décadas, esse requisito
	vem sendo relaxado, e o desafio é desenvolver estratégias de controle robustas
	à incerteza paramétrica~\cite{ref:GEROMEL}.

	Considerando o contexto descrito acima, a proposta deste trabalho é apresentar
	uma estratégia de controle de conversores conectados à rede elétrica através de
	um filtro LCL. Esta estratégia apresenta ótimo desempenho através do uso de
	Controle Multimalha, é capaz de rejeitar distúrbios e é robusta em relação à
	incertezas paramétricas em termos da indutância da rede.

	%COMENTAR SOBRE INCERTEZA PARAMÉTRICA
	%incerteza paramétrica do lado da rede é um problema frequente

	%COMENTAR O CONTROLADOR QUE SERÁ USADO
	%será usado o adaptativo porque uma vez que o controle da tensão do capacitor
	%seja realizado com sucesso, então a dinâmica anterior é desconsiderada e resta
	%apenas o capacitor como uma fonte de tensão ligada a um braço LR. Controlar
	%essa corrente com um controlador adaptativo é muito simples (embora tenha
	%incerteza paramétrica).

	% OK Controle de conversores conectados à rede

	% OK Introdução da proposta

	% OK Usar referências para introduzir multimalhas (Lee e o livro marrom)

	%Incerteza paramétrica, em algum lugar (referências?)

	%Novo capítulo, método do Lee, ajuda do Márcio para a descrição matemática
	%do controlador adaptativo, mostrar o projeto de cada malha

\section*{Objetivos}

	O objetivo deste trabalho é propor uma estratégia de controle para um conversor
	conectado à rede elétrica através de um filtro LCL. A estratégia proposta deve
	ser robusta com relação às incertezas e distúrbios da rede elétrica, e resultar
	numa dinâmica de malha fechada rápida o suficiente para permitir a rejeição de
	distúrbios e o rastreamento de possíveis referências complexas, incluindo harmônicas.

	Os objetivos específicos são:

	\begin{itemize}
		\item Determinação teórica via modelagem matemática e simulação do sistema
			em malha fechada;
		\item Comprovação da análise teórica por meio da obtenção de resultados
			experimentais;
		\item Comparação da proposta com controladores convencionalmente utilizados.
	\end{itemize}

%\section{Revisão Bibliográfica}

%	Será feito no fim.

\section*{Contribuição da Dissertação}

	Uma estratégia de controle que oferece robustez frente a distúrbios e que
	tem capacidade de lidar com incertezas paramétricas do lado da rede.

\section*{Organização do Documento}

	Este trabalho é organizado como segue: o primeiro capítulo apresenta uma introdução
	e motivação para o trabalho proposto. O segundo capítulo apresenta uma breve revisão
	bibliográfica dos elementos necessários ao desenvolvimento do trabalho. O terceiro
	capítulo apresenta a modelagem matemática do sistema.

%FIM---------------------------------------------------------------------------
