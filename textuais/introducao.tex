%TITULO------------------------------------------------------------------------

%==============================================================================
\label{introducao}
%==============================================================================

	%comentar sobre a conexão de conversores de tensão na rede e aplicações de filtragem
	%ativa. Depois, falar de geração distribuída e comentar problemas de harmônicas
	%de comutação

	A tensão fornecida pelo sistema elétrico de potência é idealmente senoidal e
	balanceada, com correntes de linha senoidais, amplitude e frequência fixas e
	fator de potência unitário. Durante a operação real do sistema, no entanto,
	é difícil manter as condições ideais. As divergências do padrão são classificadas
	como problemas de qualidade de energia e, por se tratarem de problemas, devem
	ser corrigidas.

	Problemas de qualidade de energia ocorrem com mais frequência e intensidade em
	ambientes industriais, devido ao tipo de carga instalada. Transformadores,
	fornos a arco, conversores tiristorizados e cargas semelhantes drenam correntes
	harmônicas e causam variações bruscas de energia reativa. É crescente a
	utilização de dispositivos eletrônicos de potência em equipamentos
	eletroeletrônicos, atualmente tão comuns em residências. Tais dispositivos possuem,
	em geral, um estágio de entrada sem correção do fator de potência, fazendo com que
	drenem correntes distorcidas da rede elétrica~\cite{ref:MANSOOR}. Estes fatores em
	conjunto agravam o problema de qualidade de energia, devido às suas consequências
	negativas, como	o aquecimento de condutores e transformadores devido à circulação
	de correntes reativas e o mau funcionamento de equipamentos sensíveis conectados
	ao sistema.	Tais consequências levaram à criação de normas internacionais que
	regulamentam limites para a distorção harmônica total, ou THD (do inglês \emph{
	Total Harmonic Distortion}). Como exemplos de normas pode-se citar a IEC 1000-3-2
	e a IEEE 519-1992.

	É possível mitigar estes problemas através da conexão de filtros de potência
	com a carga. Estes filtros podem ser ativos ou passivos, e a conexão pode ser
	em série, em paralelo, ou em série-paralelo. O filtro pode ser implementado
	com elementos passivos (resistores, indutores e capacitores) ou elementos
	ativos (chaves semicondutoras de potência), sendo os filtros ativos conhecidos
	como FAP's (Filtros Ativos de Potência). Embora filtros passivos sejam mais
	simples de projetar e mais baratos de construir do que filtros ativos, têm como
	desvantagem a possibilidade de oscilar com a impedância da linha e uma capacidade
	de compensação limitada, visto que para cada componente harmônica um reator
	deve ser projetado. Por isso, a partir da década de $70$, com o desenvolvimento
	da tecnologia de dispositivos semicondutores de potência, microprocessadores e
	processadores digitais de sinal, ou DSP's (do inglês \emph{Digital Signal
	Processor}) foi possível desenvolver algoritmos mais complexos de modulação,
	geração de referências e programas supervisórios, o que tornou a utilização de
	FAP's ainda mais popular~\cite{ref:SASAKI}.

	Além disso, diversos fatores têm levado à intensificação no uso de conversores
	eletrônicos	de potência nos últimos anos. As inúmeras aplicações que precisam de
	uma forma de conexão com a rede elétrica fazem uso de conversores de potência.
	Novas tecnologias, a crise energética e o aumento do efeito estufa são alguns dos
	motivos para o aumento desta demanda. Aplicações de geração distribuída, como
	células de energia,	painéis fotovoltaicos, turbinas eólicas e microturbinas são
	usadas não só para aumentar a energia disponível no sistema, mas também para
	melhorar sua confiabilidade, fornecendo energia aos consumidores mesmo durante
	uma falta na rede~\cite{ref:KARSHENAS}.	Na maioria destes geradores, a eletricidade
	está disponível em um estágio contínuo, ou é produzida em baixa frequência e,
	portanto, é convertida para um nível contínuo. Inversores de tensão são
	predominantemente utilizados para transferir energia de uma fonte contínua para
	a rede elétrica.

	Apesar de vastamente utilizados, os inversores de tensão demandam cuidado
	em sua utilização. Isso deve-se ao fato de o inversor de tensão trabalhar
	com uma frequência de comutação da ordem de $kHz$ para manter as perdas de
	comutação em níveis aceitáveis. Para manter as correntes harmônicas oriundas
	do inversor em níveis aceitáveis, de forma a respeitar os códigos de
	rede~\cite{ref:HOLMES}, existem diversas topologias de filtro que podem ser
	utilizadas~\cite{ref:RIBEIRO}. A mais comum é a aplicação de um filtro L como
	interface entre a rede e o inversor. Mais recentemente, filtros LCL começaram
	a ser utilizados para esta função~\cite{ref:LINDGREN}\cite{ref:TEODORESCU}
	\cite{ref:XU}, pois apresentam maior atenuação das frequências harmônicas sem
	aumentar significativamente o consumo de potência reativa na frequência
	fundamental da rede quando comparados a filtros L~\cite{ref:FUCHS}. Além disso,
	as dimensões do filtro LCL são significativamente menores que a de um filtro L,
	reduzindo o custo do filtro e as perdas de operação.

	A indutância da rede pode ser considerada como parte do filtro LCL. No entanto,
	a incerteza quanto ao seu valor real altera a frequência de ressonância do
	filtro e pode levar a instabilidade. Por este motivo, a incerteza quanto ao
	valor da indutância da rede precisa ser incluída no projeto do
	controlador~\cite{ref:LISERRE}. Outro ponto importante é que o controlador
	precisa rejeitar distorções de corrente de baixa ordem resultantes da
	distorção de tensão no ponto de conexão do conversor. Isto, aliado ao fato de
	que o controlador é implementado em um microcontrolador ou um DSP, torna o
	projeto bastante complexo.

	Por ser de terceira ordem, o filtro LCL apresenta um pico de amplitude em
	sua frequência de ressonância, o que faz com que a estabilidade geral do
	sistema seja reduzida dependendo principalmente de sua frequência de
	ressonância. Dessa forma, é necessário realizar o amortecimento deste pico
	de amplitude. É possível realizar este amortecimento de forma passiva
	através da inserção	de um resistor em série ou em paralelo com o capacitor
	do filtro~\cite{ref:AHMED}. Embora este amortecimento reduza consideravelmente
	o pico de amplitude na frequência de ressonância, ele aumenta o consumo de
	energia e degrada o desempenho do filtro nas altas frequências. Não é,
	portanto, uma solução aceitável para aplicações que necessitam do máximo de
	desempenho~\cite{ref:SHEN}. Outra forma de realizar este amortecimento é via
	amortecimento ativo. Isto é alcançado utilizando uma dentre várias estratégias
	de controle possíveis, tais como estruturas de controle específicas~\cite{ref:WU},
	estimação de impedância da rede~\cite{ref:BLAABJERG}, retroação de
	estados~\cite{ref:MASSING}, estratégias utilizando múltiplos laços de
	realimentação~\cite{ref:POH}, dentre outras~\cite{ref:WESSELS} \cite{ref:MORENO}
	\cite{ref:YANG}.

	Essas estratégias podem ser classificadas, de maneira geral, como analógicas
	e digitais. Os métodos de controle digital oferecem diversas vantagens sobre
	as técnicas analógicas, como reprogramabilidade, tolerância à variações
	nos componentes, suporte a multiplos modos de operação, melhor eficiência e,
	em geral, melhor desempenho. O controle analógico se limita a estruturas particulares,
	enquanto o controle digital depende apenas dos limites da taxa de amostragem,
	resolução e capacidade computacional~\cite{ref:KIMBALL}.

	Assim sendo, o enfoque recai sobre as técnicas de controle digital. Existem
	muitas técnicas diferentes para o controle de conversores. O controle \emph{PI}
	utiliza compensadores de erro do tipo proporcional-integral para produzir os
	sinais de comando de cada fase. A parte integral do controlador minimiza o
	erro em baixas frequências, enquanto a parte proporcional e a posição do zero
	influenciam na ondulação do sinal. O desafio desta técnica é realizar o
	rastreamento das referências de corrente. Isto é resolvido, em geral,
	utilizando circuitos do tipo malha de captura de fase, ou PLL (do inglês
	\emph{Phase Locked Loop}) para gerar as referências de corrente. O controlador
	\emph{PI} geralmente é implementado em eixos síncronos $dq$, de modo que as
	referências senoidais são transformadas em sinais constantes. Alternativamente,
	podem ser utilizados \emph{PI's} em eixos estacionários
	$\alpha \beta$~\cite{ref:KAZMIERKOWSKI}. Em ambos os casos, o objetivo é o
	rastreamento de referências senoidais e a rejeição de distúrbios de mesma
	natureza~\cite{ref:AREERAK}.

	Uma outra técnica é o controle de corrente usando um controlador do tipo
	\textit{Dead-Beat}. Essa é a mais rápida estratégia de controle linear que pode
	ser adotada. Teoricamente, o laço de corrente replica exatamente a corrente
	de referência com dois ciclos de atraso. O controle é baseado no modelo interno
	da carga do conversor, usado para prever o comportamento dinâmico do sistema. O
	controlador, assim sendo, é inerentemente sensível a descasamento de parâmetros
	e de modelo~\cite{ref:MALESANI}.

	Existe ainda o controlador por Histerese. Devido à sua inerente não-linearidade,
	este controlador é capaz de proporcionar a mais rápida resposta dinâmica possível.
	Utilizando esta técnica, é possível atingir o máximo aproveitamento do conversor
	de potência~\cite{ref:YAO}. O limite para a regulação de corrente, na verdade,
	é dado pelo projeto do conversor. O controle de corrente por histerese é estável
	e robusto com relação à variações na carga ou qualquer outro tipo de distúrbios
	dinâmicos~\cite{ref:TENTI}.

	O controle de realimentação é a estratégia de controle mais simples que existe
	para compensar perturbações de um processo. Embora a grande maioria das estratégias
	de controle utilizadas na prática industrial seja controle de realimentação simples,
	essa estratégia apresenta uma desvantagem bastante significativa: é preciso que
	um distúrbio se propague pelo processo, fazendo a variável controlada desviar
	do ponto de operação, para que a realimentação adote uma ação corretiva~\cite{ref:SMITH}.

	Existem aplicações, no entanto, que demandam desempenho superior, devido à alguma
	necessidade específica, dinâmica lenta ou perturbações frequentes. Quando o distúrbio
	é associado à variável controlada ou quando o elemento de controle final apresenta
	comportamento não-linear, o Controle em Multi-Malha melhora significativamente o
	desempenho em relação ao controle com realimentação simples~\cite{ref:KRISHNA}.

	Este tipo de controle pressupõe um conjunto de malhas em cascata, onde as mais
	externas geram as referências para as malhas mais internas. Dessa forma, variáveis
	intermediárias são usadas para reduzir o efeito de algumas dinâmicas no processo.
	Não é mais necessário esperar o distúrbio propagar-se pelo sistema e modificar a
	variável controlada. Uma vez que uma mudança seja detectada em uma variável
	intermediária, a ação corretiva começa imediatamente a ser aplicada na variável
	controlada, reduzindo a magnitude do impacto do distúrbio e consequentemente
	melhorando o desempenho. O único requisito para que isto aconteça é que a malha
	interna seja mais rápida que a malha externa. Quanto mais rápida, melhor, pois
	a velocidade da malha interna implica na velocidade com que mudanças na variável
	intermediária serão detectadas, o que afeta diretamente a redução do impacto
	do distúrbio na variável controlada.

	As técnicas de controle clássicas pressupõe o uso de um modelo interno do sistema
	que deve ser precisamente conhecido. Nas duas últimas décadas, este requisito
	vem sendo relaxado, e o desafio é desenvolver estratégias de controle robustas
	à incerteza paramétrica~\cite{ref:GEROMEL}.

	Considerando o contexto descrito acima, a proposta desse trabalho é apresentar
	uma estratégia de controle de conversores conectados à rede elétrica através de
	um filtro LCL. Esta estratégia apresenta ótimo desempenho através do uso de
	Controle Multimalha, é capaz de rejeitar distúrbios e é robusta em relação à
	incertezas paramétricas em termos da indutância da rede.

	%COMENTAR SOBRE INCERTEZA PARAMÉTRICA
	%incerteza paramétrica do lado da rede é um problema frequente

	%COMENTAR O CONTROLADOR QUE SERÁ USADO
	%será usado o adaptativo porque uma vez que o controle da tensão do capacitor
	%seja realizado com sucesso, então a dinâmica anterior é desconsiderada e resta
	%apenas o capacitor como uma fonte de tensão ligada a um braço LR. Controlar
	%essa corrente com um controlador adaptativo é muito simples (embora tenha
	%incerteza paramétrica).

	% OK Controle de conversores conectados à rede

	% OK Introdução da proposta

	% OK Usar referências para introduzir multimalhas (Lee e o livro marrom)

	%Incerteza paramétrica, em algum lugar (referências?)

	%Novo capítulo, método do Lee, ajuda do Márcio para a descrição matemática
	%do controlador adaptativo, mostrar o projeto de cada malha

\section*{Objetivos e Contribuições da Dissertação}

	O objetivo desse trabalho é propor uma estratégia de controle para um conversor
	conectado à rede elétrica através de um filtro LCL. A estratégia proposta deve
	ser robusta com relação às incertezas e distúrbios da rede elétrica, e resultar
	numa dinâmica de malha fechada rápida o suficiente para permitir a rejeição de
	distúrbios e o rastreamento de possíveis referências complexas, incluindo harmônicas.

	Mais especificamente, esta Dissertação visa:

	\begin{itemize}
		\item Propor um controlador adaptativo para controlar a corrente de conversores
			conectados à rede elétrica com um filtro LCL que ajuste automaticamente os
			ganhos e que garanta estabilidade para uma ampla faixa de valores de
			impedância da rede;
		\item Propor um controlador que garanta desempenho e estabilidade frente a
			distúrbios de tensão e incerteza na impedância da rede elétrica;
		\item Realizar a prova de estabilidade do controlador proposto.
	\end{itemize}

%\section{Revisão Bibliográfica}

%	Será feito no fim.

\section*{Organização do Documento}

	Esse trabalho é organizado como segue: o primeiro capítulo apresenta uma introdução
	e motivação para o trabalho proposto. O segundo capítulo apresenta uma breve revisão
	bibliográfica dos elementos necessários ao desenvolvimento do trabalho. O terceiro
	capítulo apresenta a modelagem matemática do sistema.

%FIM---------------------------------------------------------------------------
